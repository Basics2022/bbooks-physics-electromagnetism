%% Generated by Sphinx.
\def\sphinxdocclass{jupyterBook}
\documentclass[letterpaper,10pt,english]{jupyterBook}
\ifdefined\pdfpxdimen
   \let\sphinxpxdimen\pdfpxdimen\else\newdimen\sphinxpxdimen
\fi \sphinxpxdimen=.75bp\relax
\ifdefined\pdfimageresolution
    \pdfimageresolution= \numexpr \dimexpr1in\relax/\sphinxpxdimen\relax
\fi
%% let collapsible pdf bookmarks panel have high depth per default
\PassOptionsToPackage{bookmarksdepth=5}{hyperref}
%% turn off hyperref patch of \index as sphinx.xdy xindy module takes care of
%% suitable \hyperpage mark-up, working around hyperref-xindy incompatibility
\PassOptionsToPackage{hyperindex=false}{hyperref}
%% memoir class requires extra handling
\makeatletter\@ifclassloaded{memoir}
{\ifdefined\memhyperindexfalse\memhyperindexfalse\fi}{}\makeatother

\PassOptionsToPackage{warn}{textcomp}

\catcode`^^^^00a0\active\protected\def^^^^00a0{\leavevmode\nobreak\ }
\usepackage{cmap}
\usepackage{fontspec}
\defaultfontfeatures[\rmfamily,\sffamily,\ttfamily]{}
\usepackage{amsmath,amssymb,amstext}
\usepackage{polyglossia}
\setmainlanguage{english}



\setmainfont{FreeSerif}[
  Extension      = .otf,
  UprightFont    = *,
  ItalicFont     = *Italic,
  BoldFont       = *Bold,
  BoldItalicFont = *BoldItalic
]
\setsansfont{FreeSans}[
  Extension      = .otf,
  UprightFont    = *,
  ItalicFont     = *Oblique,
  BoldFont       = *Bold,
  BoldItalicFont = *BoldOblique,
]
\setmonofont{FreeMono}[
  Extension      = .otf,
  UprightFont    = *,
  ItalicFont     = *Oblique,
  BoldFont       = *Bold,
  BoldItalicFont = *BoldOblique,
]



\usepackage[Bjarne]{fncychap}
\usepackage[,numfigreset=1,mathnumfig]{sphinx}

\fvset{fontsize=\small}
\usepackage{geometry}


% Include hyperref last.
\usepackage{hyperref}
% Fix anchor placement for figures with captions.
\usepackage{hypcap}% it must be loaded after hyperref.
% Set up styles of URL: it should be placed after hyperref.
\urlstyle{same}

\addto\captionsenglish{\renewcommand{\contentsname}{Elettromagnetismo}}

\usepackage{sphinxmessages}



        % Start of preamble defined in sphinx-jupyterbook-latex %
         \usepackage[Latin,Greek]{ucharclasses}
        \usepackage{unicode-math}
        % fixing title of the toc
        \addto\captionsenglish{\renewcommand{\contentsname}{Contents}}
        \hypersetup{
            pdfencoding=auto,
            psdextra
        }
        % End of preamble defined in sphinx-jupyterbook-latex %
        

\title{My sample book}
\date{Nov 09, 2024}
\release{}
\author{basics}
\newcommand{\sphinxlogo}{\vbox{}}
\renewcommand{\releasename}{}
\makeindex
\begin{document}

\pagestyle{empty}
\sphinxmaketitle
\pagestyle{plain}
\sphinxtableofcontents
\pagestyle{normal}
\phantomsection\label{\detokenize{intro::doc}}


\sphinxAtStartPar
\sphinxstylestrong{Introduzione.}

\sphinxAtStartPar
L’elettromagnetismo si occupa dello studio dei fenomeni elettromagnetici prodotti da cariche e correnti elettriche o dalla struttura microscopica della materia (magnetismo naturale)

\sphinxAtStartPar
\sphinxstylestrong{Breve storia.} \sphinxstyleemphasis{Prime esperienze: cariche di 2 tipi diversi e legge di Coulomb;}

\sphinxAtStartPar
\sphinxstylestrong{Argomenti.}

\sphinxAtStartPar
{\hyperref[\detokenize{ch/experiments:classical-electromagnetism-first-experiments}]{\sphinxcrossref{\DUrole{std,std-ref}{Prime esperienze}}}} \sphinxstylestrong{TODO} \sphinxstyleemphasis{Prime esperienze; elettromagnetismo come teoria dei} \sphinxstylestrong{campi} \sphinxstylestrong{TODO} \sphinxstyleemphasis{aggiungere una sezione su first\sphinxhyphen{}experiments\sphinxhyphen{}revisited, dopo la presentazione dei princìpi dell’elettromagnetismo}

\sphinxAtStartPar
\sphinxstylestrong{todo} Aggiungere sezione su strumenti matematici necessari, per la formulazione di una teoria dei campi

\sphinxAtStartPar
{\hyperref[\detokenize{ch/principles:classical-electromagnetism-principles}]{\sphinxcrossref{\DUrole{std,std-ref}{Princìpi dell’elettromagnetismo}}}} \sphinxstylestrong{TODO} \sphinxstyleemphasis{Trattare prima regime stazionario \sphinxhyphen{} elettricità e magnetismo \sphinxhyphen{} e poi regime non\sphinxhyphen{}stazionario \sphinxhyphen{} elettromagnetismo}**?** \sphinxstylestrong{TODO} \sphinxstyleemphasis{Princìpi. Conservazione della carica, leggi di Maxwell, legge di Lorentz} \sphinxstylestrong{TODO} \sphinxstyleemphasis{Principi in forma integrale; princìpi in forma differenziale \sphinxhyphen{} le leggi di Maxwell}

\sphinxAtStartPar
{\hyperref[\detokenize{ch/energy:classical-electromagnetism-energy}]{\sphinxcrossref{\DUrole{std,std-ref}{Energia}}}}

\sphinxAtStartPar
{\hyperref[\detokenize{ch/waves:classical-electromagnetism-waves}]{\sphinxcrossref{\DUrole{std,std-ref}{Onde elettromagnetiche}}}}

\sphinxAtStartPar
{\hyperref[\detokenize{ch/circuits:classical-electromagnetism-circuits}]{\sphinxcrossref{\DUrole{std,std-ref}{Approssimazione circuitale}}}} \sphinxstylestrong{TODO} \sphinxstyleemphasis{Circuiti elettrici; circuiti elettromagnetici; sistemi elettro\sphinxhyphen{}meccanici. Regimi: stazionario, non\sphinxhyphen{}stazionario: regime transitorio e armonico}

\sphinxAtStartPar
\sphinxstylestrong{Extra.}

\sphinxAtStartPar
{\hyperref[\detokenize{ch/optics:classical-electromagnetism-optics}]{\sphinxcrossref{\DUrole{std,std-ref}{Ottica}}}}

\sphinxAtStartPar
\sphinxstylestrong{Elettromagnetismo e relatività} \sphinxstylestrong{todo} \sphinxstyleemphasis{Relatività a per \(v \ll c\); crisi della relatività galileiana}

\sphinxstepscope


\part{Elettromagnetismo}

\sphinxstepscope

\begin{sphinxuseclass}{sd-container-fluid}
\begin{sphinxuseclass}{sd-sphinx-override}
\begin{sphinxuseclass}{sd-p-0}
\begin{sphinxuseclass}{sd-mt-2}
\begin{sphinxuseclass}{sd-mb-4}
\begin{sphinxuseclass}{sd-row}
\begin{sphinxuseclass}{sd-row-cols-2}
\begin{sphinxuseclass}{sd-gx-2}
\begin{sphinxuseclass}{sd-gy-1}
\begin{sphinxuseclass}{sd-col}
\begin{sphinxuseclass}{sd-d-flex-row}
\begin{sphinxuseclass}{sd-align-minor-center}
\begin{sphinxuseclass}{sd-container-fluid}
\begin{sphinxuseclass}{sd-sphinx-override}
\begin{sphinxuseclass}{sd-row}
\begin{sphinxuseclass}{sd-row-cols-2}
\begin{sphinxuseclass}{sd-row-cols-xs-2}
\begin{sphinxuseclass}{sd-row-cols-sm-3}
\begin{sphinxuseclass}{sd-row-cols-md-3}
\begin{sphinxuseclass}{sd-row-cols-lg-3}
\begin{sphinxuseclass}{sd-gx-3}
\begin{sphinxuseclass}{sd-gy-1}
\begin{sphinxuseclass}{sd-col}
\begin{sphinxuseclass}{sd-col-auto}
\begin{sphinxuseclass}{sd-d-flex-row}
\begin{sphinxuseclass}{sd-align-minor-center}
\sphinxAtStartPar
basics

\end{sphinxuseclass}
\end{sphinxuseclass}
\end{sphinxuseclass}
\end{sphinxuseclass}
\begin{sphinxuseclass}{sd-col}
\begin{sphinxuseclass}{sd-col-auto}
\begin{sphinxuseclass}{sd-d-flex-row}
\begin{sphinxuseclass}{sd-align-minor-center}
\sphinxAtStartPar
Nov 09, 2024

\end{sphinxuseclass}
\end{sphinxuseclass}
\end{sphinxuseclass}
\end{sphinxuseclass}
\begin{sphinxuseclass}{sd-col}
\begin{sphinxuseclass}{sd-col-auto}
\begin{sphinxuseclass}{sd-d-flex-row}
\begin{sphinxuseclass}{sd-align-minor-center}
\sphinxAtStartPar
0 min read

\end{sphinxuseclass}
\end{sphinxuseclass}
\end{sphinxuseclass}
\end{sphinxuseclass}
\end{sphinxuseclass}
\end{sphinxuseclass}
\end{sphinxuseclass}
\end{sphinxuseclass}
\end{sphinxuseclass}
\end{sphinxuseclass}
\end{sphinxuseclass}
\end{sphinxuseclass}
\end{sphinxuseclass}
\end{sphinxuseclass}
\end{sphinxuseclass}
\end{sphinxuseclass}
\end{sphinxuseclass}
\end{sphinxuseclass}
\end{sphinxuseclass}
\end{sphinxuseclass}
\end{sphinxuseclass}
\end{sphinxuseclass}
\end{sphinxuseclass}
\end{sphinxuseclass}
\end{sphinxuseclass}
\end{sphinxuseclass}

\chapter{Prime esperienze}
\label{\detokenize{ch/experiments:prime-esperienze}}\label{\detokenize{ch/experiments:classical-electromagnetism-first-experiments}}\label{\detokenize{ch/experiments::doc}}
\sphinxstepscope

\begin{sphinxuseclass}{sd-container-fluid}
\begin{sphinxuseclass}{sd-sphinx-override}
\begin{sphinxuseclass}{sd-p-0}
\begin{sphinxuseclass}{sd-mt-2}
\begin{sphinxuseclass}{sd-mb-4}
\begin{sphinxuseclass}{sd-row}
\begin{sphinxuseclass}{sd-row-cols-2}
\begin{sphinxuseclass}{sd-gx-2}
\begin{sphinxuseclass}{sd-gy-1}
\begin{sphinxuseclass}{sd-col}
\begin{sphinxuseclass}{sd-d-flex-row}
\begin{sphinxuseclass}{sd-align-minor-center}
\begin{sphinxuseclass}{sd-container-fluid}
\begin{sphinxuseclass}{sd-sphinx-override}
\begin{sphinxuseclass}{sd-row}
\begin{sphinxuseclass}{sd-row-cols-2}
\begin{sphinxuseclass}{sd-row-cols-xs-2}
\begin{sphinxuseclass}{sd-row-cols-sm-3}
\begin{sphinxuseclass}{sd-row-cols-md-3}
\begin{sphinxuseclass}{sd-row-cols-lg-3}
\begin{sphinxuseclass}{sd-gx-3}
\begin{sphinxuseclass}{sd-gy-1}
\begin{sphinxuseclass}{sd-col}
\begin{sphinxuseclass}{sd-col-auto}
\begin{sphinxuseclass}{sd-d-flex-row}
\begin{sphinxuseclass}{sd-align-minor-center}
\sphinxAtStartPar
basics

\end{sphinxuseclass}
\end{sphinxuseclass}
\end{sphinxuseclass}
\end{sphinxuseclass}
\begin{sphinxuseclass}{sd-col}
\begin{sphinxuseclass}{sd-col-auto}
\begin{sphinxuseclass}{sd-d-flex-row}
\begin{sphinxuseclass}{sd-align-minor-center}
\sphinxAtStartPar
Nov 09, 2024

\end{sphinxuseclass}
\end{sphinxuseclass}
\end{sphinxuseclass}
\end{sphinxuseclass}
\begin{sphinxuseclass}{sd-col}
\begin{sphinxuseclass}{sd-col-auto}
\begin{sphinxuseclass}{sd-d-flex-row}
\begin{sphinxuseclass}{sd-align-minor-center}
\sphinxAtStartPar
2 min read

\end{sphinxuseclass}
\end{sphinxuseclass}
\end{sphinxuseclass}
\end{sphinxuseclass}
\end{sphinxuseclass}
\end{sphinxuseclass}
\end{sphinxuseclass}
\end{sphinxuseclass}
\end{sphinxuseclass}
\end{sphinxuseclass}
\end{sphinxuseclass}
\end{sphinxuseclass}
\end{sphinxuseclass}
\end{sphinxuseclass}
\end{sphinxuseclass}
\end{sphinxuseclass}
\end{sphinxuseclass}
\end{sphinxuseclass}
\end{sphinxuseclass}
\end{sphinxuseclass}
\end{sphinxuseclass}
\end{sphinxuseclass}
\end{sphinxuseclass}
\end{sphinxuseclass}
\end{sphinxuseclass}
\end{sphinxuseclass}

\chapter{Princìpi dell’elettromagnetismo classico}
\label{\detokenize{ch/principles:principi-dell-elettromagnetismo-classico}}\label{\detokenize{ch/principles:classical-electromagnetism-principles}}\label{\detokenize{ch/principles::doc}}
\sphinxAtStartPar
I progressi nello studio dei fenomeni elettromagnetici nel XIX secolo, permisero a James Clerk Maxwell di formulare quelle che oggi sono note con il nome di \sphinxstyleemphasis{equazioni di Maxwell} e possono essere considerate la prima formulazione consistente dei principi dell’elettromagnetismo classico, insieme alla legge di bilancio della carica e all’espressione della forza di Lorentz su una carica elettrica immersa in un campo elettromagnetico.

\sphinxAtStartPar
I principi in forma differenziale possono essere ricavati dai principi in forma integrale, più generici, se i campi soddisfano le necessarie condizioni di regolarità minima, che qualitativamente si possono enunciare come “tutte le operazioni svolte devono avere senso”.


\section{Prinicipi in forma differenziale}
\label{\detokenize{ch/principles:prinicipi-in-forma-differenziale}}
\sphinxAtStartPar
\sphinxstylestrong{Conservazione della carica elettrica.}
\begin{equation*}
\begin{split}\partial_t \rho + \nabla \cdot \mathbf{j} = 0 \ .\end{split}
\end{equation*}
\sphinxAtStartPar
\sphinxstylestrong{Equazioni di Maxwell.}
\begin{equation*}
\begin{split}\begin{cases}
 \nabla \cdot \mathbf{d} = \rho \\
 \nabla \times \mathbf{e} + \partial_t \mathbf{b} = \mathbf{0} \\ 
 \nabla \cdot \mathbf{b} = 0 \\
 \nabla \times \mathbf{h} - \partial_t \mathbf{d} = \mathbf{j} \\
\end{cases}\end{split}
\end{equation*}
\sphinxAtStartPar
con la necessità di definire delle equazioni costitutive \(\mathbf{d}(\mathbf{e}, \mathbf{b})\), \(\mathbf{h}(\mathbf{e}, \mathbf{b})\).

\sphinxAtStartPar
\sphinxstylestrong{Forza di Lorentz.} La forza per unità di volume agente sulla carica elettrica presente in un punto \(\mathbf{r}\) nello spazio è
\begin{equation*}
\begin{split}\begin{aligned}
  \mathbf{f}(\mathbf{r},t) & = \rho(\mathbf{r},t) \, \mathbf{e}(\mathbf{r},t) + \mathbf{j}(\mathbf{r},t) \times \mathbf{b}(\mathbf{r},t) = \\
                           & = \rho(\mathbf{r},t) \left[ \mathbf{e}(\mathbf{r}) + \mathbf{v}(\mathbf{r},t) \times \mathbf{b}(\mathbf{r},t) \right] =  \\
                           & = \rho(\mathbf{r},t) \, \mathbf{e}^*(\mathbf{r},t) 
\end{aligned}\end{split}
\end{equation*}
\sphinxAtStartPar
avendo definito \(\mathbf{e}^*\) il campo elettrico \sphinxstylestrong{visto dalla carica in movimento}.*


\section{Principi in forma integrale: equazioni dell’elettromagnetismo e relatività galileiana}
\label{\detokenize{ch/principles:principi-in-forma-integrale-equazioni-dell-elettromagnetismo-e-relativita-galileiana}}

\subsection{Forma integrale su volumi di controllo}
\label{\detokenize{ch/principles:forma-integrale-su-volumi-di-controllo}}
\sphinxAtStartPar
La forma integrale dei principi dell’elettromagnetismo per volumi \(V\) e superfici \(S\) fissi nello spazio viene ricavata integrando le equazioni differenziali sui domini e usando il teorema della divergenza per ottenere termini di flusso, e il teorema del rotore per ottenere termini di circuitazione.

\sphinxAtStartPar
\sphinxstylestrong{Continuità della carica elettrica.}
\begin{equation*}
\begin{split}
    \dfrac{d}{dt} \int_{V} \rho + \oint_{\partial V} \mathbf{j} \cdot \hat{\mathbf{n}} = 0
\end{split}
\end{equation*}
\sphinxAtStartPar
\sphinxstylestrong{Legge di Gauss per il campo \(\mathbf{d}(\mathbf{r},t)\).}
\begin{equation*}
\begin{split}
    \oint_{\partial V} \mathbf{d} \cdot \mathbf{\hat{n}} = \int_{V} \rho
\end{split}
\end{equation*}
\sphinxAtStartPar
\sphinxstylestrong{Legge di Gauss per il campo \(\mathbf{b}(\mathbf{r},t)\).}
\begin{equation*}
\begin{split}
    \oint_{\partial V} \mathbf{b} \cdot \mathbf{\hat{n}} = 0
\end{split}
\end{equation*}
\sphinxAtStartPar
\sphinxstylestrong{Legge di Faraday\sphinxhyphen{}Neumann\sphinxhyphen{}Lenz, per l’induzione elettromagnetica.}
\begin{equation*}
\begin{split}
    \oint_{\partial S} \mathbf{e} \cdot \hat{\mathbf{t}} + \dfrac{d}{dt} \int_{S} \mathbf{b} \cdot \hat{\mathbf{n}} = \mathbf{0}
\end{split}
\end{equation*}
\sphinxAtStartPar
\sphinxstylestrong{Legge di Ampére\sphinxhyphen{}Maxwell.}
\begin{equation*}
\begin{split}
    \oint_{\partial S} \mathbf{h} \cdot \hat{\mathbf{t}} - \dfrac{d}{dt} \int_{S} \mathbf{d} \cdot \hat{\mathbf{n}} = \int_{S} \mathbf{j} \cdot \hat{\mathbf{n}} \ .
\end{split}
\end{equation*}

\subsection{Forma integrale su volumi arbitrari}
\label{\detokenize{ch/principles:forma-integrale-su-volumi-arbitrari}}
\sphinxAtStartPar
Per la loro importanza in applicazioni fondamentali come i motori elettrici, e per evitare confusione e voli pindarici quando si tratta il fenomeno dell’induzione elettromagnetica, risulta di primaria importanza fornire l’espressione corretta dei principi dell’elettromagnetismo quando sono coinvolti volumi mobili nello spazio. Non viene solo mostrata la forma di questi principi, ma anche il procedimento corretto per ricavarli partendo dalla loro forma valida per volumi di controllo fermi nello spazio: per fare questo, vengono usate le regole di derivazione nel tempo di integrali fondamentali su domini mobili, come l’integrale su un volume di una funzione densità, il flusso di un campo vettoriale attraverso una superficie o una circuitazione lungo una curva.

\sphinxAtStartPar
Queste tre regole di derivazione recitano \sphinxstylestrong{todo} Iniziare il bbook di calcolo vettoriale, e aggiungere riferimento
\begin{equation*}
\begin{split}\dfrac{d}{dt} \int_{v_t} f = \int_{v_t} \dfrac{\partial f}{\partial t} + \oint_{\partial v_t} f \, \mathbf{u}_b \cdot \hat{\mathbf{n}}\end{split}
\end{equation*}\begin{equation*}
\begin{split}\dfrac{d}{dt} \int_{s_t} \mathbf{f} \cdot \hat{\mathbf{n}} = \int_{s_t} \dfrac{\partial \mathbf{f}}{\partial t} \cdot \hat{\mathbf{n}} + \int_{s_t} \nabla \cdot \mathbf{f} \, \mathbf{u}_b \cdot \hat{\mathbf{n}} - \oint_{\partial s_t} \mathbf{u}_b \times \mathbf{f} \cdot \hat{\mathbf{t}}\end{split}
\end{equation*}\begin{equation*}
\begin{split}\dfrac{d}{dt} \int_{\ell_t} \mathbf{f} \cdot \hat{\mathbf{t}} = \int_{\ell_t} \dfrac{\partial \mathbf{f}}{\partial t} \cdot \hat{\mathbf{t}} + \int_{\ell_t} \nabla \times \mathbf{f} \, \cdot \, \mathbf{u}_b \times \hat{\mathbf{t}} + \mathbf{f}_B \cdot \mathbf{u}_B - \mathbf{f}_A \cdot \mathbf{u}_A\end{split}
\end{equation*}
\sphinxAtStartPar
\sphinxstylestrong{Continuità della carica elettrica.}
\begin{equation*}
\begin{split}\begin{aligned}
   0 & = \dfrac{d}{dt} \int_{V} \rho + \oint_{\partial V} \mathbf{j} \cdot \hat{\mathbf{n}} = \\
   & = \dfrac{d}{dt} \int_{v_t} \rho - \oint_{\partial v_t } \rho \mathbf{u}_b \cdot \hat{\mathbf{n}} + \oint_{\partial v_t} \mathbf{j} \cdot \hat{\mathbf{n}} 
\end{aligned}\end{split}
\end{equation*}\begin{equation*}
\begin{split}
    \dfrac{d}{dt} \int_{v_t} \rho + \oint_{\partial v_t} \underbrace{\rho (\mathbf{u} - \mathbf{u}_b)}_{\mathbf{j}^*} \cdot \hat{\mathbf{n}} 
\end{split}
\end{equation*}
\sphinxAtStartPar
\sphinxstylestrong{Legge di Gauss per il campo \(\mathbf{d}(\mathbf{r},t)\).}
\begin{equation*}
\begin{split}
    \oint_{\partial v_t} \mathbf{d} \cdot \mathbf{\hat{n}} = \int_{v_t} \rho
\end{split}
\end{equation*}
\sphinxAtStartPar
\sphinxstylestrong{Legge di Gauss per il campo \(\mathbf{b}(\mathbf{r},t)\).}
\begin{equation*}
\begin{split}
    \oint_{\partial v_t} \mathbf{b} \cdot \mathbf{\hat{n}} = 0
\end{split}
\end{equation*}
\sphinxAtStartPar
\sphinxstylestrong{Legge di Faraday\sphinxhyphen{}Neumann\sphinxhyphen{}Lenz, per l’induzione elettromagnetica.}
\begin{equation*}
\begin{split}\begin{aligned}
   \mathbf{0} & = \oint_{\partial S} \mathbf{e} \cdot \hat{\mathbf{t}} + \dfrac{d}{dt} \int_{S} \mathbf{b} \cdot \hat{\mathbf{n}} = \\
    & = \oint_{\partial s_t} \mathbf{e} \cdot \hat{\mathbf{t}} + \dfrac{d}{dt} \int_{s_t} \mathbf{b} \cdot \hat{\mathbf{n}} - \int_{s_t} \underbrace{\nabla \cdot \mathbf{b}}_{=0} \, \mathbf{u}_b \cdot \hat{\mathbf{n}} + \oint_{s_t} \mathbf{u}_b \times \mathbf{b} \cdot \hat{\mathbf{t}} =  \\
\end{aligned}\end{split}
\end{equation*}\begin{equation*}
\begin{split}
    \oint_{\partial s_t} \mathbf{e}^* \cdot \hat{\mathbf{t}} + \dfrac{d}{dt} \int_{s_t} \mathbf{b} \cdot \hat{\mathbf{n}} \ ,
\end{split}
\end{equation*}
\sphinxAtStartPar
con la definizione \(\mathbf{e}^* := \mathbf{e} + \mathbf{u}_b \cdot \mathbf{b}\), già usata nell’espressione della legge di Lorentz.

\sphinxAtStartPar
\sphinxstylestrong{Legge di Ampére\sphinxhyphen{}Maxwell.}
\begin{equation*}
\begin{split}\begin{aligned}
    \mathbf{0} & = \oint_{\partial s_t} \mathbf{h} \cdot \hat{\mathbf{t}} - \dfrac{d}{dt} \int_{s_t} \mathbf{d} \cdot \hat{\mathbf{n}} - \int_{s_t} \mathbf{j} \cdot \hat{\mathbf{n}} = \\
    & = \oint_{\partial s_t} \mathbf{h} \cdot \hat{\mathbf{t}} - \dfrac{d}{dt} \int_{s_t} \mathbf{d} \cdot \hat{\mathbf{n}} + \int_{s_t} \underbrace{\nabla \cdot \mathbf{d}}_{=\rho} \, \mathbf{u}_b \cdot \hat{\mathbf{n}} - \oint_{s_t} \mathbf{u}_b \times \mathbf{d} \cdot \hat{\mathbf{t}} - \int_{s_t} \mathbf{j} \cdot \hat{\mathbf{n}} =  \\
\end{aligned}\end{split}
\end{equation*}\begin{equation*}
\begin{split}
    \oint_{\partial s_t} \mathbf{h}^* \cdot \hat{\mathbf{t}} - \dfrac{d}{dt} \int_{s_t} \mathbf{b} \cdot \hat{\mathbf{n}} = \int_{s_t} \mathbf{j}^* \cdot \hat{\mathbf{n}} \ ,
\end{split}
\end{equation*}
\sphinxAtStartPar
avendo definito \(\mathbf{h}^* := \mathbf{h} - \mathbf{u}_b \times \mathbf{d}\), e usato la definizione già introdotta in precedenza \(\mathbf{j}^* := \mathbf{j} - \rho \mathbf{u}_b\).

\sphinxAtStartPar
Aggiungendo le definizioni
\begin{equation*}
\begin{split}\rho^* = \rho\end{split}
\end{equation*}\begin{equation*}
\begin{split}\mathbf{d}^* = \mathbf{d}\end{split}
\end{equation*}\begin{equation*}
\begin{split}\mathbf{b}^* = \mathbf{b}\end{split}
\end{equation*}
\sphinxAtStartPar
si ottengono delle equazioni che hanno la stessa forma delle equazioni scritte per domini fermi nello spazio, ma che possono essere applicati a domini mobili. Le definizioni
\begin{equation*}
\begin{split}\begin{aligned}
\rho^* = \rho \qquad & , \qquad \mathbf{j}^* = \mathbf{j} - \rho \mathbf{u}_b \\
\mathbf{d}^* = \mathbf{d} \qquad & , \qquad \mathbf{e}^* = \mathbf{e} + \mathbf{u}_b \times \mathbf{b} \\
\mathbf{b}^* = \mathbf{b} \qquad & , \qquad \mathbf{h}^* = \mathbf{h} - \mathbf{u}_b \times \mathbf{d} \\
\end{aligned}\end{split}
\end{equation*}
\sphinxAtStartPar
non rappresentano altro che la trasformazione dei campi per due osservatori in moto relativo, e corrispondono al limite delle trasformazioni di Lorentz della relatività speciale per velocità \(|\mathbf{u}_b| \ll c\): in questo procedimento vengono ottenute le trasformazioni per basse velocità relative, poiché non è stata considerata nessuna trasformazione delle dimensioni spaziali e temporali, come prevede la teoria della relatività di Einstein.

\sphinxAtStartPar
\sphinxstylestrong{todo} riferimento alle trasformazioni galileiane e di Lorentz per la relatività nell’elettromagnetismo.

\sphinxstepscope

\begin{sphinxuseclass}{sd-container-fluid}
\begin{sphinxuseclass}{sd-sphinx-override}
\begin{sphinxuseclass}{sd-p-0}
\begin{sphinxuseclass}{sd-mt-2}
\begin{sphinxuseclass}{sd-mb-4}
\begin{sphinxuseclass}{sd-row}
\begin{sphinxuseclass}{sd-row-cols-2}
\begin{sphinxuseclass}{sd-gx-2}
\begin{sphinxuseclass}{sd-gy-1}
\begin{sphinxuseclass}{sd-col}
\begin{sphinxuseclass}{sd-d-flex-row}
\begin{sphinxuseclass}{sd-align-minor-center}
\begin{sphinxuseclass}{sd-container-fluid}
\begin{sphinxuseclass}{sd-sphinx-override}
\begin{sphinxuseclass}{sd-row}
\begin{sphinxuseclass}{sd-row-cols-2}
\begin{sphinxuseclass}{sd-row-cols-xs-2}
\begin{sphinxuseclass}{sd-row-cols-sm-3}
\begin{sphinxuseclass}{sd-row-cols-md-3}
\begin{sphinxuseclass}{sd-row-cols-lg-3}
\begin{sphinxuseclass}{sd-gx-3}
\begin{sphinxuseclass}{sd-gy-1}
\begin{sphinxuseclass}{sd-col}
\begin{sphinxuseclass}{sd-col-auto}
\begin{sphinxuseclass}{sd-d-flex-row}
\begin{sphinxuseclass}{sd-align-minor-center}
\sphinxAtStartPar
basics

\end{sphinxuseclass}
\end{sphinxuseclass}
\end{sphinxuseclass}
\end{sphinxuseclass}
\begin{sphinxuseclass}{sd-col}
\begin{sphinxuseclass}{sd-col-auto}
\begin{sphinxuseclass}{sd-d-flex-row}
\begin{sphinxuseclass}{sd-align-minor-center}
\sphinxAtStartPar
Nov 09, 2024

\end{sphinxuseclass}
\end{sphinxuseclass}
\end{sphinxuseclass}
\end{sphinxuseclass}
\begin{sphinxuseclass}{sd-col}
\begin{sphinxuseclass}{sd-col-auto}
\begin{sphinxuseclass}{sd-d-flex-row}
\begin{sphinxuseclass}{sd-align-minor-center}
\sphinxAtStartPar
1 min read

\end{sphinxuseclass}
\end{sphinxuseclass}
\end{sphinxuseclass}
\end{sphinxuseclass}
\end{sphinxuseclass}
\end{sphinxuseclass}
\end{sphinxuseclass}
\end{sphinxuseclass}
\end{sphinxuseclass}
\end{sphinxuseclass}
\end{sphinxuseclass}
\end{sphinxuseclass}
\end{sphinxuseclass}
\end{sphinxuseclass}
\end{sphinxuseclass}
\end{sphinxuseclass}
\end{sphinxuseclass}
\end{sphinxuseclass}
\end{sphinxuseclass}
\end{sphinxuseclass}
\end{sphinxuseclass}
\end{sphinxuseclass}
\end{sphinxuseclass}
\end{sphinxuseclass}
\end{sphinxuseclass}
\end{sphinxuseclass}

\chapter{Potenziali elettromagnetici}
\label{\detokenize{ch/potentials:potenziali-elettromagnetici}}\label{\detokenize{ch/potentials:classical-electromagnetism-potentials}}\label{\detokenize{ch/potentials::doc}}
\sphinxAtStartPar
E’ possibile dimostrare che il sistema di equazioni di Maxwell e dell’equazione del bilancio della carica elettrica è un sistema sovra\sphinxhyphen{}determinato.
In particolare, è possibile dimostrare che, nota la distribuzione di carica e di densità di corrente \sphinxhyphen{} considerate come cause generanti il campo elettrico \sphinxhyphen{}, date le leggi costitutive del materiale, sono sufficienti 4 incognite per definire le 6 incognite (3 componenti, per due campi vettoriali) del problema.
E’ possibile formulare quindi il problema in termini di un potenziale scalare \(\varphi\) e un potenziale vettore \(\mathbf{a}\) per ottenere, insieme a una condizione di gauge che elimini le due arbitrarietà (irrilevanti ai fini del calcolo dei campi fisici) restanti.


\section{Potenziale vettore e potenziale scalare}
\label{\detokenize{ch/potentials:potenziale-vettore-e-potenziale-scalare}}
\sphinxAtStartPar
Partendo dalle equazioni di Maxwell si possono definire i potenziali del campo elettromagnetico. Usando l’equazione di Gauss per il campo magnetico si può introdurre il potenziale vettore \(\mathbf{a}(\mathbf{r},t)\),
\begin{equation*}
\begin{split}0 = \nabla \cdot \mathbf{b} \qquad \rightarrow \qquad \mathbf{b} = \nabla \times \mathbf{a} \ ,\end{split}
\end{equation*}
\sphinxAtStartPar
poiché la divergenza di un rotore è identicamente nulla. Introducendo questa relazione nell’equazione di Faraday\sphinxhyphen{}Newumann\sphinxhyphen{}Lenz, nell’ipotesi di sufficiente regolarità dei campi che consenta di invertire l’ordine delle derivate,
\begin{equation*}
\begin{split}0 = \nabla \times \mathbf{e} + \partial_t \mathbf{b} = \nabla \times \mathbf{e} +  \partial_t \nabla \times \mathbf{a} = \nabla \times (\mathbf{e} + \partial_t \mathbf{a}) \qquad \rightarrow \qquad \mathbf{e} + \partial_t \mathbf{a} = - \nabla \varphi \ ,\end{split}
\end{equation*}
\sphinxAtStartPar
poichè il rotore di un gradiente è identicamente nulla. Le grandezze “fisiche” campo elettrico \(\mathbf{e}(\mathbf{r},t)\) e campo magnetico \(\mathbf{b}(mathbf{r},t)\) possono quindi essere scritte usando i pootenziali elettromagnetici come
\begin{equation*}
\begin{split}\begin{cases}
 \mathbf{e} & = - \nabla \varphi - \partial_t \mathbf{a} \\
 \mathbf{b} & = \nabla \times \mathbf{a} \\
\end{cases}\end{split}
\end{equation*}

\section{Condizioni di gauge}
\label{\detokenize{ch/potentials:condizioni-di-gauge}}
\sphinxAtStartPar
I potenziali sono definiti a meno di una condizione di gauge, un’ulteriore condzione che elimina ogni arbitrarietà nella definizione.
Ad esempio, il potenziale vettore è definito a meno del gradiente di una funzione scalare, poiché \(\nabla \times \nabla f \equiv \mathbf{0}\), e quindi il potenziale \(\tilde{\mathbf{a}} = \mathbf{a} + \nabla f\) produce lo stesso campo magnetico \(\mathbf{b}\)
\begin{equation*}
\begin{split}\nabla \times \tilde{\mathbf{a}} = \nabla \times (\mathbf{a} + \nabla f) = \nabla \times \mathbf{a} \ .\end{split}
\end{equation*}
\sphinxAtStartPar
\sphinxstylestrong{Condizione di gauge di Lorentz.} Per motivi che saranno più evidenti nella sezione sulle {\hyperref[\detokenize{ch/waves:classical-electromagnetism-waves}]{\sphinxcrossref{\DUrole{std,std-ref}{onde elettromagnetiche}}}}, una condizione di gauge conveniente è
\begin{equation*}
\begin{split}\nabla \cdot \mathbf{a} + \frac{1}{c^2} \partial_t \varphi = 0\end{split}
\end{equation*}
\sphinxAtStartPar
\sphinxstylestrong{Condizione di gauge di Coulomb.}
\begin{equation*}
\begin{split}\nabla \cdot \mathbf{a} = 0\end{split}
\end{equation*}
\sphinxstepscope

\begin{sphinxuseclass}{sd-container-fluid}
\begin{sphinxuseclass}{sd-sphinx-override}
\begin{sphinxuseclass}{sd-p-0}
\begin{sphinxuseclass}{sd-mt-2}
\begin{sphinxuseclass}{sd-mb-4}
\begin{sphinxuseclass}{sd-row}
\begin{sphinxuseclass}{sd-row-cols-2}
\begin{sphinxuseclass}{sd-gx-2}
\begin{sphinxuseclass}{sd-gy-1}
\begin{sphinxuseclass}{sd-col}
\begin{sphinxuseclass}{sd-d-flex-row}
\begin{sphinxuseclass}{sd-align-minor-center}
\begin{sphinxuseclass}{sd-container-fluid}
\begin{sphinxuseclass}{sd-sphinx-override}
\begin{sphinxuseclass}{sd-row}
\begin{sphinxuseclass}{sd-row-cols-2}
\begin{sphinxuseclass}{sd-row-cols-xs-2}
\begin{sphinxuseclass}{sd-row-cols-sm-3}
\begin{sphinxuseclass}{sd-row-cols-md-3}
\begin{sphinxuseclass}{sd-row-cols-lg-3}
\begin{sphinxuseclass}{sd-gx-3}
\begin{sphinxuseclass}{sd-gy-1}
\begin{sphinxuseclass}{sd-col}
\begin{sphinxuseclass}{sd-col-auto}
\begin{sphinxuseclass}{sd-d-flex-row}
\begin{sphinxuseclass}{sd-align-minor-center}
\sphinxAtStartPar
basics

\end{sphinxuseclass}
\end{sphinxuseclass}
\end{sphinxuseclass}
\end{sphinxuseclass}
\begin{sphinxuseclass}{sd-col}
\begin{sphinxuseclass}{sd-col-auto}
\begin{sphinxuseclass}{sd-d-flex-row}
\begin{sphinxuseclass}{sd-align-minor-center}
\sphinxAtStartPar
Nov 09, 2024

\end{sphinxuseclass}
\end{sphinxuseclass}
\end{sphinxuseclass}
\end{sphinxuseclass}
\begin{sphinxuseclass}{sd-col}
\begin{sphinxuseclass}{sd-col-auto}
\begin{sphinxuseclass}{sd-d-flex-row}
\begin{sphinxuseclass}{sd-align-minor-center}
\sphinxAtStartPar
0 min read

\end{sphinxuseclass}
\end{sphinxuseclass}
\end{sphinxuseclass}
\end{sphinxuseclass}
\end{sphinxuseclass}
\end{sphinxuseclass}
\end{sphinxuseclass}
\end{sphinxuseclass}
\end{sphinxuseclass}
\end{sphinxuseclass}
\end{sphinxuseclass}
\end{sphinxuseclass}
\end{sphinxuseclass}
\end{sphinxuseclass}
\end{sphinxuseclass}
\end{sphinxuseclass}
\end{sphinxuseclass}
\end{sphinxuseclass}
\end{sphinxuseclass}
\end{sphinxuseclass}
\end{sphinxuseclass}
\end{sphinxuseclass}
\end{sphinxuseclass}
\end{sphinxuseclass}
\end{sphinxuseclass}
\end{sphinxuseclass}

\chapter{Leggi costitutive}
\label{\detokenize{ch/media:leggi-costitutive}}\label{\detokenize{ch/media:classical-electromagnetism-media}}\label{\detokenize{ch/media::doc}}
\sphinxAtStartPar
\sphinxstylestrong{todo}


\section{Vuoto}
\label{\detokenize{ch/media:vuoto}}
\sphinxAtStartPar
“vuoto di materia, ma non di proprietà fisiche”


\section{Mezzi continui}
\label{\detokenize{ch/media:mezzi-continui}}\begin{itemize}
\item {} 
\sphinxAtStartPar
leggi costitutive generali

\item {} 
\sphinxAtStartPar
mezzi lineari non disperivi:
\begin{itemize}
\item {} 
\sphinxAtStartPar
e isotropi

\end{itemize}

\end{itemize}


\section{Polarizzazione e Magnetizzazione dei mezzi}
\label{\detokenize{ch/media:polarizzazione-e-magnetizzazione-dei-mezzi}}\begin{itemize}
\item {} 
\sphinxAtStartPar
bounded/free charges and currents

\end{itemize}


\section{Esempi}
\label{\detokenize{ch/media:esempi}}\begin{itemize}
\item {} 
\sphinxAtStartPar
conduttori

\item {} 
\sphinxAtStartPar
ferromagentici e magnetismo debole (para\sphinxhyphen{}, dia\sphinxhyphen{}, anti\sphinxhyphen{})

\end{itemize}

\sphinxstepscope

\begin{sphinxuseclass}{sd-container-fluid}
\begin{sphinxuseclass}{sd-sphinx-override}
\begin{sphinxuseclass}{sd-p-0}
\begin{sphinxuseclass}{sd-mt-2}
\begin{sphinxuseclass}{sd-mb-4}
\begin{sphinxuseclass}{sd-row}
\begin{sphinxuseclass}{sd-row-cols-2}
\begin{sphinxuseclass}{sd-gx-2}
\begin{sphinxuseclass}{sd-gy-1}
\begin{sphinxuseclass}{sd-col}
\begin{sphinxuseclass}{sd-d-flex-row}
\begin{sphinxuseclass}{sd-align-minor-center}
\begin{sphinxuseclass}{sd-container-fluid}
\begin{sphinxuseclass}{sd-sphinx-override}
\begin{sphinxuseclass}{sd-row}
\begin{sphinxuseclass}{sd-row-cols-2}
\begin{sphinxuseclass}{sd-row-cols-xs-2}
\begin{sphinxuseclass}{sd-row-cols-sm-3}
\begin{sphinxuseclass}{sd-row-cols-md-3}
\begin{sphinxuseclass}{sd-row-cols-lg-3}
\begin{sphinxuseclass}{sd-gx-3}
\begin{sphinxuseclass}{sd-gy-1}
\begin{sphinxuseclass}{sd-col}
\begin{sphinxuseclass}{sd-col-auto}
\begin{sphinxuseclass}{sd-d-flex-row}
\begin{sphinxuseclass}{sd-align-minor-center}
\sphinxAtStartPar
basics

\end{sphinxuseclass}
\end{sphinxuseclass}
\end{sphinxuseclass}
\end{sphinxuseclass}
\begin{sphinxuseclass}{sd-col}
\begin{sphinxuseclass}{sd-col-auto}
\begin{sphinxuseclass}{sd-d-flex-row}
\begin{sphinxuseclass}{sd-align-minor-center}
\sphinxAtStartPar
Nov 09, 2024

\end{sphinxuseclass}
\end{sphinxuseclass}
\end{sphinxuseclass}
\end{sphinxuseclass}
\begin{sphinxuseclass}{sd-col}
\begin{sphinxuseclass}{sd-col-auto}
\begin{sphinxuseclass}{sd-d-flex-row}
\begin{sphinxuseclass}{sd-align-minor-center}
\sphinxAtStartPar
1 min read

\end{sphinxuseclass}
\end{sphinxuseclass}
\end{sphinxuseclass}
\end{sphinxuseclass}
\end{sphinxuseclass}
\end{sphinxuseclass}
\end{sphinxuseclass}
\end{sphinxuseclass}
\end{sphinxuseclass}
\end{sphinxuseclass}
\end{sphinxuseclass}
\end{sphinxuseclass}
\end{sphinxuseclass}
\end{sphinxuseclass}
\end{sphinxuseclass}
\end{sphinxuseclass}
\end{sphinxuseclass}
\end{sphinxuseclass}
\end{sphinxuseclass}
\end{sphinxuseclass}
\end{sphinxuseclass}
\end{sphinxuseclass}
\end{sphinxuseclass}
\end{sphinxuseclass}
\end{sphinxuseclass}
\end{sphinxuseclass}

\chapter{Bilancio di energia del campo elettromagnetico}
\label{\detokenize{ch/energy:bilancio-di-energia-del-campo-elettromagnetico}}\label{\detokenize{ch/energy:classical-electromagnetism-energy}}\label{\detokenize{ch/energy::doc}}
\sphinxAtStartPar
Ispirati dalle dimensioni fisiche dei campi elettromagnetici,
\begin{equation*}
\begin{split}\begin{aligned}
\left[\mathbf{e}\right] = \frac{\text{force}}{\text{charge}} \qquad & , \qquad
[\mathbf{d}] = \frac{\text{charge}}{\text{length}^2} \\
[\mathbf{b}] = \frac{\text{force}\cdot\text{time}}{\text{charge}\cdot\text{length}} \qquad & , \qquad
[\mathbf{h}] = \frac{\text{charge}}{\text{time} \cdot \text{length}}
\end{aligned}\end{split}
\end{equation*}



\begin{equation*}
\begin{split}\begin{aligned}
\left[\mathbf{e} \cdot \mathbf{d}\right] & = \frac{\text{force}}{\text{length}^2} = \frac{\text{energy}}{\text{length}^3} = [u] \\
[\mathbf{b} \cdot \mathbf{h}] & = \frac{\text{force}}{\text{length}^2} = \frac{\text{energy}}{\text{length}^3} = [u]
\end{aligned}\end{split}
\end{equation*}
\sphinxAtStartPar
si può costruire la densità di volume di energia  (\sphinxstylestrong{todo} trovare motivazioni più convincenti, non basandosi solo sull’analisi dimensionale ma sul lavoro)
\begin{equation*}
\begin{split}u = \frac{1}{2} \left( \mathbf{e} \cdot \mathbf{d} + \mathbf{b} \cdot \mathbf{h} \right) \ .\end{split}
\end{equation*}
\sphinxAtStartPar
Si può calcolare la derivata parziale nel tempo della densità di energia, \(u\), e usare le equazioni di Maxwell per ottenere un’equazione di bilancio dell’energia del campo elettromagnetico. Per un mezzo isotropo lineare, per il quale valgono le equazioni costitutive \(\mathbf{d} = \varepsilon \mathbf{e}\), \(\mathbf{b} = \mu \mathbf{h}\), la derivata parziale nel tempo dell’energia elettromagnetica può essere riscritta sfuttando la regola di derivazione del prodotto e le equazioni di Faraday\sphinxhyphen{}Lenz\sphinxhyphen{}Neumann e Ampére\sphinxhyphen{}Maxwell,
\begin{equation*}
\begin{split}\begin{aligned}
\dfrac{\partial u}{\partial t} & = \dfrac{\partial}{\partial t}\left( \frac{1}{2} \mathbf{e} \cdot \mathbf{d} + \mathbf{b} \cdot \mathbf{h} \right) =  \qquad (...) \\
& = \mathbf{e} \cdot \partial_t \mathbf{d} + \mathbf{h} \cdot \partial_t \mathbf{b} = \\
& = \mathbf{e} \cdot (\nabla \times \mathbf{h} - \mathbf{j}) - \mathbf{h} \cdot \nabla \times \mathbf{e} \ .
\end{aligned}\end{split}
\end{equation*}
\sphinxAtStartPar
L’ultimo termine può essere ulteriormente manipolato, usando l’identità vettoriale
\begin{equation*}
\begin{split}\begin{aligned}
\mathbf{e} \cdot \nabla \times \mathbf{h} - \mathbf{h} \cdot \nabla \times \mathbf{e} & = e_i \varepsilon_{ijk} \partial_j h_k - h_i \varepsilon_{ijk} \partial_j e_k = \qquad (i \rightarrow k, k \rightarrow i)\\
& = e_i \varepsilon_{ijk} \partial_j h_k - h_k \varepsilon_{kji} \partial_j e_i = \\
& = e_i \varepsilon_{ijk} \partial_j h_k + h_k \varepsilon_{ijk} \partial_j e_i = \\
& =  \partial_j (\varepsilon_{ijk} e_i  h_k ) = \\
& =  \partial_j (\varepsilon_{jki} e_i  h_k ) = \\
& = \nabla \cdot (\mathbf{h} \times \mathbf{e}) = - \nabla \cdot (\mathbf{e} \times \mathbf{h})
\end{aligned}\end{split}
\end{equation*}
\sphinxAtStartPar
che permette di scrivere l’equazione del bilancio di energia elettromagnetica come,
\begin{equation*}
\begin{split}\frac{\partial u }{\partial t} + \nabla \cdot \mathbf{s} = - \mathbf{e} \cdot \mathbf{j} \ ,\end{split}
\end{equation*}
\sphinxAtStartPar
dove è stato definito il \sphinxstylestrong{vettore di Poynting}, o meglio il campo vettoriale di Poynting
\begin{equation*}
\begin{split}\mathbf{s}(\mathbf{r},t) := - \mathbf{e}(\mathbf{r},t) \times \mathbf{h}(\mathbf{r},t) \ ,\end{split}
\end{equation*}
\sphinxAtStartPar
che può essere identificato come un flusso di potenza per unità di superficie, comparendo sotto l’operatore di divergenza nel bilnacio di energia.

\sphinxAtStartPar
\sphinxstylestrong{todo.} Rimandare a una sezione in cui si mostra questa ultima affermazione passando dal bilancio differenziale al bilancio integrale e si usa il teorema della divergenza, \(\int_V \nabla \cdot \mathbf{s} = \oint_{\partial V} \mathbf{s} \cdot \hat{\mathbf{n}}\).

\sphinxstepscope

\begin{sphinxuseclass}{sd-container-fluid}
\begin{sphinxuseclass}{sd-sphinx-override}
\begin{sphinxuseclass}{sd-p-0}
\begin{sphinxuseclass}{sd-mt-2}
\begin{sphinxuseclass}{sd-mb-4}
\begin{sphinxuseclass}{sd-row}
\begin{sphinxuseclass}{sd-row-cols-2}
\begin{sphinxuseclass}{sd-gx-2}
\begin{sphinxuseclass}{sd-gy-1}
\begin{sphinxuseclass}{sd-col}
\begin{sphinxuseclass}{sd-d-flex-row}
\begin{sphinxuseclass}{sd-align-minor-center}
\begin{sphinxuseclass}{sd-container-fluid}
\begin{sphinxuseclass}{sd-sphinx-override}
\begin{sphinxuseclass}{sd-row}
\begin{sphinxuseclass}{sd-row-cols-2}
\begin{sphinxuseclass}{sd-row-cols-xs-2}
\begin{sphinxuseclass}{sd-row-cols-sm-3}
\begin{sphinxuseclass}{sd-row-cols-md-3}
\begin{sphinxuseclass}{sd-row-cols-lg-3}
\begin{sphinxuseclass}{sd-gx-3}
\begin{sphinxuseclass}{sd-gy-1}
\begin{sphinxuseclass}{sd-col}
\begin{sphinxuseclass}{sd-col-auto}
\begin{sphinxuseclass}{sd-d-flex-row}
\begin{sphinxuseclass}{sd-align-minor-center}
\sphinxAtStartPar
basics

\end{sphinxuseclass}
\end{sphinxuseclass}
\end{sphinxuseclass}
\end{sphinxuseclass}
\begin{sphinxuseclass}{sd-col}
\begin{sphinxuseclass}{sd-col-auto}
\begin{sphinxuseclass}{sd-d-flex-row}
\begin{sphinxuseclass}{sd-align-minor-center}
\sphinxAtStartPar
Nov 09, 2024

\end{sphinxuseclass}
\end{sphinxuseclass}
\end{sphinxuseclass}
\end{sphinxuseclass}
\begin{sphinxuseclass}{sd-col}
\begin{sphinxuseclass}{sd-col-auto}
\begin{sphinxuseclass}{sd-d-flex-row}
\begin{sphinxuseclass}{sd-align-minor-center}
\sphinxAtStartPar
0 min read

\end{sphinxuseclass}
\end{sphinxuseclass}
\end{sphinxuseclass}
\end{sphinxuseclass}
\end{sphinxuseclass}
\end{sphinxuseclass}
\end{sphinxuseclass}
\end{sphinxuseclass}
\end{sphinxuseclass}
\end{sphinxuseclass}
\end{sphinxuseclass}
\end{sphinxuseclass}
\end{sphinxuseclass}
\end{sphinxuseclass}
\end{sphinxuseclass}
\end{sphinxuseclass}
\end{sphinxuseclass}
\end{sphinxuseclass}
\end{sphinxuseclass}
\end{sphinxuseclass}
\end{sphinxuseclass}
\end{sphinxuseclass}
\end{sphinxuseclass}
\end{sphinxuseclass}
\end{sphinxuseclass}
\end{sphinxuseclass}

\chapter{Equazioni dell’elettromagnetismo e relatività galileiana}
\label{\detokenize{ch/low-speed-relativity:equazioni-dell-elettromagnetismo-e-relativita-galileiana}}\label{\detokenize{ch/low-speed-relativity:classical-electromagnetism-low-speed-relativity}}\label{\detokenize{ch/low-speed-relativity::doc}}
\sphinxstepscope

\begin{sphinxuseclass}{sd-container-fluid}
\begin{sphinxuseclass}{sd-sphinx-override}
\begin{sphinxuseclass}{sd-p-0}
\begin{sphinxuseclass}{sd-mt-2}
\begin{sphinxuseclass}{sd-mb-4}
\begin{sphinxuseclass}{sd-row}
\begin{sphinxuseclass}{sd-row-cols-2}
\begin{sphinxuseclass}{sd-gx-2}
\begin{sphinxuseclass}{sd-gy-1}
\begin{sphinxuseclass}{sd-col}
\begin{sphinxuseclass}{sd-d-flex-row}
\begin{sphinxuseclass}{sd-align-minor-center}
\begin{sphinxuseclass}{sd-container-fluid}
\begin{sphinxuseclass}{sd-sphinx-override}
\begin{sphinxuseclass}{sd-row}
\begin{sphinxuseclass}{sd-row-cols-2}
\begin{sphinxuseclass}{sd-row-cols-xs-2}
\begin{sphinxuseclass}{sd-row-cols-sm-3}
\begin{sphinxuseclass}{sd-row-cols-md-3}
\begin{sphinxuseclass}{sd-row-cols-lg-3}
\begin{sphinxuseclass}{sd-gx-3}
\begin{sphinxuseclass}{sd-gy-1}
\begin{sphinxuseclass}{sd-col}
\begin{sphinxuseclass}{sd-col-auto}
\begin{sphinxuseclass}{sd-d-flex-row}
\begin{sphinxuseclass}{sd-align-minor-center}
\sphinxAtStartPar
basics

\end{sphinxuseclass}
\end{sphinxuseclass}
\end{sphinxuseclass}
\end{sphinxuseclass}
\begin{sphinxuseclass}{sd-col}
\begin{sphinxuseclass}{sd-col-auto}
\begin{sphinxuseclass}{sd-d-flex-row}
\begin{sphinxuseclass}{sd-align-minor-center}
\sphinxAtStartPar
Nov 09, 2024

\end{sphinxuseclass}
\end{sphinxuseclass}
\end{sphinxuseclass}
\end{sphinxuseclass}
\begin{sphinxuseclass}{sd-col}
\begin{sphinxuseclass}{sd-col-auto}
\begin{sphinxuseclass}{sd-d-flex-row}
\begin{sphinxuseclass}{sd-align-minor-center}
\sphinxAtStartPar
0 min read

\end{sphinxuseclass}
\end{sphinxuseclass}
\end{sphinxuseclass}
\end{sphinxuseclass}
\end{sphinxuseclass}
\end{sphinxuseclass}
\end{sphinxuseclass}
\end{sphinxuseclass}
\end{sphinxuseclass}
\end{sphinxuseclass}
\end{sphinxuseclass}
\end{sphinxuseclass}
\end{sphinxuseclass}
\end{sphinxuseclass}
\end{sphinxuseclass}
\end{sphinxuseclass}
\end{sphinxuseclass}
\end{sphinxuseclass}
\end{sphinxuseclass}
\end{sphinxuseclass}
\end{sphinxuseclass}
\end{sphinxuseclass}
\end{sphinxuseclass}
\end{sphinxuseclass}
\end{sphinxuseclass}
\end{sphinxuseclass}

\chapter{Onde elettromagnetiche}
\label{\detokenize{ch/waves:onde-elettromagnetiche}}\label{\detokenize{ch/waves:classical-electromagnetism-waves}}\label{\detokenize{ch/waves::doc}}
\sphinxstepscope

\begin{sphinxuseclass}{sd-container-fluid}
\begin{sphinxuseclass}{sd-sphinx-override}
\begin{sphinxuseclass}{sd-p-0}
\begin{sphinxuseclass}{sd-mt-2}
\begin{sphinxuseclass}{sd-mb-4}
\begin{sphinxuseclass}{sd-row}
\begin{sphinxuseclass}{sd-row-cols-2}
\begin{sphinxuseclass}{sd-gx-2}
\begin{sphinxuseclass}{sd-gy-1}
\begin{sphinxuseclass}{sd-col}
\begin{sphinxuseclass}{sd-d-flex-row}
\begin{sphinxuseclass}{sd-align-minor-center}
\begin{sphinxuseclass}{sd-container-fluid}
\begin{sphinxuseclass}{sd-sphinx-override}
\begin{sphinxuseclass}{sd-row}
\begin{sphinxuseclass}{sd-row-cols-2}
\begin{sphinxuseclass}{sd-row-cols-xs-2}
\begin{sphinxuseclass}{sd-row-cols-sm-3}
\begin{sphinxuseclass}{sd-row-cols-md-3}
\begin{sphinxuseclass}{sd-row-cols-lg-3}
\begin{sphinxuseclass}{sd-gx-3}
\begin{sphinxuseclass}{sd-gy-1}
\begin{sphinxuseclass}{sd-col}
\begin{sphinxuseclass}{sd-col-auto}
\begin{sphinxuseclass}{sd-d-flex-row}
\begin{sphinxuseclass}{sd-align-minor-center}
\sphinxAtStartPar
basics

\end{sphinxuseclass}
\end{sphinxuseclass}
\end{sphinxuseclass}
\end{sphinxuseclass}
\begin{sphinxuseclass}{sd-col}
\begin{sphinxuseclass}{sd-col-auto}
\begin{sphinxuseclass}{sd-d-flex-row}
\begin{sphinxuseclass}{sd-align-minor-center}
\sphinxAtStartPar
Nov 09, 2024

\end{sphinxuseclass}
\end{sphinxuseclass}
\end{sphinxuseclass}
\end{sphinxuseclass}
\begin{sphinxuseclass}{sd-col}
\begin{sphinxuseclass}{sd-col-auto}
\begin{sphinxuseclass}{sd-d-flex-row}
\begin{sphinxuseclass}{sd-align-minor-center}
\sphinxAtStartPar
1 min read

\end{sphinxuseclass}
\end{sphinxuseclass}
\end{sphinxuseclass}
\end{sphinxuseclass}
\end{sphinxuseclass}
\end{sphinxuseclass}
\end{sphinxuseclass}
\end{sphinxuseclass}
\end{sphinxuseclass}
\end{sphinxuseclass}
\end{sphinxuseclass}
\end{sphinxuseclass}
\end{sphinxuseclass}
\end{sphinxuseclass}
\end{sphinxuseclass}
\end{sphinxuseclass}
\end{sphinxuseclass}
\end{sphinxuseclass}
\end{sphinxuseclass}
\end{sphinxuseclass}
\end{sphinxuseclass}
\end{sphinxuseclass}
\end{sphinxuseclass}
\end{sphinxuseclass}
\end{sphinxuseclass}
\end{sphinxuseclass}

\section{Equazioni delle onde in elettromagnetismo}
\label{\detokenize{ch/waves-equation:equazioni-delle-onde-in-elettromagnetismo}}\label{\detokenize{ch/waves-equation:classical-electromagnetism-waves-wave-equation}}\label{\detokenize{ch/waves-equation::doc}}
\sphinxAtStartPar
\sphinxstylestrong{Identità vettoriale.}
\begin{equation*}
\begin{split}\Delta \mathbf{v} = \nabla ( \nabla \cdot \mathbf{v} ) - \nabla \times \nabla \times \mathbf{v}\end{split}
\end{equation*}
\sphinxAtStartPar
\sphinxstylestrong{Dim.}
\begin{equation*}
\begin{split}\begin{aligned}
 \nabla \times \nabla \times \mathbf{v} & = \varepsilon_{ijk} \partial_j ( \varepsilon_{klm} \partial_l v_m ) = \\
 & = \varepsilon_{kij} \varepsilon_{klm} \partial_{jl} v_m = \\
 & = ( \delta_{il} \delta_{jm} - \delta_{im} \delta_{jl} )  \partial_{jl} v_m = \\
 & = \partial_{ij} v_j - \partial_{jj} v_i = \\
 & = \nabla (\nabla \cdot \mathbf{v}) - \Delta \mathbf{v} \ ,
\end{aligned}\end{split}
\end{equation*}
\sphinxAtStartPar
avendo utilizzato l’identità
\begin{equation*}
\begin{split}\varepsilon_{ijk} \varepsilon_{ilm} = \delta_{jl} \delta_{km} - \delta_{jm} \delta_{kl}\end{split}
\end{equation*}

\subsection{Potenziali elettromagnetici}
\label{\detokenize{ch/waves-equation:potenziali-elettromagnetici}}
\sphinxAtStartPar
Partendo dalle definizioni dei potenziali elettromagnetici e dalle equazioni di Maxwell, con l’aiuto di alcune identità vettoriali, è possibile (\sphinxstylestrong{TODO} \sphinxstyleemphasis{ipotesi, elencare quelle necessarie alla derivazione}) scrivere delle equazionin delle onde per il potenziale vettore e per il potenziale scalare.
\begin{equation*}
\begin{split}\begin{aligned}
 \mathbf{e} & = - \nabla \varphi - \partial_t \mathbf{a} \\
 \mathbf{b} & = \nabla \times \mathbf{a} \\
\end{aligned}\end{split}
\end{equation*}
\sphinxAtStartPar
Usando le equazioni costitutive
\begin{equation*}
\begin{split}\mathbf{d} = \varepsilon \ \mathbf{e} \qquad , \qquad
\mathbf{b} = \mu \mathbf{h} \end{split}
\end{equation*}
\sphinxAtStartPar
\sphinxstylestrong{Potenziale vettore.}
\begin{equation*}
\begin{split}\mathbf{b} = \nabla \times \mathbf{a}\end{split}
\end{equation*}\begin{equation*}
\begin{split}\begin{aligned}
\mathbf{0} & = \nabla \times \nabla \times \mathbf{a} - \nabla \times \mathbf{b} = \\
 & = - \Delta \mathbf{a} + \nabla(\nabla \cdot \mathbf{a})  - \mu \nabla \times \mathbf{h} = \\
 & = - \Delta \mathbf{a} + \nabla(\nabla \cdot \mathbf{a})  - \mu ( \partial_t \mathbf{d} + \mathbf{j} )  = \\
 & = - \Delta \mathbf{a} + \nabla(\nabla \cdot \mathbf{a})  - \mu ( \varepsilon \partial_t \mathbf{e} + \mathbf{j} )  = \\
 & = - \Delta \mathbf{a} + \nabla(\nabla \cdot \mathbf{a})  - \mu \varepsilon ( - \partial_t \nabla \varphi - \partial_{tt} \mathbf{a} ) + \mu \mathbf{j} = \\
 & = - \Delta \mathbf{a} + \nabla(\nabla \cdot \mathbf{a})  + \frac{1}{c^2} \partial_t \nabla \varphi + \dfrac{1}{c^2} \partial_{tt} \mathbf{a} - \mu \mathbf{j}  \\
\end{aligned}\end{split}
\end{equation*}
\sphinxAtStartPar
Usando la condizione di gauge di Lorentz
\begin{equation*}
\begin{split}\nabla \cdot \mathbf{a} + \frac{1}{c^2} \partial_t  \varphi = 0 \ ,\end{split}
\end{equation*}
\sphinxAtStartPar
si ottiene un’equazione delle onde per il potenziale vettore
\begin{equation*}
\begin{split} \dfrac{1}{c^2} \partial_{tt} \mathbf{a} - \Delta \mathbf{a}  =  \mu \mathbf{j}  \ .\end{split}
\end{equation*}
\sphinxAtStartPar
\sphinxstylestrong{Potenziale scalare.}
\begin{equation*}
\begin{split}\mathbf{e} = \nabla \varphi - \partial_t \mathbf{a}\end{split}
\end{equation*}
\sphinxAtStartPar
Calcolando la derivata nel tempo della condizione di gauge di Lorentz
\begin{equation*}
\begin{split}\begin{aligned}
 0 & = \partial_t (\frac{1}{c^2} \partial_t \varphi + \nabla \cdot \mathbf{a}) = \\
   & = \frac{1}{c^2} \partial_{tt} \varphi + \nabla \cdot \partial_t \mathbf{a} = \\
   & = \frac{1}{c^2} \partial_{tt} \varphi - \nabla \cdot \nabla \varphi - \nabla \cdot \mathbf{e} = \\
   & = \frac{1}{c^2} \partial_{tt} \varphi - \Delta \varphi - \frac{\rho}{\varepsilon} = \\
\end{aligned}\end{split}
\end{equation*}
\sphinxAtStartPar
si arriva all’equazione delle onde per il potenziale scalare,
\begin{equation*}
\begin{split} \frac{1}{c^2} \partial_{tt} \varphi - \Delta \varphi = \frac{\rho}{\varepsilon} \ .\end{split}
\end{equation*}

\subsection{Campo elettrico e campo magnetico}
\label{\detokenize{ch/waves-equation:campo-elettrico-e-campo-magnetico}}
\sphinxAtStartPar
Usando le definizioni dei campi fisici in termini dei potenziali elettromagnetici e la linearità (\sphinxstylestrong{TODO} \sphinxstyleemphasis{tutto deve essere lineare, anche le leggi costitutive}) delle operazioni, partendo dalle equazioni delle onde per i potenziali, si possono ricavare le equazioni delle onde per i campi fisici. \sphinxstylestrong{TODO} \sphinxstyleemphasis{Nell’ipotesi di proprietà costanti e uniformi}

\sphinxAtStartPar
\sphinxstylestrong{Campo elettrico.}
\begin{equation*}
\begin{split}\begin{aligned}
\square \mathbf{e} & = \square ( -\nabla \varphi - \partial_t \mathbf{a}) = \\
& = - \nabla \square \varphi - \partial_t \square \mathbf{a} = \\
& = - \nabla \dfrac{\rho}{\varepsilon} - \mu \partial_t \mathbf{j}  \ .
\end{aligned}\end{split}
\end{equation*}
\sphinxAtStartPar
\sphinxstylestrong{Campo magnetico.}
\begin{equation*}
\begin{split}\begin{aligned}
 \square \mathbf{b} & = \square \nabla \times \mathbf{a} = \\
 & = \nabla \times \square \mathbf{a} = \\
 & = \mu \nabla \times \mathbf{j}
\end{aligned}\end{split}
\end{equation*}
\sphinxstepscope

\begin{sphinxuseclass}{sd-container-fluid}
\begin{sphinxuseclass}{sd-sphinx-override}
\begin{sphinxuseclass}{sd-p-0}
\begin{sphinxuseclass}{sd-mt-2}
\begin{sphinxuseclass}{sd-mb-4}
\begin{sphinxuseclass}{sd-row}
\begin{sphinxuseclass}{sd-row-cols-2}
\begin{sphinxuseclass}{sd-gx-2}
\begin{sphinxuseclass}{sd-gy-1}
\begin{sphinxuseclass}{sd-col}
\begin{sphinxuseclass}{sd-d-flex-row}
\begin{sphinxuseclass}{sd-align-minor-center}
\begin{sphinxuseclass}{sd-container-fluid}
\begin{sphinxuseclass}{sd-sphinx-override}
\begin{sphinxuseclass}{sd-row}
\begin{sphinxuseclass}{sd-row-cols-2}
\begin{sphinxuseclass}{sd-row-cols-xs-2}
\begin{sphinxuseclass}{sd-row-cols-sm-3}
\begin{sphinxuseclass}{sd-row-cols-md-3}
\begin{sphinxuseclass}{sd-row-cols-lg-3}
\begin{sphinxuseclass}{sd-gx-3}
\begin{sphinxuseclass}{sd-gy-1}
\begin{sphinxuseclass}{sd-col}
\begin{sphinxuseclass}{sd-col-auto}
\begin{sphinxuseclass}{sd-d-flex-row}
\begin{sphinxuseclass}{sd-align-minor-center}
\sphinxAtStartPar
basics

\end{sphinxuseclass}
\end{sphinxuseclass}
\end{sphinxuseclass}
\end{sphinxuseclass}
\begin{sphinxuseclass}{sd-col}
\begin{sphinxuseclass}{sd-col-auto}
\begin{sphinxuseclass}{sd-d-flex-row}
\begin{sphinxuseclass}{sd-align-minor-center}
\sphinxAtStartPar
Nov 09, 2024

\end{sphinxuseclass}
\end{sphinxuseclass}
\end{sphinxuseclass}
\end{sphinxuseclass}
\begin{sphinxuseclass}{sd-col}
\begin{sphinxuseclass}{sd-col-auto}
\begin{sphinxuseclass}{sd-d-flex-row}
\begin{sphinxuseclass}{sd-align-minor-center}
\sphinxAtStartPar
0 min read

\end{sphinxuseclass}
\end{sphinxuseclass}
\end{sphinxuseclass}
\end{sphinxuseclass}
\end{sphinxuseclass}
\end{sphinxuseclass}
\end{sphinxuseclass}
\end{sphinxuseclass}
\end{sphinxuseclass}
\end{sphinxuseclass}
\end{sphinxuseclass}
\end{sphinxuseclass}
\end{sphinxuseclass}
\end{sphinxuseclass}
\end{sphinxuseclass}
\end{sphinxuseclass}
\end{sphinxuseclass}
\end{sphinxuseclass}
\end{sphinxuseclass}
\end{sphinxuseclass}
\end{sphinxuseclass}
\end{sphinxuseclass}
\end{sphinxuseclass}
\end{sphinxuseclass}
\end{sphinxuseclass}
\end{sphinxuseclass}

\section{Onde elettromagnetiche piane}
\label{\detokenize{ch/waves-plane:onde-elettromagnetiche-piane}}\label{\detokenize{ch/waves-plane:classical-electromagnetism-waves-plane-waves}}\label{\detokenize{ch/waves-plane::doc}}
\sphinxstepscope


\part{Elettrotecnica}

\sphinxstepscope

\begin{sphinxuseclass}{sd-container-fluid}
\begin{sphinxuseclass}{sd-sphinx-override}
\begin{sphinxuseclass}{sd-p-0}
\begin{sphinxuseclass}{sd-mt-2}
\begin{sphinxuseclass}{sd-mb-4}
\begin{sphinxuseclass}{sd-row}
\begin{sphinxuseclass}{sd-row-cols-2}
\begin{sphinxuseclass}{sd-gx-2}
\begin{sphinxuseclass}{sd-gy-1}
\begin{sphinxuseclass}{sd-col}
\begin{sphinxuseclass}{sd-d-flex-row}
\begin{sphinxuseclass}{sd-align-minor-center}
\begin{sphinxuseclass}{sd-container-fluid}
\begin{sphinxuseclass}{sd-sphinx-override}
\begin{sphinxuseclass}{sd-row}
\begin{sphinxuseclass}{sd-row-cols-2}
\begin{sphinxuseclass}{sd-row-cols-xs-2}
\begin{sphinxuseclass}{sd-row-cols-sm-3}
\begin{sphinxuseclass}{sd-row-cols-md-3}
\begin{sphinxuseclass}{sd-row-cols-lg-3}
\begin{sphinxuseclass}{sd-gx-3}
\begin{sphinxuseclass}{sd-gy-1}
\begin{sphinxuseclass}{sd-col}
\begin{sphinxuseclass}{sd-col-auto}
\begin{sphinxuseclass}{sd-d-flex-row}
\begin{sphinxuseclass}{sd-align-minor-center}
\sphinxAtStartPar
basics

\end{sphinxuseclass}
\end{sphinxuseclass}
\end{sphinxuseclass}
\end{sphinxuseclass}
\begin{sphinxuseclass}{sd-col}
\begin{sphinxuseclass}{sd-col-auto}
\begin{sphinxuseclass}{sd-d-flex-row}
\begin{sphinxuseclass}{sd-align-minor-center}
\sphinxAtStartPar
Nov 09, 2024

\end{sphinxuseclass}
\end{sphinxuseclass}
\end{sphinxuseclass}
\end{sphinxuseclass}
\begin{sphinxuseclass}{sd-col}
\begin{sphinxuseclass}{sd-col-auto}
\begin{sphinxuseclass}{sd-d-flex-row}
\begin{sphinxuseclass}{sd-align-minor-center}
\sphinxAtStartPar
0 min read

\end{sphinxuseclass}
\end{sphinxuseclass}
\end{sphinxuseclass}
\end{sphinxuseclass}
\end{sphinxuseclass}
\end{sphinxuseclass}
\end{sphinxuseclass}
\end{sphinxuseclass}
\end{sphinxuseclass}
\end{sphinxuseclass}
\end{sphinxuseclass}
\end{sphinxuseclass}
\end{sphinxuseclass}
\end{sphinxuseclass}
\end{sphinxuseclass}
\end{sphinxuseclass}
\end{sphinxuseclass}
\end{sphinxuseclass}
\end{sphinxuseclass}
\end{sphinxuseclass}
\end{sphinxuseclass}
\end{sphinxuseclass}
\end{sphinxuseclass}
\end{sphinxuseclass}
\end{sphinxuseclass}
\end{sphinxuseclass}

\chapter{Approssimazione circuitale}
\label{\detokenize{ch/circuits:approssimazione-circuitale}}\label{\detokenize{ch/circuits:classical-electromagnetism-circuits}}\label{\detokenize{ch/circuits::doc}}
\sphinxAtStartPar
\sphinxstylestrong{Circuiti elettrici.} \sphinxstyleemphasis{Condizioni per la validità dell’approssimazione circuitale; componenti elementari; regimi di utilizzo: stazionario, armonico (alternato), transitorio;}

\sphinxAtStartPar
\sphinxstylestrong{Circuiti elettromagnetici.} \sphinxstyleemphasis{Condizioni per la validità dell’approssimazione circuitale; es. trasformatori}

\sphinxAtStartPar
\sphinxstylestrong{Circuito elettro\sphinxhyphen{}magneto\sphinxhyphen{}meccanici.} \sphinxstyleemphasis{Es. semplici circuiti; motori elettrici e generatori}

\sphinxstepscope

\begin{sphinxuseclass}{sd-container-fluid}
\begin{sphinxuseclass}{sd-sphinx-override}
\begin{sphinxuseclass}{sd-p-0}
\begin{sphinxuseclass}{sd-mt-2}
\begin{sphinxuseclass}{sd-mb-4}
\begin{sphinxuseclass}{sd-row}
\begin{sphinxuseclass}{sd-row-cols-2}
\begin{sphinxuseclass}{sd-gx-2}
\begin{sphinxuseclass}{sd-gy-1}
\begin{sphinxuseclass}{sd-col}
\begin{sphinxuseclass}{sd-d-flex-row}
\begin{sphinxuseclass}{sd-align-minor-center}
\begin{sphinxuseclass}{sd-container-fluid}
\begin{sphinxuseclass}{sd-sphinx-override}
\begin{sphinxuseclass}{sd-row}
\begin{sphinxuseclass}{sd-row-cols-2}
\begin{sphinxuseclass}{sd-row-cols-xs-2}
\begin{sphinxuseclass}{sd-row-cols-sm-3}
\begin{sphinxuseclass}{sd-row-cols-md-3}
\begin{sphinxuseclass}{sd-row-cols-lg-3}
\begin{sphinxuseclass}{sd-gx-3}
\begin{sphinxuseclass}{sd-gy-1}
\begin{sphinxuseclass}{sd-col}
\begin{sphinxuseclass}{sd-col-auto}
\begin{sphinxuseclass}{sd-d-flex-row}
\begin{sphinxuseclass}{sd-align-minor-center}
\sphinxAtStartPar
basics

\end{sphinxuseclass}
\end{sphinxuseclass}
\end{sphinxuseclass}
\end{sphinxuseclass}
\begin{sphinxuseclass}{sd-col}
\begin{sphinxuseclass}{sd-col-auto}
\begin{sphinxuseclass}{sd-d-flex-row}
\begin{sphinxuseclass}{sd-align-minor-center}
\sphinxAtStartPar
Nov 09, 2024

\end{sphinxuseclass}
\end{sphinxuseclass}
\end{sphinxuseclass}
\end{sphinxuseclass}
\begin{sphinxuseclass}{sd-col}
\begin{sphinxuseclass}{sd-col-auto}
\begin{sphinxuseclass}{sd-d-flex-row}
\begin{sphinxuseclass}{sd-align-minor-center}
\sphinxAtStartPar
0 min read

\end{sphinxuseclass}
\end{sphinxuseclass}
\end{sphinxuseclass}
\end{sphinxuseclass}
\end{sphinxuseclass}
\end{sphinxuseclass}
\end{sphinxuseclass}
\end{sphinxuseclass}
\end{sphinxuseclass}
\end{sphinxuseclass}
\end{sphinxuseclass}
\end{sphinxuseclass}
\end{sphinxuseclass}
\end{sphinxuseclass}
\end{sphinxuseclass}
\end{sphinxuseclass}
\end{sphinxuseclass}
\end{sphinxuseclass}
\end{sphinxuseclass}
\end{sphinxuseclass}
\end{sphinxuseclass}
\end{sphinxuseclass}
\end{sphinxuseclass}
\end{sphinxuseclass}
\end{sphinxuseclass}
\end{sphinxuseclass}

\section{Circuiti elettrici}
\label{\detokenize{ch/circuits-electric:circuiti-elettrici}}\label{\detokenize{ch/circuits-electric:classical-electromagnetism-circuits-electric}}\label{\detokenize{ch/circuits-electric::doc}}
\sphinxAtStartPar
Se il sistema di interesse soddisfa alcune condizioni, è possibile ridurre la teoria di campo dell’elettromagnetismo a una teoria circuitale.
Quando possibile, cioè quando capace di descrivere adeguatamente il comportamento del sistema di interesse, l’approccio circuitale semplifica di molto la descrizione del problema, non richiedendo la soluzione di un sistema di equazioni differenziali alle derivate parziali da risolvere nello spazio, ma la soluzione di equazioni differenziali ordinarie nelle incognite circuitali, che si riduce a un sistema algebrico, spesso lineare, in regime stazionario.

\sphinxAtStartPar
\sphinxstylestrong{Giustificazione dell’approccio circuitale.}

\sphinxAtStartPar
\sphinxstylestrong{Componenti elementari di un circuito elettrico.}

\sphinxstepscope

\begin{sphinxuseclass}{sd-container-fluid}
\begin{sphinxuseclass}{sd-sphinx-override}
\begin{sphinxuseclass}{sd-p-0}
\begin{sphinxuseclass}{sd-mt-2}
\begin{sphinxuseclass}{sd-mb-4}
\begin{sphinxuseclass}{sd-row}
\begin{sphinxuseclass}{sd-row-cols-2}
\begin{sphinxuseclass}{sd-gx-2}
\begin{sphinxuseclass}{sd-gy-1}
\begin{sphinxuseclass}{sd-col}
\begin{sphinxuseclass}{sd-d-flex-row}
\begin{sphinxuseclass}{sd-align-minor-center}
\begin{sphinxuseclass}{sd-container-fluid}
\begin{sphinxuseclass}{sd-sphinx-override}
\begin{sphinxuseclass}{sd-row}
\begin{sphinxuseclass}{sd-row-cols-2}
\begin{sphinxuseclass}{sd-row-cols-xs-2}
\begin{sphinxuseclass}{sd-row-cols-sm-3}
\begin{sphinxuseclass}{sd-row-cols-md-3}
\begin{sphinxuseclass}{sd-row-cols-lg-3}
\begin{sphinxuseclass}{sd-gx-3}
\begin{sphinxuseclass}{sd-gy-1}
\begin{sphinxuseclass}{sd-col}
\begin{sphinxuseclass}{sd-col-auto}
\begin{sphinxuseclass}{sd-d-flex-row}
\begin{sphinxuseclass}{sd-align-minor-center}
\sphinxAtStartPar
basics

\end{sphinxuseclass}
\end{sphinxuseclass}
\end{sphinxuseclass}
\end{sphinxuseclass}
\begin{sphinxuseclass}{sd-col}
\begin{sphinxuseclass}{sd-col-auto}
\begin{sphinxuseclass}{sd-d-flex-row}
\begin{sphinxuseclass}{sd-align-minor-center}
\sphinxAtStartPar
Nov 09, 2024

\end{sphinxuseclass}
\end{sphinxuseclass}
\end{sphinxuseclass}
\end{sphinxuseclass}
\begin{sphinxuseclass}{sd-col}
\begin{sphinxuseclass}{sd-col-auto}
\begin{sphinxuseclass}{sd-d-flex-row}
\begin{sphinxuseclass}{sd-align-minor-center}
\sphinxAtStartPar
1 min read

\end{sphinxuseclass}
\end{sphinxuseclass}
\end{sphinxuseclass}
\end{sphinxuseclass}
\end{sphinxuseclass}
\end{sphinxuseclass}
\end{sphinxuseclass}
\end{sphinxuseclass}
\end{sphinxuseclass}
\end{sphinxuseclass}
\end{sphinxuseclass}
\end{sphinxuseclass}
\end{sphinxuseclass}
\end{sphinxuseclass}
\end{sphinxuseclass}
\end{sphinxuseclass}
\end{sphinxuseclass}
\end{sphinxuseclass}
\end{sphinxuseclass}
\end{sphinxuseclass}
\end{sphinxuseclass}
\end{sphinxuseclass}
\end{sphinxuseclass}
\end{sphinxuseclass}
\end{sphinxuseclass}
\end{sphinxuseclass}

\subsection{Validità dell’approccio circuitale}
\label{\detokenize{ch/circuits-electric-approximation:validita-dell-approccio-circuitale}}\label{\detokenize{ch/circuits-electric-approximation:classical-electromagnetism-circuits-electric-approximation}}\label{\detokenize{ch/circuits-electric-approximation::doc}}
\sphinxAtStartPar
L’approccio circuitale consente di ridurre il problema elettromagnetico, in generale un problema di campo che richiede la soluzione di PDE, a un approccio “ai morsetti” \sphinxstylestrong{todo}, che richiede la soluzione di ODE.

\sphinxAtStartPar
Una rivisitazione dell’{\hyperref[\detokenize{ch/energy:classical-electromagnetism-energy}]{\sphinxcrossref{\DUrole{std,std-ref}{equazione dell’energia}}}} permette di valutare i regimi in cui è possibile usare un approccio circuitale a un sistema elettromagnetico.

\sphinxAtStartPar
In particolare, nell’equazione di bilancio dell’energia elettromagnetica
\begin{equation*}
\begin{split}\dfrac{d}{dt} \int_V u = \oint_{\partial V} \mathbf{s} \cdot \hat{\mathbf{n}} - \int_V \mathbf{j} \cdot \mathbf{e} \ ,\end{split}
\end{equation*}
\sphinxAtStartPar
viene indagato il termine di flusso alla frontiera, ricordando la definizione di vettore di Poynting \(\mathbf{s} := \mathbf{e} \times \mathbf{h}\), e riscrivendo i campi elettrico e magnetico in funzione dei potenziali elettromagnetici, \(\mathbf{b} = \nabla \times \mathbf{a}\), \(\mathbf{e} = - \nabla \varphi - \partial_t \mathbf{a} \ ,\)
\begin{equation*}
\begin{split}\begin{aligned}
  - \oint_{\partial V} \mathbf{s} \cdot \mathbf{\hat{n}}
  & = - \oint_{\partial V} \left(\mathbf{e} \times \mathbf{h} \right) \cdot \mathbf{\hat{n}} = \\
  & =   \oint_{\partial V} \left(\nabla \varphi + \partial_t \mathbf{a} \right) \times \mathbf{h}  \cdot \mathbf{\hat{n}} = \\
  & = ... \\
  & = \underbrace{\oint_{\partial V} \hat{\mathbf{n}} \cdot \nabla \times ( \varphi \mathbf{h} )}_{=0 \text{ (Stokes'thm **todo** check)}} - \oint_{\partial V} \varphi \hat{\mathbf{n}} \cdot \underbrace{\nabla \times \mathbf{h}}_{\partial_t \mathbf{d} + \mathbf{j}} + \oint_{\partial V} \hat{\mathbf{n}} \cdot \partial_t \mathbf{a} \times \mathbf{h} = \\
  & = - \oint_{\partial V} \varphi \mathbf{j} \cdot \hat{\mathbf{n}} - \oint_{\partial V} \hat{\mathbf{n}} \cdot \left( \partial_t \mathbf{d} + \mathbf{h} \times \partial_t \mathbf{a} \right) \ , 
\end{aligned}\end{split}
\end{equation*}
\sphinxAtStartPar
e assumendo che il flusso di carica elettrica avvenga solo in corrispondenza di un numero finito di sezioni \(S_k \in \partial V\) equipotenziali a potenziale \(v_k = -\varphi_k\), costante sulle sezioni, e riconoscento il flusso di carica elettrica attraverso la sezione \(S_k\) come la corrente \(i_k = \int_{S_k} \mathbf{j} \cdot \hat{\mathbf{n}}\), si può scrivere
\begin{equation*}
\begin{split}- \oint_{\partial V} \mathbf{s} \cdot \hat{\mathbf{n}} = \sum_k v_k \, i_k - \oint_{\partial V} \hat{\mathbf{n}} \cdot \left( \partial_t \mathbf{d} + \mathbf{h} \times \partial_t \mathbf{a} \right) \ .\end{split}
\end{equation*}
\sphinxAtStartPar
Il bilancio di energia elettromagnetica del sistema può quindi essere riscritto come
\begin{equation*}
\begin{split}\frac{d}{dt} \int_V u = \sum_k v_k \, i_k - \int_{V} \mathbf{j} \cdot \mathbf{e} - \oint_{\partial V} \hat{\mathbf{n}} \cdot \left( \partial_t \mathbf{d} + \mathbf{h} \times \partial_t \mathbf{a} \right) \ .\end{split}
\end{equation*}
\sphinxAtStartPar
Nelle condizioni in cui l’ultimo termine è nullo o trascurabile (\sphinxstylestrong{todo} \sphinxstyleemphasis{quali? Spendere due parole sulla validità dell’approssimazione, con analisi dimensionale? Fare esempio in cui l’approssimazione non funziona}), la variazione di energia interna al sistema è dovuta alla differenza della potenza in ingresso ai morsetti, e la dissipazione all’interno del volume (ad esempio dovuta alla conduzione non ideale in conduttori con resistività finita),
\begin{equation*}
\begin{split}\dot{E}^{em} = P^{ext, vi} - \dot{D} \ ,\end{split}
\end{equation*}
\sphinxAtStartPar
con \(\dot{D} \ge 0\) per il secondo principio della termodinamica \sphinxstylestrong{todo} \sphinxstyleemphasis{aggiungere riferimento, e discussione.}

\sphinxstepscope

\begin{sphinxuseclass}{sd-container-fluid}
\begin{sphinxuseclass}{sd-sphinx-override}
\begin{sphinxuseclass}{sd-p-0}
\begin{sphinxuseclass}{sd-mt-2}
\begin{sphinxuseclass}{sd-mb-4}
\begin{sphinxuseclass}{sd-row}
\begin{sphinxuseclass}{sd-row-cols-2}
\begin{sphinxuseclass}{sd-gx-2}
\begin{sphinxuseclass}{sd-gy-1}
\begin{sphinxuseclass}{sd-col}
\begin{sphinxuseclass}{sd-d-flex-row}
\begin{sphinxuseclass}{sd-align-minor-center}
\begin{sphinxuseclass}{sd-container-fluid}
\begin{sphinxuseclass}{sd-sphinx-override}
\begin{sphinxuseclass}{sd-row}
\begin{sphinxuseclass}{sd-row-cols-2}
\begin{sphinxuseclass}{sd-row-cols-xs-2}
\begin{sphinxuseclass}{sd-row-cols-sm-3}
\begin{sphinxuseclass}{sd-row-cols-md-3}
\begin{sphinxuseclass}{sd-row-cols-lg-3}
\begin{sphinxuseclass}{sd-gx-3}
\begin{sphinxuseclass}{sd-gy-1}
\begin{sphinxuseclass}{sd-col}
\begin{sphinxuseclass}{sd-col-auto}
\begin{sphinxuseclass}{sd-d-flex-row}
\begin{sphinxuseclass}{sd-align-minor-center}
\sphinxAtStartPar
basics

\end{sphinxuseclass}
\end{sphinxuseclass}
\end{sphinxuseclass}
\end{sphinxuseclass}
\begin{sphinxuseclass}{sd-col}
\begin{sphinxuseclass}{sd-col-auto}
\begin{sphinxuseclass}{sd-d-flex-row}
\begin{sphinxuseclass}{sd-align-minor-center}
\sphinxAtStartPar
Nov 09, 2024

\end{sphinxuseclass}
\end{sphinxuseclass}
\end{sphinxuseclass}
\end{sphinxuseclass}
\begin{sphinxuseclass}{sd-col}
\begin{sphinxuseclass}{sd-col-auto}
\begin{sphinxuseclass}{sd-d-flex-row}
\begin{sphinxuseclass}{sd-align-minor-center}
\sphinxAtStartPar
1 min read

\end{sphinxuseclass}
\end{sphinxuseclass}
\end{sphinxuseclass}
\end{sphinxuseclass}
\end{sphinxuseclass}
\end{sphinxuseclass}
\end{sphinxuseclass}
\end{sphinxuseclass}
\end{sphinxuseclass}
\end{sphinxuseclass}
\end{sphinxuseclass}
\end{sphinxuseclass}
\end{sphinxuseclass}
\end{sphinxuseclass}
\end{sphinxuseclass}
\end{sphinxuseclass}
\end{sphinxuseclass}
\end{sphinxuseclass}
\end{sphinxuseclass}
\end{sphinxuseclass}
\end{sphinxuseclass}
\end{sphinxuseclass}
\end{sphinxuseclass}
\end{sphinxuseclass}
\end{sphinxuseclass}
\end{sphinxuseclass}

\subsection{Induzione elettromagnetica nell’approssimazione circuitale}
\label{\detokenize{ch/circuits-electric-induction:induzione-elettromagnetica-nell-approssimazione-circuitale}}\label{\detokenize{ch/circuits-electric-induction:classical-electromagnetism-circuits-electric-induction}}\label{\detokenize{ch/circuits-electric-induction::doc}}
\sphinxAtStartPar
E’ possibile applicare l’approssimazione circuitale anche in presenza di regioni in cui non è possibile trascurare il termine \(\partial_t \mathbf{b}\), come ad esempio circuiti elettromagnetici che coinvolgono trasformatori e/o motori o generatori elettrici.

\sphinxAtStartPar
In queste situazioni, se è possibile identificare una regione \(V_0\) dello spazio connessa nella quale il termine \(\partial_t \mathbf{b} = \mathbf{0}\), e quindi \(\nabla \times \mathbf{e} = \mathbf{0}\), in \(V_0\) è possibile definire il campo elettrico in termini di un potenziale \(\varphi\),
\begin{equation*}
\begin{split}\mathbf{e} = - \nabla \varphi \qquad , \qquad \mathbf{r} \in V_0 \ .\end{split}
\end{equation*}
\sphinxAtStartPar
E’ possibile calcolare le differenze di potenziale ai morsetti di un sistema in cui \(\delta_t \mathbf{b} \ne 0\), racchiuso nel volume \(V_k\), con la legge di Faraday,
\begin{equation*}
\begin{split}\oint_{\ell_k} \mathbf{e} \cdot \hat{\mathbf{t}} = - \frac{d}{dt} \int_{S_k} \mathbf{b} \cdot \hat{\mathbf{n}} \ ,\end{split}
\end{equation*}
\sphinxAtStartPar
dove il percorso chiuso \(\ell_k = \ell_k^{cond} \cup \ell_k^{mors}\) descrive il conduttore in \(V_k\) chiuso dalla linea geometrica tra i morsetti. Se si può trascurare la resistività del conduttore in \(V_k\), \(\int_{\ell_k^{cond}} \mathbf{e} \cdot \hat{\mathbf{t}} = 0\), la differenza di tensione ai morsetti vale
\begin{equation*}
\begin{split}\Delta v_k = \int_{\ell^{mors}_k} \mathbf{e} \cdot \hat{\mathbf{t}} = - \frac{d}{dt} \int_{S_k} \mathbf{b} \cdot \hat{\mathbf{n}}\end{split}
\end{equation*}
\sphinxstepscope

\begin{sphinxuseclass}{sd-container-fluid}
\begin{sphinxuseclass}{sd-sphinx-override}
\begin{sphinxuseclass}{sd-p-0}
\begin{sphinxuseclass}{sd-mt-2}
\begin{sphinxuseclass}{sd-mb-4}
\begin{sphinxuseclass}{sd-row}
\begin{sphinxuseclass}{sd-row-cols-2}
\begin{sphinxuseclass}{sd-gx-2}
\begin{sphinxuseclass}{sd-gy-1}
\begin{sphinxuseclass}{sd-col}
\begin{sphinxuseclass}{sd-d-flex-row}
\begin{sphinxuseclass}{sd-align-minor-center}
\begin{sphinxuseclass}{sd-container-fluid}
\begin{sphinxuseclass}{sd-sphinx-override}
\begin{sphinxuseclass}{sd-row}
\begin{sphinxuseclass}{sd-row-cols-2}
\begin{sphinxuseclass}{sd-row-cols-xs-2}
\begin{sphinxuseclass}{sd-row-cols-sm-3}
\begin{sphinxuseclass}{sd-row-cols-md-3}
\begin{sphinxuseclass}{sd-row-cols-lg-3}
\begin{sphinxuseclass}{sd-gx-3}
\begin{sphinxuseclass}{sd-gy-1}
\begin{sphinxuseclass}{sd-col}
\begin{sphinxuseclass}{sd-col-auto}
\begin{sphinxuseclass}{sd-d-flex-row}
\begin{sphinxuseclass}{sd-align-minor-center}
\sphinxAtStartPar
basics

\end{sphinxuseclass}
\end{sphinxuseclass}
\end{sphinxuseclass}
\end{sphinxuseclass}
\begin{sphinxuseclass}{sd-col}
\begin{sphinxuseclass}{sd-col-auto}
\begin{sphinxuseclass}{sd-d-flex-row}
\begin{sphinxuseclass}{sd-align-minor-center}
\sphinxAtStartPar
Nov 09, 2024

\end{sphinxuseclass}
\end{sphinxuseclass}
\end{sphinxuseclass}
\end{sphinxuseclass}
\begin{sphinxuseclass}{sd-col}
\begin{sphinxuseclass}{sd-col-auto}
\begin{sphinxuseclass}{sd-d-flex-row}
\begin{sphinxuseclass}{sd-align-minor-center}
\sphinxAtStartPar
1 min read

\end{sphinxuseclass}
\end{sphinxuseclass}
\end{sphinxuseclass}
\end{sphinxuseclass}
\end{sphinxuseclass}
\end{sphinxuseclass}
\end{sphinxuseclass}
\end{sphinxuseclass}
\end{sphinxuseclass}
\end{sphinxuseclass}
\end{sphinxuseclass}
\end{sphinxuseclass}
\end{sphinxuseclass}
\end{sphinxuseclass}
\end{sphinxuseclass}
\end{sphinxuseclass}
\end{sphinxuseclass}
\end{sphinxuseclass}
\end{sphinxuseclass}
\end{sphinxuseclass}
\end{sphinxuseclass}
\end{sphinxuseclass}
\end{sphinxuseclass}
\end{sphinxuseclass}
\end{sphinxuseclass}
\end{sphinxuseclass}

\subsection{Componenti elementari dei circuiti elettrici}
\label{\detokenize{ch/circuits-electric-components:componenti-elementari-dei-circuiti-elettrici}}\label{\detokenize{ch/circuits-electric-components:classical-electromagnetism-circuits-electric-components}}\label{\detokenize{ch/circuits-electric-components::doc}}

\subsubsection{Resistore ohmico}
\label{\detokenize{ch/circuits-electric-components:resistore-ohmico}}
\sphinxAtStartPar
Un resistore di Ohm risulta dall’approssimazione circuitale di un materiale con equazione costitutiva lineare
\begin{equation*}
\begin{split}\mathbf{e} = \rho_R \, \mathbf{j} \ ,\end{split}
\end{equation*}
\sphinxAtStartPar
tra il campo elettrico \(\mathbf{e}\) e la densità di corrente \(\mathbf{j}\), tramite la costante di proporzionalità \(\rho_R\), la \sphinxstylestrong{resistività} del materiale. La corrente elettrica attraverso una sezione del componente è definita come il flusso di carica attraverso una sua sezione
\begin{equation*}
\begin{split}i = \int_S \mathbf{j} \cdot \hat{\mathbf{t}} \simeq j \, A \ ,\end{split}
\end{equation*}
\sphinxAtStartPar
Nell’ipotesi che il vettore densità di corrente si allineato con l’asse del componente e uniforme sulla sezione \(A\), “piccola”.
Se il materiale non è in grado di accumulare carica, il bilancio di carica elettrica si traduce nella continuità della corrente elettrica attraverso le sezioni del conduttore.

\sphinxAtStartPar
Utilizzando l’equazione costitutiva su un elemento di lunghezza elementare \(d\mathbf{r} =\hat{\mathbf{t}} \, d \ell \), e assumendo che il campo elettrico sia allineato con l’asse del componente, \(\mathbf{e} = e \hat{\mathbf{t}}\) si può scrivere il lavoro elementare per unità di carica come
\begin{equation*}
\begin{split}\delta v = \mathbf{e} \cdot d \mathbf{r} =  e \, d\ell = \rho_R \, j \, d\ell =  \frac{\rho_R \, d\ell}{A} i \ .\end{split}
\end{equation*}
\sphinxAtStartPar
Da questa ultima equazione seguono le due leggi di Ohm, per resistori lineari.

\sphinxAtStartPar
\sphinxstylestrong{Prima legge di Ohm.} La differenza di potenziale tra due sezioni di un resistore lineare è proporzionale alla corrente che passa attraverso di esso,
\begin{equation*}
\begin{split}\delta v = dR \, i \ .\end{split}
\end{equation*}
\sphinxAtStartPar
\sphinxstylestrong{Seconda legge di Ohm.} La costante di proporzionalità che lega la differenza di potenziale e la corrente all’interno di un resistore ohmico, la \sphinxstylestrong{resistenza} del resistore, è proporzionale alla resistività e alla lunghezza del resistore, e inversamente proporzionale alla sua sezione,
\begin{equation*}
\begin{split}dR = \frac{\rho_R \ d\ell}{A} \ .\end{split}
\end{equation*}
\sphinxAtStartPar
Se le proprietà sono uniformi nel resistore, si possono integrare le relazioni elementari per ottenere la relazione tra grandezze finite,
\begin{equation*}
\begin{split}\Delta V = R \, i \end{split}
\end{equation*}\begin{equation*}
\begin{split}R = \frac{\rho_R \ \ell}{A}\end{split}
\end{equation*}
\sphinxAtStartPar
\sphinxstylestrong{todo} (perché si può usare il potenziale? Nelle mie note avevo usato il simbolo \(v^*\), come se fosse una definizione leggermente diversa per incorporare movimento e instazionarietà, che si riduce a \(v\) nel caso stazionario).

\sphinxAtStartPar
\sphinxstylestrong{Condensatore.}

\sphinxAtStartPar
\sphinxstylestrong{Induttore.}

\sphinxAtStartPar
\sphinxstylestrong{Generatore di tensione.}

\sphinxAtStartPar
\sphinxstylestrong{Generatore di corrente.}

\sphinxstepscope

\begin{sphinxuseclass}{sd-container-fluid}
\begin{sphinxuseclass}{sd-sphinx-override}
\begin{sphinxuseclass}{sd-p-0}
\begin{sphinxuseclass}{sd-mt-2}
\begin{sphinxuseclass}{sd-mb-4}
\begin{sphinxuseclass}{sd-row}
\begin{sphinxuseclass}{sd-row-cols-2}
\begin{sphinxuseclass}{sd-gx-2}
\begin{sphinxuseclass}{sd-gy-1}
\begin{sphinxuseclass}{sd-col}
\begin{sphinxuseclass}{sd-d-flex-row}
\begin{sphinxuseclass}{sd-align-minor-center}
\begin{sphinxuseclass}{sd-container-fluid}
\begin{sphinxuseclass}{sd-sphinx-override}
\begin{sphinxuseclass}{sd-row}
\begin{sphinxuseclass}{sd-row-cols-2}
\begin{sphinxuseclass}{sd-row-cols-xs-2}
\begin{sphinxuseclass}{sd-row-cols-sm-3}
\begin{sphinxuseclass}{sd-row-cols-md-3}
\begin{sphinxuseclass}{sd-row-cols-lg-3}
\begin{sphinxuseclass}{sd-gx-3}
\begin{sphinxuseclass}{sd-gy-1}
\begin{sphinxuseclass}{sd-col}
\begin{sphinxuseclass}{sd-col-auto}
\begin{sphinxuseclass}{sd-d-flex-row}
\begin{sphinxuseclass}{sd-align-minor-center}
\sphinxAtStartPar
basics

\end{sphinxuseclass}
\end{sphinxuseclass}
\end{sphinxuseclass}
\end{sphinxuseclass}
\begin{sphinxuseclass}{sd-col}
\begin{sphinxuseclass}{sd-col-auto}
\begin{sphinxuseclass}{sd-d-flex-row}
\begin{sphinxuseclass}{sd-align-minor-center}
\sphinxAtStartPar
Nov 09, 2024

\end{sphinxuseclass}
\end{sphinxuseclass}
\end{sphinxuseclass}
\end{sphinxuseclass}
\begin{sphinxuseclass}{sd-col}
\begin{sphinxuseclass}{sd-col-auto}
\begin{sphinxuseclass}{sd-d-flex-row}
\begin{sphinxuseclass}{sd-align-minor-center}
\sphinxAtStartPar
0 min read

\end{sphinxuseclass}
\end{sphinxuseclass}
\end{sphinxuseclass}
\end{sphinxuseclass}
\end{sphinxuseclass}
\end{sphinxuseclass}
\end{sphinxuseclass}
\end{sphinxuseclass}
\end{sphinxuseclass}
\end{sphinxuseclass}
\end{sphinxuseclass}
\end{sphinxuseclass}
\end{sphinxuseclass}
\end{sphinxuseclass}
\end{sphinxuseclass}
\end{sphinxuseclass}
\end{sphinxuseclass}
\end{sphinxuseclass}
\end{sphinxuseclass}
\end{sphinxuseclass}
\end{sphinxuseclass}
\end{sphinxuseclass}
\end{sphinxuseclass}
\end{sphinxuseclass}
\end{sphinxuseclass}
\end{sphinxuseclass}

\subsection{Regimi di funzionamento in circuiti elettrici}
\label{\detokenize{ch/circuits-electric-regimes:regimi-di-funzionamento-in-circuiti-elettrici}}\label{\detokenize{ch/circuits-electric-regimes:classical-electromagnetism-circuits-electric-regimes}}\label{\detokenize{ch/circuits-electric-regimes::doc}}
\sphinxstepscope

\begin{sphinxuseclass}{sd-container-fluid}
\begin{sphinxuseclass}{sd-sphinx-override}
\begin{sphinxuseclass}{sd-p-0}
\begin{sphinxuseclass}{sd-mt-2}
\begin{sphinxuseclass}{sd-mb-4}
\begin{sphinxuseclass}{sd-row}
\begin{sphinxuseclass}{sd-row-cols-2}
\begin{sphinxuseclass}{sd-gx-2}
\begin{sphinxuseclass}{sd-gy-1}
\begin{sphinxuseclass}{sd-col}
\begin{sphinxuseclass}{sd-d-flex-row}
\begin{sphinxuseclass}{sd-align-minor-center}
\begin{sphinxuseclass}{sd-container-fluid}
\begin{sphinxuseclass}{sd-sphinx-override}
\begin{sphinxuseclass}{sd-row}
\begin{sphinxuseclass}{sd-row-cols-2}
\begin{sphinxuseclass}{sd-row-cols-xs-2}
\begin{sphinxuseclass}{sd-row-cols-sm-3}
\begin{sphinxuseclass}{sd-row-cols-md-3}
\begin{sphinxuseclass}{sd-row-cols-lg-3}
\begin{sphinxuseclass}{sd-gx-3}
\begin{sphinxuseclass}{sd-gy-1}
\begin{sphinxuseclass}{sd-col}
\begin{sphinxuseclass}{sd-col-auto}
\begin{sphinxuseclass}{sd-d-flex-row}
\begin{sphinxuseclass}{sd-align-minor-center}
\sphinxAtStartPar
basics

\end{sphinxuseclass}
\end{sphinxuseclass}
\end{sphinxuseclass}
\end{sphinxuseclass}
\begin{sphinxuseclass}{sd-col}
\begin{sphinxuseclass}{sd-col-auto}
\begin{sphinxuseclass}{sd-d-flex-row}
\begin{sphinxuseclass}{sd-align-minor-center}
\sphinxAtStartPar
Nov 09, 2024

\end{sphinxuseclass}
\end{sphinxuseclass}
\end{sphinxuseclass}
\end{sphinxuseclass}
\begin{sphinxuseclass}{sd-col}
\begin{sphinxuseclass}{sd-col-auto}
\begin{sphinxuseclass}{sd-d-flex-row}
\begin{sphinxuseclass}{sd-align-minor-center}
\sphinxAtStartPar
1 min read

\end{sphinxuseclass}
\end{sphinxuseclass}
\end{sphinxuseclass}
\end{sphinxuseclass}
\end{sphinxuseclass}
\end{sphinxuseclass}
\end{sphinxuseclass}
\end{sphinxuseclass}
\end{sphinxuseclass}
\end{sphinxuseclass}
\end{sphinxuseclass}
\end{sphinxuseclass}
\end{sphinxuseclass}
\end{sphinxuseclass}
\end{sphinxuseclass}
\end{sphinxuseclass}
\end{sphinxuseclass}
\end{sphinxuseclass}
\end{sphinxuseclass}
\end{sphinxuseclass}
\end{sphinxuseclass}
\end{sphinxuseclass}
\end{sphinxuseclass}
\end{sphinxuseclass}
\end{sphinxuseclass}
\end{sphinxuseclass}

\section{Circuiti elettromagnetici}
\label{\detokenize{ch/circuits-electromagnetic:circuiti-elettromagnetici}}\label{\detokenize{ch/circuits-electromagnetic:classical-electromagnetism-circuits-electromagnetic}}\label{\detokenize{ch/circuits-electromagnetic::doc}}
\sphinxAtStartPar
Sotto opportune ipotesi è possibile usare un modello circuitale anche per sistemi elettromagnetici, come ad esempio i trasformatori, o i motori elettrici.
\begin{itemize}
\item {} 
\sphinxAtStartPar
legge di Gauss per il campo magnetico
\begin{equation*}
\begin{split}\nabla \cdot \mathbf{b} = 0\end{split}
\end{equation*}
\item {} 
\sphinxAtStartPar
legge di Ampére\sphinxhyphen{}Maxwell
\begin{equation*}
\begin{split}\nabla \times \mathbf{h} - \partial_t \mathbf{d} = \mathbf{j}\end{split}
\end{equation*}
\end{itemize}

\sphinxAtStartPar
Si aggiungono le seguenti ipotesi:
\begin{itemize}
\item {} 
\sphinxAtStartPar
materiali lineari non\sphinxhyphen{}dissipativi e non\sphinxhyphen{}dispersivi \(\mathbf{b} = \mu \mathbf{h}\) \sphinxstylestrong{todo} discutere questa ipotesi, insieme a isteresi materiali, cicli di magnetizzazione,….

\item {} 
\sphinxAtStartPar
variazioni del campo \(\mathbf{d}\) nel tempo trascurabili, \(\partial_t \mathbf{d} = \mathbf{0}\).

\end{itemize}

\sphinxAtStartPar
La legge di Gauss del campo magnetico in forma integrale permette di scrivere la \sphinxstylestrong{legge ai nodi} del flusso del campo magnetico per i circuiti magnetici,
\begin{equation*}
\begin{split}0 = \oint_{\partial V} \mathbf{b} \cdot \hat{\mathbf{n}} = \sum_k \phi_k \ .\end{split}
\end{equation*}
\sphinxAtStartPar
La legge di Ampére\sphinxhyphen{}Maxwell in forma integrale considerando:
\begin{itemize}
\item {} 
\sphinxAtStartPar
un percorso incatenato con il solo induttore
\begin{equation*}
\begin{split}\int_{\ell_{ind}} \mathbf{h} \cdot \hat{\mathbf{t}} + \int_{\ell_{12}} \mathbf{h} \cdot \hat{\mathbf{t}} = \oint_{\ell_{1}} \mathbf{h} \cdot \hat{\mathbf{t}} = \int_{S^{ind}} \mathbf{j} \cdot \hat{\mathbf{n}} =  N i =: m\end{split}
\end{equation*}
\item {} 
\sphinxAtStartPar
un percorso incatenato con il traferro, aggirando l’induttore
\begin{equation*}
\begin{split}0 = \int_{\ell_{traf}} \mathbf{h} \cdot \hat{\mathbf{t}} + \int_{\ell_{21}} \hat{h} \cdot \hat{\mathbf{t}} = \sum_{k} h_k \ell_k + \int_{\ell_{21}} \hat{h} \cdot \hat{\mathbf{t}}\end{split}
\end{equation*}
\end{itemize}

\sphinxAtStartPar
e sommando le due equazioni, riconoscendo che i due integrali di linea sullo stesso percorsoin versi opposti si annullano, si ottiene la \sphinxstylestrong{legge alle maglie} per i circuiti magnetici
\begin{equation*}
\begin{split}\begin{aligned}
  m & = \int_{\ell_{ind}} \mathbf{h} \cdot \hat{\mathbf{t}} + \int_{\ell_{traf}} \mathbf{h} \cdot \hat{\mathbf{t}} = \\
    & \approx \sum_{k \in \ell} h_k \, \ell_k 
      = \sum_{k \in \ell} \frac{b_k}{\mu_k} \, \ell_k 
      = \sum_{k \in \ell} \frac{\ell_k}{\mu_k \, A_k} \, \phi_k  \ .
\end{aligned}\end{split}
\end{equation*}
\sphinxAtStartPar
Le leggi di Kirchhoff per i circuiti magnetici sono quindi
\begin{equation*}
\begin{split}\begin{cases}
  \sum_{k \in N_j} \phi_k = 0 \\ \\
  m_{\ell_i} = \sum_{k \in \ell_i} \theta_k \phi_k \ ,
\end{cases}\end{split}
\end{equation*}
\sphinxAtStartPar
avendo introdotto la riluttanza \(\theta_k = \frac{\ell_k}{\mu_k \, A_k}\), l’inverso della permeanza \(\Lambda_k = \theta_k^{-1}\).

\sphinxstepscope

\begin{sphinxuseclass}{sd-container-fluid}
\begin{sphinxuseclass}{sd-sphinx-override}
\begin{sphinxuseclass}{sd-p-0}
\begin{sphinxuseclass}{sd-mt-2}
\begin{sphinxuseclass}{sd-mb-4}
\begin{sphinxuseclass}{sd-row}
\begin{sphinxuseclass}{sd-row-cols-2}
\begin{sphinxuseclass}{sd-gx-2}
\begin{sphinxuseclass}{sd-gy-1}
\begin{sphinxuseclass}{sd-col}
\begin{sphinxuseclass}{sd-d-flex-row}
\begin{sphinxuseclass}{sd-align-minor-center}
\begin{sphinxuseclass}{sd-container-fluid}
\begin{sphinxuseclass}{sd-sphinx-override}
\begin{sphinxuseclass}{sd-row}
\begin{sphinxuseclass}{sd-row-cols-2}
\begin{sphinxuseclass}{sd-row-cols-xs-2}
\begin{sphinxuseclass}{sd-row-cols-sm-3}
\begin{sphinxuseclass}{sd-row-cols-md-3}
\begin{sphinxuseclass}{sd-row-cols-lg-3}
\begin{sphinxuseclass}{sd-gx-3}
\begin{sphinxuseclass}{sd-gy-1}
\begin{sphinxuseclass}{sd-col}
\begin{sphinxuseclass}{sd-col-auto}
\begin{sphinxuseclass}{sd-d-flex-row}
\begin{sphinxuseclass}{sd-align-minor-center}
\sphinxAtStartPar
basics

\end{sphinxuseclass}
\end{sphinxuseclass}
\end{sphinxuseclass}
\end{sphinxuseclass}
\begin{sphinxuseclass}{sd-col}
\begin{sphinxuseclass}{sd-col-auto}
\begin{sphinxuseclass}{sd-d-flex-row}
\begin{sphinxuseclass}{sd-align-minor-center}
\sphinxAtStartPar
Nov 09, 2024

\end{sphinxuseclass}
\end{sphinxuseclass}
\end{sphinxuseclass}
\end{sphinxuseclass}
\begin{sphinxuseclass}{sd-col}
\begin{sphinxuseclass}{sd-col-auto}
\begin{sphinxuseclass}{sd-d-flex-row}
\begin{sphinxuseclass}{sd-align-minor-center}
\sphinxAtStartPar
1 min read

\end{sphinxuseclass}
\end{sphinxuseclass}
\end{sphinxuseclass}
\end{sphinxuseclass}
\end{sphinxuseclass}
\end{sphinxuseclass}
\end{sphinxuseclass}
\end{sphinxuseclass}
\end{sphinxuseclass}
\end{sphinxuseclass}
\end{sphinxuseclass}
\end{sphinxuseclass}
\end{sphinxuseclass}
\end{sphinxuseclass}
\end{sphinxuseclass}
\end{sphinxuseclass}
\end{sphinxuseclass}
\end{sphinxuseclass}
\end{sphinxuseclass}
\end{sphinxuseclass}
\end{sphinxuseclass}
\end{sphinxuseclass}
\end{sphinxuseclass}
\end{sphinxuseclass}
\end{sphinxuseclass}
\end{sphinxuseclass}

\subsection{Trasformatore}
\label{\detokenize{ch/circuits-electromagnetic-transformer:trasformatore}}\label{\detokenize{ch/circuits-electromagnetic-transformer:classical-electromagnetism-circuits-electromagnetic-transformer}}\label{\detokenize{ch/circuits-electromagnetic-transformer::doc}}\begin{itemize}
\item {} 
\sphinxAtStartPar
flusso del campo magnetico, nell’ipotesi di campo uniforme, o in termini del campo medio
\begin{equation*}
\begin{split}\phi = b \, A\end{split}
\end{equation*}
\item {} 
\sphinxAtStartPar
flusso del campo magnetico concatenato a \(N\) avvolgimenti
\begin{equation*}
\begin{split}\psi = N \, \phi\end{split}
\end{equation*}
\item {} 
\sphinxAtStartPar
relazione tra tensione ai morsetti dell’induttore e flusso concatenato, applicando la {\hyperref[\detokenize{ch/circuits-electric-induction:classical-electromagnetism-circuits-electric-induction}]{\sphinxcrossref{\DUrole{std,std-ref}{legge di Faraday solo in parte irrotazionali}}}}
\begin{equation*}
\begin{split}v = \dot{\psi}\end{split}
\end{equation*}
\end{itemize}


\subsubsection{Trasformatore ideale}
\label{\detokenize{ch/circuits-electromagnetic-transformer:trasformatore-ideale}}
\sphinxAtStartPar
In assenza di flussi dispersi e riluttanza nel traferro, la legge alle maglie nel traferro implica
\begin{equation*}
\begin{split}0 = m_1 + m_2 = N_1 \, i_1 + N_2 \, i_2\end{split}
\end{equation*}
\sphinxAtStartPar
Il flusso del campo magnetico può essere scritto in funzione del flusso concatenato agli avvolgimenti,
\begin{equation*}
\begin{split}\phi = \frac{\psi_1}{N_1} = \frac{\psi_2}{N_2}\end{split}
\end{equation*}
\sphinxAtStartPar
La derivata nel tempo di questa relazione, con numero di avvolgimenti costanti nel tempo, implica
\begin{equation*}
\begin{split}\frac{v_2}{N_2} = \frac{v_1}{N_1} \ .\end{split}
\end{equation*}

\subsubsection{Trasformatore con flussi dispersi}
\label{\detokenize{ch/circuits-electromagnetic-transformer:trasformatore-con-flussi-dispersi}}\begin{equation*}
\begin{split}\begin{cases}
 & \phi_1 - \phi_{1,d} = \phi \\
 & \phi_2 - \phi_{2,d} = \phi \\
 & m_1 = \theta_{1,d} \phi_{1,d} \\
 & m_2 = \theta_{2,d} \phi_{2,d} \\
 & m_1 + m_2 = 0
\end{cases}\end{split}
\end{equation*}\begin{equation*}
\begin{split}\rightarrow \qquad 0 = m_1 + m_2 = N_1 \, i_1 + N_2 \, i_2\end{split}
\end{equation*}\begin{equation*}
\begin{split}\begin{aligned}
  0 & = \phi_2 - \phi_1 - \phi_{2,d} + \phi_{1,d} \\
    & = \phi_2 - \phi_1 - \frac{m_2}{\theta_{2,d}} + \frac{m_1}{\theta_{1,d}} \\
\end{aligned}\end{split}
\end{equation*}\begin{equation*}
\begin{split}\rightarrow \qquad \frac{\psi_2}{N_2} - \frac{m_2}{\theta_{2,d}} = \frac{\psi_1}{N_1} - \frac{m_1}{\theta_{1,d}} \ .\end{split}
\end{equation*}\begin{equation*}
\begin{split}\rightarrow \qquad \frac{1}{N_2} \left( v_2 - \frac{N_2^2}{\theta_{2,d}} \dfrac{d i_2}{d t} \right) =  
                     \frac{1}{N_1} \left( v_1 - \frac{N_1^2}{\theta_{1,d}} \dfrac{d i_1}{d t} \right)  \ .\end{split}
\end{equation*}

\subsubsection{Trasformatore con flussi dispersi e riluttanza \protect\(\theta_{Fe}\protect\) nel traferro}
\label{\detokenize{ch/circuits-electromagnetic-transformer:trasformatore-con-flussi-dispersi-e-riluttanza-theta-fe-nel-traferro}}\begin{equation*}
\begin{split}\begin{cases}
 & \phi_{1} - \phi_{1,d} = \phi \\
 & \phi_{2} - \phi_{2,d} = \phi \\
 & m_{1} = \theta_{1,d} \phi_{1,d} \\
 & m_{2} = \theta_{2,d} \phi_{2,d} \\
 & m_1   + m_{2} = \theta_{Fe} \, \phi
\end{cases}\end{split}
\end{equation*}
\sphinxAtStartPar
\sphinxstylestrong{todo} finire e controllare i conti; disegnare circuito equivalente



\sphinxstepscope

\begin{sphinxuseclass}{sd-container-fluid}
\begin{sphinxuseclass}{sd-sphinx-override}
\begin{sphinxuseclass}{sd-p-0}
\begin{sphinxuseclass}{sd-mt-2}
\begin{sphinxuseclass}{sd-mb-4}
\begin{sphinxuseclass}{sd-row}
\begin{sphinxuseclass}{sd-row-cols-2}
\begin{sphinxuseclass}{sd-gx-2}
\begin{sphinxuseclass}{sd-gy-1}
\begin{sphinxuseclass}{sd-col}
\begin{sphinxuseclass}{sd-d-flex-row}
\begin{sphinxuseclass}{sd-align-minor-center}
\begin{sphinxuseclass}{sd-container-fluid}
\begin{sphinxuseclass}{sd-sphinx-override}
\begin{sphinxuseclass}{sd-row}
\begin{sphinxuseclass}{sd-row-cols-2}
\begin{sphinxuseclass}{sd-row-cols-xs-2}
\begin{sphinxuseclass}{sd-row-cols-sm-3}
\begin{sphinxuseclass}{sd-row-cols-md-3}
\begin{sphinxuseclass}{sd-row-cols-lg-3}
\begin{sphinxuseclass}{sd-gx-3}
\begin{sphinxuseclass}{sd-gy-1}
\begin{sphinxuseclass}{sd-col}
\begin{sphinxuseclass}{sd-col-auto}
\begin{sphinxuseclass}{sd-d-flex-row}
\begin{sphinxuseclass}{sd-align-minor-center}
\sphinxAtStartPar
basics

\end{sphinxuseclass}
\end{sphinxuseclass}
\end{sphinxuseclass}
\end{sphinxuseclass}
\begin{sphinxuseclass}{sd-col}
\begin{sphinxuseclass}{sd-col-auto}
\begin{sphinxuseclass}{sd-d-flex-row}
\begin{sphinxuseclass}{sd-align-minor-center}
\sphinxAtStartPar
Nov 09, 2024

\end{sphinxuseclass}
\end{sphinxuseclass}
\end{sphinxuseclass}
\end{sphinxuseclass}
\begin{sphinxuseclass}{sd-col}
\begin{sphinxuseclass}{sd-col-auto}
\begin{sphinxuseclass}{sd-d-flex-row}
\begin{sphinxuseclass}{sd-align-minor-center}
\sphinxAtStartPar
2 min read

\end{sphinxuseclass}
\end{sphinxuseclass}
\end{sphinxuseclass}
\end{sphinxuseclass}
\end{sphinxuseclass}
\end{sphinxuseclass}
\end{sphinxuseclass}
\end{sphinxuseclass}
\end{sphinxuseclass}
\end{sphinxuseclass}
\end{sphinxuseclass}
\end{sphinxuseclass}
\end{sphinxuseclass}
\end{sphinxuseclass}
\end{sphinxuseclass}
\end{sphinxuseclass}
\end{sphinxuseclass}
\end{sphinxuseclass}
\end{sphinxuseclass}
\end{sphinxuseclass}
\end{sphinxuseclass}
\end{sphinxuseclass}
\end{sphinxuseclass}
\end{sphinxuseclass}
\end{sphinxuseclass}
\end{sphinxuseclass}

\section{Circuiti elettromeccanici}
\label{\detokenize{ch/circuits-electromechanic:circuiti-elettromeccanici}}\label{\detokenize{ch/circuits-electromechanic:classical-electromagnetism-circuits-electromechanic}}\label{\detokenize{ch/circuits-electromechanic::doc}}
\sphinxAtStartPar
Alcuni sistemi di interesse e di enorme diffusione nella società moderna sfruttano le interazioni tra componenti fenomeni elettromagnetici e meccanici: un esempio fondamentale sono le macchine elettriche, alcune delle quali possono operare sia come motore (con la potenza fornita dal sistema elettrico e convertita in potenza meccanica) sia come generatore di energia elettrica (convertendo potenza meccanica in potenza elettrica).

\sphinxAtStartPar
In un sistema di induttori con mutua influenza, la differenza di tensione ai capi dell’induttore “potenziato” \(i\) è
\begin{equation*}
\begin{split}v_i = \dot{\psi}_i = \dfrac{d}{dt} \left( N_i \, \phi_i \right) \ .\end{split}
\end{equation*}
\sphinxAtStartPar
Il flusso concatenato dipende dall’effetto di tutti gli induttori del sistema (e del campo magnetico generato da eventuali cause esterne al sistema),
\begin{equation*}
\begin{split}\phi_i = \sum_{k} \phi_{ik} = \sum_{k} \frac{1}{\theta_{ik}} \, m_k \ ,\end{split}
\end{equation*}
\sphinxAtStartPar
avendo indicato con \(\theta_{ik}\) la riluttanza del circuito tra l’induttore potenziante \(k\) e l’induttore potenziato \(i\). Usando l’espressione della forza magneto\sphinxhyphen{}motrice \(m_k = N_k \, i_k\), si può riscrivere l’espressione della differenza di tensione
\begin{equation*}
\begin{split}v_i = \sum_k \frac{d}{dt} \left( \frac{N_i \, N_k}{\theta_{ik}} i_k \right) = \sum_k \frac{d}{dt} \left( L_{ik} \, i_k \right) \ .\end{split}
\end{equation*}
\sphinxAtStartPar
In genereale, in circuiti elettromeccanici le riluttanze non sono dei parametri costanti del sistema ma dipendono dallo stato “meccanico” del sistema, descritto qui dalle variabili \(\mathbf{x}\),
\begin{equation*}
\begin{split}v_i = \sum_k \frac{d}{dt} \left( \frac{N_i \, N_k}{\theta_{ik}(\mathbf{x})} i_k \right) = \sum_k \frac{d}{dt} \left( L_{ik} (\mathbf{x}) \, i_k \right) \ .\end{split}
\end{equation*}\begin{equation*}
\begin{split}\mathbf{v}(t) = \dfrac{d}{dt} \Big( \mathbf{L}(\mathbf{x}(t)) \, \mathbf{i}(t) \Big) \ .\end{split}
\end{equation*}
\sphinxAtStartPar
La matrice di induttanza \(\mathbf{L}\) è simmetrica \sphinxstylestrong{todo} \sphinxstyleemphasis{Dimostrazione}


\subsection{Sistemi elettromeccanici conservativi}
\label{\detokenize{ch/circuits-electromechanic:sistemi-elettromeccanici-conservativi}}
\sphinxAtStartPar
Le equazioni che governano il sistema elettromeccanico, senza condensatori, in generale possono essere scritte come
\begin{equation*}
\begin{split}\begin{cases}
 \mathbf{M} \ddot{\mathbf{x}} + \mathbf{D} \dot{\mathbf{x}} + \mathbf{K} \mathbf{x} = \mathbf{f}^{ext} + \mathbf{f}^{em} \\
 \dfrac{d}{dt} \left( \mathbf{L} \mathbf{i} \right) + \mathbf{R} \mathbf{i} = \mathbf{e}
\end{cases}\end{split}
\end{equation*}
\sphinxAtStartPar
In termini di energia,
\begin{equation*}
\begin{split}
0 = \dot{\mathbf{x}}^T \left[ \mathbf{M} \ddot{\mathbf{x}} + \mathbf{D} \dot{\mathbf{x}} + \mathbf{K} \mathbf{x} - \mathbf{f}^{ext} - \mathbf{f}^{em} \right] + \mathbf{i}^T \left[ \dfrac{d}{dt} \left( \mathbf{L} \mathbf{i} \right) + \mathbf{R} \mathbf{i} - \mathbf{e} \right]
\end{split}
\end{equation*}
\sphinxAtStartPar
Nel caso di matrici di massa, smorzamento e rigidzza costanti, e usando la derivata del prodotto per ottenere un termine di derivata dell’energia degli induttori sfruttando la simmetria di \(\mathbf{L}\),
\begin{equation}\label{equation:ch/circuits-electromechanic:classical-electromagnetism:circuits-electromechanic:energy-mech-0}
\begin{split} \begin{aligned}
\dfrac{d}{dt} \left[ \frac{1}{2} \mathbf{i}^T \mathbf{L} \mathbf{i} \right] 
  & = \mathbf{i}^T \dfrac{d}{dt} \left( \mathbf{L} \, \mathbf{i} \right) + \frac{1}{2} \mathbf{i}^T \dfrac{d \mathbf{L}}{dt} \mathbf{i} = \\
  & = \mathbf{i}^T \dfrac{d}{dt} \left( \mathbf{L} \, \mathbf{i} \right) + \sum_{a} \frac{1}{2} \mathbf{i}^T \dfrac{\partial \mathbf{L}}{\partial x_a} \mathbf{i} \, \dot{x}_a = \\
  & = \mathbf{i}^T \dfrac{d}{dt} \left( \mathbf{L} \, \mathbf{i} \right) + \nabla \left( \frac{1}{2} \mathbf{i}^T \mathbf{L} \mathbf{i} \right) \dot{\mathbf{x}}  \ .
\end{aligned}\end{split}
\end{equation}
\sphinxAtStartPar
si può scrivere un’equazione di bilancio dell’energia meccanica macroscopica, \(E^{mec, int}\)
\begin{equation*}
\begin{split}
0 & = \dfrac{d}{dt} \left[ \dfrac{1}{2} \dot{\mathbf{x}}^T \mathbf{M} \dot{\mathbf{x}} + \dfrac{1}{2} \mathbf{x}^T \mathbf{K} \mathbf{x} + \dfrac{1}{2} \mathbf{i}^T \mathbf{L} \mathbf{i} \right] - \dot{\mathbf{x}}^T \left( \mathbf{f}^{em} - \nabla E^{ind}(\mathbf{x}, \mathbf{i})  \right) + \\
  & - \dot{\mathbf{x}}^T \mathbf{f}^{ext} - \mathbf{i}^T \mathbf{e} + \\
  & + \dot{\mathbf{x}}^T \mathbf{C} \dot{\mathbf{x}} + \mathbf{i}^T \mathbf{R} \mathbf{i} \ .
\end{split}
\end{equation*}
\sphinxAtStartPar
Nell’ipotesi che il processo sia conservativo, si ricava la forma delle forze dovute ai fenomeni elettromagnetici,
\begin{equation}\label{equation:ch/circuits-electromechanic:classical-electromagnetism:circuits-electromechanic:f-em}
\begin{split}\mathbf{f}^{em} = \nabla_{\mathbf{x}} E^{ind}(\mathbf{x}, \mathbf{i}) \ .\end{split}
\end{equation}

\subsection{Equazioni di governo}
\label{\detokenize{ch/circuits-electromechanic:equazioni-di-governo}}
\sphinxAtStartPar
Usando l’espressione \eqref{equation:ch/circuits-electromechanic:classical-electromagnetism:circuits-electromechanic:f-em} delle azioni meccaniche dovute agli effetti elettromagnetici, del sistema sono
\begin{equation*}
\begin{split}\begin{cases}
  \mathbf{M} \ddot{\mathbf{x}} + \mathbf{D} \dot{\mathbf{x}} + \mathbf{K} \mathbf{x} - \nabla_{\mathbf{x}} E^{ind}(\mathbf{x}, \mathbf{i})  = \mathbf{f}^{ext} \\
  \frac{d}{dt} \left( \mathbf{L}(\mathbf{x}) \mathbf{i} \right) + \mathbf{R} \mathbf{i} = \mathbf{e}
\end{cases}\end{split}
\end{equation*}
\sphinxAtStartPar
o nel caso generale
\begin{equation*}
\begin{split}\begin{cases}
  \mathbf{M} \ddot{\mathbf{x}} - \nabla_{\mathbf{x}} E^{ind} ( \mathbf{x}, \mathbf{i}) = \mathbf{f}^{ext} \\
  \frac{d}{dt} \left( \mathbf{L}(\mathbf{x}) \mathbf{i} \right) + \mathbf{R} \mathbf{i} = \mathbf{e}
\end{cases}\end{split}
\end{equation*}

\subsection{Bilancio energetico}
\label{\detokenize{ch/circuits-electromechanic:bilancio-energetico}}

\subsubsection{Energia meccanica macroscopica}
\label{\detokenize{ch/circuits-electromechanic:energia-meccanica-macroscopica}}
\sphinxAtStartPar
Usando l’espressione \eqref{equation:ch/circuits-electromechanic:classical-electromagnetism:circuits-electromechanic:f-em} delle azioni meccaniche dovute ai fenomeni elettromagnetici, si può riscrivere la relazione \eqref{equation:ch/circuits-electromechanic:classical-electromagnetism:circuits-electromechanic:energy-mech-0}, come un bilancio di energia meccanica macroscopica del sistema,
\begin{equation*}
\begin{split}\dfrac{d}{dt} \left[ \dfrac{1}{2} \dot{\mathbf{x}}^T \mathbf{M} \dot{\mathbf{x}} + \dfrac{1}{2} \mathbf{x}^T \mathbf{K} \mathbf{x} + \dfrac{1}{2} \mathbf{i}^T \mathbf{L} \mathbf{i} \right] = \dot{\mathbf{x}}^T \mathbf{f}^{ext} + \mathbf{i}^T \mathbf{e} - \dot{\mathbf{x}}^T \mathbf{D} \dot{\mathbf{x}} - \mathbf{i}^T \mathbf{R} \mathbf{i} \ , \end{split}
\end{equation*}
\sphinxAtStartPar
e quindi
\begin{equation*}
\begin{split}\dot{E}^{mec} = P^{ext} - \dot{D} \ .\end{split}
\end{equation*}

\subsubsection{Energia cinetica}
\label{\detokenize{ch/circuits-electromechanic:energia-cinetica}}
\sphinxAtStartPar
L’energia meccanica macroscopica può essere scritta come la somma dell’energia cinetica e dell’energia potenziale interna del sistema, \(E^{mec} = K + V^{int}\). La derivata nel tempo dell’energia potenziale delle azioni interne è l’opposto della potenza delle azioni interne conservative, \(P^{int, c} = - \dot{V}^{int}\); la dissipazione è l’opposto della potenza delle azioni interne non\sphinxhyphen{}conservative, \(P^{int, nc} = - \dot{D}\). La potenza complessiva delle azioni interne può quindi essere scritta come
\begin{equation*}
\begin{split}P^{int} = P^{int, c} + P^{int, nc} = - \dot{V}^{int} - \dot{D} \ ,\end{split}
\end{equation*}\begin{equation*}
\begin{split}\dot{K} = \dot{E}^{mec} - \dot{V}^{int} = P^{ext} \underbrace{- \dot{D} - \dot{V}^{int}}_{=P^{int}} \  \end{split}
\end{equation*}

\subsubsection{Energia totale}
\label{\detokenize{ch/circuits-electromechanic:energia-totale}}
\sphinxAtStartPar
Il primo principio della termodinamica fornisce l’equazione di bilancio dell’energia totale di un sistema chiuso,
\begin{equation*}
\begin{split}\dot{E}^{tot} = P^{ext} + \dot{Q}^{ext} \ .\end{split}
\end{equation*}

\subsubsection{Energia interna}
\label{\detokenize{ch/circuits-electromechanic:energia-interna}}
\sphinxAtStartPar
L’energia interna di un sistema è definita come la differenza dell’energia totale e dell’energia cinetica macroscopica, \(E := E^{tot} - K\). L’equazione di bilancio dell’energia interna di un sistema chiuso è
\begin{equation*}
\begin{split}\dot{E} = Q^{ext} - P^{int} \ .\end{split}
\end{equation*}

\subsubsection{Energia interna termica (microscopica)}
\label{\detokenize{ch/circuits-electromechanic:energia-interna-termica-microscopica}}
\sphinxAtStartPar
Se si definisce l’energia interna termica, corrispondente all’energia cinetica associata alle dinamiche microscopiche, come differenza tra energia interna e energia potenziale interna, o differenza di energia totale ed energia meccanica macrsocopica,
\begin{equation*}
\begin{split}\begin{aligned}
  E^{th} & = E - V^{int} = \\
         & = E^{tot} - E^{mec} \ ,
\end{aligned}\end{split}
\end{equation*}
\sphinxAtStartPar
l’equazione di bilancio dell’energia interna termica è
\begin{equation*}
\begin{split}   \dot{E}^{th} = \dot{Q}^{ext} + \dot{D} \ . \end{split}
\end{equation*}\subsubsection*{Dimostrazione}
\begin{equation*}
\begin{split}\begin{aligned}
  \dot{E}^{th} = \dot{E} - V^{int}
    & = \dot{Q}^{ext} - P^{int} - V^{int} = \\
    & = \dot{Q}^{ext} + \dot{D} + \dot{V}^{int} - \dot{V}^{int} = \\
    & = \dot{Q}^{ext} + \dot{D} \ .
\end{aligned}\end{split}
\end{equation*}
\sphinxAtStartPar
\sphinxstylestrong{Con condensatori.} \sphinxstylestrong{todo}
\subsubsection*{Equazioni}
\begin{itemize}
\item {} 
\sphinxAtStartPar
\sphinxstylestrong{Leggi ai nodi.}
\begin{equation*}
\begin{split}0 = \sum_{k \in B_j} \alpha_{jk} \, i_{jk}\end{split}
\end{equation*}\begin{equation*}
\begin{split}\mathbf{A} \mathbf{i} = \mathbf{0}\end{split}
\end{equation*}
\item {} 
\sphinxAtStartPar
\sphinxstylestrong{Differenza di potenziale nodi\sphinxhyphen{}lati.}
\begin{equation*}
\begin{split}\mathbf{A}^T \mathbf{v}_{n} = \mathbf{v}\end{split}
\end{equation*}
\item {} 
\sphinxAtStartPar
\sphinxstylestrong{Nodo a terra.}
\begin{equation*}
\begin{split}\mathbf{v}_{\perp} = \mathbf{v}_0 \ .\end{split}
\end{equation*}
\item {} 
\sphinxAtStartPar
\sphinxstylestrong{Equazioni costitutive.}
\begin{equation*}
\begin{split}\begin{aligned}
    \mathbf{0} & = \mathbf{v}_R - \mathbf{R} \mathbf{i}_R & \text{resistenze} \\
    \mathbf{0} & = \mathbf{v}_L - \frac{d}{dt} \left( \mathbf{L} \mathbf{i}_L \right) & \text{induttanze} \\
    \mathbf{0} & = \frac{d}{dt} \left( C \mathbf{v}_C \right) - \mathbf{i}_C & \text{condensatori} \\
  \end{aligned}\end{split}
\end{equation*}
\end{itemize}

\sphinxstepscope


\part{Metodi numerici}

\sphinxstepscope

\begin{sphinxuseclass}{sd-container-fluid}
\begin{sphinxuseclass}{sd-sphinx-override}
\begin{sphinxuseclass}{sd-p-0}
\begin{sphinxuseclass}{sd-mt-2}
\begin{sphinxuseclass}{sd-mb-4}
\begin{sphinxuseclass}{sd-row}
\begin{sphinxuseclass}{sd-row-cols-2}
\begin{sphinxuseclass}{sd-gx-2}
\begin{sphinxuseclass}{sd-gy-1}
\begin{sphinxuseclass}{sd-col}
\begin{sphinxuseclass}{sd-d-flex-row}
\begin{sphinxuseclass}{sd-align-minor-center}
\begin{sphinxuseclass}{sd-container-fluid}
\begin{sphinxuseclass}{sd-sphinx-override}
\begin{sphinxuseclass}{sd-row}
\begin{sphinxuseclass}{sd-row-cols-2}
\begin{sphinxuseclass}{sd-row-cols-xs-2}
\begin{sphinxuseclass}{sd-row-cols-sm-3}
\begin{sphinxuseclass}{sd-row-cols-md-3}
\begin{sphinxuseclass}{sd-row-cols-lg-3}
\begin{sphinxuseclass}{sd-gx-3}
\begin{sphinxuseclass}{sd-gy-1}
\begin{sphinxuseclass}{sd-col}
\begin{sphinxuseclass}{sd-col-auto}
\begin{sphinxuseclass}{sd-d-flex-row}
\begin{sphinxuseclass}{sd-align-minor-center}
\sphinxAtStartPar
basics

\end{sphinxuseclass}
\end{sphinxuseclass}
\end{sphinxuseclass}
\end{sphinxuseclass}
\begin{sphinxuseclass}{sd-col}
\begin{sphinxuseclass}{sd-col-auto}
\begin{sphinxuseclass}{sd-d-flex-row}
\begin{sphinxuseclass}{sd-align-minor-center}
\sphinxAtStartPar
Nov 09, 2024

\end{sphinxuseclass}
\end{sphinxuseclass}
\end{sphinxuseclass}
\end{sphinxuseclass}
\begin{sphinxuseclass}{sd-col}
\begin{sphinxuseclass}{sd-col-auto}
\begin{sphinxuseclass}{sd-d-flex-row}
\begin{sphinxuseclass}{sd-align-minor-center}
\sphinxAtStartPar
1 min read

\end{sphinxuseclass}
\end{sphinxuseclass}
\end{sphinxuseclass}
\end{sphinxuseclass}
\end{sphinxuseclass}
\end{sphinxuseclass}
\end{sphinxuseclass}
\end{sphinxuseclass}
\end{sphinxuseclass}
\end{sphinxuseclass}
\end{sphinxuseclass}
\end{sphinxuseclass}
\end{sphinxuseclass}
\end{sphinxuseclass}
\end{sphinxuseclass}
\end{sphinxuseclass}
\end{sphinxuseclass}
\end{sphinxuseclass}
\end{sphinxuseclass}
\end{sphinxuseclass}
\end{sphinxuseclass}
\end{sphinxuseclass}
\end{sphinxuseclass}
\end{sphinxuseclass}
\end{sphinxuseclass}
\end{sphinxuseclass}

\chapter{Metodi numerici}
\label{\detokenize{ch/numerical-methods:metodi-numerici}}\label{\detokenize{ch/numerical-methods:classical-electromagnetism-numerics}}\label{\detokenize{ch/numerical-methods::doc}}

\section{Elettrostatica}
\label{\detokenize{ch/numerical-methods:elettrostatica}}
\sphinxAtStartPar
I problemi dell’elettrostatica sono governate dalle due equazioni di Maxwell per i campi \(\mathbf{e}\), \(\mathbf{d}\),
\begin{equation*}
\begin{split}\begin{cases}
  \nabla \cdot \mathbf{d} = \rho \\ \\
  \nabla \times \mathbf{e} = \mathbf{0 \ ,}
\end{cases}\end{split}
\end{equation*}
\sphinxAtStartPar
dotate delle opportune condizioni al contorno ed equazioni costitutive. Per un materiale lineare isotropo, ad esempio, \(\mathbf{d} = \varepsilon \mathbf{e}\). La condizione di irrotazionalità del campo elettrico, permette di scriverlo come gradiente di un potenziale scalare, \(\mathbf{e} = - \nabla v\), e di ottenere l’equazione di Poisson,
\begin{equation*}
\begin{split}-\nabla \cdot (\varepsilon \nabla v ) = \rho \ .\end{split}
\end{equation*}

\subsection{Sorgente}
\label{\detokenize{ch/numerical-methods:sorgente}}\begin{equation*}
\begin{split}\mathbf{e}(r) = \frac{q_i}{4 \pi \varepsilon}\frac{\mathbf{r} - \mathbf{r}_i}{|\mathbf{r} - \mathbf{r}_i|^3}\end{split}
\end{equation*}\begin{equation*}
\begin{split}\mathbf{e}(\mathbf{r}) = - \nabla_{\mathbf{r}} v(\mathbf{r})\end{split}
\end{equation*}\begin{equation*}
\begin{split}\varepsilon \, v(\mathbf{r}) = \frac{q_i}{4 \pi}\frac{1}{|\mathbf{r} - \mathbf{r}_i|}\end{split}
\end{equation*}

\subsection{Dipolo}
\label{\detokenize{ch/numerical-methods:dipolo}}
\sphinxAtStartPar
Un dipolo è definito come due cariche di intensità uguale e contraria \(-q_2 = q_1 = q > 0\), nei punti dello spazio \(P_1\), \(P_2 = P_1 + \mathbf{l}\), nelle condizioni limite \(|\mathbf{l}| \rightarrow 0\), \(q \rightarrow \infty\), in modo tale da avere \(q |\mathbf{l}|\) finito, \(\mathbf{p} = q \mathbf{l}\).

\sphinxAtStartPar
Il potenziale del dipolo è dato dal principio di sovrapposizione delle cause e degli effetti,
\begin{equation*}
\begin{split}\begin{aligned}
  \varepsilon \, v(\mathbf{r})
  & = - \frac{q}{4 \pi }\frac{1}{\left|\mathbf{r} - \mathbf{r}_0 + \frac{\mathbf{l}}{2} \right|} 
      + \frac{q}{4 \pi }\frac{1}{\left|\mathbf{r} - \mathbf{r}_0 - \frac{\mathbf{l}}{2} \right|} = \\
  & = \ ... \\
  & = \frac{q}{4 \pi} \left( 
  - \frac{1}{\left|\mathbf{r} - \mathbf{r}_0 \right|} + \frac{\mathbf{r} - \mathbf{r}_0}{\left|\mathbf{r} - \mathbf{r}_0 \right|^3} \cdot \frac{\mathbf{l}}{2}
  + \frac{1}{\left|\mathbf{r} - \mathbf{r}_0 \right|} + \frac{\mathbf{r} - \mathbf{r}_0}{\left|\mathbf{r} - \mathbf{r}_0 \right|^3} \cdot \frac{\mathbf{l}}{2} + o(|\mathbf{l}|) \right) = \\
  & = \ ... \\
  & = \frac{1}{4 \pi}
 \frac{\mathbf{r} - \mathbf{r}_0}{\left|\mathbf{r} - \mathbf{r}_0 \right|^3} \cdot \mathbf{P} \ ,
\end{aligned}\end{split}
\end{equation*}
\sphinxAtStartPar
avendo definito il vettore momento dipolo \(\mathbf{P} = q \mathbf{l}\).

\sphinxAtStartPar
\sphinxstylestrong{Polariazazione \sphinxhyphen{} Potenziale generato da una distribuzione di dipoli.}
\begin{equation*}
\begin{split}d \mathbf{P} = \mathbf{p} \, \Delta V\end{split}
\end{equation*}\begin{equation*}
\begin{split}\varepsilon v_P(\mathbf{r}) = \int_{\mathbf{r}_0 \in V_0} \frac{1}{4 \pi}
 \frac{\mathbf{r} - \mathbf{r}_0}{\left|\mathbf{r} - \mathbf{r}_0 \right|^3} \cdot \mathbf{p}(\mathbf{r}_0) \, dV_0 \end{split}
\end{equation*}\begin{equation*}
\begin{split}\begin{aligned}
\partial_i |\mathbf{r}|^2 & = 2 x_i \\
                          & = 2 |\mathbf{r}| \partial_i |\mathbf{r}|
\end{aligned}
\qquad \rightarrow \qquad \partial_i |\mathbf{r}| = \frac{x_i}{|\mathbf{r}|}\end{split}
\end{equation*}\begin{equation*}
\begin{split}\partial_i |\mathbf{r}|^n = n |\mathbf{r}|^{n-1} \, \partial_i |\mathbf{r}| = n x_i |\mathbf{r}|^{n-2}\end{split}
\end{equation*}\begin{equation*}
\begin{split}\frac{\mathbf{r}-\mathbf{r}_0}{|\mathbf{r}-\mathbf{r}_0|^3} = \nabla_{\mathbf{r}_0} \frac{1}{|\mathbf{r}-\mathbf{r}_0|}\end{split}
\end{equation*}\begin{equation*}
\begin{split}\begin{aligned}
\frac{\mathbf{r}- \mathbf{r}_0}{|\mathbf{r}- \mathbf{r}_0|^3} \cdot \mathbf{p}(\mathbf{r}_0) 
 & = \nabla_{\mathbf{r}_0} \frac{1}{|\mathbf{r}-\mathbf{r}_0|} \cdot \mathbf{p}(\mathbf{r}_0) = \\
 & = \nabla_{\mathbf{r}_0} \cdot \left( \frac{1}{|\mathbf{r}-\mathbf{r}_0|} \mathbf{p}(\mathbf{r}_0) \right) - \frac{1}{|\mathbf{r}- \mathbf{r}_0|} \nabla_{\mathbf{r}_0} \cdot \mathbf{p}(\mathbf{r}_0) = \\
\end{aligned}\end{split}
\end{equation*}
\sphinxAtStartPar
e quindi
\begin{equation*}
\begin{split}4 \, \pi \, \varepsilon v_P(\mathbf{r}) = \oint_{\mathbf{r}_0 \in \partial V_0} \frac{\hat{\mathbf{n}}(\mathbf{r}_0) \cdot \mathbf{p}(\mathbf{r}_0)}{|\mathbf{r}-\mathbf{r}_0|} - \oint_{\mathbf{r}_0 \in V_0} \frac{\nabla_{\mathbf{r}_0} \cdot \mathbf{p}(\mathbf{r}_0)}{|\mathbf{r} - \mathbf{r}_0|}\end{split}
\end{equation*}
\sphinxAtStartPar
I due contributi hanno la forma di sorgenti, essendo termini proporzionali a \(\frac{1}{|\mathbf{r}-\mathbf{r}_0|}\).
Il potenziale dovuto alla densità di volume di dipoli equivale alla somma dei due contributi delle cariche di:
\begin{itemize}
\item {} 
\sphinxAtStartPar
polarizzazione di superficie \(\sigma_p =   \hat{\mathbf{n}} \cdot \mathbf{p}\)

\item {} 
\sphinxAtStartPar
polarizzazione di volume     \(\rho_p   = - \nabla \cdot \mathbf{p}\)

\end{itemize}

\sphinxAtStartPar
\sphinxstylestrong{Oss.} Se la polarizzazione è uniforme nel volume, il contributo della polarizzazione nel volume si annulla e rimane solo il contributo della polarizzazione sul contorno del volume.

\sphinxAtStartPar
\sphinxstylestrong{Oss.} Legge di Gauss per il campo elettrico,
\begin{equation*}
\begin{split}\begin{aligned}
  \nabla \cdot \mathbf{e} & = \frac{1}{\varepsilon_0} \rho = \\
                          & = \frac{1}{\varepsilon_0} \left( \rho_l + \rho_p \right) = \\
                          & = \frac{1}{\varepsilon_0} \left( \rho_l - \nabla \cdot \mathbf{p} \right) \\
  \nabla \cdot \left( \varepsilon_0 \mathbf{e} + \mathbf{p} \right) & = \rho_l \\
  \nabla \cdot  \mathbf{d} & = \rho_l
\end{aligned}\end{split}
\end{equation*}
\sphinxstepscope


\part{Appendici}

\sphinxstepscope

\begin{sphinxuseclass}{sd-container-fluid}
\begin{sphinxuseclass}{sd-sphinx-override}
\begin{sphinxuseclass}{sd-p-0}
\begin{sphinxuseclass}{sd-mt-2}
\begin{sphinxuseclass}{sd-mb-4}
\begin{sphinxuseclass}{sd-row}
\begin{sphinxuseclass}{sd-row-cols-2}
\begin{sphinxuseclass}{sd-gx-2}
\begin{sphinxuseclass}{sd-gy-1}
\begin{sphinxuseclass}{sd-col}
\begin{sphinxuseclass}{sd-d-flex-row}
\begin{sphinxuseclass}{sd-align-minor-center}
\begin{sphinxuseclass}{sd-container-fluid}
\begin{sphinxuseclass}{sd-sphinx-override}
\begin{sphinxuseclass}{sd-row}
\begin{sphinxuseclass}{sd-row-cols-2}
\begin{sphinxuseclass}{sd-row-cols-xs-2}
\begin{sphinxuseclass}{sd-row-cols-sm-3}
\begin{sphinxuseclass}{sd-row-cols-md-3}
\begin{sphinxuseclass}{sd-row-cols-lg-3}
\begin{sphinxuseclass}{sd-gx-3}
\begin{sphinxuseclass}{sd-gy-1}
\begin{sphinxuseclass}{sd-col}
\begin{sphinxuseclass}{sd-col-auto}
\begin{sphinxuseclass}{sd-d-flex-row}
\begin{sphinxuseclass}{sd-align-minor-center}
\sphinxAtStartPar
basics

\end{sphinxuseclass}
\end{sphinxuseclass}
\end{sphinxuseclass}
\end{sphinxuseclass}
\begin{sphinxuseclass}{sd-col}
\begin{sphinxuseclass}{sd-col-auto}
\begin{sphinxuseclass}{sd-d-flex-row}
\begin{sphinxuseclass}{sd-align-minor-center}
\sphinxAtStartPar
Nov 09, 2024

\end{sphinxuseclass}
\end{sphinxuseclass}
\end{sphinxuseclass}
\end{sphinxuseclass}
\begin{sphinxuseclass}{sd-col}
\begin{sphinxuseclass}{sd-col-auto}
\begin{sphinxuseclass}{sd-d-flex-row}
\begin{sphinxuseclass}{sd-align-minor-center}
\sphinxAtStartPar
0 min read

\end{sphinxuseclass}
\end{sphinxuseclass}
\end{sphinxuseclass}
\end{sphinxuseclass}
\end{sphinxuseclass}
\end{sphinxuseclass}
\end{sphinxuseclass}
\end{sphinxuseclass}
\end{sphinxuseclass}
\end{sphinxuseclass}
\end{sphinxuseclass}
\end{sphinxuseclass}
\end{sphinxuseclass}
\end{sphinxuseclass}
\end{sphinxuseclass}
\end{sphinxuseclass}
\end{sphinxuseclass}
\end{sphinxuseclass}
\end{sphinxuseclass}
\end{sphinxuseclass}
\end{sphinxuseclass}
\end{sphinxuseclass}
\end{sphinxuseclass}
\end{sphinxuseclass}
\end{sphinxuseclass}
\end{sphinxuseclass}

\chapter{Ottica}
\label{\detokenize{ch/optics:ottica}}\label{\detokenize{ch/optics:classical-electromagnetism-optics}}\label{\detokenize{ch/optics::doc}}






\renewcommand{\indexname}{Index}
\printindex
\end{document}