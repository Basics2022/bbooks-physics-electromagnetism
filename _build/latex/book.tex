%% Generated by Sphinx.
\def\sphinxdocclass{jupyterBook}
\documentclass[letterpaper,10pt,italian]{jupyterBook}
\ifdefined\pdfpxdimen
   \let\sphinxpxdimen\pdfpxdimen\else\newdimen\sphinxpxdimen
\fi \sphinxpxdimen=.75bp\relax
\ifdefined\pdfimageresolution
    \pdfimageresolution= \numexpr \dimexpr1in\relax/\sphinxpxdimen\relax
\fi
%% let collapsible pdf bookmarks panel have high depth per default
\PassOptionsToPackage{bookmarksdepth=5}{hyperref}
%% turn off hyperref patch of \index as sphinx.xdy xindy module takes care of
%% suitable \hyperpage mark-up, working around hyperref-xindy incompatibility
\PassOptionsToPackage{hyperindex=false}{hyperref}
%% memoir class requires extra handling
\makeatletter\@ifclassloaded{memoir}
{\ifdefined\memhyperindexfalse\memhyperindexfalse\fi}{}\makeatother

\PassOptionsToPackage{warn}{textcomp}

\catcode`^^^^00a0\active\protected\def^^^^00a0{\leavevmode\nobreak\ }
\usepackage{cmap}
\usepackage{fontspec}
\defaultfontfeatures[\rmfamily,\sffamily,\ttfamily]{}
\usepackage{amsmath,amssymb,amstext}
\usepackage{polyglossia}
\setmainlanguage{italian}



\setmainfont{FreeSerif}[
  Extension      = .otf,
  UprightFont    = *,
  ItalicFont     = *Italic,
  BoldFont       = *Bold,
  BoldItalicFont = *BoldItalic
]
\setsansfont{FreeSans}[
  Extension      = .otf,
  UprightFont    = *,
  ItalicFont     = *Oblique,
  BoldFont       = *Bold,
  BoldItalicFont = *BoldOblique,
]
\setmonofont{FreeMono}[
  Extension      = .otf,
  UprightFont    = *,
  ItalicFont     = *Oblique,
  BoldFont       = *Bold,
  BoldItalicFont = *BoldOblique,
]



\usepackage[Sonny]{fncychap}
\ChNameVar{\Large\normalfont\sffamily}
\ChTitleVar{\Large\normalfont\sffamily}
\usepackage[,numfigreset=1,mathnumfig]{sphinx}

\fvset{fontsize=\small}
\usepackage{geometry}


% Include hyperref last.
\usepackage{hyperref}
% Fix anchor placement for figures with captions.
\usepackage{hypcap}% it must be loaded after hyperref.
% Set up styles of URL: it should be placed after hyperref.
\urlstyle{same}

\addto\captionsitalian{\renewcommand{\contentsname}{Elettromagnetismo}}

\usepackage{sphinxmessages}



        % Start of preamble defined in sphinx-jupyterbook-latex %
         \usepackage[Latin,Greek]{ucharclasses}
        \usepackage{unicode-math}
        % fixing title of the toc
        \addto\captionsenglish{\renewcommand{\contentsname}{Contents}}
        \hypersetup{
            pdfencoding=auto,
            psdextra
        }
        % End of preamble defined in sphinx-jupyterbook-latex %
        

\title{electromagnetism}
\date{26 apr 2025}
\release{}
\author{basics}
\newcommand{\sphinxlogo}{\vbox{}}
\renewcommand{\releasename}{}
\makeindex
\begin{document}

\pagestyle{empty}
\sphinxmaketitle
\pagestyle{plain}
\sphinxtableofcontents
\pagestyle{normal}
\phantomsection\label{\detokenize{intro::doc}}


\sphinxAtStartPar
This material is part of the \sphinxhref{https://basics2022.github.io/bbooks}{\sphinxstylestrong{basics\sphinxhyphen{}books project}}. It is also available as a \DUrole{xref,download,myst}{.pdf document}.

\sphinxAtStartPar
\sphinxstylestrong{Introduzione.}

\sphinxAtStartPar
L’elettromagnetismo si occupa dello studio dei fenomeni elettromagnetici prodotti da cariche e correnti elettriche o dalla struttura microscopica della materia (magnetismo naturale)

\sphinxAtStartPar
\sphinxstylestrong{Breve storia.} \sphinxstyleemphasis{Prime esperienze: cariche di 2 tipi diversi e legge di Coulomb;}

\sphinxAtStartPar
\sphinxstylestrong{Argomenti.}

\sphinxAtStartPar
{\hyperref[\detokenize{ch/experiments:classical-electromagnetism-first-experiments}]{\sphinxcrossref{\DUrole{std,std-ref}{Prime esperienze}}}} \sphinxstylestrong{TODO} \sphinxstyleemphasis{Prime esperienze; elettromagnetismo come teoria dei} \sphinxstylestrong{campi} \sphinxstylestrong{TODO} \sphinxstyleemphasis{aggiungere una sezione su first\sphinxhyphen{}experiments\sphinxhyphen{}revisited, dopo la presentazione dei princìpi dell’elettromagnetismo}

\sphinxAtStartPar
\sphinxstylestrong{todo} Aggiungere sezione su strumenti matematici necessari, per la formulazione di una teoria dei campi

\sphinxAtStartPar
{\hyperref[\detokenize{ch/principles:classical-electromagnetism-principles}]{\sphinxcrossref{\DUrole{std,std-ref}{Princìpi dell’elettromagnetismo}}}} \sphinxstylestrong{TODO} \sphinxstyleemphasis{Trattare prima regime stazionario \sphinxhyphen{} elettricità e magnetismo \sphinxhyphen{} e poi regime non\sphinxhyphen{}stazionario \sphinxhyphen{} elettromagnetismo}**?** \sphinxstylestrong{TODO} \sphinxstyleemphasis{Princìpi. Conservazione della carica, leggi di Maxwell, legge di Lorentz} \sphinxstylestrong{TODO} \sphinxstyleemphasis{Principi in forma integrale; princìpi in forma differenziale \sphinxhyphen{} le leggi di Maxwell}

\sphinxAtStartPar
{\hyperref[\detokenize{ch/energy-linear:classical-electromagnetism-energy}]{\sphinxcrossref{\DUrole{std,std-ref}{Energia}}}}

\sphinxAtStartPar
{\hyperref[\detokenize{ch/waves:classical-electromagnetism-waves}]{\sphinxcrossref{\DUrole{std,std-ref}{Onde elettromagnetiche}}}}

\sphinxAtStartPar
{\hyperref[\detokenize{ch/circuits:classical-electromagnetism-circuits}]{\sphinxcrossref{\DUrole{std,std-ref}{Approssimazione circuitale}}}} \sphinxstylestrong{TODO} \sphinxstyleemphasis{Circuiti elettrici; circuiti elettromagnetici; sistemi elettro\sphinxhyphen{}meccanici. Regimi: stazionario, non\sphinxhyphen{}stazionario: regime transitorio e armonico}

\sphinxAtStartPar
\sphinxstylestrong{Extra.}

\sphinxAtStartPar
{\hyperref[\detokenize{ch/optics:classical-electromagnetism-optics}]{\sphinxcrossref{\DUrole{std,std-ref}{Ottica}}}}

\sphinxAtStartPar
\sphinxstylestrong{Elettromagnetismo e relatività} \sphinxstylestrong{todo} \sphinxstyleemphasis{Relatività a per \(v \ll c\); crisi della relatività galileiana}

\sphinxstepscope


\part{Elettromagnetismo}

\sphinxstepscope




\chapter{Brief history of Electromagnetism}
\label{\detokenize{ch/experiments:brief-history-of-electromagnetism}}\label{\detokenize{ch/experiments:classical-electromagnetism-first-experiments}}\label{\detokenize{ch/experiments::doc}}
\sphinxstepscope




\chapter{Principles of Classical Electromagnetism}
\label{\detokenize{ch/principles:principles-of-classical-electromagnetism}}\label{\detokenize{ch/principles:classical-electromagnetism-principles}}\label{\detokenize{ch/principles::doc}}
\sphinxAtStartPar
The progress in the study of electromagnetic phenomena during the 19th century allowed James Clerk Maxwell to formulate what are now known as \sphinxstyleemphasis{Maxwell’s equations}, which can be considered the first consistent formulation of the principles of classical electromagnetism, together with the charge conservation law and the expression for the Lorentz force on an electric charge immersed in an electromagnetic field.

\sphinxAtStartPar
The principles in differential form can be derived from the more general integral form, provided the fields satisfy the necessary minimal regularity conditions, which can be qualitatively stated as «all operations must make sense.»


\section{Principles in Differential Form}
\label{\detokenize{ch/principles:principles-in-differential-form}}\label{\detokenize{ch/principles:classical-electromagnetism-principles-differential}}
\sphinxAtStartPar
\sphinxstylestrong{Conservation of Electric Charge.}
\begin{equation*}
\begin{split}\partial_t \rho + \nabla \cdot \mathbf{j} = 0 \ .\end{split}
\end{equation*}
\sphinxAtStartPar
\sphinxstylestrong{Maxwell’s Equations.}
\begin{equation*}
\begin{split}\begin{cases}
 \nabla \cdot \mathbf{d} = \rho \\
 \nabla \times \mathbf{e} + \partial_t \mathbf{b} = \mathbf{0} \\ 
 \nabla \cdot \mathbf{b} = 0 \\
 \nabla \times \mathbf{h} - \partial_t \mathbf{d} = \mathbf{j} \\
\end{cases}\end{split}
\end{equation*}
\sphinxAtStartPar
with the need to define constitutive equations \(\mathbf{d}(\mathbf{e}, \mathbf{b})\), \(\mathbf{h}(\mathbf{e}, \mathbf{b})\).

\sphinxAtStartPar
\sphinxstylestrong{Lorentz Force.} The force per unit volume acting on the electric charge present at a point \(\mathbf{r}\) in space is
\begin{equation*}
\begin{split}\begin{aligned}
  \mathbf{f}(\mathbf{r},t) & = \rho(\mathbf{r},t) \, \mathbf{e}(\mathbf{r},t) + \mathbf{j}(\mathbf{r},t) \times \mathbf{b}(\mathbf{r},t) = \\
                           & = \rho(\mathbf{r},t) \left[ \mathbf{e}(\mathbf{r}) + \mathbf{v}(\mathbf{r},t) \times \mathbf{b}(\mathbf{r},t) \right] =  \\
                           & = \rho(\mathbf{r},t) \, \mathbf{e}^*(\mathbf{r},t) 
\end{aligned}\end{split}
\end{equation*}
\sphinxAtStartPar
having defined \(\mathbf{e}^*\) as the electric field \sphinxstylestrong{seen by the moving charge}.


\section{Principles in Integral Form: Electromagnetic Equations and Galilean Relativity}
\label{\detokenize{ch/principles:principles-in-integral-form-electromagnetic-equations-and-galilean-relativity}}\label{\detokenize{ch/principles:classical-electromagnetism-principles-integral}}

\subsection{Integral Form on Control Volumes}
\label{\detokenize{ch/principles:integral-form-on-control-volumes}}\label{\detokenize{ch/principles:classical-electromagnetism-principles-integral-control-volume}}
\sphinxAtStartPar
The integral form of the principles of electromagnetism for fixed volumes \(V\) and surfaces \(S\) in space is obtained by integrating the differential equations over the domains and using the divergence theorem to obtain flux terms, and Stokes” theorem to obtain circulation terms.

\sphinxAtStartPar
\sphinxstylestrong{Continuity of Electric Charge.}
\begin{equation*}
\begin{split}
    \dfrac{d}{dt} \int_{V} \rho + \oint_{\partial V} \mathbf{j} \cdot \hat{\mathbf{n}} = 0
\end{split}
\end{equation*}
\sphinxAtStartPar
\sphinxstylestrong{Gauss’s Law for the Field \(\mathbf{d}(\mathbf{r},t)\).}
\begin{equation*}
\begin{split}
    \oint_{\partial V} \mathbf{d} \cdot \mathbf{\hat{n}} = \int_{V} \rho
\end{split}
\end{equation*}
\sphinxAtStartPar
\sphinxstylestrong{Gauss’s Law for the Field \(\mathbf{b}(\mathbf{r},t)\).}
\begin{equation*}
\begin{split}
    \oint_{\partial V} \mathbf{b} \cdot \mathbf{\hat{n}} = 0
\end{split}
\end{equation*}
\sphinxAtStartPar
\sphinxstylestrong{Faraday–Neumann–Lenz Law for Electromagnetic Induction.}
\begin{equation*}
\begin{split}
    \oint_{\partial S} \mathbf{e} \cdot \hat{\mathbf{t}} + \dfrac{d}{dt} \int_{S} \mathbf{b} \cdot \hat{\mathbf{n}} = \mathbf{0}
\end{split}
\end{equation*}
\sphinxAtStartPar
\sphinxstylestrong{Ampère–Maxwell Law.}
\begin{equation*}
\begin{split}
    \oint_{\partial S} \mathbf{h} \cdot \hat{\mathbf{t}} - \dfrac{d}{dt} \int_{S} \mathbf{d} \cdot \hat{\mathbf{n}} = \int_{S} \mathbf{j} \cdot \hat{\mathbf{n}} \ .
\end{split}
\end{equation*}

\subsection{Integral Form on Arbitrary Volumes}
\label{\detokenize{ch/principles:integral-form-on-arbitrary-volumes}}\label{\detokenize{ch/principles:classical-electromagnetism-principles-integral-arbitrary-volume}}
\sphinxAtStartPar
Due to their importance in fundamental applications such as electric motors, and to avoid confusion or leaps in logic when dealing with electromagnetic induction, it is crucial to provide the correct expression of the electromagnetic principles when moving volumes are involved in space. Not only is the form of these principles shown, but also the correct procedure to derive them starting from the fixed\sphinxhyphen{}control\sphinxhyphen{}volume version. This is done using rules for \sphinxhref{https://basics2022.github.io/bbooks-math-miscellanea/ch/tensor-algebra-calculus/time-derivative-of-integrals.html}{time derivative for fundamental integrals over moving domains}, such as the integral of a density function over a volume, the flux of a vector field through a surface, or the circulation along a curve.

\sphinxAtStartPar
These three derivative rules are listed here and proved in the material about \sphinxhref{https://basics2022.github.io/bbooks-math-miscellanea/intro.html}{Mathematics}:Vector and Tensor Algebra and Calculus:\sphinxhref{https://basics2022.github.io/bbooks-math-miscellanea/ch/tensor-algebra-calculus/time-derivative-of-integrals.html}{Time derivatives of integrals over moving domains}
\begin{equation*}
\begin{split}\dfrac{d}{dt} \int_{v_t} f = \int_{v_t} \dfrac{\partial f}{\partial t} + \oint_{\partial v_t} f \, \mathbf{u}_b \cdot \hat{\mathbf{n}}\end{split}
\end{equation*}\begin{equation*}
\begin{split}\dfrac{d}{dt} \int_{s_t} \mathbf{f} \cdot \hat{\mathbf{n}} = \int_{s_t} \dfrac{\partial \mathbf{f}}{\partial t} \cdot \hat{\mathbf{n}} + \int_{s_t} \nabla \cdot \mathbf{f} \, \mathbf{u}_b \cdot \hat{\mathbf{n}} - \oint_{\partial s_t} \mathbf{u}_b \times \mathbf{f} \cdot \hat{\mathbf{t}}\end{split}
\end{equation*}\begin{equation*}
\begin{split}\dfrac{d}{dt} \int_{\ell_t} \mathbf{f} \cdot \hat{\mathbf{t}} = \int_{\ell_t} \dfrac{\partial \mathbf{f}}{\partial t} \cdot \hat{\mathbf{t}} + \int_{\ell_t} \nabla \times \mathbf{f} \, \cdot \, \mathbf{u}_b \times \hat{\mathbf{t}} + \mathbf{f}_B \cdot \mathbf{u}_B - \mathbf{f}_A \cdot \mathbf{u}_A\end{split}
\end{equation*}
\sphinxAtStartPar
\sphinxstylestrong{Continuity of Electric Charge.}
\begin{equation*}
\begin{split}\begin{aligned}
   0 & = \dfrac{d}{dt} \int_{V} \rho + \oint_{\partial V} \mathbf{j} \cdot \hat{\mathbf{n}} = \\
   & = \dfrac{d}{dt} \int_{v_t} \rho - \oint_{\partial v_t } \rho \mathbf{u}_b \cdot \hat{\mathbf{n}} + \oint_{\partial v_t} \mathbf{j} \cdot \hat{\mathbf{n}} 
\end{aligned}\end{split}
\end{equation*}\begin{equation*}
\begin{split}
    \dfrac{d}{dt} \int_{v_t} \rho + \oint_{\partial v_t} \underbrace{\rho (\mathbf{u} - \mathbf{u}_b)}_{\mathbf{j}^*} \cdot \hat{\mathbf{n}} 
\end{split}
\end{equation*}
\sphinxAtStartPar
\sphinxstylestrong{Gauss’s Law for the Field \(\mathbf{d}(\mathbf{r},t)\).}
\begin{equation*}
\begin{split}
    \oint_{\partial v_t} \mathbf{d} \cdot \mathbf{\hat{n}} = \int_{v_t} \rho
\end{split}
\end{equation*}
\sphinxAtStartPar
\sphinxstylestrong{Gauss’s Law for the Field \(\mathbf{b}(\mathbf{r},t)\).}
\begin{equation*}
\begin{split}
    \oint_{\partial v_t} \mathbf{b} \cdot \mathbf{\hat{n}} = 0
\end{split}
\end{equation*}
\sphinxAtStartPar
\sphinxstylestrong{Faraday–Neumann–Lenz Law for Electromagnetic Induction.}
\begin{equation*}
\begin{split}\begin{aligned}
   \mathbf{0} & = \oint_{\partial S} \mathbf{e} \cdot \hat{\mathbf{t}} + \dfrac{d}{dt} \int_{S} \mathbf{b} \cdot \hat{\mathbf{n}} = \\
    & = \oint_{\partial s_t} \mathbf{e} \cdot \hat{\mathbf{t}} + \dfrac{d}{dt} \int_{s_t} \mathbf{b} \cdot \hat{\mathbf{n}} - \int_{s_t} \underbrace{\nabla \cdot \mathbf{b}}_{=0} \, \mathbf{u}_b \cdot \hat{\mathbf{n}} + \oint_{s_t} \mathbf{u}_b \times \mathbf{b} \cdot \hat{\mathbf{t}} =  \\
\end{aligned}\end{split}
\end{equation*}\begin{equation*}
\begin{split}
    \oint_{\partial s_t} \mathbf{e}^* \cdot \hat{\mathbf{t}} + \dfrac{d}{dt} \int_{s_t} \mathbf{b} \cdot \hat{\mathbf{n}} \ ,
\end{split}
\end{equation*}
\sphinxAtStartPar
with the definition \(\mathbf{e}^* := \mathbf{e} + \mathbf{u}_b \cdot \mathbf{b}\), already used in the expression of the Lorentz force law.

\sphinxAtStartPar
\sphinxstylestrong{Ampère–Maxwell Law.}
\begin{equation*}
\begin{split}\begin{aligned}
    \mathbf{0} & = \oint_{\partial s_t} \mathbf{h} \cdot \hat{\mathbf{t}} - \dfrac{d}{dt} \int_{s_t} \mathbf{d} \cdot \hat{\mathbf{n}} - \int_{s_t} \mathbf{j} \cdot \hat{\mathbf{n}} = \\
    & = \oint_{\partial s_t} \mathbf{h} \cdot \hat{\mathbf{t}} - \dfrac{d}{dt} \int_{s_t} \mathbf{d} \cdot \hat{\mathbf{n}} + \int_{s_t} \underbrace{\nabla \cdot \mathbf{d}}_{=\rho} \, \mathbf{u}_b \cdot \hat{\mathbf{n}} - \oint_{s_t} \mathbf{u}_b \times \mathbf{d} \cdot \hat{\mathbf{t}} - \int_{s_t} \mathbf{j} \cdot \hat{\mathbf{n}} =  \\
\end{aligned}\end{split}
\end{equation*}\begin{equation*}
\begin{split}
    \oint_{\partial s_t} \mathbf{h}^* \cdot \hat{\mathbf{t}} - \dfrac{d}{dt} \int_{s_t} \mathbf{b} \cdot \hat{\mathbf{n}} = \int_{s_t} \mathbf{j}^* \cdot \hat{\mathbf{n}} \ ,
\end{split}
\end{equation*}
\sphinxAtStartPar
having defined \(\mathbf{h}^* := \mathbf{h} - \mathbf{u}_b \times \mathbf{d}\), and using the previously introduced definition \(\mathbf{j}^* := \mathbf{j} - \rho \mathbf{u}_b\).

\sphinxAtStartPar
Adding the definitions:
\begin{equation*}
\begin{split}\rho^* = \rho\end{split}
\end{equation*}\begin{equation*}
\begin{split}\mathbf{d}^* = \mathbf{d}\end{split}
\end{equation*}\begin{equation*}
\begin{split}\mathbf{b}^* = \mathbf{b}\end{split}
\end{equation*}
\sphinxAtStartPar
one obtains equations having the same form as those written for stationary domains in space, but which can be applied to moving domains. The definitions:
\begin{equation*}
\begin{split}\begin{aligned}
\rho^* = \rho \qquad & , \qquad \mathbf{j}^* = \mathbf{j} - \rho \mathbf{u}_b \\
\mathbf{d}^* = \mathbf{d} \qquad & , \qquad \mathbf{e}^* = \mathbf{e} + \mathbf{u}_b \times \mathbf{b} \\
\mathbf{b}^* = \mathbf{b} \qquad & , \qquad \mathbf{h}^* = \mathbf{h} - \mathbf{u}_b \times \mathbf{d} \\
\end{aligned}\end{split}
\end{equation*}
\sphinxAtStartPar
are nothing more than the transformation of the fields for two observers in relative motion, and correspond to the low\sphinxhyphen{}speed limit of Lorentz transformations from special relativity for velocities \(|\mathbf{u}_b| \ll c\): in this procedure, the transformations for low relative speeds are obtained, as no transformation of spatial and temporal dimensions has been considered, unlike Einstein’s theory of relativity.

\sphinxAtStartPar
\sphinxstylestrong{todo} Reference Galilean and Lorentz transformations for relativity in electromagnetism.

\sphinxstepscope

\begin{sphinxuseclass}{sd-container-fluid}
\begin{sphinxuseclass}{sd-sphinx-override}
\begin{sphinxuseclass}{sd-p-0}
\begin{sphinxuseclass}{sd-mt-2}
\begin{sphinxuseclass}{sd-mb-4}
\begin{sphinxuseclass}{sd-row}
\begin{sphinxuseclass}{sd-row-cols-2}
\begin{sphinxuseclass}{sd-gx-2}
\begin{sphinxuseclass}{sd-gy-1}
\begin{sphinxuseclass}{sd-col}
\begin{sphinxuseclass}{sd-d-flex-row}
\begin{sphinxuseclass}{sd-align-minor-center}
\begin{sphinxuseclass}{sd-container-fluid}
\begin{sphinxuseclass}{sd-sphinx-override}
\begin{sphinxuseclass}{sd-row}
\begin{sphinxuseclass}{sd-row-cols-2}
\begin{sphinxuseclass}{sd-row-cols-xs-2}
\begin{sphinxuseclass}{sd-row-cols-sm-3}
\begin{sphinxuseclass}{sd-row-cols-md-3}
\begin{sphinxuseclass}{sd-row-cols-lg-3}
\begin{sphinxuseclass}{sd-gx-3}
\begin{sphinxuseclass}{sd-gy-1}
\begin{sphinxuseclass}{sd-col}
\begin{sphinxuseclass}{sd-col-auto}
\begin{sphinxuseclass}{sd-d-flex-row}
\begin{sphinxuseclass}{sd-align-minor-center}
\sphinxAtStartPar
basics

\end{sphinxuseclass}
\end{sphinxuseclass}
\end{sphinxuseclass}
\end{sphinxuseclass}
\begin{sphinxuseclass}{sd-col}
\begin{sphinxuseclass}{sd-col-auto}
\begin{sphinxuseclass}{sd-d-flex-row}
\begin{sphinxuseclass}{sd-align-minor-center}
\sphinxAtStartPar
26 apr 2025

\end{sphinxuseclass}
\end{sphinxuseclass}
\end{sphinxuseclass}
\end{sphinxuseclass}
\begin{sphinxuseclass}{sd-col}
\begin{sphinxuseclass}{sd-col-auto}
\begin{sphinxuseclass}{sd-d-flex-row}
\begin{sphinxuseclass}{sd-align-minor-center}
\sphinxAtStartPar
1 min read

\end{sphinxuseclass}
\end{sphinxuseclass}
\end{sphinxuseclass}
\end{sphinxuseclass}
\end{sphinxuseclass}
\end{sphinxuseclass}
\end{sphinxuseclass}
\end{sphinxuseclass}
\end{sphinxuseclass}
\end{sphinxuseclass}
\end{sphinxuseclass}
\end{sphinxuseclass}
\end{sphinxuseclass}
\end{sphinxuseclass}
\end{sphinxuseclass}
\end{sphinxuseclass}
\end{sphinxuseclass}
\end{sphinxuseclass}
\end{sphinxuseclass}
\end{sphinxuseclass}
\end{sphinxuseclass}
\end{sphinxuseclass}
\end{sphinxuseclass}
\end{sphinxuseclass}
\end{sphinxuseclass}
\end{sphinxuseclass}

\chapter{Potenziali elettromagnetici}
\label{\detokenize{ch/potentials:potenziali-elettromagnetici}}\label{\detokenize{ch/potentials:classical-electromagnetism-potentials}}\label{\detokenize{ch/potentials::doc}}
\sphinxAtStartPar
E” possibile dimostrare che il sistema di equazioni di Maxwell e dell’equazione del bilancio della carica elettrica è un sistema sovra\sphinxhyphen{}determinato.
In particolare, è possibile dimostrare che, nota la distribuzione di carica e di densità di corrente \sphinxhyphen{} considerate come cause generanti il campo elettrico \sphinxhyphen{}, date le leggi costitutive del materiale, sono sufficienti 4 incognite per definire le 6 incognite (3 componenti, per due campi vettoriali) del problema.
E” possibile formulare quindi il problema in termini di un potenziale scalare \(\varphi\) e un potenziale vettore \(\mathbf{a}\) per ottenere, insieme a una condizione di gauge che elimini le due arbitrarietà (irrilevanti ai fini del calcolo dei campi fisici) restanti.


\section{Potenziale vettore e potenziale scalare}
\label{\detokenize{ch/potentials:potenziale-vettore-e-potenziale-scalare}}
\sphinxAtStartPar
Partendo dalle equazioni di Maxwell si possono definire i potenziali del campo elettromagnetico. Usando l’equazione di Gauss per il campo magnetico si può introdurre il potenziale vettore \(\mathbf{a}(\mathbf{r},t)\),
\begin{equation*}
\begin{split}0 = \nabla \cdot \mathbf{b} \qquad \rightarrow \qquad \mathbf{b} = \nabla \times \mathbf{a} \ ,\end{split}
\end{equation*}
\sphinxAtStartPar
poiché la divergenza di un rotore è identicamente nulla. Introducendo questa relazione nell’equazione di Faraday\sphinxhyphen{}Newumann\sphinxhyphen{}Lenz, nell’ipotesi di sufficiente regolarità dei campi che consenta di invertire l’ordine delle derivate,
\begin{equation*}
\begin{split}0 = \nabla \times \mathbf{e} + \partial_t \mathbf{b} = \nabla \times \mathbf{e} +  \partial_t \nabla \times \mathbf{a} = \nabla \times (\mathbf{e} + \partial_t \mathbf{a}) \qquad \rightarrow \qquad \mathbf{e} + \partial_t \mathbf{a} = - \nabla \varphi \ ,\end{split}
\end{equation*}
\sphinxAtStartPar
poichè il rotore di un gradiente è identicamente nulla. Le grandezze «fisiche» campo elettrico \(\mathbf{e}(\mathbf{r},t)\) e campo magnetico \(\mathbf{b}(mathbf{r},t)\) possono quindi essere scritte usando i pootenziali elettromagnetici come
\begin{equation*}
\begin{split}\begin{cases}
 \mathbf{e} & = - \nabla \varphi - \partial_t \mathbf{a} \\
 \mathbf{b} & = \nabla \times \mathbf{a} \\
\end{cases}\end{split}
\end{equation*}

\section{Condizioni di gauge}
\label{\detokenize{ch/potentials:condizioni-di-gauge}}
\sphinxAtStartPar
I potenziali sono definiti a meno di una condizione di gauge, un’ulteriore condzione che elimina ogni arbitrarietà nella definizione.
Ad esempio, il potenziale vettore è definito a meno del gradiente di una funzione scalare, poiché \(\nabla \times \nabla f \equiv \mathbf{0}\), e quindi il potenziale \(\tilde{\mathbf{a}} = \mathbf{a} + \nabla f\) produce lo stesso campo magnetico \(\mathbf{b}\)
\begin{equation*}
\begin{split}\nabla \times \tilde{\mathbf{a}} = \nabla \times (\mathbf{a} + \nabla f) = \nabla \times \mathbf{a} \ .\end{split}
\end{equation*}
\sphinxAtStartPar
\sphinxstylestrong{Condizione di gauge di Lorentz.} Per motivi che saranno più evidenti nella sezione sulle {\hyperref[\detokenize{ch/waves:classical-electromagnetism-waves}]{\sphinxcrossref{\DUrole{std,std-ref}{onde elettromagnetiche}}}}, una condizione di gauge conveniente è
\begin{equation*}
\begin{split}\nabla \cdot \mathbf{a} + \frac{1}{c^2} \partial_t \varphi = 0\end{split}
\end{equation*}
\sphinxAtStartPar
\sphinxstylestrong{Condizione di gauge di Coulomb.}
\begin{equation*}
\begin{split}\nabla \cdot \mathbf{a} = 0\end{split}
\end{equation*}
\sphinxstepscope




\chapter{Elettromagnetismo nella materia}
\label{\detokenize{ch/media:elettromagnetismo-nella-materia}}\label{\detokenize{ch/media:classical-electromagnetism-media}}\label{\detokenize{ch/media::doc}}
\sphinxAtStartPar
\sphinxstylestrong{todo}


\section{Vuoto}
\label{\detokenize{ch/media:vuoto}}
\sphinxAtStartPar
I fenomeni elettromagnetici nel vuoto sono governati dalle equazioni di Maxwell nel vuoto,
\begin{equation*}
\begin{split}\begin{cases}
\nabla \cdot \mathbf{e} = \frac{\rho}{\varepsilon_0} \\
\nabla \times \mathbf{e} + \partial_t \mathbf{b} = \mathbf{0} \\
\nabla \cdot \mathbf{b} = 0 \\
\nabla \times \mathbf{b} - \mu_0 \varepsilon_0 \partial_t \mathbf{e} = \mu_0 \mathbf{j}
\end{cases}\end{split}
\end{equation*}
\sphinxAtStartPar
e dall’equazione della continuità della carica elettrica,
\begin{equation*}
\begin{split}\partial_t \rho + \nabla \cdot \mathbf{j} = 0 \ .\end{split}
\end{equation*}

\section{Mezzi continui}
\label{\detokenize{ch/media:mezzi-continui}}
\sphinxAtStartPar
In generale, alcuni materiali rispondono a un campo elettromagnetico «esterno» imposto, con una polarizzazione e una magnetizzazione. In particolare, la polarizzazione elettrica di un materiale corrisponde a una separazione locale delle cariche elettriche dal punto di vista macroscopico equivalente a una densità di volume di dipoli, \(\mathbf{p}(\mathbf{r}_0)\); la magnetizzazione corrisponde a un orientamento degli assi delle spire delle correnti amperiane dal punto di vista macroscopico equivalente a una densità di momento magnetico \(\mathbf{m}(\mathbf{r}_0)\).


\section{Polarizzazione}
\label{\detokenize{ch/media:polarizzazione}}

\subsection{Singolo dipolo elettrico}
\label{\detokenize{ch/media:singolo-dipolo-elettrico}}
\sphinxAtStartPar
Un dipolo elettrico discreto è formato da due cariche elettriche uguali e opposte \(q\), \(-q\), nei punti \(P_+\), \(P_- = P_+ \mathbf{l}\), nel limite \(q \rightarrow +\infty\), \(|\mathbf{l}| \rightarrow 0\) con \(q |\mathbf{e}|\) finito.

\sphinxAtStartPar
Il campo elettrico (stazioario \sphinxstylestrong{todo} \sphinxstyleemphasis{controllare cosa succede nel caso non stazionario. Magari dopo aver derivato la soluzione generale del problema, come soluzione delle equazioni delle onde  in termini dei potenziali EM}) generato nel punto dello spazio \(\mathbf{r}\) da un dipolo elettrico nel punto \(\mathbf{r}_0\) viene calcolato come limite del campo elettrico generato da due cariche uguali e opposte \(q^{\mp}\) nei punti \(\mathbf{r}_0 \mp \frac{\mathbf{l}}{2}\),
\begin{equation*}
\begin{split}\mathbf{e}(\mathbf{r}) = -\frac{q}{4 \pi \varepsilon_0} \frac{\mathbf{r} - \left( \mathbf{r}_0 - \frac{\mathbf{l}}{2} \right)}{\left|\mathbf{r} - \left( \mathbf{r}_0 - \frac{\mathbf{l}}{2} \right)\right|^3} + \frac{q}{4 \pi \varepsilon_0} \frac{\mathbf{r} - \left( \mathbf{r}_0 + \frac{\mathbf{l}}{2} \right)}{\left|\mathbf{r} - \left( \mathbf{r}_0 + \frac{\mathbf{l}}{2} \right)\right|^3} \ .\end{split}
\end{equation*}
\sphinxAtStartPar
Usando la formula per la derivata dei termini
\begin{equation*}
\begin{split}\partial_{\ell_k} \frac{x_i \pm \frac{\ell_i}{2}}{\left|\mathbf{x} \pm \frac{\mathbf{l}}{2} \right|^3} = \frac{1}{2} \left[ \pm \frac{\delta_{ik}}{r^3} - 3 r^{-4} \left( \pm \frac{x_k \pm \frac{\ell_k}{2}}{r} \right) \right]\end{split}
\end{equation*}\begin{equation*}
\begin{split}\left. \partial_{\ell_k} \frac{x_i \pm \frac{\ell_i}{2}}{\left|\mathbf{x} \pm \frac{\mathbf{l}}{2} \right|^3} \right|_{\mathbf{l} = \mathbf{0}} = \mp \frac{1}{2} \left[ - \frac{\delta_{ik}}{|\mathbf{x}|^3} + 3 \left( \frac{x_k}{r^5} \right) \right] = \mp \frac{1}{2} \partial_{r_{0 k}} \frac{r_i - r_{0 i}}{|\mathbf{r} - \mathbf{r}_0|^3} = \mp \frac{1}{2} \nabla_{\mathbf{r}_0} \frac{\mathbf{r} - \mathbf{r}_0}{|\mathbf{r} - \mathbf{r}_0|^3} \end{split}
\end{equation*}
\sphinxAtStartPar
si ricava l’apporssimazione al primo ordine in \(\mathbf{l}\) dei due termini
\begin{equation*}
\begin{split}\begin{aligned}
  \frac{\mathbf{r} - \left( \mathbf{r}_0 \mp \frac{\mathbf{l}}{2} \right)}{\left|\mathbf{r} - \left( \mathbf{r}_0 \mp \frac{\mathbf{l}}{2} \right)\right|^3} 
  & = \frac{\mathbf{r} - \mathbf{r}_0 }{\left|\mathbf{r} - \mathbf{r}_0 \right|^3} \pm \mathbf{l} \cdot \frac{1}{2} \nabla_{\mathbf{r}_0} \left( \frac{\mathbf{r} - \mathbf{r}_0}{|\mathbf{r} - \mathbf{r}_0|^3} \right) + o(|\mathbf{l}|)\\
\end{aligned}\end{split}
\end{equation*}
\sphinxAtStartPar
e, definendo l’intensità del dipolo \(\mathbf{P}_0 := q \mathbf{l}\) e facendo tendere le grandezze al limite desiderato, quella del campo elettrico
\begin{equation*}
\begin{split}\begin{aligned}
  \mathbf{e}(\mathbf{r})
  & = - \frac{1}{4\pi \varepsilon_0} \, \mathbf{P}_0 \cdot \nabla_{\mathbf{r}_0}  \left( \frac{\mathbf{r} - \mathbf{r}_0}{|\mathbf{r} - \mathbf{r}_0|^3} \right)   = \\
  & = - \frac{1}{4\pi \varepsilon_0} \left[ \frac{(\mathbf{r}-\mathbf{r}_0)(\mathbf{r}-\mathbf{r}_0)}{|\mathbf{r}-\mathbf{r}_0|^5} \cdot \mathbf{P}_0 - \frac{\mathbf{P}_0}{|\mathbf{r}-\mathbf{r}_0|^3} \right] = \\
  & = - \frac{1}{4\pi \varepsilon_0} \left[ \frac{(\mathbf{r}-\mathbf{r}_0) \otimes (\mathbf{r}-\mathbf{r}_0)}{|\mathbf{r}-\mathbf{r}_0|^5} - \frac{\mathbb{I}}{|\mathbf{r}-\mathbf{r}_0|^3} \right] \cdot \mathbf{P}_0 \ .
\end{aligned}\end{split}
\end{equation*}
\sphinxAtStartPar
\sphinxstylestrong{todo} nel caso generale sarebbe necessario prestare attenzione all’ordine dei fattori nel prodotto tra vettori e tensori, ma in questo caso si può sfruttare la simmetria del tensore del secondo ordine (o delle operazioni).


\subsection{Distribuzione continua di dipoli}
\label{\detokenize{ch/media:distribuzione-continua-di-dipoli}}
\sphinxAtStartPar
Una distribuzione di dipoli con densità di volume \(\mathbf{p}(\mathbf{r_0})\), che produce il dipolo elementare \(\Delta \mathbf{P}(\mathbf{r}_0) = \mathbf{p}(\mathbf{r}_0) dV_0\) nel volume \(d V_0\), produce il campo elettrico
\begin{equation*}
\begin{split}\mathbf{e}(\mathbf{r}) = \int_{\mathbf{r}_0 \in V_0} \frac{1}{4 \pi \varepsilon_0} \mathbf{p}(\mathbf{r}_0) \cdot \nabla_{\mathbf{r}_0}  \left( \frac{\mathbf{r} - \mathbf{r}_0}{|\mathbf{r} - \mathbf{r}_0|^3} \right) \ , \end{split}
\end{equation*}
\sphinxAtStartPar
la cui espressione può essere riscritta usando le regole di integrazione per parti
\begin{equation*}
\begin{split}\begin{aligned}
\mathbf{e}(\mathbf{r})
  & = \int_{\mathbf{r}_0 \in V_0} \frac{1}{4 \pi \varepsilon_0} \mathbf{p}(\mathbf{r}_0) \cdot \nabla_{\mathbf{r}_0}  \left( \frac{\mathbf{r} - \mathbf{r}_0}{|\mathbf{r} - \mathbf{r}_0|^3} \right) = \\
  & = \oint_{\mathbf{r}_0 \in \partial V_0} \frac{1}{4 \pi \varepsilon_0}  \frac{\mathbf{r} - \mathbf{r}_0}{|\mathbf{r} - \mathbf{r}_0|^3} \underbrace{ \hat{\mathbf{n}}(\mathbf{r}_0) \cdot \mathbf{p}(\mathbf{r}_0) }_{ =: \sigma_P(\mathbf{r}_0)}  + \int_{\mathbf{r}_0 \in V_0} \frac{1}{4 \pi \varepsilon_0} \frac{\mathbf{r} - \mathbf{r}_0}{|\mathbf{r} - \mathbf{r}_0|^3} \underbrace{ \left( - \nabla_{\mathbf{r}_0} \cdot \mathbf{p}(\mathbf{r}_0) \right)}_{ =: \rho_P(\mathbf{r}_0) } \ , \\
\end{aligned}\end{split}
\end{equation*}
\sphinxAtStartPar
avendo definito le densità di carica di polarizzazione superficiale \(\sigma_P\) e di volume \(\rho_P\) come le intensità delle sorgenti distribuite di campo elettrico, in analogia con l’espressione della legge di Coulomb.


\subsection{Riformulazione delle equazioni di Maxwell e della continuità della carica}
\label{\detokenize{ch/media:riformulazione-delle-equazioni-di-maxwell-e-della-continuita-della-carica}}
\sphinxAtStartPar
L’equazione di Gauss determina la densità di flusso nel volume del campo elettrico \(\mathbf{e}\),
\begin{equation*}
\begin{split}\nabla \cdot \mathbf{e} = \frac{\rho}{\varepsilon_0} \ .\end{split}
\end{equation*}
\sphinxAtStartPar
Scomponendo la densità di carica come somma delle \sphinxstylestrong{cariche libere} \(\rho_f\) e delle \sphinxstylestrong{cariche di polarizzazione} \(\rho_P := - \nabla \cdot \mathbf{p}\), si può rielaborare l’equazione di Gauss,
\begin{equation*}
\begin{split}\begin{aligned}
 & \nabla \cdot \mathbf{e} = \frac{\rho_f + \rho_P}{\varepsilon_0} \\
 & \nabla \cdot \left( \varepsilon_0 \mathbf{e} + \mathbf{p} \right) = \rho_f \\ \\
 & \nabla \cdot \mathbf{d} = \rho_f \ ,
\end{aligned}\end{split}
\end{equation*}
\sphinxAtStartPar
avendo introdotto il \sphinxstylestrong{campo di spostamento}, \(\mathbf{d} := \varepsilon_0 \mathbf{e} + \mathbf{p}\).

\sphinxAtStartPar
La scomposizione della corrente elettrica come somma \(\mathbf{j} = \mathbf{j}_f + \mathbf{j}_P\) della corrente libera \(\mathbf{j}_f\) e corrente di polarizzazione \(\mathbf{j}_P\), permette di rielaborare l’equazione della continuità della carica elettrica
\begin{equation*}
\begin{split}\begin{aligned}
  0 & = \partial_t \rho + \nabla \cdot \mathbf{j} = \\
    & = \partial_t (\rho_f + \rho_P) + \nabla \cdot \left( \mathbf{j}_f + \mathbf{j}_P \right) = \\
    & = \partial_t \rho_f + \nabla \cdot  \mathbf{j}_f + \partial_t \rho_P + \nabla \cdot \mathbf{j}_P \ ,
\end{aligned}\end{split}
\end{equation*}
\sphinxAtStartPar
e scrivere le equazioni di continuità per le due distribuzioni di carica (di natura diversa, si suppone che entrambe devono soddisfare la continuità della carica in maniera indipendente, se le cariche libere rimangono libere e le cariche di polarizzazione rimangono di polarizzazione),
\begin{equation*}
\begin{split}\begin{aligned}
  & \partial_t \rho_f + \nabla \cdot \mathbf{j}_f = 0 \\
  & \partial_t \rho_P + \nabla \cdot \mathbf{j}_P = 0 \qquad \rightarrow \qquad 0 = \nabla \cdot (-\partial_t \mathbf{p} + \mathbf{j}_P) \qquad \rightarrow \qquad \mathbf{j}_P = \partial_t \mathbf{p}
\end{aligned}\end{split}
\end{equation*}
\sphinxAtStartPar
\sphinxstylestrong{todo} \sphinxstyleemphasis{giustificare assenza di campo costante}


\section{Magnetizzazione}
\label{\detokenize{ch/media:magnetizzazione}}

\subsection{Singolo momento magnetico (limite di una spira elementare)}
\label{\detokenize{ch/media:singolo-momento-magnetico-limite-di-una-spira-elementare}}
\sphinxAtStartPar
Usando la legge di Biot\sphinxhyphen{}Savart, specializzato a un conduttore percorso da corrente \(i(\mathbf{r}_0)\)
\begin{equation*}
\begin{split}\begin{aligned}
  d \mathbf{b}(\mathbf{r})
  & = - \frac{\mu}{4 \pi} \frac{\mathbf{r} - \mathbf{r}_0}{|\mathbf{r} - \mathbf{r}_0|^3} \times \mathbf{j}(\mathbf{r}_0) d V_0 = \\
  & = - \frac{\mu}{4 \pi} i(\mathbf{r}_0) \frac{\mathbf{r} - \mathbf{r}_0}{|\mathbf{r} - \mathbf{r}_0|^3} \times \hat{\mathbf{t}}(\mathbf{r}_0) d \ell_0 \ ,
\end{aligned}\end{split}
\end{equation*}
\sphinxAtStartPar
si può calcolare il campo magnetico generato da una spira con percorso \(\ell_0 = \partial S_0\) sfruttando il PSCE
\begin{equation*}
\begin{split}\begin{aligned}
  \mathbf{b}(\mathbf{r})
  & = \oint_{\ell_0} d \mathbf{b}(\mathbf{r}_0) = \\
  & = - \frac{\mu}{4 \pi} i_0 \oint_{\mathbf{r}_0 \in \ell_0} \frac{\mathbf{r} - \mathbf{r}_0}{|\mathbf{r} - \mathbf{r}_0|^3} \times \hat{\mathbf{t}}(\mathbf{r}_0)  = \\
  & =   \frac{\mu}{4 \pi} i_0 \int_{\mathbf{r}_0 \in S_0} \hat{\mathbf{n}}(\mathbf{r}_0) \cdot \nabla_{\mathbf{r}_0} \left( \frac{\mathbf{r} - \mathbf{r}_0}{|\mathbf{r} - \mathbf{r}_0|^3} \right)
\end{aligned}\end{split}
\end{equation*}
\sphinxAtStartPar
Il campo generato da spira elementare di superficie \(S_0\) con normale \(\hat{\mathbf{n}}_0\), usando il teorema della media, è
\begin{equation*}
\begin{split}\mathbf{b}(\mathbf{r}) = \frac{\mu}{4 \pi} i_0 S_0 \hat{\mathbf{n}}_0 \cdot \nabla_{\mathbf{r}_0} \left( \frac{\mathbf{r} - \mathbf{r}_0}{|\mathbf{r} - \mathbf{r}_0|^3} \right) + o(S_0)\end{split}
\end{equation*}
\sphinxAtStartPar
e al tendere di \(i_0 \rightarrow \infty\), \(S_0 \rightarrow 0\) in modo tale da avere \(\mathbf{M}_0 := i_0 S_0 \hat{\mathbf{n}}_0\)
\begin{equation*}
\begin{split}\begin{aligned}
 \mathbf{b}(\mathbf{r}) 
  & = \frac{\mu}{4 \pi} \mathbf{M}_0 \cdot \nabla_{\mathbf{r}_0} \left( \frac{\mathbf{r} - \mathbf{r}_0}{|\mathbf{r} - \mathbf{r}_0|^3} \right) \\
  & = - \frac{\mu_0}{4\pi} \left[ \frac{(\mathbf{r}-\mathbf{r}_0)(\mathbf{r}-\mathbf{r}_0)}{|\mathbf{r}-\mathbf{r}_0|^5} \cdot \mathbf{M}_0 - \frac{\mathbf{M}_0}{|\mathbf{r}-\mathbf{r}_0|^3} \right] = \\
  & = - \frac{\mu_0}{4\pi} \left[ \frac{(\mathbf{r}-\mathbf{r}_0) \otimes (\mathbf{r}-\mathbf{r}_0)}{|\mathbf{r}-\mathbf{r}_0|^5} - \frac{\mathbb{I}}{|\mathbf{r}-\mathbf{r}_0|^3} \right] \cdot \mathbf{M}_0 \ .
\end{aligned}\end{split}
\end{equation*}
\sphinxAtStartPar
\sphinxstylestrong{todo} Analogia con il campo elettrico prodotto da una distribuzione di dipoli.
\subsubsection*{Dettagli}
\begin{equation*}
\begin{split}\oint_{\partial S} A \, t_i = \int_S \varepsilon_{ijk} \, n_j \, \partial_k A
\qquad , \qquad 
  \oint_{\partial S} A \, \hat{\mathbf{t}} = \int_S \hat{\mathbf{n}} \times \nabla A\end{split}
\end{equation*}\begin{equation*}
\begin{split}\begin{aligned}
\oint_{\mathbf{r}_0 \in \ell_0} \frac{\mathbf{r} - \mathbf{r}_0}{|\mathbf{r} - \mathbf{r}_0|^3} \times \hat{\mathbf{t}}(\mathbf{r}_0) d \ell_0 
  & = \oint_{\mathbf{r}_0 \in \ell_0} \varepsilon_{ijk} \frac{r_j - r_{0,j}}{|\mathbf{r} - \mathbf{r}_0|^3} \ t_k = \\
  & = \int_{\mathbf{r}_0 \in S_0} \varepsilon_{krs} n_r \partial^0_s \left( \varepsilon_{ijk} \frac{r_j - r_{0,j}}{|\mathbf{r} - \mathbf{r}_0|^3} \right) = \\
  & = \int_{\mathbf{r}_0 \in S_0} \left( \delta_{ir} \delta_{js} - \delta_{is} \delta_{jr} \right) n_r \partial^0_s \left( \frac{r_j - r_{0,j}}{|\mathbf{r} - \mathbf{r}_0|^3} \right) = \\
  & = \int_{\mathbf{r}_0 \in S_0} \left\{ n_i \underbrace{\partial^0_j \left( \frac{r_j - r_{0,j}}{|\mathbf{r} - \mathbf{r}_0|^3} \right)}_{=0} - n_j \partial^0_i \left( \frac{r_j - r_{0,j}}{|\mathbf{r} - \mathbf{r}_0|^3} \right) \right\} = \\
  & = - \int_{\mathbf{r}_0 \in S_0} n_j \partial^0_i \left( \frac{r_j - r_{0,j}}{|\mathbf{r} - \mathbf{r}_0|^3} \right) \ .
\end{aligned}\end{split}
\end{equation*}

\subsection{Distribuzione continua di momento magnetico}
\label{\detokenize{ch/media:distribuzione-continua-di-momento-magnetico}}
\sphinxAtStartPar
Per calcolare il campo magnetico generato da una distribuzione di volume di momento magnetico si può procedere in analogia con quanto fatto per calcolare il campo elettrico generato da una distribuzione di dipoli
\begin{equation*}
\begin{split}\begin{aligned}
\mathbf{b}(\mathbf{r})
  & = \int_{\mathbf{r}_0 \in V_0} \frac{\mu_0}{4 \pi } \mathbf{m}(\mathbf{r}_0) \cdot \nabla_{\mathbf{r}_0}  \left( \frac{\mathbf{r} - \mathbf{r}_0}{|\mathbf{r} - \mathbf{r}_0|^3} \right) = \\
  & = \oint_{\mathbf{r}_0 \in \partial V_0} \frac{\mu_0}{4 \pi}  \frac{\mathbf{r} - \mathbf{r}_0}{|\mathbf{r} - \mathbf{r}_0|^3} \hat{\mathbf{n}}(\mathbf{r}_0) \cdot \mathbf{m}(\mathbf{r}_0) + \int_{\mathbf{r}_0 \in V_0} \frac{\mu_0}{4 \pi} \frac{\mathbf{r} - \mathbf{r}_0}{|\mathbf{r} - \mathbf{r}_0|^3} \,\left( - \nabla_{\mathbf{r}_0} \cdot \mathbf{m}(\mathbf{r}_0) \right) \ , \\
\end{aligned}\end{split}
\end{equation*}
\sphinxAtStartPar
ma senza ottenere un’analogia con l’espressione della legge di Biot\sphinxhyphen{}Savart che prevede il prodotto vettore tra il termine \(\frac{\mathbf{r}- \mathbf{r}_0}{|\mathbf{r} - \mathbf{r}_0|^3}\) con una densità di corrente \(\mathbf{j}(\mathbf{r}_0)\).
\subsubsection*{Dettagli}

\sphinxAtStartPar
Si può riscrivere
\begin{equation*}
\begin{split}\begin{aligned}
  \oint_{\mathbf{r}_0 \in \partial V_0} & \frac{\mathbf{r} - \mathbf{r}_0}{|\mathbf{r} - \mathbf{r}_0|^3} \times \left( \hat{\mathbf{n}}(\mathbf{r}_0) \times \mathbf{m}(\mathbf{r}_0) \right) \\
  & = \oint_{\mathbf{r}_0 \in \partial V_0} \varepsilon_{ijk} \frac{r_j - r_{0,j}}{|\mathbf{r}-\mathbf{r}_0|^3} \varepsilon_{krs} n_r m_s = \\
  & = \int_{\mathbf{r}_0 \in V_0} \left( \delta_{ir} \delta_{js} - \delta_{is} \delta_{jr} \right) \partial^0_r \left( \frac{r_j - r_{0,j}}{|\mathbf{r}-\mathbf{r}_0|^3} m_s \right) = \\
  & = \int_{\mathbf{r}_0 \in V_0} \left\{ \partial^0_i \left( \frac{r_j - r_{0,j}}{|\mathbf{r}-\mathbf{r}_0|^3} m_j \right) - \partial^0_j \left( \frac{r_j - r_{0,j}}{|\mathbf{r}-\mathbf{r}_0|^3} m_i \right)  \right\} = \\
  & = \int_{\mathbf{r}_0 \in V_0}
  \left\{ \partial^0_i \frac{r_j - r_{0,j}}{|\mathbf{r}-\mathbf{r}_0|^3} m_j 
         + \frac{r_j - r_{0,j}}{|\mathbf{r}-\mathbf{r}_0|^3} \partial^0_i m_j
         - \frac{r_j - r_{0,j}}{|\mathbf{r}-\mathbf{r}_0|^3} \partial^0_j m_i 
         - \underbrace{ \partial^0_j \frac{r_j - r_{0,j}}{|\mathbf{r}-\mathbf{r}_0|^3}}_{=0} m_i 
  \right\} = \\
  & = \int_{\mathbf{r}_0 \in V_0}
  \left\{ \partial^0_i \frac{r_j - r_{0,j}}{|\mathbf{r}-\mathbf{r}_0|^3} m_j 
         + \varepsilon_{ijk} \varepsilon_{krs} \frac{r_j - r_{0,j}}{|\mathbf{r}-\mathbf{r}_0|^3} \partial^0_r m_s
  \right\} = \\
  & = \int_{\mathbf{r}_0 \in V_0}
  \left\{ \nabla_{\mathbf{r}_0} \frac{\mathbf{r} - \mathbf{r}_0}{|\mathbf{r}-\mathbf{r}_0|^3} \cdot \mathbf{m}(\mathbf{r}_0) 
         + \frac{\mathbf{r} - \mathbf{r}_0}{|\mathbf{r}-\mathbf{r}_0|^3} \times \left( \nabla_{\mathbf{r}_0} \times \mathbf{m}(\mathbf{r}_0) \right)
  \right\} = \\
\end{aligned}\end{split}
\end{equation*}
\sphinxAtStartPar
usando le identità del calcolo vettoriale,
\begin{equation*}
\begin{split}\begin{aligned}
  \mathbf{a} \times (\mathbf{b} \times \mathbf{c}) & = \varepsilon_{ijk} a_j \varepsilon_{krs} b_r c_s = \\
  & = (\delta_{ir} \delta_{js} - \delta_{is} \delta_{jr}) a_j b_r c_s = \\
  & = a_j b_i c_j - c_i b_j a_j = \mathbf{b}(\mathbf{a} \cdot \mathbf{c}) - \mathbf{c} (\mathbf{a} \cdot \mathbf{b})
\end{aligned}\end{split}
\end{equation*}\begin{equation*}
\begin{split}\begin{aligned}
 a_j \partial_i m_j - a_j \partial_j m_i
 & = (\delta_{ir} \delta_{js} - \delta_{is} \delta_{jr}) a_j \partial_r m_s = \\
 & = \varepsilon_{ijk} \varepsilon_{krs} a_j \partial_r m_s = \\
 & = \mathbf{a} \times \left( \nabla \times \mathbf{m} \right)
\end{aligned}\end{split}
\end{equation*}
\sphinxAtStartPar
Il campo magnetico generato da una distribuzione di momento magnetico può quindi essere riscritto come
\begin{equation*}
\begin{split}\begin{aligned}
\mathbf{b}(\mathbf{r})
  & = \int_{\mathbf{r}_0 \in V_0} \frac{\mu_0}{4 \pi } \mathbf{m}(\mathbf{r}_0) \cdot \nabla_{\mathbf{r}_0}  \left( \frac{\mathbf{r} - \mathbf{r}_0}{|\mathbf{r} - \mathbf{r}_0|^3} \right) = \\
  & = - \frac{\mu_0}{4\pi} \oint_{\mathbf{r}_0 \in \partial V_0} \frac{\mathbf{r} - \mathbf{r}_0}{|\mathbf{r} - \mathbf{r}_0|^3} \times \underbrace{ \left( - \hat{\mathbf{n}}(\mathbf{r}_0) \times \mathbf{m}(\mathbf{r}_0) \right) }_{\mathbf{j}^s_M}
  - \frac{\mu_0}{4 \pi} \int_{\mathbf{r}_0 \in V_0} \frac{\mathbf{r} - \mathbf{r}_0}{|\mathbf{r}-\mathbf{r}_0|^3} \times \underbrace{ \left(\nabla_{\mathbf{r}_0} \times \mathbf{m}(\mathbf{r}_0) \right)}_{\mathbf{j}_M} \ ,
\end{aligned}\end{split}
\end{equation*}
\sphinxAtStartPar
avendo definito le densità di corrente di magnetizzazione superficiale \(\mathbf{j}^s_M\) e di colume \(\mathbf{j}_M\) come le intensità delle singolarità distribuite, in analogia con l’espressione della legge di Biot\sphinxhyphen{}Savart.


\subsection{Riformulazione delle equazioni di Maxwell e della continuità della carica}
\label{\detokenize{ch/media:id1}}
\sphinxAtStartPar
La legge di Ampére\sphinxhyphen{}Maxwell può essere riscritta
\begin{equation*}
\begin{split}\begin{aligned}
 & \nabla \times \mathbf{b} - \mu_0 \varepsilon_0 \partial_t \mathbf{e} = \mu_0 \mathbf{j} \\
 & \nabla \times \mathbf{b} - \mu_0 \partial_t \left( \mathbf{d} - \mathbf{p} \right) = \mu_0 \left( \mathbf{j}_f + \mathbf{j}_P + \mathbf{j}_M \right) \\
 & \nabla \times \underbrace{\left( \mathbf{b} - \mu_0 \mathbf{m} \right)}_{=: \mu_0 \mathbf{h}} - \mu_0 \partial_t \mathbf{d} + \mu_0 \underbrace{\left( \partial_t \mathbf{p} - \mathbf{j}_P \right)}_{= \mathbf{0}} = \mu_0 \mathbf{j}_f  \\ \\
 & \nabla \times \mathbf{h} - \partial_t \mathbf{d} = \mathbf{j}_f
\end{aligned}\end{split}
\end{equation*}
\sphinxAtStartPar
Dalla legge di continuità della corrente elettrica,
\begin{equation*}
\begin{split}\partial_t \rho + \nabla \cdot \mathbf{j} = 0 \ ,\end{split}
\end{equation*}
\sphinxAtStartPar
si ricava l’equazione di continuità per le cariche di magnetizzazione
\begin{equation*}
\begin{split}\begin{aligned}
  0 & = \partial_t \rho_M + \nabla \cdot \mathbf{j}_M = \\
    & = \partial_t \rho_M + \underbrace{ \nabla \cdot \nabla \times \mathbf{m}}_{ \equiv \mathbf{0} } \ .
\end{aligned}\end{split}
\end{equation*}

\section{Esempi}
\label{\detokenize{ch/media:esempi}}\begin{itemize}
\item {} 
\sphinxAtStartPar
conduttori

\item {} 
\sphinxAtStartPar
ferromagentici e magnetismo debole (para\sphinxhyphen{}, dia\sphinxhyphen{}, anti\sphinxhyphen{})

\end{itemize}


\section{Jump conditions}
\label{\detokenize{ch/media:jump-conditions}}\label{\detokenize{ch/media:classical-electromagnetism-media-jump}}
\sphinxAtStartPar
Differential form of Maxwell’s equations
\begin{equation*}
\begin{split}\begin{cases}
 \nabla \cdot \mathbf{d} = \rho_f \\
 \nabla \times \mathbf{e} + \partial_t \mathbf{b} = \mathbf{0} \\
 \nabla \cdot \mathbf{b} = 0 \\
 \nabla \times \mathbf{h} - \partial_t \mathbf{d} = \mathbf{j}_f
\end{cases}\end{split}
\end{equation*}
\sphinxAtStartPar
Integral form of Maxwell’s equations
\begin{equation*}
\begin{split}\begin{cases}
 \displaystyle \oint_{\partial V} \mathbf{d} \cdot \hat{\mathbf{n}} = \int_{V} \rho_f \\
 \displaystyle \oint_{\partial S} \mathbf{e} \cdot \hat{\mathbf{t}} + \dfrac{d}{dt} \int_S \mathbf{b} \cdot \hat{\mathbf{n}} = 0 \\
 \displaystyle \oint_{\partial V} \mathbf{b} \cdot \hat{\mathbf{n}} = 0 \\
 \displaystyle \oint_{\partial S} \mathbf{h} \cdot \hat{\mathbf{t}} - \dfrac{d}{dt} \int_S \mathbf{d} \cdot \hat{\mathbf{n}} = \int_{S} \mathbf{j}_f \cdot \hat{\mathbf{n}} \\
\end{cases}\end{split}
\end{equation*}
\sphinxAtStartPar
Letting \(V\) and \(S\) «collapsing on a discontinuity»…
\begin{equation}\label{equation:ch/media:eq:em-jump}
\begin{split}\begin{cases}
  [ d_n ] = \sigma_f \\
  [ e_t ] = 0 \\
  [ b_n ] = 0 \\
  [ h_t ] = \iota_f \ ,
\end{cases}\end{split}
\end{equation}
\sphinxAtStartPar
being \(\sigma_f\) and \(\iota_f\) surface charge and current density, with physical dimension \(\frac{\text{charge}}{\text{surface}}\), and \(\frac{\text{current}}{\text{surface}}\) respectively. These contributions can be thought as Dirac delta contributions in volume density, namely
\begin{equation*}
\begin{split}\rho(\mathbf{r},t) = \rho_0(\mathbf{r},t) + \sigma(\mathbf{r}_s,t) \delta_{1}(\mathbf{r}-\mathbf{r}_s) \ ,\end{split}
\end{equation*}
\sphinxAtStartPar
being \(\rho(\mathbf{r},t)\) the regular part of the volume density in all the points of the domain \(\mathbf{r} \in V\), \(\sigma(\mathbf{r}_s,t)\) the surface density on 2\sphinxhyphen{}dimensional surfaces \(\mathbf{r}_s \in S\), \(\delta_1()\) the Dirac’s delta with physical dimension \(\frac{1}{\text{length}}\).

\sphinxAtStartPar
If there’s no free surface charge and currents, jump conditions form linear media become
\begin{equation}\label{equation:ch/media:eq:em-jump:no-surf-density}
\begin{split}\begin{cases}
  [ d_n ] = 0 \\
  [ e_t ] = 0 \\
  [ b_n ] = 0 \\
  [ h_t ] = 0 \ ,
\end{cases}
\qquad \rightarrow \qquad
\begin{cases}
  d_{n,1} = d_{n,2}  \quad \rightarrow \quad \varepsilon_1 e_{n,1} = \varepsilon_2 e_{n,2} \\
  e_{t,1} = e_{t,2}  \\
  b_{n,1} = b_{n,2}  \\
  h_{t,1} = h_{t,2}  \quad \rightarrow \quad \frac{1}{\mu_1} b_{t,1} = \frac{1}{\mu_2} b_{t,2} \\
\end{cases}
\end{split}
\end{equation}
\sphinxstepscope


\chapter{Electrostatics}
\label{\detokenize{ch/electrostatics:electrostatics}}\label{\detokenize{ch/electrostatics:classical-electromagnetism-electrostatics}}\label{\detokenize{ch/electrostatics::doc}}

\section{Zero electric field inside a conductor}
\label{\detokenize{ch/electrostatics:zero-electric-field-inside-a-conductor}}
\sphinxAtStartPar
Studying the transient of the electric charge distribution inside a conductor,
\begin{equation*}
\begin{split}\vec{e} = \rho_R \vec{j} \ ,\end{split}
\end{equation*}
\sphinxAtStartPar
whose constitutive equation is
\begin{equation*}
\begin{split}\vec{d} = \varepsilon \vec{e} \ ,\end{split}
\end{equation*}
\sphinxAtStartPar
with free electric charge continuity equation
\begin{equation*}
\begin{split}\partial_t \rho_f + \nabla \cdot \vec{j}_f = 0 \ ,\end{split}
\end{equation*}
\sphinxAtStartPar
and Gauss equation for the displacement field
\begin{equation*}
\begin{split}\nabla \cdot \vec{d} = \rho_f \ .\end{split}
\end{equation*}\begin{equation*}
\begin{split}\begin{aligned}
  \partial_t \rho_f
  & = - \nabla \cdot \vec{j}_f = \\
  & = - \nabla \cdot \left( \frac{1}{\rho_R} \vec{e} \right) = \\
  & = - \frac{1}{\rho_R \varepsilon} \nabla \cdot \vec{d} = \\
  & = - \frac{1}{\rho_R \varepsilon} \rho_f \ ,
\end{aligned}\end{split}
\end{equation*}
\sphinxAtStartPar
having assumed uniform properties. The differential equation in the volume of the conductor provides the evolution of the electric charge in the volume \(\rho(\mathbf{r},t)\), given the initial condition \(\rho(\mathbf{r},0) = \rho_{f,0}(\mathbf{r})\)
\begin{equation*}
\begin{split}\partial_t \rho_f = - \frac{1}{\rho_R \varepsilon} \rho_f\end{split}
\end{equation*}\begin{equation*}
\begin{split}\rho_f(\mathbf{r},t) = \rho_{f,0}(\mathbf{r}) \exp\left[ - \dfrac{t}{\rho_R \varepsilon} \right] \ .\end{split}
\end{equation*}
\sphinxAtStartPar
For a conductor:
\begin{itemize}
\item {} 
\sphinxAtStartPar
\(\varepsilon \sim \varepsilon_0 = 8.85 \cdot 10^{-12} \text{F} \text{m}^-1\)

\item {} 
\sphinxAtStartPar
\(\rho_R \sim 10^{-7}  \Omega \, \text{m}\)

\end{itemize}

\sphinxAtStartPar
so that the time constant (that can be thought as a characteristic time) of the process is
\begin{equation*}
\begin{split}\tau = \rho_R \varepsilon \sim 8.85 \cdot 10^{-19} \, \text{s} \ , \end{split}
\end{equation*}
\sphinxAtStartPar
and thus, after a very short period of time the volume charge density is approximately zero everywhere in the volume: it accumulates in a very thin surface layer.
\subsubsection*{Proof}
\begin{equation*}
\begin{split}\partial_t \left( \rho_f e^{\frac{t}{\rho_R \varepsilon}} \right) = 0\end{split}
\end{equation*}\begin{equation*}
\begin{split}\rho_f(\mathbf{r},t) e^{\frac{r}{\rho_R \varepsilon}} = a(\mathbf{r})\end{split}
\end{equation*}
\sphinxAtStartPar
and appylying initial conditions in all the points of the domain, \(\rho_{f}(\mathbf{r},0) = \rho_{f,0}(\mathbf{r})\), function \(a(\mathbf{r})\) must be equal to \(\rho_{f,0}(\mathbf{r})\) and the solution reads
\begin{equation*}
\begin{split}\rho_f(\mathbf{r},t) = \rho_{f,0}(\mathbf{r}) \exp \left[ -\dfrac{t}{\rho_R \varepsilon} \right]\end{split}
\end{equation*}
\sphinxstepscope




\chapter{Energy balance in electromagnetism}
\label{\detokenize{ch/energy:energy-balance-in-electromagnetism}}\label{\detokenize{ch/energy:classical-electromagnetism-energy}}\label{\detokenize{ch/energy::doc}}

\section{Force, moment, and power on elementary components}
\label{\detokenize{ch/energy:force-moment-and-power-on-elementary-components}}

\subsection{Force, moment and power on a point electric charge}
\label{\detokenize{ch/energy:force-moment-and-power-on-a-point-electric-charge}}
\sphinxAtStartPar
Point electric charge with charge \(q\) in a point \(\vec{r}_P(t)\) at time \(t\) where electromagnetic field is \(\vec{e}(\vec{r},t)\), \(\vec{b}(\vec{r},t)\):
\begin{itemize}
\item {} 
\sphinxAtStartPar
Lorentz’s force
\begin{equation*}
\begin{split}\vec{F} = q \left( \vec{e}(\vec{r}_P(t), t) - \vec{b}(\vec{r}_P(t),t) \times \vec{v}_P(t) \right) \ ,\end{split}
\end{equation*}
\item {} 
\sphinxAtStartPar
zero moment, since it has no dimension (and assumed uniform or symmetric or… distribution of electric charge)

\item {} 
\sphinxAtStartPar
power
\begin{equation*}
\begin{split}\begin{aligned}
     P & = \vec{v}_P(t) \cdot \vec{F} = \\
       & = \vec{v}_P(t) \cdot \, q \, \left( \vec{e}(\vec{r}_P(t), t) - \vec{b}(\vec{r}_P(t), t) \times \vec{v}_P(t) \right) = q \, \vec{v}_P(t) \cdot \vec{e}(\vec{r}_P(t),t) \ .
   \end{aligned}\end{split}
\end{equation*}
\end{itemize}


\subsection{Force, moment and power on a electric dipole}
\label{\detokenize{ch/energy:force-moment-and-power-on-a-electric-dipole}}
\sphinxAtStartPar
Electric dipole with center \(\vec{r}_C(t)\), axis \(\vec{\ell}\), so that the positive charge \(q\) is in \(P_+ = C + \dfrac{\vec{\ell}}{2}\) and the negative charge is in \(P_- = C - \dfrac{\vec{\ell}}{2}\), with \(q \rightarrow +\infty\), \(|\vec{\ell}| \rightarrow 0\), s.t. \(q|\vec{\ell}| = |\vec{d}|\) finite.

\sphinxAtStartPar
\sphinxstylestrong{Kinematics and expansion of the field}
\begin{equation*}
\begin{split}\vec{v}_{\pm} = \vec{v}_C \pm \vec{\omega} \times \frac{\vec{\ell}}{2}\end{split}
\end{equation*}\begin{equation*}
\begin{split}\vec{e}(P_{\pm}) = \vec{e}\left( C \pm \dfrac{\vec{\ell}}{2} \right) = \vec{e}(C) \pm \dfrac{\vec{\ell}}{2} \cdot \nabla \vec{e}(C) + o(|\vec{\ell}|)\end{split}
\end{equation*}\begin{equation*}
\begin{split}\vec{b}(P_{\pm}) = \vec{b}\left( C \pm \dfrac{\vec{\ell}}{2} \right) = \vec{b}(C) \pm \dfrac{\vec{\ell}}{2} \cdot \nabla \vec{b}(C) + o(|\vec{\ell}|)\end{split}
\end{equation*}
\sphinxAtStartPar
\sphinxstylestrong{Net force.}
\begin{equation*}
\begin{split}\begin{aligned}
  \vec{F} & = \vec{F}_+ + \vec{F}_- = \\
   & = q \left[ \vec{e}(P_+) - \vec{b}(P_+) \times \vec{v}_{+} \right] - q \left[ \vec{e}(P_-) - \vec{b}(P_-) \times \vec{v}_{-} \right] = \\
   & = q \left[ \vec{e}_C + \dfrac{\vec{\ell}}{2} \cdot \nabla \vec{e}_C - \left( \vec{b}_C + \dfrac{\vec{\ell}}{2} \cdot \nabla \vec{b}_C \right) \times \left( \vec{v}_C + \vec{\omega} \times \dfrac{\vec{\ell}}{2} \right) \right] + \\ 
   & - q \left[ \vec{e}_C - \dfrac{\vec{\ell}}{2} \cdot \nabla \vec{e}_C - \left( \vec{b}_C - \dfrac{\vec{\ell}}{2} \cdot \nabla \vec{b}_C \right) \times \left( \vec{v}_C - \vec{\omega} \times \dfrac{\vec{\ell}}{2} \right) \right] = \\
   & = q \vec{\ell} \cdot \nabla \vec{e}(C) - \left( q \vec{\ell} \cdot \nabla \vec{b}(C) \right) \times \vec{v}_C + \vec{b}(C) \times \left(  \vec{\omega} \times q \vec{\ell} \right) + o(|\vec{\ell}|)
\end{aligned}\end{split}
\end{equation*}
\sphinxAtStartPar
\sphinxstylestrong{Net moment, w.r.t. \(C\).}
\begin{equation*}
\begin{split}\begin{aligned}
  \vec{M}_C
   & = \frac{\vec{\ell}}{2} \times \vec{F}_+ - \frac{\vec{\ell}}{2} \times \vec{F}_- = \\
   & = q \frac{\vec{\ell}}{2} \times \left[ \vec{e}(P_+) - \vec{b}(P_+) \times \vec{v}_{+} \right] + q \frac{\vec{\ell}}{2} \times \left[ \vec{e}(P_-) - \vec{b}(P_-) \times \vec{v}_{-} \right] = \\
   & = q \frac{\vec{\ell}}{2} \times \left[ \vec{e}_C + \dfrac{\vec{\ell}}{2} \cdot \nabla \vec{e}_C - \left( \vec{b}_C + \dfrac{\vec{\ell}}{2} \cdot \nabla \vec{b}_C \right) \times \left( \vec{v}_C + \vec{\omega} \times \dfrac{\vec{\ell}}{2} \right) \right] + \\ 
   & + q \frac{\vec{\ell}}{2} \times \left[ \vec{e}_C - \dfrac{\vec{\ell}}{2} \cdot \nabla \vec{e}_C - \left( \vec{b}_C - \dfrac{\vec{\ell}}{2} \cdot \nabla \vec{b}_C \right) \times \left( \vec{v}_C - \vec{\omega} \times \dfrac{\vec{\ell}}{2} \right) \right] = \\
   & = q \vec{\ell} \times \left[ \vec{e}_C - \vec{b}_C \times \vec{v}_C \right] + o(|\vec{\ell}|) \ .
\end{aligned}\end{split}
\end{equation*}
\sphinxAtStartPar
\sphinxstylestrong{Power.}
\begin{equation*}
\begin{split}\begin{aligned}
  P & = P_+ + P_- = \\
  & = \vec{F}_+ \cdot \vec{v}_+ + \vec{F}_- \cdot \vec{v}_- = \\
  & = q \, \left[ \vec{e}(P_+) - \vec{b}(P_+) \times \vec{v}_{+}  \right] \cdot \vec{v}_{+} 
    - q \, \left[ \vec{e}(P_-) - \vec{b}(P_-) \times \vec{v}_{-}  \right] \cdot \vec{v}_{-} = \\
  & = q \, \vec{e}(P_+) \cdot \vec{v}_{+} 
    - q \, \vec{e}(P_-) \cdot \vec{v}_{-} = \\
  & = q \, \left[ \vec{e}_C + \dfrac{\vec{\ell}}{2} \cdot \nabla \vec{e}_C  \right] \cdot \left[ \vec{v}_C + \vec{\omega} \times \dfrac{\vec{\ell}}{2} \right] 
    - q \, \left[ \vec{e}_C - \dfrac{\vec{\ell}}{2} \cdot \nabla \vec{e}_C  \right] \cdot \left[ \vec{v}_C - \vec{\omega} \times \dfrac{\vec{\ell}}{2} \right] = \\
  & = \vec{e}_C \cdot \left( \vec{\omega} \times q \vec{\ell} \right) + \left( q \vec{\ell} \cdot \nabla \vec{e}_C \right) \cdot \vec{v}_C + o(|\vec{\ell}|^2) \ .
\end{aligned}\end{split}
\end{equation*}

\subsection{Force, moment and power on a magnetic dipole}
\label{\detokenize{ch/energy:force-moment-and-power-on-a-magnetic-dipole}}
\sphinxAtStartPar
On an elementary magnetic dipole, modeled as a «small» circuit with current \(i\) enclosing area \(S\) and center \(C\), with \(S \rightarrow 0\), \(i \rightarrow + \infty\) so that \(i S \hat{n} := \vec{m}\) finite

\sphinxAtStartPar
\sphinxstylestrong{Force.}
\begin{equation*}
\begin{split}\dots\end{split}
\end{equation*}\begin{equation*}
\begin{split}\vec{F} = \nabla \vec{b}(C) \cdot \vec{m}\end{split}
\end{equation*}
\sphinxAtStartPar
\sphinxstylestrong{Moment.}
\begin{equation*}
\begin{split}\dots\end{split}
\end{equation*}\begin{equation*}
\begin{split}\vec{M}_C = \vec{m} \times \vec{b}(C)\end{split}
\end{equation*}
\sphinxAtStartPar
\sphinxstylestrong{Power.}
\begin{equation*}
\begin{split}P = \vec{v}_C \cdot \nabla \vec{b}(C) \cdot \vec{m} + \vec{\omega} \cdot \vec{m} \times \vec{b}(C) \ .\end{split}
\end{equation*}

\section{Energy balance}
\label{\detokenize{ch/energy:energy-balance}}
\sphinxAtStartPar
\sphinxstylestrong{todo} \sphinxstyleemphasis{Check and put charges, currents, and dipoles together with the electromagnetic field}

\sphinxAtStartPar
Ispirati dalle dimensioni fisiche dei campi elettromagnetici,
\begin{equation*}
\begin{split}\begin{aligned}
\left[\mathbf{e}\right] = \frac{\text{force}}{\text{charge}} \qquad & , \qquad
[\mathbf{d}] = \frac{\text{charge}}{\text{length}^2} \\
[\mathbf{b}] = \frac{\text{force}\cdot\text{time}}{\text{charge}\cdot\text{length}} \qquad & , \qquad
[\mathbf{h}] = \frac{\text{charge}}{\text{time} \cdot \text{length}}
\end{aligned}\end{split}
\end{equation*}



\begin{equation*}
\begin{split}\begin{aligned}
\left[\mathbf{e} \cdot \mathbf{d}\right] & = \frac{\text{force}}{\text{length}^2} = \frac{\text{energy}}{\text{length}^3} = [u] \\
[\mathbf{b} \cdot \mathbf{h}] & = \frac{\text{force}}{\text{length}^2} = \frac{\text{energy}}{\text{length}^3} = [u]
\end{aligned}\end{split}
\end{equation*}
\sphinxAtStartPar
si può costruire la densità di volume di energia  (\sphinxstylestrong{todo} trovare motivazioni più convincenti, non basandosi solo sull’analisi dimensionale ma sul lavoro)
\begin{equation*}
\begin{split}u = \frac{1}{2} \left( \mathbf{e} \cdot \mathbf{d} + \mathbf{b} \cdot \mathbf{h} \right) \ .\end{split}
\end{equation*}
\sphinxAtStartPar
Si può calcolare la derivata parziale nel tempo della densità di energia, \(u\), e usare le equazioni di Maxwell per ottenere un’equazione di bilancio dell’energia del campo elettromagnetico. Per un mezzo isotropo lineare, per il quale valgono le equazioni costitutive \(\mathbf{d} = \varepsilon \mathbf{e}\), \(\mathbf{b} = \mu \mathbf{h}\), la derivata parziale nel tempo dell’energia elettromagnetica può essere riscritta sfuttando la regola di derivazione del prodotto e le equazioni di Faraday\sphinxhyphen{}Lenz\sphinxhyphen{}Neumann e Ampére\sphinxhyphen{}Maxwell,
\begin{equation*}
\begin{split}\begin{aligned}
\dfrac{\partial u}{\partial t} & = \dfrac{\partial}{\partial t}\left( \frac{1}{2} \mathbf{e} \cdot \mathbf{d} + \mathbf{b} \cdot \mathbf{h} \right) =  \qquad (...) \\
& = \mathbf{e} \cdot \partial_t \mathbf{d} + \mathbf{h} \cdot \partial_t \mathbf{b} = \\
& = \mathbf{e} \cdot (\nabla \times \mathbf{h} - \mathbf{j}) - \mathbf{h} \cdot \nabla \times \mathbf{e} \ .
\end{aligned}\end{split}
\end{equation*}
\sphinxAtStartPar
L’ultimo termine può essere ulteriormente manipolato, usando l’identità vettoriale
\begin{equation*}
\begin{split}\begin{aligned}
\mathbf{e} \cdot \nabla \times \mathbf{h} - \mathbf{h} \cdot \nabla \times \mathbf{e} & = e_i \varepsilon_{ijk} \partial_j h_k - h_i \varepsilon_{ijk} \partial_j e_k = \qquad (i \rightarrow k, k \rightarrow i)\\
& = e_i \varepsilon_{ijk} \partial_j h_k - h_k \varepsilon_{kji} \partial_j e_i = \\
& = e_i \varepsilon_{ijk} \partial_j h_k + h_k \varepsilon_{ijk} \partial_j e_i = \\
& =  \partial_j (\varepsilon_{ijk} e_i  h_k ) = \\
& =  \partial_j (\varepsilon_{jki} e_i  h_k ) = \\
& = \nabla \cdot (\mathbf{h} \times \mathbf{e}) = - \nabla \cdot (\mathbf{e} \times \mathbf{h})
\end{aligned}\end{split}
\end{equation*}
\sphinxAtStartPar
che permette di scrivere l’equazione del bilancio di energia elettromagnetica come,
\begin{equation*}
\begin{split}\frac{\partial u }{\partial t} + \nabla \cdot \mathbf{s} = - \mathbf{e} \cdot \mathbf{j} \ ,\end{split}
\end{equation*}
\sphinxAtStartPar
dove è stato definito il \sphinxstylestrong{vettore di Poynting}, o meglio il campo vettoriale di Poynting,
\begin{equation*}
\begin{split}\mathbf{s}(\mathbf{r},t) := \mathbf{e}(\mathbf{r},t) \times \mathbf{h}(\mathbf{r},t) \ ,\end{split}
\end{equation*}
\sphinxAtStartPar
che può essere identificato come un flusso di potenza per unità di superficie, comparendo sotto l’operatore di divergenza nel bilnacio di energia.

\sphinxAtStartPar
\sphinxstylestrong{todo.} Rimandare a una sezione in cui si mostra questa ultima affermazione passando dal bilancio differenziale al bilancio integrale e si usa il teorema della divergenza, \(\int_V \nabla \cdot \mathbf{s} = \oint_{\partial V} \mathbf{s} \cdot \hat{\mathbf{n}}\).
\subsubsection*{Bilancio di energia di cariche nel vuoto, o i materiali senza polarizzazione o magnetizzazione}

\sphinxAtStartPar
\sphinxstylestrong{Moto di cariche puntiformi.}
L’equazione del moto di carica puntiforme \(q_k\) nella posizione \(\mathbf{r}_k(t)\) al tempo \(t\) è
\begin{equation*}
\begin{split}m_k \ddot{\mathbf{r}}_k = \mathbf{f}_k + \mathbf{f}_k^{em} \ ,\end{split}
\end{equation*}
\sphinxAtStartPar
avendo riconosciuto i contributi di forza dovuti al campo elettromagnetico come \(\mathbf{f}_k^{em}\) dagli altri. L’espressione della forza dovuta al campo elettromagnetico sulla carica \(k\) è data dalla forza di Lorentz,
\begin{equation*}
\begin{split}\mathbf{f}_k^{em}(t) = q_k \left[ \mathbf{e}(\mathbf{r_k}(t), t) - \mathbf{b}(\mathbf{r}_k(t), t) \times \dot{\mathbf{r}}_k(t) \right]\end{split}
\end{equation*}
\sphinxAtStartPar
\sphinxstylestrong{Continuità della carica elettrica.} La densità di carica e di corrente elettrica di un insieme di cariche libere puntiformi macroscopiche può essere scritta come
\begin{equation*}
\begin{split}\begin{aligned}
  \rho(\mathbf{r},t) & = \sum_k q_k \delta(\mathbf{r} - \mathbf{r}_k(t)) \\
  \mathbf{j}(\mathbf{r},t) & = \sum_k q_k \dot{\mathbf{r}}_k(t) \delta(\mathbf{r} - \mathbf{r}_k(t)) \ .
\end{aligned}\end{split}
\end{equation*}
\sphinxAtStartPar
L’equazione di continuità della carica, \(\partial_t \rho + \nabla \cdot \mathbf{j} = 0\), risulta quindi soddisfatta,
\begin{equation*}
\begin{split}\begin{aligned}
  \partial_t \rho &  = - \sum_k q_k \, \partial_i \delta(\mathbf{r} - \mathbf{r}_k(t)) \, \dot{r}_{k,i} \\
  \partial_i j_i  &  =   \sum_k q_k \, \dot{r}_{k,i} \, \partial_i \delta(\mathbf{r} - \mathbf{r}_k(t)) \\
\end{aligned}\end{split}
\end{equation*}\subsubsection*{Procedimento alternativo (e più generale?)}

\sphinxAtStartPar
\sphinxstylestrong{todo} \sphinxstyleemphasis{In caso questo procedimento sia più generale, o più corretto, sostituire il procedimento precedente.}

\sphinxAtStartPar
La carica elementare in un volumetto \(\Delta V\) è data da dal prodotto tra il volume e la densità volumetrica di carica, \(\rho \Delta V\); la velocità media locale della carica elettrica è \(\mathbf{v}\); la forza agente sulla carica elementare immersa in un campo elettromagnetico è determinata dalla formula di Lorentz, \(\mathbf{f} \Delta V = \Delta V \rho \left( \mathbf{e} - \mathbf{b} \times \mathbf{v} \right)\). La potenza di questa forza è il prodotto scalare con la velocità media delle cariche, \(\Delta V \mathbf{f} \cdot \mathbf{v}\)

\sphinxAtStartPar
La potenza del campo elettromagnetico sul moto della carica elettrica per unità di volume è quindi
\begin{equation*}
\begin{split}\mathbf{v} \cdot \mathbf{f} = \rho \mathbf{v} \cdot \left( \mathbf{e} - \mathbf{b} \times \mathbf{v} \right) = \rho \mathbf{v} \cdot \mathbf{e} = \mathbf{j} \cdot \mathbf{e} \ .\end{split}
\end{equation*}
\sphinxAtStartPar
\sphinxstylestrong{todo}
\begin{itemize}
\item {} 
\sphinxAtStartPar
discutere questo termine del bilancio di energia cinetica nel moto della carica elettrica

\item {} 
\sphinxAtStartPar
questo termine compare con segno opposto nel bilancio dell’energia elettromagnetica del sistema

\item {} 
\sphinxAtStartPar
dove compare la non\sphinxhyphen{}conservatività del problema in presenza di materiali dissipativi (come resistenza elettrica con \(\mathbf{e} = \rho_R \mathbf{j}\)?

\end{itemize}

\sphinxAtStartPar
Il termine \(\mathbf{e} \cdot \mathbf{j}\) può essere manipolato usando le equazioni di Maxwell, e le relazioni
\begin{equation*}
\begin{split}\begin{cases}
  \mathbf{d} = \varepsilon_0 \mathbf{e} + \mathbf{p} \\
  \mathbf{h} = \frac{\mathbf{b}}{\mu_0} - \mathbf{m} \\
\end{cases}\end{split}
\end{equation*}\begin{equation*}
\begin{split}\begin{aligned}
  \mathbf{e} \cdot \mathbf{j} 
    & = \mathbf{e} \cdot \left( \nabla \times \mathbf{h} - \partial_t \mathbf{d} \right) = \\
    & = - \nabla \cdot \left( \mathbf{e} \times \mathbf{h} \right) + \mathbf{h} \cdot \nabla \times \mathbf{e} - \mathbf{e} \cdot \partial_t \mathbf{d} = \\
    & = - \nabla \cdot \left( \mathbf{e} \times \mathbf{h} \right) - \mathbf{h} \cdot \partial_t \mathbf{b} - \mathbf{e} \cdot \partial_t \mathbf{d} 
\end{aligned}\end{split}
\end{equation*}
\sphinxAtStartPar
Gli ultimi due termini possono essere manipolati in diverse maniere,
\begin{equation*}
\begin{split}\begin{aligned}
  \mathbf{e} \cdot \partial_t \mathbf{d}
    = \mathbf{e} \cdot \partial_t \left( \varepsilon_0 \mathbf{e} + \mathbf{p} \right) 
  & = \partial_t \left( \frac{1}{2} \varepsilon_0 \mathbf{e} \cdot \mathbf{e} \right) + \mathbf{e} \cdot \partial_t \mathbf{p} \\
  & = \partial_t \left( \frac{1}{2} \mathbf{e} \cdot \mathbf{d} \right) + \frac{1}{2} \left( \mathbf{e} \cdot \partial_t \mathbf{p} - \mathbf{p} \cdot \partial_t \mathbf{e} \right) \\
  & = \partial_t \left( \frac{1}{2 \varepsilon_0} \mathbf{d} \cdot \mathbf{d} \right) - \frac{\mathbf{p}}{\varepsilon_0} \cdot \partial_t \mathbf{d} \\
\end{aligned}\end{split}
\end{equation*}\begin{equation*}
\begin{split}\begin{aligned}
  \mathbf{h} \cdot \partial_t \mathbf{b}
    = \mathbf{h} \cdot \partial_t \left( \mu_0 \mathbf{h} + \mu_0 \mathbf{m} \right) 
  & = \partial_t \left( \frac{1}{2} \mu_0 \mathbf{h} \cdot \mathbf{h} \right) + \mu_0 \mathbf{h} \cdot \partial_t \mathbf{m} \\
  & = \partial_t \left( \frac{1}{2} \mathbf{b} \cdot \mathbf{h} \right) + \frac{1}{2} \mu_0 \left( \mathbf{h} \cdot \partial_t \mathbf{m} - \mathbf{m} \cdot \partial_t \mathbf{h} \right) \\
  & = \partial_t \left( \frac{1}{2 \mu_0} \mathbf{b} \cdot \mathbf{b} \right) - \mathbf{m} \cdot \partial_t \mathbf{b} \\
\end{aligned}\end{split}
\end{equation*}
\sphinxAtStartPar
Nel vuoto o in mezzi lineari \(\mathbf{e} \cdot \partial_t \mathbf{p} - \mathbf{p} \cdot \partial_t \mathbf{e} = \mathbf{0}\), \(\mathbf{h} \cdot \partial_t \mathbf{m} - \mathbf{m} \cdot \partial_t \mathbf{h} = \mathbf{0}\). Usando le seconde espressioni, si può riscrivere l’equazione dell’energia del campo elettromagnetico come
\begin{equation*}
\begin{split}\begin{aligned}
  \partial_t \left( \frac{1}{2} \mathbf{e} \cdot \mathbf{d} + \frac{1}{2} \mathbf{b} \cdot \mathbf{h} \right) + \nabla \cdot \left( \mathbf{e} \times \mathbf{h} \right) & = - \ \mathbf{e} \cdot \mathbf{j} \ + \\
   & \quad - \frac{1}{2} \left[ \mathbf{e} \cdot \partial_t \mathbf{p} - \mathbf{p} \cdot \partial_t \mathbf{e} + \mu_0 \left(  \mathbf{h} \cdot \partial_t \mathbf{m} - \mathbf{m} \cdot \partial_t \mathbf{h} \right) \right]
\end{aligned}\end{split}
\end{equation*}
\sphinxAtStartPar
o, usando le definizioni di densità di energia elettromagnetica \(u\) e vettore di Poynting \(\mathbf{s}\),
\begin{equation*}
\begin{split}
  \partial_t u + \nabla \cdot \mathbf{s} =
    - \ \mathbf{e} \cdot \mathbf{j} \
    - \frac{1}{2} \left[ \mathbf{e} \cdot \partial_t \mathbf{p} - \mathbf{p} \cdot \partial_t \mathbf{e} + \mu_0 \left(  \mathbf{h} \cdot \partial_t \mathbf{m} - \mathbf{m} \cdot \partial_t \mathbf{h} \right) \right]
\end{split}
\end{equation*}
\sphinxstepscope


\chapter{Energy and momentum balance in linear, local, isotropic, non\sphinxhyphen{}dispersive media}
\label{\detokenize{ch/energy-linear:energy-and-momentum-balance-in-linear-local-isotropic-non-dispersive-media}}\label{\detokenize{ch/energy-linear:classical-electromagnetism-energy}}\label{\detokenize{ch/energy-linear::doc}}\begin{equation*}
\begin{split}\begin{cases}
  \mathbf{d} = \varepsilon_0 \mathbf{e} + \mathbf{p} \\
  \mathbf{h} = \dfrac{1}{\mu_0} \mathbf{b} - \mathbf{m} \ .
\end{cases}\end{split}
\end{equation*}
\sphinxAtStartPar
with
\begin{equation*}
\begin{split}\mathbf{d} = \varepsilon \mathbf{e} \qquad , \qquad \mathbf{h} = \dfrac{\mathbf{b}}{\mu}\end{split}
\end{equation*}
\sphinxAtStartPar
Let \(r\) be mass density, and \(\vec{v}\) be charge velocity field, the equation of motion \sphinxhyphen{} momentum equation \sphinxhyphen{} of electric charges reads
\begin{equation*}
\begin{split}r \frac{D \mathbf{v}}{D t} = \mathbf{f} \ ,\end{split}
\end{equation*}
\sphinxAtStartPar
and the kinetic energy equation becomes
\begin{equation*}
\begin{split}\mathbf{v} \cdot \mathbf{f} = r \mathbf{v} \cdot \frac{D \mathbf{v}}{D t} = r \dfrac{D}{Dt} \dfrac{|\mathbf{v}|^2}{2} \ ,\end{split}
\end{equation*}
\sphinxAtStartPar
or using continuity equation for \(r\), it can be recast in conservative form. The same term can be recast using the expression of Lorentz’s force on electric charges in electromagnetic field (\sphinxstylestrong{is this the right way to evaluate power of bounded charges and currents? check it!})
\begin{equation*}
\begin{split}\mathbf{v} \cdot \mathbf{f} = \mathbf{v} \cdot \left[ \rho ( \mathbf{e} - \mathbf{b} \times \mathbf{v} ) \right] = \rho \mathbf{v} \cdot \mathbf{e} = \mathbf{e} \cdot \mathbf{j} \ ,\end{split}
\end{equation*}
\sphinxAtStartPar
and furthered manipulated writing \(\mathbf{j} = \mathbf{j}_f + \mathbf{j}_p + \mathbf{j}_m\) and using Maxwell’s equations
\begin{equation*}
\begin{split}\begin{aligned}
  \mathbf{e} \cdot \mathbf{j} 
  & = \mathbf{e} \cdot \left( \mathbf{j}_f + \mathbf{j}_p + \mathbf{j}_m \right) = \\
  & = \mathbf{e} \cdot \left( \nabla \times \mathbf{h} - \partial_t \mathbf{d} \right) + \mathbf{e} \cdot \partial_t \mathbf{p} + \mathbf{e} \cdot \nabla \times \mathbf{m} = \\
  & = \mathbf{e} \cdot \nabla \times \left( \mathbf{h} + \mathbf{m} \right) - \mathbf{e} \cdot \partial_t \left( \mathbf{d} - \mathbf{p} \right)  = \\
  & = \dfrac{1}{\mu_0} \mathbf{e} \cdot \nabla \times \mathbf{b} - \varepsilon_0 \mathbf{e} \cdot \partial_t \mathbf{e} = \\
  & = - \dfrac{1}{\mu_0} \nabla \cdot \left( \mathbf{e} \times \mathbf{b} \right) - \dfrac{\mathbf{b}}{\mu_0} \cdot \partial_t \mathbf{b} - \varepsilon_0 \mathbf{e} \cdot \partial_t \mathbf{e} = \\
\end{aligned}\end{split}
\end{equation*}\begin{equation*}
\begin{split}\mathbf{e} \cdot \nabla \times \mathbf{h} = e_i \varepsilon_{ijk} \partial_j h_k = \partial_j \left( \varepsilon_{jki} h_k e_i \right) - \varepsilon_{ijk} h_k \partial_j e_i = - \nabla \cdot (\mathbf{e} \times \mathbf{h} ) + \mathbf{h} \cdot \nabla \times \mathbf{e} = - \nabla \cdot (\mathbf{e} \times \mathbf{h} ) - \mathbf{h} \cdot \partial_t \mathbf{b}\end{split}
\end{equation*}\begin{equation*}
\begin{split}\begin{aligned}
  \mathbf{e} \cdot \mathbf{j}_f & = - \nabla \cdot ( \mathbf{e} \times \mathbf{h} ) - \mathbf{e} \cdot \partial_t \mathbf{d} - \mathbf{h} \cdot \partial_t \mathbf{b}
\end{aligned}\end{split}
\end{equation*}

\section{Linear media \sphinxhyphen{} energy}
\label{\detokenize{ch/energy-linear:linear-media-energy}}
\sphinxAtStartPar
For linear media, the energy of the electromagnetic field per unit volume reads
\begin{equation*}
\begin{split}u = \dfrac{1}{2} \left( \mathbf{e} \cdot \mathbf{d} + \mathbf{h} \cdot \mathbf{b} \right)\end{split}
\end{equation*}
\sphinxAtStartPar
so that the differential balance equation for the eneergy of the electromagnetic field becomes
\begin{equation*}
\begin{split}\begin{aligned}
 \partial_t u + \nabla \cdot \mathbf{s} = - \mathbf{e} \cdot \mathbf{j} \ ,
\end{aligned}\end{split}
\end{equation*}
\sphinxAtStartPar
with Poynting vector \(\mathbf{s} := \mathbf{e} \times \mathbf{h}\), namely the momentum density of the electromagnetic field.


\section{Linear media \sphinxhyphen{} momentum}
\label{\detokenize{ch/energy-linear:linear-media-momentum}}\begin{equation*}
\begin{split}\partial_t \mathbf{s} = \partial_t s_i = \partial_t \left( \varepsilon_{ijk} e_j h_k \right)\end{split}
\end{equation*}\begin{equation*}
\begin{split}\begin{aligned}
  \varepsilon_{ijk} \partial_t e_j h_k
  & = \dfrac{1}{\varepsilon} \varepsilon_{ijk} \partial_t d_j h_k \\
  & = \dfrac{1}{\varepsilon} \varepsilon_{ijk} \left(\varepsilon_{jlm} \partial_l h_m - j^f_j \right) h_k \\
  & = - \dfrac{1}{\varepsilon} \varepsilon_{ijk} \, j^f_j \, h_k + \dfrac{1}{\varepsilon} \varepsilon_{ijk} \varepsilon_{jlm} h_k \partial_l h_m \\
  & = - \dfrac{1}{\varepsilon} \varepsilon_{ijk} \, j^f_j \, h_k + \dfrac{1}{\varepsilon} \left( \delta_{im} \delta_{kl} - \delta_{il} \delta_{km} \right) h_k \partial_l h_m =  \\
  & = - \dfrac{1}{\varepsilon} \varepsilon_{ijk} \, j^f_j \, h_k + \dfrac{1}{\varepsilon} \left( h_m \partial_m h_i - h_m \partial_i h_m \right) =  \\
  & = - \dfrac{1}{\varepsilon} \varepsilon_{ijk} \, j^f_j \, h_k + \dfrac{1}{\varepsilon} \left[ \partial_m ( h_m  h_i ) - \partial_m h_m \, h_i - \partial_i \left( \frac{h_m h_m}{2} \right) \right] =  \\
  & = \dfrac{1}{\varepsilon \mu} \varepsilon_{ijk} \, b_j \, j^f_k + \dfrac{1}{\varepsilon \mu} \left[ \partial_m ( b_m  h_i ) - \underbrace{\partial_m b_m}_{=0} \, h_i - \partial_i \left( \frac{h_m b_m}{2} \right) \right] =  \\
\end{aligned}\end{split}
\end{equation*}\begin{equation*}
\begin{split}\begin{aligned}
  \varepsilon_{ijk} e_j \partial_t h_k
  & =   \dfrac{1}{\mu} \varepsilon_{ijk} e_j \partial_t b_k = \\
  & = - \dfrac{1}{\mu} \varepsilon_{ijk} e_j \left( \varepsilon_{klm} \partial_l e_m \right) = \\
  & = - \dfrac{1}{\mu} \left( \delta_{il} \delta_{jm} - \delta_{im} \delta_{jl} \right) e_j \partial_l e_m =  \\
  & = - \dfrac{1}{\mu} \left( e_m \partial_i e_m - e_m \partial_m e_i \right) =  \\
  & = - \dfrac{1}{\mu} \left[ \partial_i \left(\frac{e_m e_m}{2}\right) -  \partial_m \left( e_m e_i \right) + \partial_m e_m \, e_i \right] = \\
  & = - \dfrac{1}{\varepsilon \mu} \left[ \partial_i \left(\frac{d_m e_m}{2}\right) - \partial_m \left( d_m e_i \right) + \rho^f \, e_i \right] \ .
\end{aligned}\end{split}
\end{equation*}
\sphinxAtStartPar
so that
\begin{equation*}
\begin{split}\partial_t s_i + c^2 \partial_m \left[ \dfrac{1}{2}\left( d_n e_n + h_n b_n \right) \delta_{mi} - \left( h_m b_i + d_m e_i \right) \right] = - c^2 \rho^f e_i + c^2 \varepsilon_{ijk} b_j j_k^f \end{split}
\end{equation*}
\sphinxAtStartPar
or
\begin{equation*}
\begin{split}\partial_t \mathbf{s} + c^2 \nabla \cdot \left[ \, \dfrac{1}{2} \left( \mathbf{d} \cdot \mathbf{e} + \mathbf{h} \cdot \mathbf{b} \right) \mathbb{I} - \left( \mathbf{d} \otimes \mathbf{e} + \mathbf{h} \otimes \mathbf{b} \right) \, \right] = - c^2 \left( \rho^f \mathbf{e} - \mathbf{b} \times \mathbf{j}^f \right)\end{split}
\end{equation*}\begin{equation*}
\begin{split}\begin{cases}
& \partial_t u + \nabla \cdot \mathbf{s} = - \mathbf{e} \cdot \mathbf{j}^f \\
& \partial_t \mathbf{s} + c^2 \nabla \cdot \left[ \, u \mathbb{I} - \left( \mathbf{d} \otimes \mathbf{e} + \mathbf{h} \otimes \mathbf{b} \right) \, \right] = - c^2 \left( \mathbf{e} \, \rho^f - \mathbf{b} \times \mathbf{j}^f \right)
\end{cases}\end{split}
\end{equation*}
\sphinxAtStartPar
\sphinxstylestrong{todo} \sphinxstyleemphasis{use this system to derive the 4\sphinxhyphen{}d formulation of special relativity in modern physics}

\sphinxstepscope




\chapter{Equazioni dell’elettromagnetismo e relatività galileiana}
\label{\detokenize{ch/low-speed-relativity:equazioni-dell-elettromagnetismo-e-relativita-galileiana}}\label{\detokenize{ch/low-speed-relativity:classical-electromagnetism-low-speed-relativity}}\label{\detokenize{ch/low-speed-relativity::doc}}
\sphinxstepscope




\chapter{Onde elettromagnetiche}
\label{\detokenize{ch/waves:onde-elettromagnetiche}}\label{\detokenize{ch/waves:classical-electromagnetism-waves}}\label{\detokenize{ch/waves::doc}}
\sphinxstepscope

\begin{sphinxuseclass}{sd-container-fluid}
\begin{sphinxuseclass}{sd-sphinx-override}
\begin{sphinxuseclass}{sd-p-0}
\begin{sphinxuseclass}{sd-mt-2}
\begin{sphinxuseclass}{sd-mb-4}
\begin{sphinxuseclass}{sd-row}
\begin{sphinxuseclass}{sd-row-cols-2}
\begin{sphinxuseclass}{sd-gx-2}
\begin{sphinxuseclass}{sd-gy-1}
\begin{sphinxuseclass}{sd-col}
\begin{sphinxuseclass}{sd-d-flex-row}
\begin{sphinxuseclass}{sd-align-minor-center}
\begin{sphinxuseclass}{sd-container-fluid}
\begin{sphinxuseclass}{sd-sphinx-override}
\begin{sphinxuseclass}{sd-row}
\begin{sphinxuseclass}{sd-row-cols-2}
\begin{sphinxuseclass}{sd-row-cols-xs-2}
\begin{sphinxuseclass}{sd-row-cols-sm-3}
\begin{sphinxuseclass}{sd-row-cols-md-3}
\begin{sphinxuseclass}{sd-row-cols-lg-3}
\begin{sphinxuseclass}{sd-gx-3}
\begin{sphinxuseclass}{sd-gy-1}
\begin{sphinxuseclass}{sd-col}
\begin{sphinxuseclass}{sd-col-auto}
\begin{sphinxuseclass}{sd-d-flex-row}
\begin{sphinxuseclass}{sd-align-minor-center}
\sphinxAtStartPar
basics

\end{sphinxuseclass}
\end{sphinxuseclass}
\end{sphinxuseclass}
\end{sphinxuseclass}
\begin{sphinxuseclass}{sd-col}
\begin{sphinxuseclass}{sd-col-auto}
\begin{sphinxuseclass}{sd-d-flex-row}
\begin{sphinxuseclass}{sd-align-minor-center}
\sphinxAtStartPar
26 apr 2025

\end{sphinxuseclass}
\end{sphinxuseclass}
\end{sphinxuseclass}
\end{sphinxuseclass}
\begin{sphinxuseclass}{sd-col}
\begin{sphinxuseclass}{sd-col-auto}
\begin{sphinxuseclass}{sd-d-flex-row}
\begin{sphinxuseclass}{sd-align-minor-center}
\sphinxAtStartPar
1 min read

\end{sphinxuseclass}
\end{sphinxuseclass}
\end{sphinxuseclass}
\end{sphinxuseclass}
\end{sphinxuseclass}
\end{sphinxuseclass}
\end{sphinxuseclass}
\end{sphinxuseclass}
\end{sphinxuseclass}
\end{sphinxuseclass}
\end{sphinxuseclass}
\end{sphinxuseclass}
\end{sphinxuseclass}
\end{sphinxuseclass}
\end{sphinxuseclass}
\end{sphinxuseclass}
\end{sphinxuseclass}
\end{sphinxuseclass}
\end{sphinxuseclass}
\end{sphinxuseclass}
\end{sphinxuseclass}
\end{sphinxuseclass}
\end{sphinxuseclass}
\end{sphinxuseclass}
\end{sphinxuseclass}
\end{sphinxuseclass}

\section{Equazioni delle onde in elettromagnetismo}
\label{\detokenize{ch/waves-equation:equazioni-delle-onde-in-elettromagnetismo}}\label{\detokenize{ch/waves-equation:classical-electromagnetism-waves-wave-equation}}\label{\detokenize{ch/waves-equation::doc}}
\sphinxAtStartPar
\sphinxstylestrong{Identità vettoriale.}
\begin{equation*}
\begin{split}\Delta \mathbf{v} = \nabla ( \nabla \cdot \mathbf{v} ) - \nabla \times \nabla \times \mathbf{v}\end{split}
\end{equation*}
\sphinxAtStartPar
\sphinxstylestrong{Dim.}
\begin{equation*}
\begin{split}\begin{aligned}
 \nabla \times \nabla \times \mathbf{v} & = \varepsilon_{ijk} \partial_j ( \varepsilon_{klm} \partial_l v_m ) = \\
 & = \varepsilon_{kij} \varepsilon_{klm} \partial_{jl} v_m = \\
 & = ( \delta_{il} \delta_{jm} - \delta_{im} \delta_{jl} )  \partial_{jl} v_m = \\
 & = \partial_{ij} v_j - \partial_{jj} v_i = \\
 & = \nabla (\nabla \cdot \mathbf{v}) - \Delta \mathbf{v} \ ,
\end{aligned}\end{split}
\end{equation*}
\sphinxAtStartPar
avendo utilizzato l’identità
\begin{equation*}
\begin{split}\varepsilon_{ijk} \varepsilon_{ilm} = \delta_{jl} \delta_{km} - \delta_{jm} \delta_{kl}\end{split}
\end{equation*}

\subsection{Potenziali elettromagnetici}
\label{\detokenize{ch/waves-equation:potenziali-elettromagnetici}}
\sphinxAtStartPar
Partendo dalle definizioni dei potenziali elettromagnetici e dalle equazioni di Maxwell, con l’aiuto di alcune identità vettoriali, è possibile (\sphinxstylestrong{TODO} \sphinxstyleemphasis{ipotesi, elencare quelle necessarie alla derivazione}) scrivere delle equazionin delle onde per il potenziale vettore e per il potenziale scalare.
\begin{equation*}
\begin{split}\begin{aligned}
 \mathbf{e} & = - \nabla \varphi - \partial_t \mathbf{a} \\
 \mathbf{b} & = \nabla \times \mathbf{a} \\
\end{aligned}\end{split}
\end{equation*}
\sphinxAtStartPar
Usando le equazioni costitutive
\begin{equation*}
\begin{split}\mathbf{d} = \varepsilon \ \mathbf{e} \qquad , \qquad
\mathbf{b} = \mu \mathbf{h} \end{split}
\end{equation*}
\sphinxAtStartPar
\sphinxstylestrong{Potenziale vettore.}
\begin{equation*}
\begin{split}\mathbf{b} = \nabla \times \mathbf{a}\end{split}
\end{equation*}\begin{equation*}
\begin{split}\begin{aligned}
\mathbf{0} & = \nabla \times \nabla \times \mathbf{a} - \nabla \times \mathbf{b} = \\
 & = - \Delta \mathbf{a} + \nabla(\nabla \cdot \mathbf{a})  - \mu \nabla \times \mathbf{h} = \\
 & = - \Delta \mathbf{a} + \nabla(\nabla \cdot \mathbf{a})  - \mu ( \partial_t \mathbf{d} + \mathbf{j} )  = \\
 & = - \Delta \mathbf{a} + \nabla(\nabla \cdot \mathbf{a})  - \mu ( \varepsilon \partial_t \mathbf{e} + \mathbf{j} )  = \\
 & = - \Delta \mathbf{a} + \nabla(\nabla \cdot \mathbf{a})  - \mu \varepsilon ( - \partial_t \nabla \varphi - \partial_{tt} \mathbf{a} ) + \mu \mathbf{j} = \\
 & = - \Delta \mathbf{a} + \nabla(\nabla \cdot \mathbf{a})  + \frac{1}{c^2} \partial_t \nabla \varphi + \dfrac{1}{c^2} \partial_{tt} \mathbf{a} - \mu \mathbf{j}  \\
\end{aligned}\end{split}
\end{equation*}
\sphinxAtStartPar
Usando la condizione di gauge di Lorentz
\begin{equation*}
\begin{split}\nabla \cdot \mathbf{a} + \frac{1}{c^2} \partial_t  \varphi = 0 \ ,\end{split}
\end{equation*}
\sphinxAtStartPar
si ottiene un’equazione delle onde per il potenziale vettore
\begin{equation*}
\begin{split} \dfrac{1}{c^2} \partial_{tt} \mathbf{a} - \Delta \mathbf{a}  =  \mu \mathbf{j}  \ .\end{split}
\end{equation*}
\sphinxAtStartPar
\sphinxstylestrong{Potenziale scalare.}
\begin{equation*}
\begin{split}\mathbf{e} = \nabla \varphi - \partial_t \mathbf{a}\end{split}
\end{equation*}
\sphinxAtStartPar
Calcolando la derivata nel tempo della condizione di gauge di Lorentz
\begin{equation*}
\begin{split}\begin{aligned}
 0 & = \partial_t (\frac{1}{c^2} \partial_t \varphi + \nabla \cdot \mathbf{a}) = \\
   & = \frac{1}{c^2} \partial_{tt} \varphi + \nabla \cdot \partial_t \mathbf{a} = \\
   & = \frac{1}{c^2} \partial_{tt} \varphi - \nabla \cdot \nabla \varphi - \nabla \cdot \mathbf{e} = \\
   & = \frac{1}{c^2} \partial_{tt} \varphi - \Delta \varphi - \frac{\rho}{\varepsilon} = \\
\end{aligned}\end{split}
\end{equation*}
\sphinxAtStartPar
si arriva all’equazione delle onde per il potenziale scalare,
\begin{equation*}
\begin{split} \frac{1}{c^2} \partial_{tt} \varphi - \Delta \varphi = \frac{\rho}{\varepsilon} \ .\end{split}
\end{equation*}

\subsection{Campo elettrico e campo magnetico}
\label{\detokenize{ch/waves-equation:campo-elettrico-e-campo-magnetico}}
\sphinxAtStartPar
Usando le definizioni dei campi fisici in termini dei potenziali elettromagnetici e la linearità (\sphinxstylestrong{TODO} \sphinxstyleemphasis{tutto deve essere lineare, anche le leggi costitutive}) delle operazioni, partendo dalle equazioni delle onde per i potenziali, si possono ricavare le equazioni delle onde per i campi fisici. \sphinxstylestrong{TODO} \sphinxstyleemphasis{Nell’ipotesi di proprietà costanti e uniformi}

\sphinxAtStartPar
\sphinxstylestrong{Campo elettrico.}
\begin{equation*}
\begin{split}\begin{aligned}
\square \mathbf{e} & = \square ( -\nabla \varphi - \partial_t \mathbf{a}) = \\
& = - \nabla \square \varphi - \partial_t \square \mathbf{a} = \\
& = - \nabla \dfrac{\rho}{\varepsilon} - \mu \partial_t \mathbf{j}  \ .
\end{aligned}\end{split}
\end{equation*}
\sphinxAtStartPar
\sphinxstylestrong{Campo magnetico.}
\begin{equation*}
\begin{split}\begin{aligned}
 \square \mathbf{b} & = \square \nabla \times \mathbf{a} = \\
 & = \nabla \times \square \mathbf{a} = \\
 & = \mu \nabla \times \mathbf{j}
\end{aligned}\end{split}
\end{equation*}
\sphinxstepscope




\section{Onde elettromagnetiche piane}
\label{\detokenize{ch/waves-plane:onde-elettromagnetiche-piane}}\label{\detokenize{ch/waves-plane:classical-electromagnetism-waves-plane-waves}}\label{\detokenize{ch/waves-plane::doc}}
\sphinxAtStartPar
Harmonic decomposition of the electromagnetic field. EM field can be written as the superposition of plane waves (Fourier decomposition)
\begin{equation*}
\begin{split}\begin{aligned}
  \mathbf{e}(\mathbf{r},t) & = \mathbf{E} e^{i(\mathbf{k} \cdot \mathbf{r} - \omega t)} \\
  \mathbf{b}(\mathbf{r},t) & = \mathbf{B} e^{i(\mathbf{k} \cdot \mathbf{r} - \omega t)} \\
\end{aligned}\end{split}
\end{equation*}
\sphinxAtStartPar
Introudcing this decomposiiton in Maxwell’s equations with no free charge and current
\begin{equation*}
\begin{split}
\begin{cases}
 \nabla \cdot \mathbf{d} = 0 \\
 \nabla \times \mathbf{e} + \partial_t \mathbf{b} = \mathbf{0} \\
 \nabla \cdot \mathbf{b} = 0 \\
 \nabla \times \mathbf{h} - \partial_t \mathbf{d} = \mathbf{0}
\end{cases}
\end{split}
\end{equation*}\begin{equation*}
\begin{split}
\begin{cases}
 i \mathbf{k} \cdot \mathbf{D} = 0 \\
 i \mathbf{k} \times \mathbf{E} - i \omega \mathbf{B} = \mathbf{0} \\
 i \mathbf{k} \cdot \mathbf{B} = 0 \\
 i \mathbf{k} \times \mathbf{H} + i \omega \mathbf{D} = \mathbf{0}
\end{cases}
\quad \rightarrow \quad
\begin{cases}
 i \varepsilon \mathbf{k} \cdot \mathbf{E} = 0 \\
 i \mathbf{k} \times \mathbf{E} - i \omega \mathbf{B} = \mathbf{0} \\
 i \mathbf{k} \cdot \mathbf{B} = 0 \\
 i \dfrac{1}{\mu} \mathbf{k} \times \mathbf{B} + i \omega \varepsilon \mathbf{E} = \mathbf{0}
\end{cases}
\end{split}
\end{equation*}\begin{itemize}
\item {} 
\sphinxAtStartPar
From Gauss” equations for the electric and the magnetic field
\begin{equation*}
\begin{split}\mathbf{k} \perp \mathbf{E} \quad , \quad \mathbf{k} \perp \mathbf{B}\end{split}
\end{equation*}
\item {} 
\sphinxAtStartPar
From Faraday and Ampére\sphinxhyphen{}Maxwell equations
\begin{equation*}
\begin{split}\mathbf{B} = \dfrac{\mathbf{k}}{\omega} \times \mathbf{E}\end{split}
\end{equation*}\begin{equation*}
\begin{split}\mathbf{E} = - \dfrac{1}{\mu \varepsilon}\dfrac{\mathbf{k}}{\omega} \times \mathbf{B}\end{split}
\end{equation*}
\end{itemize}

\sphinxAtStartPar
It follows that:
\begin{itemize}
\item {} 
\sphinxAtStartPar
\(\mathbf{k}\), \(\mathbf{E}\), \(\mathbf{B}\) are orthogonal «RHS» set of vectors

\item {} 
\sphinxAtStartPar
Relations between \(\mathbf{E}\), \(\mathbf{B}\), and \(\mathbf{k}\) and the speed of light
\begin{equation*}
\begin{split}\begin{aligned}
      \mathbf{B} & = \dfrac{1}{c} \, \hat{\mathbf{k}} \times \mathbf{E} \\
      \mathbf{E} & = - c \, \hat{\mathbf{k}} \times \mathbf{B} \\
    \end{aligned}\end{split}
\end{equation*}
\sphinxAtStartPar
hold, with speed of light \(c = \dfrac{1}{\sqrt{\mu \varepsilon}} = \dfrac{\omega}{|\mathbf{k}|}\), and unit vector \(\hat{\mathbf{k}} = \dfrac{\mathbf{k}}{|\mathbf{k}|}\).

\end{itemize}
\subsubsection*{Proof using vector algebra identity}

\sphinxAtStartPar
Recalling \(c^2 = \frac{1}{\mu \varepsilon}\) and
\begin{equation*}
\begin{split}\mathbf{B} = \dfrac{\mathbf{k}}{\omega} \times \mathbf{E} = \dfrac{\mathbf{k}}{\omega} \times \left[ - c^2 \dfrac{\mathbf{k}}{\omega} \times \mathbf{B} \right] = - \dfrac{c^2 |\mathbf{k}|^2}{\omega^2} \hat{\mathbf{k}} \times \left( \hat{\mathbf{k}} \times \mathbf{B} \right)\end{split}
\end{equation*}
\sphinxAtStartPar
Vector identity
\begin{equation*}
\begin{split}\mathbf{a} \times (\mathbf{b} \times \mathbf{c}) = \varepsilon_{ijk} a_j \varepsilon_{klm} b_l c_m = \left( \delta_{il} \delta_{jm} - \delta_{im} \delta_{jl} \right) a_j \, b_l \, c_m = b_i a_m c_m - c_i a_m b_m = (\mathbf{a} \cdot \mathbf{c}) \mathbf{b} - (\mathbf{a} \cdot \mathbf{b}) \mathbf{c}\end{split}
\end{equation*}
\sphinxAtStartPar
applied to \(\hat{\mathbf{k}} \times \left( \hat{\mathbf{k}} \times \mathbf{B} \right)\) gives
\begin{equation*}
\begin{split}
  \hat{\mathbf{k}} \times \left( \hat{\mathbf{k}} \times \mathbf{B} \right) = \underbrace{\left( \hat{\mathbf{k}} \dot \mathbf{B} \right)}_{=0 \text{ since $\mathbf{k} \perp \mathbf{B}$}} \hat{\mathbf{k}} - \underbrace{\left( \hat{\mathbf{k}} \cdot \hat{\mathbf{k}} \right)}_{= 1} \mathbf{B} = - \mathbf{B} \ , 
\end{split}
\end{equation*}
\sphinxAtStartPar
and the original relation gives
\begin{equation*}
\begin{split}\mathbf{B} = \mathbf{B} \dfrac{c^2 |\mathbf{k}|^2}{\omega^2} \ ,\end{split}
\end{equation*}
\sphinxAtStartPar
and the relation between pulsation \(\omega\), wave vector \(\mathbf{k}\) and speed of light (EM radiation) \(c\),
\begin{equation*}
\begin{split}c = \dfrac{\omega}{|\mathbf{k}|} \ .\end{split}
\end{equation*}

\subsection{Snell’s law at an interface}
\label{\detokenize{ch/waves-plane:snell-s-law-at-an-interface}}\label{\detokenize{ch/waves-plane:classical-electromagnetism-waves-plane-waves-snell}}
\sphinxAtStartPar
Snell’s law is derived here assuming isotrpoic linear media, so that
\begin{equation*}
\begin{split}\begin{cases}
  \mathbf{d}(\mathbf{r},t) = \varepsilon \mathbf{e}(\mathbf{r},t) \\
  \mathbf{b}(\mathbf{r},t) = \mu         \mathbf{h}(\mathbf{r},t)
\end{cases}\end{split}
\end{equation*}
\sphinxAtStartPar
and for harmonic plane EM waves
\begin{equation*}
\begin{split}\begin{cases}
 \mathbf{e}(\mathbf{r}, t) = \mathbf{E}_{a} \, e^{i \left( \mathbf{k}_a \cdot \mathbf{r} - \omega t \right)} \\
 \mathbf{b}(\mathbf{r}, t) = \mathbf{B}_{a} \, e^{i \left( \mathbf{k}_a \cdot \mathbf{r} - \omega t \right)} \\
\end{cases}\end{split}
\end{equation*}\begin{equation*}
\begin{split}\begin{aligned}
  \mathbf{B}_a & = \dfrac{1}{c} \, \hat{\mathbf{k}}_a \times \mathbf{E}_a \\
  \mathbf{E}_a & = - c \, \hat{\mathbf{k}}_a \times \mathbf{B}_a \\
\end{aligned}\end{split}
\end{equation*}
\sphinxAtStartPar
being index \(a\) representing the media involved: \(a = 1\) for the medium with incident and reflected waves, \(a = 2\) for the medium withthe refracted wave.

\sphinxAtStartPar
{\hyperref[\detokenize{ch/media:classical-electromagnetism-media-jump}]{\sphinxcrossref{\DUrole{std,std-ref}{Jump conditions of electromagnetic field at an interface}}}} with no charge or current surface density are given by conditions \eqref{equation:ch/media:eq:em-jump:no-surf-density},
\begin{equation*}
\begin{split}\begin{cases}
  \varepsilon_1 e_{n,1} = \varepsilon_2 e_{n,2} \\
  e_{t_{\alpha},1} = e_{t_{\alpha},2}                                    & \quad , \quad \alpha=1:2 \\
  b_{n,1} = b_{n,2}  \\
  \dfrac{1}{\mu_1} b_{t_{\alpha},1} = \dfrac{1}{\mu_2} b_{t_{\alpha},2}  & \quad , \quad \alpha=1:2 \\
\end{cases}\end{split}
\end{equation*}
\sphinxAtStartPar
Definition of some vectors: \(\hat{\mathbf{n}}\) unit normal vector, \(\mathbf{k}\) wave vector, \(\hat{\mathbf{b}} = \dfrac{\hat{\mathbf{n}} \times \mathbf{k}}{|\hat{\mathbf{n}} \times \mathbf{k}|}\) (singular only for normal incident ray), \(\hat{\mathbf{c}} = \dfrac{\hat{\mathbf{b}} \times \mathbf{k}}{|\hat{\mathbf{b}} \times \mathbf{k}|}\), \(\hat{\mathbf{t}} = \dfrac{\hat{\mathbf{b}} \times \hat{\mathbf{n}}}{|\hat{\mathbf{b}} \times \hat{\mathbf{n}}|}\)

\sphinxAtStartPar
Incindent angle \(\theta_{1,i}\) is the angle between \(\hat{\mathbf{n}}\) and \(\mathbf{k}\), s.t. \(\hat{\mathbf{n}} \times \mathbf{k} = \hat{\mathbf{b}} \, k \, \sin \theta_{1,i}\).
\begin{equation*}
\begin{split}\begin{cases}
  \hat{\mathbf{k}} = \quad \cos \theta_{1,i} \hat{\mathbf{n}} + \sin \theta_{1,i} \hat{\mathbf{t}} \\
  \hat{\mathbf{c}} =      -\sin \theta_{1,i} \hat{\mathbf{n}} + \cos \theta_{1,i} \hat{\mathbf{t}}
\end{cases}
\quad , \quad
\begin{cases}
  \hat{\mathbf{n}} = \cos \theta_{1,i} \hat{\mathbf{k}} - \sin \theta_{1,i} \hat{\mathbf{c}} \\
  \hat{\mathbf{t}} = \sin \theta_{1,i} \hat{\mathbf{k}} + \cos \theta_{1,i} \hat{\mathbf{c}}
\end{cases}\end{split}
\end{equation*}
\sphinxAtStartPar
The electromagnetic field can be written as
\begin{equation*}
\begin{split}\begin{aligned}
  \mathbf{E} & = E_b \hat{\mathbf{b}} + E_c \hat{\mathbf{c}} = \\
             & = E_b \hat{\mathbf{b}} - E_c \sin \theta_{1,i} \hat{\mathbf{n}} + E_c \cos \theta_{1,i} \hat{\mathbf{t}} \\
  \mathbf{B} & = B_b \hat{\mathbf{b}} + B_c \hat{\mathbf{c}} = \\
             & = \frac{E_c}{c} \hat{\mathbf{b}} - \frac{E_b}{c} \hat{\mathbf{c}} = \\
             & = \frac{E_c}{c} \hat{\mathbf{b}} + \frac{E_b}{c} \sin \theta_{1,i} \hat{\mathbf{n}} - \frac{E_b}{c} \cos \theta_{1,i} \hat{\mathbf{t}} \ .
\end{aligned}\end{split}
\end{equation*}
\sphinxAtStartPar
so that jump relations become
\begin{equation*}
\begin{split}\begin{cases}
  b: & \quad E_{b,1} = E_{b,2} \\
  n: & \quad \dots \\
  t: & \quad \dots \\
\end{cases}
\quad , \quad
\begin{cases}
  b: & \quad \dots \\
  n: & \quad \frac{E_{b,1}}{c_1} \sin \theta_{1,i} = \frac{E_{b,2}}{c_2} \sin \theta_{2,i}  \\
  t: & \quad \dots \\
\end{cases}\end{split}
\end{equation*}
\sphinxAtStartPar
thus \sphinxstylestrong{Snell’s law} follows
\begin{equation*}
\begin{split}\frac{\sin \theta_{1,i}}{\sin \theta_{2,t}} = \frac{c_2}{c_1} = \frac{n_1}{n_2} \ .\end{split}
\end{equation*}
\sphinxAtStartPar
\sphinxstylestrong{Incident, relfected and refracted wave.} Wave at interface in medium 1 ha the contribution of the incoming incident wave, and the reflected one.
\begin{equation*}
\begin{split}\begin{aligned}
\mathbf{e}_1(\mathbf{r},t) 
& = \mathbf{e}_i(\mathbf{r},t) + \mathbf{e}_r(\mathbf{r},t) = \\
& = \mathbf{E}_{i} e^{i \left( \mathbf{k}_i \cdot \mathbf{r} - \omega t \right)} + \mathbf{E}_{r} e^{i \left( \mathbf{k}_r \cdot \mathbf{r} - \omega t \right)} = \\
& = \left( \mathbf{E}_{i} e^{i \mathbf{k}_i \cdot \mathbf{r}} + \mathbf{E}_{r} e^{i \mathbf{k}_r \cdot \mathbf{r} } \right) e^{-i \omega t} 
\end{aligned}\end{split}
\end{equation*}
\sphinxAtStartPar
with
\begin{equation*}
\begin{split}\begin{aligned}
  \mathbf{k}_i & = k_{i,n} \hat{\mathbf{n}} + k_{i,t} \hat{\mathbf{t}} \\
  \mathbf{k}_r & = k_{r,n} \hat{\mathbf{n}} + k_{r,t} \hat{\mathbf{t}} \\
\end{aligned}\end{split}
\end{equation*}
\sphinxAtStartPar
At the interface, \(\mathbf{r}_s \cdot \hat{\mathbf{n}} = 0\), and thus
\begin{equation*}
\begin{split}\begin{aligned}
  \mathbf{e}_1(\mathbf{r}_s, t) & = \left( \mathbf{E}_i e^{i k_{i,t} x_t} + \mathbf{E}_r e^{i k_{r,t} x_t} \right) e^{-i\omega t} \\
  \mathbf{e}_2(\mathbf{r}_s, t) & =        \mathbf{E}_t e^{i k_{t,t} x_t} e^{-i \omega t} \\
\end{aligned}\end{split}
\end{equation*}
\sphinxAtStartPar
In order for the boundary conditions to be satisfied at all the points of the interface at each time,
\begin{equation*}
\begin{split}k_{i,t} = k_{r,t} = k_{t,t} \ .\end{split}
\end{equation*}
\sphinxAtStartPar
Exploiting the relation between the pulsation, the wave\sphinxhyphen{}length and the speed of light in media, \(c_a = \frac{\omega}{|\mathbf{k}_a|} = \frac{c}{n_a}\),
\begin{equation*}
\begin{split}|\mathbf{k}_i| = |\mathbf{k}_r| \qquad \rightarrow \qquad k_{r,n} = - k_{i,n}\end{split}
\end{equation*}\begin{equation*}
\begin{split}\frac{|\mathbf{k}_2|}{|\mathbf{k}_1|} = \frac{c_1}{c_2}\end{split}
\end{equation*}\begin{equation*}
\begin{split}\frac{k_{t,t}^2 + k_{t,n}^2}{k_{i,t}^2 + k_{i,n}^2} = \frac{c_1^2}{c_2^2}\end{split}
\end{equation*}\begin{equation*}
\begin{split}
\begin{aligned}
  k_{i,n} & = \ \ \ |\mathbf{k}_i| \, \cos \theta_i \\
  k_{r,n} & = -     |\mathbf{k}_r| \, \cos \theta_r \\
  k_{t,n} & = \ \ \ |\mathbf{k}_t| \, \cos \theta_t \\
\end{aligned}
\quad , \quad 
\begin{aligned}
  k_{i,t} & = \ \ \ |\mathbf{k}_i| \, \sin \theta_i \\
  k_{r,t} & = \ \ \ |\mathbf{k}_r| \, \sin \theta_r \\
  k_{t,t} & = \ \ \ |\mathbf{k}_t| \, \sin \theta_t \\
\end{aligned}
\end{split}
\end{equation*}\begin{equation*}
\begin{split}\begin{cases}
 E_n: & \quad \varepsilon_1   \left( E_{i,c} \sin \theta_i + E_{r,c} \sin \theta_r \right) = \varepsilon_2   E_{t,c} \sin \theta_{t} \\
 E_t: & \quad                        E_{i,c} \cos \theta_i - E_{r,c} \cos \theta_r         =                 E_{t,c} \cos \theta_{t} \\
 E_b: & \quad                        E_{i,b}               + E_{r,b}                       =                 E_{t,b}                 \\
 B_n: & \quad                        B_{i,c} \sin \theta_i + B_{r,c} \sin \theta_r         =                 B_{t,c} \sin \theta_{t} \\
 B_t: & \quad \frac{1}{\mu_1} \left( B_{i,c} \cos \theta_i - B_{r,c} \cos \theta_r \right) = \frac{1}{\mu_2} B_{t,c} \cos \theta_{t} \\
 B_b: & \quad \frac{1}{\mu_1} \left( B_{i,b}               + B_{r,b}               \right) = \frac{1}{\mu_2} B_{t,b}                 \\
\end{cases}\end{split}
\end{equation*}
\sphinxAtStartPar
Writing the magnetic field as a function of the wave\sphinxhyphen{}vector and the magnetic field, it’s possible to write 2 decoupled systems of equations
\begin{equation*}
\begin{split}\begin{cases}
 E_n: & \quad \varepsilon_1   \left( E_{i,c} \sin \theta_i + E_{r,c} \sin \theta_r \right) = \varepsilon_2   E_{t,c} \sin \theta_{t} \\
 E_t: & \quad                        E_{i,c} \cos \theta_i - E_{r,c} \cos \theta_r         =                 E_{t,c} \cos \theta_{t} \\
 B_b: & \quad \frac{1}{\mu_1} \left( \frac{E_{i,c}}{c_1}   + \frac{E_{r,c}}{c_1}   \right) = \frac{1}{\mu_2} \frac{E_{t,c}}{c_2}     \\
\end{cases}\end{split}
\end{equation*}\begin{equation*}
\begin{split}\begin{cases}
 E_b: & \quad                        E_{i,b}               + E_{r,b}                       =                 E_{t,b}                 \\
 B_n: & \quad                        \frac{E_{i,b}}{c_1} \sin \theta_i + \frac{E_{r,b}}{c_1} \sin \theta_r         =                 \frac{E_{t,b}}{c_2} \sin \theta_{t} \\
 B_t: & \quad \frac{1}{\mu_1} \left( \frac{E_{i,b}}{c_1} \cos \theta_i - \frac{E_{r,b}}{c_1} \cos \theta_r \right) = \frac{1}{\mu_2} \frac{E_{t,b}}{c_2} \cos \theta_{t} \\
\end{cases}\end{split}
\end{equation*}
\sphinxAtStartPar
The equations \(E_n\) and \(B_b\) are equivalent; \(E_b\) and \(B_n\) are equivalent as well, because of Snell’s law. Thus, defining
\begin{equation*}
\begin{split}
\begin{aligned}
  r_c & := \dfrac{E_{r,c}}{E_{i,c}} \\
  t_c & := \dfrac{E_{t,c}}{E_{i,c}} \\
\end{aligned}
\quad , \quad
\begin{aligned}
  r_b & := \dfrac{E_{r,b}}{E_{i,b}} \\
  t_b & := \dfrac{E_{t,b}}{E_{i,b}} \\
\end{aligned}
\end{split}
\end{equation*}
\sphinxAtStartPar
and \(\alpha_i := \frac{1}{\mu_i c_i}\). These system of equations can be written as two uncoupled linear systems of equations,

\sphinxAtStartPar
(for P\sphinxhyphen{}polarization \sphinxstylestrong{todo} \sphinxstyleemphasis{change index from \(c\) to \(p\)}; for S\sphinxhyphen{}polarization \sphinxstylestrong{todo} \sphinxstyleemphasis{change index from \(b\) to \(s\)})
\begin{equation*}
\begin{split}
& \begin{cases}
 E_t: & \quad  \cos \theta_i -  \cos \theta_r \, r_c = \cos \theta_{t} \, t_c \\
 B_b: & \quad  \alpha_1      +  \alpha_1      \, r_c =  \alpha_2       \, t_c \\
\end{cases}
\\
& \begin{cases}
 E_b: & \quad                          1 +                   r_b =  t_b \\
 B_t: & \quad  \alpha_1 \, \cos \theta_i -  \alpha_1 \, \cos \theta_r \, r_b = \alpha_2 \, \cos \theta_{t} \, t_b \\
\end{cases}
\end{split}
\end{equation*}
\sphinxAtStartPar
Calling \(\theta_i = \theta_r = \theta_1\), \(\theta_2 = \theta_t\), these linear systems can be written using matrix formalism,
\begin{equation*}
\begin{split}
& \begin{bmatrix} -1 & 1 \\ 1 & \frac{\alpha_2}{\alpha_1} \frac{\cos \theta_2}{\cos \theta_1} \end{bmatrix}
 \begin{bmatrix} r_b \\ t_b \end{bmatrix} = \begin{bmatrix} 1 \\ 1 \end{bmatrix}
\\
& \begin{bmatrix} 1 & \frac{\cos \theta_2}{\cos \theta_1} \\ -1 & \frac{\alpha_2}{\alpha_1} \end{bmatrix}
 \begin{bmatrix} r_c \\ t_c \end{bmatrix} = \begin{bmatrix} 1 \\ 1 \end{bmatrix}
\end{split}
\end{equation*}
\sphinxAtStartPar
\sphinxstylestrong{todo} \sphinxstyleemphasis{Analysis of the total reflection, forcing \(t_x = 0\). Check signs before}
\$\(
\begin{bmatrix} 1 & \frac{\cos \theta_2}{\cos \theta_1} \\ -1 & \frac{\alpha_2}{\alpha_1} \end{bmatrix}
 \begin{bmatrix} r_c \\ t_c \end{bmatrix} = \begin{bmatrix} 1 \\ 1 \end{bmatrix}
\qquad \rightarrow \qquad
\begin{bmatrix} r_c \\ t_c \end{bmatrix} = \dfrac{1}{\frac{\alpha_2}{\alpha_1} + \frac{\cos \theta_2}{\cos \theta_1}} \begin{bmatrix} \frac{\alpha_2}{\alpha_1} & - \frac{\cos \theta_2}{\cos \theta_1} \\ 1 & 1  \end{bmatrix} \begin{bmatrix} 1 \\ 1 \end{bmatrix} 
= \begin{bmatrix} \frac{\alpha_2 \cos \theta_1 - \alpha_1 \cos \theta_2}{\alpha_2 \cos \theta_1 + \alpha_1 \cos \theta_2} \\ \frac{2 \alpha_1 \cos \theta_1}{\alpha_2 \cos \theta_1 + \alpha_1 \cos \theta_2} \end{bmatrix}
\)\$
\begin{equation*}
\begin{split}
\begin{bmatrix} -1 & 1 \\ 1 & \frac{\alpha_2}{\alpha_1} \frac{\cos \theta_2}{\cos \theta_1} \end{bmatrix}
 \begin{bmatrix} r_b \\ t_b \end{bmatrix} = \begin{bmatrix} 1 \\ 1 \end{bmatrix}
\qquad \rightarrow \qquad
\begin{bmatrix} r_b \\ t_b \end{bmatrix} = \dfrac{1}{-\frac{\alpha_2}{\alpha_1} \frac{\cos \theta_2}{\cos \theta_1} - 1} \begin{bmatrix}  \frac{\alpha_2}{\alpha_1} \frac{\cos \theta_2}{\cos \theta_1} & -1 \\ -1 & -1  \end{bmatrix} \begin{bmatrix} 1 \\ 1 \end{bmatrix} 
= \begin{bmatrix} \frac{\alpha_1 \cos \theta_1 - \alpha_2 \cos \theta_2}{\alpha_1 \cos \theta_1 + \alpha_2 \cos \theta_2} \\ \frac{2 \alpha_1 \cos \theta_1}{\alpha_1 \cos \theta_1 + \alpha_2 \cos \theta_2} \end{bmatrix}
\end{split}
\end{equation*}


\sphinxAtStartPar
that can be recast with the wave impedance \(Z\),
\begin{equation*}
\begin{split}\alpha_1 = \frac{1}{\mu_1 c_1} = \frac{\sqrt{\mu_1 \varepsilon_1}}{\mu_1} = \sqrt{\dfrac{\varepsilon_1}{\mu_1}} =: \frac{1}{Z_1} \ ,\end{split}
\end{equation*}\begin{equation*}
\begin{split}
\begin{bmatrix} r_c \\ t_c \end{bmatrix} = \begin{bmatrix} \frac{Z_1 \cos \theta_1 - Z_2 \cos \theta_2}{Z_1 \cos \theta_1 + Z_2 \cos \theta_2} \\ \frac{2 Z_2 \cos \theta_1}{Z_1 \cos \theta_1 + Z_2 \cos \theta_2} \end{bmatrix}
\end{split}
\end{equation*}\begin{equation*}
\begin{split}
\begin{bmatrix} r_b \\ t_b \end{bmatrix} = \begin{bmatrix} \frac{Z_2 \cos \theta_1 - Z_1 \cos \theta_2}{Z_2 \cos \theta_1 + Z_1 \cos \theta_2} \\ \frac{2 Z_2 \cos \theta_1}{Z_2 \cos \theta_1 + Z_1 \cos \theta_2} \end{bmatrix}
\end{split}
\end{equation*}
\sphinxAtStartPar
\sphinxstylestrong{Energy balance and transmission coefficients.} Energy balance for a domain collapsing on the interface reduces to power flux balance, namely
\begin{equation*}
\begin{split}\oint_{\partial V} \mathbf{s} \cdot \hat{\mathbf{n}} = 0 \ ,\end{split}
\end{equation*}
\sphinxAtStartPar
with \(\mathbf{s} = \mathbf{e} \times \mathbf{h}\) the Poynting vector. For harmonic plane waves,
\begin{equation*}
\begin{split}\begin{aligned}
  \mathbf{s}(\mathbf{r},t) 
  & = \mathbf{e}(\mathbf{r},t) \times \mathbf{h}(\mathbf{r},t) = \\
  & = \frac{1}{\mu} \left[ \mathbf{E} e^{i(\mathbf{k} \cdot \mathbf{r} - \omega t)} + \mathbf{E}^* e^{-i(\mathbf{k} \cdot \mathbf{r} - \omega t)} \right] \times \left[ \mathbf{B} e^{i(\mathbf{k} \cdot \mathbf{r} - \omega t)} + \mathbf{B}^* e^{-i(\mathbf{k} \cdot \mathbf{r} - \omega t)}  \right] = \\
  & = \frac{1}{\mu} \left[ \, \mathbf{E} \times \mathbf{B} \, e^{i 2(\mathbf{k} \cdot \mathbf{r} - \omega t)} + c.c. \, \right] + \frac{1}{\mu} \left[ \, \mathbf{E} \times \mathbf{B}^* + c.c. \, \right] = \\
  & = \dots + \frac{1}{\mu} \mathbf{E} \times \left( \frac{1}{c} \hat{\mathbf{k}} \times \mathbf{E} \right)^* = \\
  & = \dots + \frac{1}{\mu c} \left( \mathbf{E} \cdot \mathbf{E}^* \right) \hat{\mathbf{k}} = \\
  & = \dots + \frac{1}{\mu c} | \mathbf{E} |^2 \hat{\mathbf{k}} \ .
  & = \dots + \alpha | \mathbf{E} |^2 \hat{\mathbf{k}} \ .
\end{aligned}\end{split}
\end{equation*}
\sphinxAtStartPar
For each one of the two polarizations, the following holds (\(\cos \theta\) comes from the doct product \(\hat{k} \cdot \hat{n}\) appearing in the surface integral),
\begin{equation*}
\begin{split}\alpha_1 \cos \theta_1 = \alpha_1 r_x^2 \, \cos \theta_1 + \alpha_2 t_x^2 \, \cos \theta_2 \ ,\end{split}
\end{equation*}
\sphinxAtStartPar
i.e. the sum of reflected and transmitted power equals the incident power.
\subsubsection*{Proof of the power balance, for P\sphinxhyphen{}polarization}

\sphinxAtStartPar
\sphinxstylestrong{todo} Here \(P\) is index \(c\)

\sphinxAtStartPar
Dividing by \(\alpha_1 \cos  \theta_1\)
\begin{equation*}
\begin{split}\begin{aligned}
 & \frac{1}{\alpha_1 \cos \theta_1} \left( \alpha_1 r_p^2 \, \cos \theta_1 + \alpha_2 t_p^2 \, \cos \theta_2 \right) = \\
 & = \frac{\left(\alpha_1 \cos \theta_1 - \alpha_2 \cos \theta_2\right)^2}{\left(\alpha_1 \cos \theta_1 + \alpha_2 \cos \theta_2\right)^2} + \frac{\alpha_2 \cos \theta_2}{\alpha_1 \cos \theta_1} \frac{\left( 2 \alpha_1 \cos \theta_1 \right)^2}{\left( \alpha_1 \cos \theta_1 + \alpha_2 \cos \theta_2 \right)^2} = \\
  & = \dfrac{1}{\left( \alpha_1 \cos \theta_1 + \alpha_2 \cos \theta_2 \right)^2} \left[ \alpha_1^2 \cos^2 \theta_1 - 2 \alpha_1 \alpha_2 \cos \theta_1 \cos \theta_2 + \alpha_2^2 \cos^2 \theta_2 + 4 \alpha_1 \alpha_2 \cos \theta_1 \cos \theta_2 \right] = \\
  & = 1 \ .
\end{aligned}\end{split}
\end{equation*}
\sphinxstepscope


\part{Elettrotecnica}

\sphinxstepscope




\chapter{Approssimazione circuitale}
\label{\detokenize{ch/circuits:approssimazione-circuitale}}\label{\detokenize{ch/circuits:classical-electromagnetism-circuits}}\label{\detokenize{ch/circuits::doc}}
\sphinxAtStartPar
\sphinxstylestrong{Circuiti elettrici.} \sphinxstyleemphasis{Condizioni per la validità dell’approssimazione circuitale; componenti elementari; regimi di utilizzo: stazionario, armonico (alternato), transitorio;}

\sphinxAtStartPar
\sphinxstylestrong{Circuiti elettromagnetici.} \sphinxstyleemphasis{Condizioni per la validità dell’approssimazione circuitale; es. trasformatori}

\sphinxAtStartPar
\sphinxstylestrong{Circuito elettro\sphinxhyphen{}magneto\sphinxhyphen{}meccanici.} \sphinxstyleemphasis{Es. semplici circuiti; motori elettrici e generatori}

\sphinxstepscope




\section{Circuiti elettrici}
\label{\detokenize{ch/circuits-electric:circuiti-elettrici}}\label{\detokenize{ch/circuits-electric:classical-electromagnetism-circuits-electric}}\label{\detokenize{ch/circuits-electric::doc}}
\sphinxAtStartPar
Se il sistema di interesse soddisfa alcune condizioni, è possibile ridurre la teoria di campo dell’elettromagnetismo a una teoria circuitale.
Quando possibile, cioè quando capace di descrivere adeguatamente il comportamento del sistema di interesse, l’approccio circuitale semplifica di molto la descrizione del problema, non richiedendo la soluzione di un sistema di equazioni differenziali alle derivate parziali da risolvere nello spazio, ma la soluzione di equazioni differenziali ordinarie nelle incognite circuitali, che si riduce a un sistema algebrico, spesso lineare, in regime stazionario.

\sphinxAtStartPar
\sphinxstylestrong{Giustificazione dell’approccio circuitale.}

\sphinxAtStartPar
\sphinxstylestrong{Componenti elementari di un circuito elettrico.}

\sphinxstepscope




\subsection{Validità dell’approccio circuitale}
\label{\detokenize{ch/circuits-electric-approximation:validita-dell-approccio-circuitale}}\label{\detokenize{ch/circuits-electric-approximation:classical-electromagnetism-circuits-electric-approximation}}\label{\detokenize{ch/circuits-electric-approximation::doc}}
\sphinxAtStartPar
L’approccio circuitale consente di ridurre il problema elettromagnetico, in generale un problema di campo che richiede la soluzione di PDE, a un approccio «ai morsetti» \sphinxstylestrong{todo}, che richiede la soluzione di ODE.

\sphinxAtStartPar
Una rivisitazione dell”{\hyperref[\detokenize{ch/energy-linear:classical-electromagnetism-energy}]{\sphinxcrossref{\DUrole{std,std-ref}{equazione dell’energia}}}} permette di valutare i regimi in cui è possibile usare un approccio circuitale a un sistema elettromagnetico.

\sphinxAtStartPar
In particolare, nell’equazione di bilancio dell’energia elettromagnetica
\begin{equation*}
\begin{split}\dfrac{d}{dt} \int_V u = \oint_{\partial V} \mathbf{s} \cdot \hat{\mathbf{n}} - \int_V \mathbf{j} \cdot \mathbf{e} \ ,\end{split}
\end{equation*}
\sphinxAtStartPar
viene indagato il termine di flusso alla frontiera, ricordando la definizione di vettore di Poynting \(\mathbf{s} := \mathbf{e} \times \mathbf{h}\), e riscrivendo i campi elettrico e magnetico in funzione dei potenziali elettromagnetici, \(\mathbf{b} = \nabla \times \mathbf{a}\), \(\mathbf{e} = - \nabla \varphi - \partial_t \mathbf{a} \ ,\)
\begin{equation*}
\begin{split}\begin{aligned}
  - \oint_{\partial V} \mathbf{s} \cdot \mathbf{\hat{n}}
  & = - \oint_{\partial V} \left(\mathbf{e} \times \mathbf{h} \right) \cdot \mathbf{\hat{n}} = \\
  & =   \oint_{\partial V} \left(\nabla \varphi + \partial_t \mathbf{a} \right) \times \mathbf{h}  \cdot \mathbf{\hat{n}} = \\
  & = ... \\
  & = \underbrace{\oint_{\partial V} \hat{\mathbf{n}} \cdot \nabla \times ( \varphi \mathbf{h} )}_{=0 \text{ (Stokes'thm **todo** check)}} - \oint_{\partial V} \varphi \hat{\mathbf{n}} \cdot \underbrace{\nabla \times \mathbf{h}}_{\partial_t \mathbf{d} + \mathbf{j}} + \oint_{\partial V} \hat{\mathbf{n}} \cdot \partial_t \mathbf{a} \times \mathbf{h} = \\
  & = - \oint_{\partial V} \varphi \mathbf{j} \cdot \hat{\mathbf{n}} - \oint_{\partial V} \hat{\mathbf{n}} \cdot \left( \partial_t \mathbf{d} + \mathbf{h} \times \partial_t \mathbf{a} \right) \ , 
\end{aligned}\end{split}
\end{equation*}
\sphinxAtStartPar
e assumendo che il flusso di carica elettrica avvenga solo in corrispondenza di un numero finito di sezioni \(S_k \in \partial V\) equipotenziali a potenziale \(v_k = -\varphi_k\), costante sulle sezioni, e riconoscento il flusso di carica elettrica attraverso la sezione \(S_k\) come la corrente \(i_k = \int_{S_k} \mathbf{j} \cdot \hat{\mathbf{n}}\), si può scrivere
\begin{equation*}
\begin{split}- \oint_{\partial V} \mathbf{s} \cdot \hat{\mathbf{n}} = \sum_k v_k \, i_k - \oint_{\partial V} \hat{\mathbf{n}} \cdot \left( \partial_t \mathbf{d} + \mathbf{h} \times \partial_t \mathbf{a} \right) \ .\end{split}
\end{equation*}
\sphinxAtStartPar
Il bilancio di energia elettromagnetica del sistema può quindi essere riscritto come
\begin{equation*}
\begin{split}\frac{d}{dt} \int_V u = \sum_k v_k \, i_k - \int_{V} \mathbf{j} \cdot \mathbf{e} - \oint_{\partial V} \hat{\mathbf{n}} \cdot \left( \partial_t \mathbf{d} + \mathbf{h} \times \partial_t \mathbf{a} \right) \ .\end{split}
\end{equation*}
\sphinxAtStartPar
Nelle condizioni in cui l’ultimo termine è nullo o trascurabile (\sphinxstylestrong{todo} \sphinxstyleemphasis{quali? Spendere due parole sulla validità dell’approssimazione, con analisi dimensionale? Fare esempio in cui l’approssimazione non funziona}), la variazione di energia interna al sistema è dovuta alla differenza della potenza in ingresso ai morsetti, e la dissipazione all’interno del volume (ad esempio dovuta alla conduzione non ideale in conduttori con resistività finita),
\begin{equation*}
\begin{split}\dot{E}^{em} = P^{ext, vi} - \dot{D} \ ,\end{split}
\end{equation*}
\sphinxAtStartPar
con \(\dot{D} \ge 0\) per il secondo principio della termodinamica \sphinxstylestrong{todo} \sphinxstyleemphasis{aggiungere riferimento, e discussione.}

\sphinxstepscope




\subsection{Induzione elettromagnetica nell’approssimazione circuitale}
\label{\detokenize{ch/circuits-electric-induction:induzione-elettromagnetica-nell-approssimazione-circuitale}}\label{\detokenize{ch/circuits-electric-induction:classical-electromagnetism-circuits-electric-induction}}\label{\detokenize{ch/circuits-electric-induction::doc}}
\sphinxAtStartPar
E” possibile applicare l’approssimazione circuitale anche in presenza di regioni in cui non è possibile trascurare il termine \(\partial_t \mathbf{b}\), come ad esempio circuiti elettromagnetici che coinvolgono trasformatori e/o motori o generatori elettrici.

\sphinxAtStartPar
In queste situazioni, se è possibile identificare una regione \(V_0\) dello spazio connessa nella quale il termine \(\partial_t \mathbf{b} = \mathbf{0}\), e quindi \(\nabla \times \mathbf{e} = \mathbf{0}\), in \(V_0\) è possibile definire il campo elettrico in termini di un potenziale \(\varphi\),
\begin{equation*}
\begin{split}\mathbf{e} = - \nabla \varphi \qquad , \qquad \mathbf{r} \in V_0 \ .\end{split}
\end{equation*}
\sphinxAtStartPar
E” possibile calcolare le differenze di potenziale ai morsetti di un sistema in cui \(\delta_t \mathbf{b} \ne 0\), racchiuso nel volume \(V_k\), con la legge di Faraday,
\begin{equation*}
\begin{split}\oint_{\ell_k} \mathbf{e} \cdot \hat{\mathbf{t}} = - \frac{d}{dt} \int_{S_k} \mathbf{b} \cdot \hat{\mathbf{n}} \ ,\end{split}
\end{equation*}
\sphinxAtStartPar
dove il percorso chiuso \(\ell_k = \ell_k^{cond} \cup \ell_k^{mors}\) descrive il conduttore in \(V_k\) chiuso dalla linea geometrica tra i morsetti. Se si può trascurare la resistività del conduttore in \(V_k\), \(\int_{\ell_k^{cond}} \mathbf{e} \cdot \hat{\mathbf{t}} = 0\), la differenza di tensione ai morsetti vale
\begin{equation*}
\begin{split}\Delta v_k = \int_{\ell^{mors}_k} \mathbf{e} \cdot \hat{\mathbf{t}} = - \frac{d}{dt} \int_{S_k} \mathbf{b} \cdot \hat{\mathbf{n}}\end{split}
\end{equation*}
\sphinxstepscope




\subsection{Componenti elementari dei circuiti elettrici}
\label{\detokenize{ch/circuits-electric-components:componenti-elementari-dei-circuiti-elettrici}}\label{\detokenize{ch/circuits-electric-components:classical-electromagnetism-circuits-electric-components}}\label{\detokenize{ch/circuits-electric-components::doc}}

\subsubsection{Resistore ohmico}
\label{\detokenize{ch/circuits-electric-components:resistore-ohmico}}
\sphinxAtStartPar
Un resistore di Ohm risulta dall’approssimazione circuitale di un materiale con equazione costitutiva lineare
\begin{equation*}
\begin{split}\mathbf{e} = \rho_R \, \mathbf{j} \ ,\end{split}
\end{equation*}
\sphinxAtStartPar
tra il campo elettrico \(\mathbf{e}\) e la densità di corrente \(\mathbf{j}\), tramite la costante di proporzionalità \(\rho_R\), la \sphinxstylestrong{resistività} del materiale. La corrente elettrica attraverso una sezione del componente è definita come il flusso di carica attraverso una sua sezione
\begin{equation*}
\begin{split}i = \int_S \mathbf{j} \cdot \hat{\mathbf{t}} \simeq j \, A \ ,\end{split}
\end{equation*}
\sphinxAtStartPar
Nell’ipotesi che il vettore densità di corrente si allineato con l’asse del componente e uniforme sulla sezione \(A\), «piccola».
Se il materiale non è in grado di accumulare carica, il bilancio di carica elettrica si traduce nella continuità della corrente elettrica attraverso le sezioni del conduttore.

\sphinxAtStartPar
Utilizzando l’equazione costitutiva su un elemento di lunghezza elementare \(d\mathbf{r} =\hat{\mathbf{t}} \, d \ell \), e assumendo che il campo elettrico sia allineato con l’asse del componente, \(\mathbf{e} = e \hat{\mathbf{t}}\) si può scrivere il lavoro elementare per unità di carica come
\begin{equation*}
\begin{split}\delta v = \mathbf{e} \cdot d \mathbf{r} =  e \, d\ell = \rho_R \, j \, d\ell =  \frac{\rho_R \, d\ell}{A} i \ .\end{split}
\end{equation*}
\sphinxAtStartPar
Da questa ultima equazione seguono le due leggi di Ohm, per resistori lineari.

\sphinxAtStartPar
\sphinxstylestrong{Prima legge di Ohm.} La differenza di potenziale tra due sezioni di un resistore lineare è proporzionale alla corrente che passa attraverso di esso,
\begin{equation*}
\begin{split}\delta v = dR \, i \ .\end{split}
\end{equation*}
\sphinxAtStartPar
\sphinxstylestrong{Seconda legge di Ohm.} La costante di proporzionalità che lega la differenza di potenziale e la corrente all’interno di un resistore ohmico, la \sphinxstylestrong{resistenza} del resistore, è proporzionale alla resistività e alla lunghezza del resistore, e inversamente proporzionale alla sua sezione,
\begin{equation*}
\begin{split}dR = \frac{\rho_R \ d\ell}{A} \ .\end{split}
\end{equation*}
\sphinxAtStartPar
Se le proprietà sono uniformi nel resistore, si possono integrare le relazioni elementari per ottenere la relazione tra grandezze finite,
\begin{equation*}
\begin{split}\Delta V = R \, i \end{split}
\end{equation*}\begin{equation*}
\begin{split}R = \frac{\rho_R \ \ell}{A}\end{split}
\end{equation*}
\sphinxAtStartPar
\sphinxstylestrong{todo} (perché si può usare il potenziale? Nelle mie note avevo usato il simbolo \(v^*\), come se fosse una definizione leggermente diversa per incorporare movimento e instazionarietà, che si riduce a \(v\) nel caso stazionario).

\sphinxAtStartPar
\sphinxstylestrong{Condensatore.}

\sphinxAtStartPar
\sphinxstylestrong{Induttore.}

\sphinxAtStartPar
\sphinxstylestrong{Generatore di tensione.}

\sphinxAtStartPar
\sphinxstylestrong{Generatore di corrente.}

\sphinxstepscope




\subsection{Regimi di funzionamento in circuiti elettrici}
\label{\detokenize{ch/circuits-electric-regimes:regimi-di-funzionamento-in-circuiti-elettrici}}\label{\detokenize{ch/circuits-electric-regimes:classical-electromagnetism-circuits-electric-regimes}}\label{\detokenize{ch/circuits-electric-regimes::doc}}
\sphinxstepscope




\section{Circuiti elettromagnetici}
\label{\detokenize{ch/circuits-electromagnetic:circuiti-elettromagnetici}}\label{\detokenize{ch/circuits-electromagnetic:classical-electromagnetism-circuits-electromagnetic}}\label{\detokenize{ch/circuits-electromagnetic::doc}}
\sphinxAtStartPar
Sotto opportune ipotesi è possibile usare un modello circuitale anche per sistemi elettromagnetici, come ad esempio i trasformatori, o i motori elettrici.
\begin{itemize}
\item {} 
\sphinxAtStartPar
legge di Gauss per il campo magnetico
\begin{equation*}
\begin{split}\nabla \cdot \mathbf{b} = 0\end{split}
\end{equation*}
\item {} 
\sphinxAtStartPar
legge di Ampére\sphinxhyphen{}Maxwell
\begin{equation*}
\begin{split}\nabla \times \mathbf{h} - \partial_t \mathbf{d} = \mathbf{j}\end{split}
\end{equation*}
\end{itemize}

\sphinxAtStartPar
Si aggiungono le seguenti ipotesi:
\begin{itemize}
\item {} 
\sphinxAtStartPar
materiali lineari non\sphinxhyphen{}dissipativi e non\sphinxhyphen{}dispersivi \(\mathbf{b} = \mu \mathbf{h}\) \sphinxstylestrong{todo} discutere questa ipotesi, insieme a isteresi materiali, cicli di magnetizzazione,….

\item {} 
\sphinxAtStartPar
variazioni del campo \(\mathbf{d}\) nel tempo trascurabili, \(\partial_t \mathbf{d} = \mathbf{0}\).

\end{itemize}

\sphinxAtStartPar
La legge di Gauss del campo magnetico in forma integrale permette di scrivere la \sphinxstylestrong{legge ai nodi} del flusso del campo magnetico per i circuiti magnetici,
\begin{equation*}
\begin{split}0 = \oint_{\partial V} \mathbf{b} \cdot \hat{\mathbf{n}} = \sum_k \phi_k \ .\end{split}
\end{equation*}
\sphinxAtStartPar
La legge di Ampére\sphinxhyphen{}Maxwell in forma integrale considerando:
\begin{itemize}
\item {} 
\sphinxAtStartPar
un percorso incatenato con il solo induttore
\begin{equation*}
\begin{split}\int_{\ell_{ind}} \mathbf{h} \cdot \hat{\mathbf{t}} + \int_{\ell_{12}} \mathbf{h} \cdot \hat{\mathbf{t}} = \oint_{\ell_{1}} \mathbf{h} \cdot \hat{\mathbf{t}} = \int_{S^{ind}} \mathbf{j} \cdot \hat{\mathbf{n}} =  N i =: m\end{split}
\end{equation*}
\item {} 
\sphinxAtStartPar
un percorso incatenato con il traferro, aggirando l’induttore
\begin{equation*}
\begin{split}0 = \int_{\ell_{traf}} \mathbf{h} \cdot \hat{\mathbf{t}} + \int_{\ell_{21}} \hat{h} \cdot \hat{\mathbf{t}} = \sum_{k} h_k \ell_k + \int_{\ell_{21}} \hat{h} \cdot \hat{\mathbf{t}}\end{split}
\end{equation*}
\end{itemize}

\sphinxAtStartPar
e sommando le due equazioni, riconoscendo che i due integrali di linea sullo stesso percorsoin versi opposti si annullano, si ottiene la \sphinxstylestrong{legge alle maglie} per i circuiti magnetici
\begin{equation*}
\begin{split}\begin{aligned}
  m & = \int_{\ell_{ind}} \mathbf{h} \cdot \hat{\mathbf{t}} + \int_{\ell_{traf}} \mathbf{h} \cdot \hat{\mathbf{t}} = \\
    & \approx \sum_{k \in \ell} h_k \, \ell_k 
      = \sum_{k \in \ell} \frac{b_k}{\mu_k} \, \ell_k 
      = \sum_{k \in \ell} \frac{\ell_k}{\mu_k \, A_k} \, \phi_k  \ .
\end{aligned}\end{split}
\end{equation*}
\sphinxAtStartPar
Le leggi di Kirchhoff per i circuiti magnetici sono quindi
\begin{equation*}
\begin{split}\begin{cases}
  \sum_{k \in N_j} \phi_k = 0 \\ \\
  m_{\ell_i} = \sum_{k \in \ell_i} \theta_k \phi_k \ ,
\end{cases}\end{split}
\end{equation*}
\sphinxAtStartPar
avendo introdotto la riluttanza \(\theta_k = \frac{\ell_k}{\mu_k \, A_k}\), l’inverso della permeanza \(\Lambda_k = \theta_k^{-1}\).

\sphinxstepscope




\subsection{Trasformatore}
\label{\detokenize{ch/circuits-electromagnetic-transformer:trasformatore}}\label{\detokenize{ch/circuits-electromagnetic-transformer:classical-electromagnetism-circuits-electromagnetic-transformer}}\label{\detokenize{ch/circuits-electromagnetic-transformer::doc}}\begin{itemize}
\item {} 
\sphinxAtStartPar
flusso del campo magnetico, nell’ipotesi di campo uniforme, o in termini del campo medio
\begin{equation*}
\begin{split}\phi = b \, A\end{split}
\end{equation*}
\item {} 
\sphinxAtStartPar
flusso del campo magnetico concatenato a \(N\) avvolgimenti
\begin{equation*}
\begin{split}\psi = N \, \phi\end{split}
\end{equation*}
\item {} 
\sphinxAtStartPar
relazione tra tensione ai morsetti dell’induttore e flusso concatenato, applicando la {\hyperref[\detokenize{ch/circuits-electric-induction:classical-electromagnetism-circuits-electric-induction}]{\sphinxcrossref{\DUrole{std,std-ref}{legge di Faraday solo in parte irrotazionali}}}}
\begin{equation*}
\begin{split}v = \dot{\psi}\end{split}
\end{equation*}
\end{itemize}


\subsubsection{Trasformatore ideale}
\label{\detokenize{ch/circuits-electromagnetic-transformer:trasformatore-ideale}}
\sphinxAtStartPar
In assenza di flussi dispersi e riluttanza nel traferro, la legge alle maglie nel traferro implica
\begin{equation*}
\begin{split}0 = m_1 + m_2 = N_1 \, i_1 + N_2 \, i_2\end{split}
\end{equation*}
\sphinxAtStartPar
Il flusso del campo magnetico può essere scritto in funzione del flusso concatenato agli avvolgimenti,
\begin{equation*}
\begin{split}\phi = \frac{\psi_1}{N_1} = \frac{\psi_2}{N_2}\end{split}
\end{equation*}
\sphinxAtStartPar
La derivata nel tempo di questa relazione, con numero di avvolgimenti costanti nel tempo, implica
\begin{equation*}
\begin{split}\frac{v_2}{N_2} = \frac{v_1}{N_1} \ .\end{split}
\end{equation*}

\subsubsection{Trasformatore con flussi dispersi}
\label{\detokenize{ch/circuits-electromagnetic-transformer:trasformatore-con-flussi-dispersi}}\begin{equation*}
\begin{split}\begin{cases}
 & \phi_1 - \phi_{1,d} = \phi \\
 & \phi_2 - \phi_{2,d} = \phi \\
 & m_1 = \theta_{1,d} \phi_{1,d} \\
 & m_2 = \theta_{2,d} \phi_{2,d} \\
 & m_1 + m_2 = 0
\end{cases}\end{split}
\end{equation*}\begin{equation*}
\begin{split}\rightarrow \qquad 0 = m_1 + m_2 = N_1 \, i_1 + N_2 \, i_2\end{split}
\end{equation*}\begin{equation*}
\begin{split}\begin{aligned}
  0 & = \phi_2 - \phi_1 - \phi_{2,d} + \phi_{1,d} \\
    & = \phi_2 - \phi_1 - \frac{m_2}{\theta_{2,d}} + \frac{m_1}{\theta_{1,d}} \\
\end{aligned}\end{split}
\end{equation*}\begin{equation*}
\begin{split}\rightarrow \qquad \frac{\psi_2}{N_2} - \frac{m_2}{\theta_{2,d}} = \frac{\psi_1}{N_1} - \frac{m_1}{\theta_{1,d}} \ .\end{split}
\end{equation*}\begin{equation*}
\begin{split}\rightarrow \qquad \frac{1}{N_2} \left( v_2 - \frac{N_2^2}{\theta_{2,d}} \dfrac{d i_2}{d t} \right) =  
                     \frac{1}{N_1} \left( v_1 - \frac{N_1^2}{\theta_{1,d}} \dfrac{d i_1}{d t} \right)  \ .\end{split}
\end{equation*}

\subsubsection{Trasformatore con flussi dispersi e riluttanza \protect\(\theta_{Fe}\protect\) nel traferro}
\label{\detokenize{ch/circuits-electromagnetic-transformer:trasformatore-con-flussi-dispersi-e-riluttanza-theta-fe-nel-traferro}}\begin{equation*}
\begin{split}\begin{cases}
 & \phi_{1} - \phi_{1,d} = \phi \\
 & \phi_{2} - \phi_{2,d} = \phi \\
 & m_{1} = \theta_{1,d} \phi_{1,d} \\
 & m_{2} = \theta_{2,d} \phi_{2,d} \\
 & m_1   + m_{2} = \theta_{Fe} \, \phi
\end{cases}\end{split}
\end{equation*}
\sphinxAtStartPar
\sphinxstylestrong{todo} finire e controllare i conti; disegnare circuito equivalente



\sphinxstepscope




\section{Circuiti elettromeccanici}
\label{\detokenize{ch/circuits-electromechanic:circuiti-elettromeccanici}}\label{\detokenize{ch/circuits-electromechanic:classical-electromagnetism-circuits-electromechanic}}\label{\detokenize{ch/circuits-electromechanic::doc}}
\sphinxAtStartPar
Alcuni sistemi di interesse e di enorme diffusione nella società moderna sfruttano le interazioni tra componenti fenomeni elettromagnetici e meccanici: un esempio fondamentale sono le macchine elettriche, alcune delle quali possono operare sia come motore (con la potenza fornita dal sistema elettrico e convertita in potenza meccanica) sia come generatore di energia elettrica (convertendo potenza meccanica in potenza elettrica).

\sphinxAtStartPar
In un sistema di induttori con mutua influenza, la differenza di tensione ai capi dell’induttore «potenziato» \(i\) è
\begin{equation*}
\begin{split}v_i = \dot{\psi}_i = \dfrac{d}{dt} \left( N_i \, \phi_i \right) \ .\end{split}
\end{equation*}
\sphinxAtStartPar
Il flusso concatenato dipende dall’effetto di tutti gli induttori del sistema (e del campo magnetico generato da eventuali cause esterne al sistema),
\begin{equation*}
\begin{split}\phi_i = \sum_{k} \phi_{ik} = \sum_{k} \frac{1}{\theta_{ik}} \, m_k \ ,\end{split}
\end{equation*}
\sphinxAtStartPar
avendo indicato con \(\theta_{ik}\) la riluttanza del circuito tra l’induttore potenziante \(k\) e l’induttore potenziato \(i\). Usando l’espressione della forza magneto\sphinxhyphen{}motrice \(m_k = N_k \, i_k\), si può riscrivere l’espressione della differenza di tensione
\begin{equation*}
\begin{split}v_i = \sum_k \frac{d}{dt} \left( \frac{N_i \, N_k}{\theta_{ik}} i_k \right) = \sum_k \frac{d}{dt} \left( L_{ik} \, i_k \right) \ .\end{split}
\end{equation*}
\sphinxAtStartPar
In genereale, in circuiti elettromeccanici le riluttanze non sono dei parametri costanti del sistema ma dipendono dallo stato «meccanico» del sistema, descritto qui dalle variabili \(\mathbf{x}\),
\begin{equation*}
\begin{split}v_i = \sum_k \frac{d}{dt} \left( \frac{N_i \, N_k}{\theta_{ik}(\mathbf{x})} i_k \right) = \sum_k \frac{d}{dt} \left( L_{ik} (\mathbf{x}) \, i_k \right) \ .\end{split}
\end{equation*}\begin{equation*}
\begin{split}\mathbf{v}(t) = \dfrac{d}{dt} \Big( \mathbf{L}(\mathbf{x}(t)) \, \mathbf{i}(t) \Big) \ .\end{split}
\end{equation*}
\sphinxAtStartPar
La matrice di induttanza \(\mathbf{L}\) è simmetrica \sphinxstylestrong{todo} \sphinxstyleemphasis{Dimostrazione}
\label{ch/circuits-electromechanic:example-0}
\begin{sphinxadmonition}{note}{Example 10.3.1}



\sphinxAtStartPar
Given an constant and uniform magnetic field \(\mathbf{b}(r) = \mathbf{B}\) in a region of space where a simple electric circuit is placed. The electric circuit consists in a simple circuit with a resistance \(R\) as a lumped load, and has a rectangular shape. Three sides are fixed, and the distance between the pair of parallel fixed sides is \(\ell\); the fourth side can move and its distance between the parallel fixed side is \(x\). The unit vector orthogonal to the rectangular surface enclosed in the circuit is \(\hat{\mathbf{n}}\).

\sphinxAtStartPar
A mechanical system provides the prescribed motion \(x(t) = x_0 + \Delta x \sin(\Omega t)\) to the moving side. It’s asked to evaluate and discuss:
\begin{itemize}
\item {} 
\sphinxAtStartPar
voltage at the electric port of the load

\item {} 
\sphinxAtStartPar
energy balance

\end{itemize}


\begin{savenotes}\sphinxattablestart
\centering
\begin{tabulary}{\linewidth}[t]{|T|}
\hline

\sphinxAtStartPar
\sphinxincludegraphics{{ex-00-loop}.jpg}
\\
\hline
\end{tabulary}
\par
\sphinxattableend\end{savenotes}

\sphinxAtStartPar
\sphinxstylestrong{Without considering the inductance of the simple circuit.} Faraday’s law
\begin{equation*}
\begin{split}\Gamma_{\partial s_t}(\mathbf{e}) + \dot{\Phi}_{s_t}(\mathbf{b}) = 0 \ ,\end{split}
\end{equation*}
\sphinxAtStartPar
provides the relation between the time derivative of the magnetic flux though two points of the electric circuit on opposite sides of the moving side of the circuit, corresponding to the voltage at the electric port of the load
\begin{equation*}
\begin{split}v = - \int_{\ell_0} \mathbf{e} \cdot \hat{t} = - \dot{\Phi}_{s_t}(\mathbf{b}) = - \dfrac{d}{dt} \left( N B A \right) = - B \ell \dot{x} \ ,\end{split}
\end{equation*}
\sphinxAtStartPar
being \(N = 1\), and \(B\) constant and uniform if self\sphinxhyphen{}inductance is not considered.
If the inductance of the circuit is neglected, from the constitutive equation of the resistance, \(v = R i\), and voltage Kirchhoff law, it follows that the current in the simple circuit is
\begin{equation*}
\begin{split}i = \frac{v}{R} = - \dot{\Phi}_{s_t}(\mathbf{b}) = - \frac{B_n \dot{A}}{R} = - \frac{B_n \, \ell \dot{x}}{R} = - \frac{B_n \, \ell \, \Delta x}{R} \, \Omega \cos(\Omega t) \ .\end{split}
\end{equation*}
\sphinxAtStartPar
The force acting on a wire conducting electric current \(i\) in a uniform magnetic field \(\mathbf{B}\) is
\begin{equation*}
\begin{split}\mathbf{F} = - i \mathbf{B} \times \mathbf{l} \ .\end{split}
\end{equation*}
\sphinxAtStartPar
Calling \(y\) the «positive» direction of the moving side, and assuming \(\mathbf{B} = B \hat{\mathbf{z}}\), with \(\hat{\mathbf{z}} = \hat{\mathbf{x}} \times \hat{\mathbf{y}}\),
\begin{equation*}
\begin{split}\mathbf{F} = i B \ell \hat{\mathbf{x}} \ .\end{split}
\end{equation*}
\sphinxAtStartPar
Assuming negligible mass of the moving wire, the second principle of dynamics reduces to force equilibrium, so that the external force provided to the wire must be opposite to the force acting on the wire due to the EM field
\begin{equation*}
\begin{split}\mathbf{F}^e = - \mathbf{F} \ ,\end{split}
\end{equation*}
\sphinxAtStartPar
and the external power reads
\begin{equation*}
\begin{split}P^e = \dot{\mathbf{x}} \cdot \mathbf{F}^e = - i B \ell \dot{x} = \frac{B^2 \ell^2 \dot{x}^2}{R} = \frac{B^2 \ell^2 \left(\Delta x\right)^2}{R} \Omega^2 \cos^2(\Omega t) \ .\end{split}
\end{equation*}

\begin{savenotes}\sphinxattablestart
\centering
\begin{tabular}[t]{|\X{38}{100}|\X{62}{100}|}
\hline

\sphinxAtStartPar
\sphinxincludegraphics{{ex-00-circuit}.jpg}
&
\sphinxAtStartPar
\sphinxincludegraphics{{ex-00-force-em}.jpg}
\\
\hline
\end{tabular}
\par
\sphinxattableend\end{savenotes}

\sphinxAtStartPar
\sphinxstylestrong{Considering the inductance of the circuit and inertia of the wire.} Considering the self\sphinxhyphen{}induced magnetic flux \(\phi\),
\begin{equation*}
\begin{split}v = - \dfrac{d}{dt} \left( N \left( \phi + B A \right) \right) \ ,\end{split}
\end{equation*}
\sphinxAtStartPar
with \(\phi = \dfrac{m}{\theta} = \dfrac{N}{\theta} i\). The expression of the voltage a the port of the circuit can be recast as
\begin{equation*}
\begin{split}v = - \dfrac{d}{dt} \left( N B A \right) - \dfrac{d}{dt} \left( \frac{N^2}{\theta} i \right) = - \dfrac{d}{dt} \left( N B \ell x \right) - \dfrac{d}{dt} \left( L i \right) \ .\end{split}
\end{equation*}
\sphinxAtStartPar
Now, assuming everything constant except for the \(x\) and \(i\), and connecting this circuit to the load with constitutive equation, \(v = R i\), the dynamical equation of the electric circuit becomes
\begin{equation*}
\begin{split}L \dfrac{d i}{d t} + R i = - N B \ell \dfrac{d x}{d t} \ .\end{split}
\end{equation*}
\sphinxAtStartPar
The dynamical equation of the wire is
\begin{equation*}
\begin{split}\begin{aligned}
 m \dfrac{d^2 x}{d t^2} 
 & = F^{ext} + F^{EM} = \\
 & = F^{ext} + i B \ell \ .
\end{aligned}\end{split}
\end{equation*}
\sphinxAtStartPar
\sphinxstylestrong{Energy balance} immidiately follows after multiplying the circuit equation by \(i\), the dynamical equation by \(\dot{x}\) and summing,
\begin{equation*}
\begin{split}\dfrac{d}{dt} \underbrace{\left( \dfrac{1}{2} m |\dot{x}|^2 + \dfrac{1}{2} L i^2 \right)}_{\text{energy: kin.+em.}} + \underbrace{R i^2}_{\text{dissipation}} = \underbrace{F^{ext} \dot{x}}_{\text{ext. power done on the sys}} \ .\end{split}
\end{equation*}\end{sphinxadmonition}


\subsection{Sistemi elettromeccanici conservativi}
\label{\detokenize{ch/circuits-electromechanic:sistemi-elettromeccanici-conservativi}}
\sphinxAtStartPar
Le equazioni che governano il sistema elettromeccanico, senza condensatori, in generale possono essere scritte come
\begin{equation*}
\begin{split}\begin{cases}
 \mathbf{M} \ddot{\mathbf{x}} + \mathbf{D} \dot{\mathbf{x}} + \mathbf{K} \mathbf{x} = \mathbf{f}^{ext} + \mathbf{f}^{em} \\
 \dfrac{d}{dt} \left( \mathbf{L} \mathbf{i} \right) + \mathbf{R} \mathbf{i} = \mathbf{e}
\end{cases}\end{split}
\end{equation*}
\sphinxAtStartPar
In termini di energia,
\begin{equation*}
\begin{split}
0 = \dot{\mathbf{x}}^T \left[ \mathbf{M} \ddot{\mathbf{x}} + \mathbf{D} \dot{\mathbf{x}} + \mathbf{K} \mathbf{x} - \mathbf{f}^{ext} - \mathbf{f}^{em} \right] + \mathbf{i}^T \left[ \dfrac{d}{dt} \left( \mathbf{L} \mathbf{i} \right) + \mathbf{R} \mathbf{i} - \mathbf{e} \right]
\end{split}
\end{equation*}
\sphinxAtStartPar
Nel caso di matrici di massa, smorzamento e rigidzza costanti, e usando la derivata del prodotto per ottenere un termine di derivata dell’energia degli induttori sfruttando la simmetria di \(\mathbf{L}\),
\begin{equation}\label{equation:ch/circuits-electromechanic:classical-electromagnetism:circuits-electromechanic:energy-mech-0}
\begin{split} \begin{aligned}
\dfrac{d}{dt} \left[ \frac{1}{2} \mathbf{i}^T \mathbf{L} \mathbf{i} \right] 
  & = \mathbf{i}^T \dfrac{d}{dt} \left( \mathbf{L} \, \mathbf{i} \right) + \frac{1}{2} \mathbf{i}^T \dfrac{d \mathbf{L}}{dt} \mathbf{i} = \\
  & = \mathbf{i}^T \dfrac{d}{dt} \left( \mathbf{L} \, \mathbf{i} \right) + \sum_{a} \frac{1}{2} \mathbf{i}^T \dfrac{\partial \mathbf{L}}{\partial x_a} \mathbf{i} \, \dot{x}_a = \\
  & = \mathbf{i}^T \dfrac{d}{dt} \left( \mathbf{L} \, \mathbf{i} \right) + \nabla \left( \frac{1}{2} \mathbf{i}^T \mathbf{L} \mathbf{i} \right) \dot{\mathbf{x}}  \ .
\end{aligned}\end{split}
\end{equation}
\sphinxAtStartPar
si può scrivere un’equazione di bilancio dell’energia meccanica macroscopica, \(E^{mec, int}\)
\begin{equation*}
\begin{split}
0 & = \dfrac{d}{dt} \left[ \dfrac{1}{2} \dot{\mathbf{x}}^T \mathbf{M} \dot{\mathbf{x}} + \dfrac{1}{2} \mathbf{x}^T \mathbf{K} \mathbf{x} + \dfrac{1}{2} \mathbf{i}^T \mathbf{L} \mathbf{i} \right] - \dot{\mathbf{x}}^T \left( \mathbf{f}^{em} - \nabla E^{ind}(\mathbf{x}, \mathbf{i})  \right) + \\
  & - \dot{\mathbf{x}}^T \mathbf{f}^{ext} - \mathbf{i}^T \mathbf{e} + \\
  & + \dot{\mathbf{x}}^T \mathbf{C} \dot{\mathbf{x}} + \mathbf{i}^T \mathbf{R} \mathbf{i} \ .
\end{split}
\end{equation*}
\sphinxAtStartPar
Nell’ipotesi che il processo sia conservativo, si ricava la forma delle forze dovute ai fenomeni elettromagnetici,
\begin{equation}\label{equation:ch/circuits-electromechanic:classical-electromagnetism:circuits-electromechanic:f-em}
\begin{split}\mathbf{f}^{em} = \nabla_{\mathbf{x}} E^{ind}(\mathbf{x}, \mathbf{i}) \ .\end{split}
\end{equation}

\subsection{Equazioni di governo}
\label{\detokenize{ch/circuits-electromechanic:equazioni-di-governo}}
\sphinxAtStartPar
Usando l’espressione \eqref{equation:ch/circuits-electromechanic:classical-electromagnetism:circuits-electromechanic:f-em} delle azioni meccaniche dovute agli effetti elettromagnetici, del sistema sono
\begin{equation*}
\begin{split}\begin{cases}
  \mathbf{M} \ddot{\mathbf{x}} + \mathbf{D} \dot{\mathbf{x}} + \mathbf{K} \mathbf{x} - \nabla_{\mathbf{x}} E^{ind}(\mathbf{x}, \mathbf{i})  = \mathbf{f}^{ext} \\
  \frac{d}{dt} \left( \mathbf{L}(\mathbf{x}) \mathbf{i} \right) + \mathbf{R} \mathbf{i} = \mathbf{e}
\end{cases}\end{split}
\end{equation*}
\sphinxAtStartPar
o nel caso generale
\begin{equation*}
\begin{split}\begin{cases}
  \mathbf{M} \ddot{\mathbf{x}} - \nabla_{\mathbf{x}} E^{ind} ( \mathbf{x}, \mathbf{i}) = \mathbf{f}^{ext} \\
  \frac{d}{dt} \left( \mathbf{L}(\mathbf{x}) \mathbf{i} \right) + \mathbf{R} \mathbf{i} = \mathbf{e}
\end{cases}\end{split}
\end{equation*}

\subsection{Bilancio energetico}
\label{\detokenize{ch/circuits-electromechanic:bilancio-energetico}}

\subsubsection{Energia meccanica macroscopica}
\label{\detokenize{ch/circuits-electromechanic:energia-meccanica-macroscopica}}
\sphinxAtStartPar
Usando l’espressione \eqref{equation:ch/circuits-electromechanic:classical-electromagnetism:circuits-electromechanic:f-em} delle azioni meccaniche dovute ai fenomeni elettromagnetici, si può riscrivere la relazione \eqref{equation:ch/circuits-electromechanic:classical-electromagnetism:circuits-electromechanic:energy-mech-0}, come un bilancio di energia meccanica macroscopica del sistema,
\begin{equation*}
\begin{split}\dfrac{d}{dt} \left[ \dfrac{1}{2} \dot{\mathbf{x}}^T \mathbf{M} \dot{\mathbf{x}} + \dfrac{1}{2} \mathbf{x}^T \mathbf{K} \mathbf{x} + \dfrac{1}{2} \mathbf{i}^T \mathbf{L} \mathbf{i} \right] = \dot{\mathbf{x}}^T \mathbf{f}^{ext} + \mathbf{i}^T \mathbf{e} - \dot{\mathbf{x}}^T \mathbf{D} \dot{\mathbf{x}} - \mathbf{i}^T \mathbf{R} \mathbf{i} \ , \end{split}
\end{equation*}
\sphinxAtStartPar
e quindi
\begin{equation*}
\begin{split}\dot{E}^{mec} = P^{ext} - \dot{D} \ .\end{split}
\end{equation*}

\subsubsection{Energia cinetica}
\label{\detokenize{ch/circuits-electromechanic:energia-cinetica}}
\sphinxAtStartPar
L’energia meccanica macroscopica può essere scritta come la somma dell’energia cinetica e dell’energia potenziale interna del sistema, \(E^{mec} = K + V^{int}\). La derivata nel tempo dell’energia potenziale delle azioni interne è l’opposto della potenza delle azioni interne conservative, \(P^{int, c} = - \dot{V}^{int}\); la dissipazione è l’opposto della potenza delle azioni interne non\sphinxhyphen{}conservative, \(P^{int, nc} = - \dot{D}\). La potenza complessiva delle azioni interne può quindi essere scritta come
\begin{equation*}
\begin{split}P^{int} = P^{int, c} + P^{int, nc} = - \dot{V}^{int} - \dot{D} \ ,\end{split}
\end{equation*}\begin{equation*}
\begin{split}\dot{K} = \dot{E}^{mec} - \dot{V}^{int} = P^{ext} \underbrace{- \dot{D} - \dot{V}^{int}}_{=P^{int}} \  \end{split}
\end{equation*}

\subsubsection{Energia totale}
\label{\detokenize{ch/circuits-electromechanic:energia-totale}}
\sphinxAtStartPar
Il primo principio della termodinamica fornisce l’equazione di bilancio dell’energia totale di un sistema chiuso,
\begin{equation*}
\begin{split}\dot{E}^{tot} = P^{ext} + \dot{Q}^{ext} \ .\end{split}
\end{equation*}

\subsubsection{Energia interna}
\label{\detokenize{ch/circuits-electromechanic:energia-interna}}
\sphinxAtStartPar
L’energia interna di un sistema è definita come la differenza dell’energia totale e dell’energia cinetica macroscopica, \(E := E^{tot} - K\). L’equazione di bilancio dell’energia interna di un sistema chiuso è
\begin{equation*}
\begin{split}\dot{E} = Q^{ext} - P^{int} \ .\end{split}
\end{equation*}

\subsubsection{Energia interna termica (microscopica)}
\label{\detokenize{ch/circuits-electromechanic:energia-interna-termica-microscopica}}
\sphinxAtStartPar
Se si definisce l’energia interna termica, corrispondente all’energia cinetica associata alle dinamiche microscopiche, come differenza tra energia interna e energia potenziale interna, o differenza di energia totale ed energia meccanica macrsocopica,
\begin{equation*}
\begin{split}\begin{aligned}
  E^{th} & = E - V^{int} = \\
         & = E^{tot} - E^{mec} \ ,
\end{aligned}\end{split}
\end{equation*}
\sphinxAtStartPar
l’equazione di bilancio dell’energia interna termica è
\begin{equation*}
\begin{split}   \dot{E}^{th} = \dot{Q}^{ext} + \dot{D} \ . \end{split}
\end{equation*}\subsubsection*{Dimostrazione}
\begin{equation*}
\begin{split}\begin{aligned}
  \dot{E}^{th} = \dot{E} - V^{int}
    & = \dot{Q}^{ext} - P^{int} - V^{int} = \\
    & = \dot{Q}^{ext} + \dot{D} + \dot{V}^{int} - \dot{V}^{int} = \\
    & = \dot{Q}^{ext} + \dot{D} \ .
\end{aligned}\end{split}
\end{equation*}
\sphinxAtStartPar
\sphinxstylestrong{Con condensatori.} \sphinxstylestrong{todo}
\subsubsection*{Equazioni}
\begin{itemize}
\item {} 
\sphinxAtStartPar
\sphinxstylestrong{Leggi ai nodi.}
\begin{equation*}
\begin{split}0 = \sum_{k \in B_j} \alpha_{jk} \, i_{jk}\end{split}
\end{equation*}\begin{equation*}
\begin{split}\mathbf{A} \mathbf{i} = \mathbf{0}\end{split}
\end{equation*}
\item {} 
\sphinxAtStartPar
\sphinxstylestrong{Differenza di potenziale nodi\sphinxhyphen{}lati.}
\begin{equation*}
\begin{split}\mathbf{A}^T \mathbf{v}_{n} = \mathbf{v}\end{split}
\end{equation*}
\item {} 
\sphinxAtStartPar
\sphinxstylestrong{Nodo a terra.}
\begin{equation*}
\begin{split}\mathbf{v}_{\perp} = \mathbf{v}_0 \ .\end{split}
\end{equation*}
\item {} 
\sphinxAtStartPar
\sphinxstylestrong{Equazioni costitutive.}
\begin{equation*}
\begin{split}\begin{aligned}
    \mathbf{0} & = \mathbf{v}_R - \mathbf{R} \mathbf{i}_R & \text{resistenze} \\
    \mathbf{0} & = \mathbf{v}_L - \frac{d}{dt} \left( \mathbf{L} \mathbf{i}_L \right) & \text{induttanze} \\
    \mathbf{0} & = \frac{d}{dt} \left( C \mathbf{v}_C \right) - \mathbf{i}_C & \text{condensatori} \\
  \end{aligned}\end{split}
\end{equation*}
\end{itemize}

\sphinxstepscope


\section{Network analysis of linear circuits}
\label{\detokenize{ch/electrical-engineering-networks:network-analysis-of-linear-circuits}}\label{\detokenize{ch/electrical-engineering-networks:classical-electromagnetism-electrical-engineering-newtork-analysis}}\label{\detokenize{ch/electrical-engineering-networks::doc}}
\sphinxAtStartPar
Dynamical equations of a linear circuit can be written as a general linear state\sphinxhyphen{}space model
\begin{equation*}
\begin{split}\begin{cases}
  \mathbf{M} \dot{\mathbf{x}} = \mathbf{A} \mathbf{x} + \mathbf{B} \mathbf{u} \\
  \mathbf{y} = \mathbf{C} \mathbf{x} + \mathbf{D} \mathbf{u} \\
\end{cases}\end{split}
\end{equation*}
\sphinxAtStartPar
The mathematical problem is a system of DAE (dynamical\sphinxhyphen{}algebraic equations), as it includes:
\begin{itemize}
\item {} 
\sphinxAtStartPar
constitutive equations of the linear components

\item {} 
\sphinxAtStartPar
Kirchhoff laws for current at nodes and voltage in loops

\end{itemize}

\sphinxAtStartPar
Thus matrix \(\mathbf{M}\) is likely to be singular, here vector \(\mathbf{x}\) contains both dynamical (like voltage across a capacitor or current through an inductor) and algebraic grid variables, current and voltages whose time derivative doesn’t appear explicitly in the system of DAE.

\sphinxAtStartPar
\sphinxstylestrong{Different representations.} Possible choices of the unknowns:
\begin{enumerate}
\sphinxsetlistlabels{\arabic}{enumi}{enumii}{}{.}%
\item {} 
\sphinxAtStartPar
current through any side, voltage at any node

\item {} 
\sphinxAtStartPar
loop currents, voltage drops across any side.

\item {} 
\sphinxAtStartPar
… \sphinxstyleemphasis{any other (linear) combination on the physical quantities}

\end{enumerate}


\subsection{Thevenin equivalent}
\label{\detokenize{ch/electrical-engineering-networks:thevenin-equivalent}}\label{\detokenize{ch/electrical-engineering-networks:classical-electromagnetism-electrical-engineering-newtork-analysis-thevenin}}
\sphinxAtStartPar
\sphinxstylestrong{One\sphinxhyphen{}port.} Thevenin’s theorem states that any linear circuit can be reduced to a single voltage source and a single impedance in series.


\subsubsection{One\sphinxhyphen{}port circuit}
\label{\detokenize{ch/electrical-engineering-networks:one-port-circuit}}\label{\detokenize{ch/electrical-engineering-networks:classical-electromagnetism-electrical-engineering-newtork-analysis-thevenin-1-port}}
\sphinxAtStartPar
As the goal of Thevenin’s theorem is to find the constitutive equation of the network as \(v(i)\), the network is connected to an external current generator that prescribes \(i\) and the voltage \(v\) at the port is evaluated.

\sphinxAtStartPar
The input of the extended network is
\begin{equation*}
\begin{split}\mathbf{u} = ( \mathbf{u}_{gen}, i ) \ ,\end{split}
\end{equation*}
\sphinxAtStartPar
while the output is, or at least contains, the voltage \(v\)
\begin{equation*}
\begin{split}\mathbf{y} = \mathbf{C} \mathbf{x} + \mathbf{D} \mathbf{u} \ .\end{split}
\end{equation*}
\sphinxAtStartPar
The linear system can be written in Laplace domain as
\begin{equation*}
\begin{split}\begin{cases}
  s \mathbf{M} \mathbf{x} - \mathbf{M} \mathbf{x}_0  = \mathbf{A} \mathbf{x} + \mathbf{B} \mathbf{u} \\
  \mathbf{y} = \mathbf{C} \mathbf{x} + \mathbf{D} \mathbf{u} \\
\end{cases}\end{split}
\end{equation*}
\sphinxAtStartPar
The state and the output are the sum of the free response to non\sphinxhyphen{}zero initial conditions and forced response,
\begin{equation*}
\begin{split}\begin{cases}
  \mathbf{x} = (s \mathbf{M} - \mathbf{A})^{-1} \mathbf{M} \mathbf{x}_0 + (s \mathbf{M} - \mathbf{A})^{-1} \mathbf{B} \mathbf{u} \\
  \mathbf{y} = \mathbf{C} (s \mathbf{M} - \mathbf{A})^{-1} \mathbf{M} \mathbf{x}_0 + \left[ \mathbf{C} (s \mathbf{M} - \mathbf{A})^{-1} \mathbf{B} + \mathbf{D} \right] \mathbf{u} \\
\end{cases}\end{split}
\end{equation*}
\sphinxAtStartPar
Forced response can be further manipulated exploiting PSCE, evaluating the effect of one input at a time, setting all the other inputs equal to zero.
\begin{itemize}
\item {} 
\sphinxAtStartPar
the effect of setting the input of the external current generator, \(i = 0\), is equivalent to evaluate the system with an open circuit at the port

\item {} 
\sphinxAtStartPar
the effect of setting equal to zero a tension generator, \(e = 0\), is equivalent to a short\sphinxhyphen{}circuit on the same side

\item {} 
\sphinxAtStartPar
the effect of setting equal to zero a current generator, \(a = 0\), is equivalent to an open circuit on the same side

\end{itemize}

\sphinxAtStartPar
If the system is \sphinxstylestrong{asymptotically stable}, the free response is approximately zero when the \sphinxstylestrong{transient dynamics is over}, and the output equals the forced output. Introducing the transfer function
\begin{equation*}
\begin{split}\mathbf{G}(s) = [ \ \mathbf{G}_{gen}(s) \quad \mathbf{G}_i(s) \ ] \ ,\end{split}
\end{equation*}
\sphinxAtStartPar
the input\sphinxhyphen{}output relation reads
\begin{equation*}
\begin{split}\begin{aligned}
   v = \mathbf{G}(s) \mathbf{u}
   & = \mathbf{G}_{gen}(s) \mathbf{u}_{gen} + G_i(s) \, i = \\
   & = v_{Th}(s) - Z_{Th}(s) i(s) \ ,
\end{aligned}\end{split}
\end{equation*}
\sphinxAtStartPar
having recast it as Thevenin’s theorem defining the voltage \(v_{Th}\) and the impedance \(Z_{Th}\) of the equivalent circuit,
\begin{equation*}
\begin{split}\begin{aligned}
   v_{Th} & := \mathbf{G}_{gen}(s) \mathbf{u}_{gen}(s) \\
   Z_{Th}(s) & := - G_i(s)
\end{aligned}\end{split}
\end{equation*}

\subsubsection{Many\sphinxhyphen{}port circuit}
\label{\detokenize{ch/electrical-engineering-networks:many-port-circuit}}\label{\detokenize{ch/electrical-engineering-networks:classical-electromagnetism-electrical-engineering-newtork-analysis-thevenin-n-port}}\begin{equation*}
\begin{split}\mathbf{v} = \mathbf{G}_{gen}(s) \mathbf{u}_{gen} + \mathbf{G}_i(s) \mathbf{i} = \mathbf{v}_{Th} - \mathbf{Z}_{Th} \mathbf{i} \ .\end{split}
\end{equation*}

\subsection{Norton equivalent}
\label{\detokenize{ch/electrical-engineering-networks:norton-equivalent}}\label{\detokenize{ch/electrical-engineering-networks:classical-electromagnetism-electrical-engineering-newtork-analysis-norton}}
\sphinxstepscope


\section{Network analysis of linear circuits \sphinxhyphen{} harmonic regime}
\label{\detokenize{ch/electrical-engineering-networks-harmonic:network-analysis-of-linear-circuits-harmonic-regime}}\label{\detokenize{ch/electrical-engineering-networks-harmonic:classical-electromagnetism-electrical-engineering-newtork-analysis-harmonic}}\label{\detokenize{ch/electrical-engineering-networks-harmonic::doc}}
\sphinxAtStartPar
The harmonic dynamics of a linear circuit can be evaluated in Fourier domain, or using complex numbers to represent harmonic functions,
\begin{equation*}
\begin{split}\begin{aligned}
  v(t) & = V_{max} \cos (\Omega t + \varphi_v) = \text{re} \{ V_{max} e^{i (\Omega t + \varphi_v)} \} = \\
       & = \sqrt{2} V \cos (\Omega t + \varphi_v) = \sqrt{2} \, \text{re} \{ V e^{i (\Omega t + \varphi_v)} \} = \sqrt{2} \, \text{re} \{ v e^{i \Omega t} \} \\
  i(t) & = I_{max} \cos (\Omega t + \varphi_i) = \text{re} \{ I_{max} e^{j (\Omega t + \varphi_i)} \} = \\
       & = \sqrt{2} I \cos (\Omega t + \varphi_i) = \sqrt{2} \, \text{re} \{ I e^{j (\Omega t + \varphi_i)} \} = \sqrt{2} \, \text{re} \{ i e^{j \Omega t} \} \\
\end{aligned}\end{split}
\end{equation*}
\sphinxAtStartPar
having anticipated the definition {\hyperref[\detokenize{ch/electrical-engineering-networks-harmonic:harmonic:effective-values}]{\sphinxcrossref{Definition 10.5.1}}} of effective tension \(V\) and current \(I\).


\subsection{Power}
\label{\detokenize{ch/electrical-engineering-networks-harmonic:power}}\label{\detokenize{ch/electrical-engineering-networks-harmonic:classical-electromagnetism-electrical-engineering-newtork-analysis-harmonic-power}}
\sphinxAtStartPar
\sphinxstylestrong{Instantaneous power.}
\begin{equation}\label{equation:ch/electrical-engineering-networks-harmonic:eq:harmonic:power:instantaneous}
\begin{split}\begin{aligned}
  P(t) 
  & = v(t) i(t) = \\
  & = V_{max} I_{max}  \cos (\Omega t )  \cos (\Omega t - \varphi_i) = \\ 
  & = \frac{1}{2} V_{max} I_{max} \left[ \cos \varphi_i +  \cos ( 2 \Omega t ) \right]  \\ 
\end{aligned}\end{split}
\end{equation}
\sphinxAtStartPar
having used \sphinxhref{https://basics2022.github.io/bbooks-math-miscellanea-hs/ch/trigonometry.html\#werner}{Werner’s formula},
\begin{equation*}
\begin{split}\cos x \cos y = \frac{1}{2} \left[ \cos(x-y) + \cos(x+y) \right] \ .\end{split}
\end{equation*}
\sphinxAtStartPar
and the property \(\cos(-x) = \cos x\).

\sphinxAtStartPar
\sphinxstylestrong{Average power on a period.} Over a period \(T = \frac{1}{f} = \frac{2 \pi}{\Omega}\)
\begin{equation*}
\begin{split}\overline{P} = \frac{1}{T} \int_{t=t_0}^{t_0+T} P(t) \, dt = \frac{V_{max} I_{max}}{2} = V I\ ,\end{split}
\end{equation*}
\sphinxAtStartPar
as the integral of the harmonic term with period \(\frac{T}{2}\) of the instantaneous power \eqref{equation:ch/electrical-engineering-networks-harmonic:eq:harmonic:power:instantaneous} is identically zero, and with the definition of the \sphinxstylestrong{effective voltage and current}
\label{ch/electrical-engineering-networks-harmonic:harmonic:effective-values}
\begin{sphinxadmonition}{note}{Definition 10.5.1 (Effective voltage and current in AC)}



\sphinxAtStartPar
Effective voltage and currents
\begin{equation*}
\begin{split}V := \frac{V_{max}}{\sqrt{2}} \quad , \quad I := \frac{I_{max}}{\sqrt{2}} \ , \end{split}
\end{equation*}
\sphinxAtStartPar
are defined as those voltage and current in DC providing the same value of average power.
\end{sphinxadmonition}

\sphinxAtStartPar
\sphinxstylestrong{Complex power.} Complex power of a dipole with impedence \(Z\), \(v =  Z i\)
\begin{equation*}
\begin{split}\begin{aligned}
  S 
  & := v i^* = |v|e^{j \varphi_v} |i| e^{-j \varphi_i} = |v| |i| e^{j(\varphi_v - \varphi_i)} = \\
  & = Z i i^* = Z |i|^2 = (R + j X ) |i|^2 = |Z||i|^2 e^{j \varphi_Z} = P + j Q \ ,
\end{aligned}\end{split}
\end{equation*}
\sphinxAtStartPar
with the active power \(P\) and the reactive power \(Q\)
\begin{equation*}
\begin{split}\begin{aligned}
  P & = \text{re}\{ S \} && = && |S| \cos \varphi_Z && = && \dots \\
  Q & = \text{im}\{ S \} && = && |S| \sin \varphi_Z && = && \dots \\
\end{aligned}\end{split}
\end{equation*}


\sphinxstepscope


\section{Three\sphinxhyphen{}phase circuits}
\label{\detokenize{ch/electrical-engineering-three-phase:three-phase-circuits}}\label{\detokenize{ch/electrical-engineering-three-phase:classical-electromagnetism-electrical-engineering-three-phase}}\label{\detokenize{ch/electrical-engineering-three-phase::doc}}

\subsection{Star\sphinxhyphen{}star network}
\label{\detokenize{ch/electrical-engineering-three-phase:star-star-network}}\label{\detokenize{ch/electrical-engineering-three-phase:classical-electromagnetism-electrical-engineering-three-phase-star-star}}

\begin{savenotes}\sphinxattablestart
\centering
\begin{tabulary}{\linewidth}[t]{|T|T|}
\hline

\sphinxAtStartPar
\sphinxincludegraphics{{star-star}.png}
&
\sphinxAtStartPar
\sphinxincludegraphics{{star-star-e1}.png}
\\
\hline
\end{tabulary}
\par
\sphinxattableend\end{savenotes}


\subsubsection{General solution}
\label{\detokenize{ch/electrical-engineering-three-phase:general-solution}}
\sphinxAtStartPar
Tension \(v_{AB}\) between the centers of the stars \(A\), \(B\)
\begin{equation*}
\begin{split}v_{AB} = \dfrac{\sum_{g=1}^{3} Y_g e_g}{\sum_{i=1}^{4} Y_i} \ .\end{split}
\end{equation*}\subsubsection*{Proof.}

\sphinxAtStartPar
PSCE is used on the linear network, leaving only one tension generator on at a time, and then combining the results.

\sphinxAtStartPar
\sphinxstylestrong{Tension generator \(e_1\) on, \(e_2 = e_3 = 0\) off.} Leaving \(e_1\) on, and switching off \(e_2 = e_3 = 0\), tension generator sees an equivalent impedance
\begin{equation*}
\begin{split}\begin{aligned}
  Z_{eq,1}
  & = Z_1 + (Z_2 \parallel Z_3 \parallel Z_4) \\
  & = \dfrac{1}{Y_1} + \dfrac{1}{Y_2 + Y_3 + Y_4} = \dfrac{Y_{1234}}{Y_1 Y_{234}}  \ ,
\end{aligned}\end{split}
\end{equation*}
\sphinxAtStartPar
so that:
\begin{itemize}
\item {} 
\sphinxAtStartPar
the current through the generator reads
\begin{equation*}
\begin{split}i_{1,1} = \dfrac{e_1}{Z_{eq,1}} = \frac{Y_1 Y_{234}}{Y_{1234}} e_1\end{split}
\end{equation*}
\item {} 
\sphinxAtStartPar
the currents through the other sides (acting as current dividers are):
\begin{equation*}
\begin{split}\begin{aligned}
      i_{2,1} & = -     \frac{Y_2}{Y_{234}} \, i_{1,1} = -     \frac{Y_1 Y_2}{Y_{1234}} e_1  \\
      i_{3,1} & = -     \frac{Y_3}{Y_{234}} \, i_{1,1} = -     \frac{Y_1 Y_3}{Y_{1234}} e_1  \\
      i_{4,1} & = \ \ \ \frac{Y_4}{Y_{234}} \, i_{1,1} = \ \ \ \frac{Y_1 Y_4}{Y_{1234}} e_1  \\
   \end{aligned}\end{split}
\end{equation*}
\item {} 
\sphinxAtStartPar
tension \(v_{AB}\)
\begin{equation*}
\begin{split}v_{AB,1} = e_1 - Z_1 i_{1,1} = \left( 1 - \frac{Y_{234}}{Y_{1234}} \right) e_1 = \frac{Y_1 e_1}{\sum_{k=1}^4 Y_k} \ . \end{split}
\end{equation*}
\end{itemize}

\sphinxAtStartPar
\sphinxstylestrong{PSCE.} Exploiting the PSCE and the symmetry of the system, the expressions of currents in the phases, in the neutral and the center\sphinxhyphen{}center voltage seamlessly follow
\begin{equation*}
\begin{split}\begin{aligned}
  i_1    & = \frac{Y_1 Y_{234}}{Y_{1234}} e_1 - \frac{Y_1 Y_2}{Y_{1234}} e_2 - \frac{Y_1 Y_3}{Y_{1234}} e_3 = \\
         & = Y_1 e_1 - \frac{Y_1}{Y_{1234}} \sum_{g=1}^{3} Y_g \, e_g \\
  i_2    & = Y_2 e_2 - \frac{Y_2}{Y_{1234}} \sum_{g=1}^{3} Y_g \, e_g \\
  i_3    & = Y_3 e_3 - \frac{Y_3}{Y_{1234}} \sum_{g=1}^{3} Y_g \, e_g \\
  i_4    & = \frac{Y_4}{Y_{1234}} \sum_{g=1}^{3} Y_g \, e_g \\
  v_{AB} & =  \frac{\sum_{g=1}^{3} Y_g \, e_g}{\sum_{k=1}^4 Y_k} \\
\end{aligned}\end{split}
\end{equation*}

\subsubsection{Equilibrated generation and loads}
\label{\detokenize{ch/electrical-engineering-three-phase:equilibrated-generation-and-loads}}

\subsubsection{Extra connections}
\label{\detokenize{ch/electrical-engineering-three-phase:extra-connections}}

\paragraph{Phase\sphinxhyphen{}neutral connections}
\label{\detokenize{ch/electrical-engineering-three-phase:phase-neutral-connections}}
\sphinxAtStartPar
Connections of a phase with the neutral result in parallel impedence with the generators and/or the loads

\begin{figure}[htbp]
\centering

\noindent\sphinxincludegraphics{{star-star-parallel-connections}.png}
\end{figure}


\paragraph{Phase\sphinxhyphen{}phase connections}
\label{\detokenize{ch/electrical-engineering-three-phase:phase-phase-connections}}
\sphinxAtStartPar
Phase\sphinxhyphen{}phase connections don’t influence the voltage \(v_{AB}\) between the centers \(A\), \(B\).

\sphinxAtStartPar
\sphinxstylestrong{todo} \sphinxstyleemphasis{Write the proof.}

\begin{figure}[htbp]
\centering

\noindent\sphinxincludegraphics{{star-star-extra-connections}.png}
\end{figure}

\sphinxstepscope


\section{Exercises}
\label{\detokenize{ch/electrical-engineering-exercises:exercises}}\label{\detokenize{ch/electrical-engineering-exercises:classical-electromagnetism-electrical-engineering-exercises}}\label{\detokenize{ch/electrical-engineering-exercises::doc}}
\sphinxAtStartPar
\sphinxstylestrong{Topics}: Thevenin and Norton equivalent;…

\sphinxAtStartPar
\sphinxstylestrong{Electric circuits}:
\begin{itemize}
\item {} 
\sphinxAtStartPar
Type a: transient dynamics of systems with 1 dynamic component (either capacitor or inductor);

\item {} 
\sphinxAtStartPar
Type b: harmonic dynamics of linear systems: phasor algebra, complex power,…

\item {} 
\sphinxAtStartPar
Type c: three\sphinxhyphen{}phase circuits, triangles and stars,…

\end{itemize}

\sphinxAtStartPar
\sphinxstylestrong{Electromagnetic circuits}:
\begin{itemize}
\item {} 
\sphinxAtStartPar
Type d: circuit approximation of magnetic circuit,…

\end{itemize}

\sphinxAtStartPar
\sphinxstylestrong{Exams.}
\subsubsection*{2025\sphinxhyphen{}02\sphinxhyphen{}11}
\begin{enumerate}
\sphinxsetlistlabels{\arabic}{enumi}{enumii}{}{.}%
\item {} 
\sphinxAtStartPar
Type a. \hyperref[\detokenize{ch/electrical-engineering-exercises-transient-1-dynamic:exam-25-02-11-exe-01}]{Exercise \ref{\detokenize{ch/electrical-engineering-exercises-transient-1-dynamic:exam-25-02-11-exe-01}}}

\item {} 
\sphinxAtStartPar
Type b. \hyperref[\detokenize{ch/electrical-engineering-exercises-harmonic:exam-25-02-11-exe-02}]{Exercise \ref{\detokenize{ch/electrical-engineering-exercises-harmonic:exam-25-02-11-exe-02}}}

\item {} 
\sphinxAtStartPar
Type b. \hyperref[\detokenize{ch/electrical-engineering-exercises-harmonic:exam-25-02-11-exe-03}]{Exercise \ref{\detokenize{ch/electrical-engineering-exercises-harmonic:exam-25-02-11-exe-03}}}

\item {} 
\sphinxAtStartPar
Theory: electrical line. Electro\sphinxhyphen{}thermal model of the cable,…

\end{enumerate}
\subsubsection*{2025\sphinxhyphen{}01\sphinxhyphen{}22}
\begin{enumerate}
\sphinxsetlistlabels{\arabic}{enumi}{enumii}{}{.}%
\item {} 
\sphinxAtStartPar
Type a. \hyperref[\detokenize{ch/electrical-engineering-exercises-transient-1-dynamic:exam-25-01-22-exe-01}]{Exercise \ref{\detokenize{ch/electrical-engineering-exercises-transient-1-dynamic:exam-25-01-22-exe-01}}}

\item {} 
\sphinxAtStartPar
Type b. \hyperref[\detokenize{ch/electrical-engineering-exercises-harmonic:exam-25-01-22-exe-02}]{Exercise \ref{\detokenize{ch/electrical-engineering-exercises-harmonic:exam-25-01-22-exe-02}}}

\item {} 
\sphinxAtStartPar
Type d. \hyperref[\detokenize{ch/electrical-engineering-exercises-electromagnetic:exam-25-01-22-exe-03}]{Exercise \ref{\detokenize{ch/electrical-engineering-exercises-electromagnetic:exam-25-01-22-exe-03}}}

\item {} 
\sphinxAtStartPar
Theory: transformer

\end{enumerate}
\subsubsection*{2024\sphinxhyphen{}09\sphinxhyphen{}06}
\begin{enumerate}
\sphinxsetlistlabels{\arabic}{enumi}{enumii}{}{.}%
\item {} 
\sphinxAtStartPar
Type a. \hyperref[\detokenize{ch/electrical-engineering-exercises-transient-1-dynamic:exam-24-09-06-exe-01}]{Exercise \ref{\detokenize{ch/electrical-engineering-exercises-transient-1-dynamic:exam-24-09-06-exe-01}}}

\item {} 
\sphinxAtStartPar
Type b. \hyperref[\detokenize{ch/electrical-engineering-exercises-harmonic:exam-24-09-06-exe-02}]{Exercise \ref{\detokenize{ch/electrical-engineering-exercises-harmonic:exam-24-09-06-exe-02}}}

\item {} 
\sphinxAtStartPar
Type c. \hyperref[\detokenize{ch/electrical-engineering-exercises-three-phase:exam-24-09-06-exe-03}]{Exercise \ref{\detokenize{ch/electrical-engineering-exercises-three-phase:exam-24-09-06-exe-03}}}

\item {} 
\sphinxAtStartPar
Theory: overload in cables

\end{enumerate}
\subsubsection*{2024\sphinxhyphen{}07\sphinxhyphen{}22}
\begin{enumerate}
\sphinxsetlistlabels{\arabic}{enumi}{enumii}{}{.}%
\item {} 
\sphinxAtStartPar
Type a. \hyperref[\detokenize{ch/electrical-engineering-exercises-transient-1-dynamic:exam-24-07-22-exe-01}]{Exercise \ref{\detokenize{ch/electrical-engineering-exercises-transient-1-dynamic:exam-24-07-22-exe-01}}}

\item {} 
\sphinxAtStartPar
Type b. \hyperref[\detokenize{ch/electrical-engineering-exercises-harmonic:exam-24-07-22-exe-02}]{Exercise \ref{\detokenize{ch/electrical-engineering-exercises-harmonic:exam-24-07-22-exe-02}}}

\item {} 
\sphinxAtStartPar
Type c. \hyperref[\detokenize{ch/electrical-engineering-exercises-three-phase:exam-24-07-22-exe-03}]{Exercise \ref{\detokenize{ch/electrical-engineering-exercises-three-phase:exam-24-07-22-exe-03}}}

\end{enumerate}
\subsubsection*{2024\sphinxhyphen{}06\sphinxhyphen{}19}
\begin{enumerate}
\sphinxsetlistlabels{\arabic}{enumi}{enumii}{}{.}%
\item {} 
\sphinxAtStartPar
Type c. \hyperref[\detokenize{ch/electrical-engineering-exercises-three-phase:exam-24-06-19-exe-01}]{Exercise \ref{\detokenize{ch/electrical-engineering-exercises-three-phase:exam-24-06-19-exe-01}}}

\item {} 
\sphinxAtStartPar
Type d. \hyperref[\detokenize{ch/electrical-engineering-exercises-electromagnetic:exam-24-06-19-exe-02}]{Exercise \ref{\detokenize{ch/electrical-engineering-exercises-electromagnetic:exam-24-06-19-exe-02}}}

\end{enumerate}
\subsubsection*{2024\sphinxhyphen{}02\sphinxhyphen{}13}
\begin{enumerate}
\sphinxsetlistlabels{\arabic}{enumi}{enumii}{}{.}%
\item {} 
\sphinxAtStartPar
Type d.+a. \hyperref[\detokenize{ch/electrical-engineering-exercises-electromagnetic:exam-24-02-13-exe-01-a}]{Exercise \ref{\detokenize{ch/electrical-engineering-exercises-electromagnetic:exam-24-02-13-exe-01-a}}}

\item {} 
\sphinxAtStartPar
Type a.    \hyperref[\detokenize{ch/electrical-engineering-exercises-transient-1-dynamic:exam-24-02-13-exe-01-b}]{Exercise \ref{\detokenize{ch/electrical-engineering-exercises-transient-1-dynamic:exam-24-02-13-exe-01-b}}}

\item {} 
\sphinxAtStartPar
Type c.    \hyperref[\detokenize{ch/electrical-engineering-exercises-three-phase:exam-24-02-13-exe-02}]{Exercise \ref{\detokenize{ch/electrical-engineering-exercises-three-phase:exam-24-02-13-exe-02}}}

\end{enumerate}

\sphinxstepscope


\subsection{Transient dynamics of linear electrical grids with one dynamic component}
\label{\detokenize{ch/electrical-engineering-exercises-transient-1-dynamic:transient-dynamics-of-linear-electrical-grids-with-one-dynamic-component}}\label{\detokenize{ch/electrical-engineering-exercises-transient-1-dynamic:classical-electromagnetism-electrical-engineering-exercises-transient-1-dynamic}}\label{\detokenize{ch/electrical-engineering-exercises-transient-1-dynamic::doc}}
\begin{sphinxadmonition}{note}{Guidelines for solution}

\sphinxAtStartPar
A many\sphinxhyphen{}port Thevenin equivalent circuit of the resistive part of the circuit is found, with two ports for interfacing with the dynamical component (A) and with the switch (B), exploiting PSCE,
\begin{equation*}
\begin{split}\begin{aligned}
  v_A & = v_{0,A}(\mathbf{e},\mathbf{a}) + R_{AA} i_A + R_{AB} i_B \\
  v_B & = v_{0,B}(\mathbf{e},\mathbf{a}) + R_{BA} i_A + R_{BB} i_B \\
  i   & =   i_{0}(\mathbf{e},\mathbf{a}) + i_{/i_A} i_A + i_{/i_B} i_B
\end{aligned}\end{split}
\end{equation*}
\sphinxAtStartPar
The constitutive equation of the dynamical equation is used to evaluate the time evolution of the system, given the initial conditions \sphinxhyphen{} usually steady conditions with switches open. If the dynamical element is a capacitor,
\begin{equation*}
\begin{split}i_A = - C \dfrac{d v_A}{dt} \ .\end{split}
\end{equation*}
\sphinxAtStartPar
Typically, (1) the generators provides steady inputs, (2) for \(t = 0^{-}\) the switch is open, and closes at \(t = 0\). At \(t=0\) an impulsive forcing, and thus a jump, in some physical quantity may occur.

\sphinxAtStartPar
\sphinxstylestrong{Steady conditions for \(t \le 0^-\).} Steady conditions with the switch is open imply
\begin{equation*}
\begin{split}\begin{aligned}
       i_B & = 0 && \text{open switch} \\
       i_A & = 0 && \text{steady conditions and $i_A = C \dot{v}_A = 0$} \\
   \end{aligned}\end{split}
\end{equation*}
\sphinxAtStartPar
\sphinxstylestrong{Transient dynamics.} At \(t = 0\) the switch closes, so that the tension at the switch immediately goes from the steady value evaluated for \(t \le 0^-\) to the value
\begin{equation*}
\begin{split}v_B(t) = 0 \qquad , \qquad \text{for $t > 0$} \ ,\end{split}
\end{equation*}
\sphinxAtStartPar
and thus, its value and its time derivative w.r.t. time, \(t \in (-\infty, +\infty)\) can be written as
\begin{equation*}
\begin{split}\begin{aligned}
        v_B  (t) & = v_B(0^-) h(-t) \\
    \dot{v}_B(t) & = - v_B(0^-) \delta(t) \ ,
  \end{aligned}\end{split}
\end{equation*}
\sphinxAtStartPar
having used the step function \(h(t)\) and Dirac’s delta \(\delta(t)\).

\sphinxAtStartPar
\sphinxstylestrong{Method 1. Dynamical equation for the state variable, here \(v_A\).} Writing the dynamical equation for the state variable should give a dynamical equation with no impulsive forcing and thus no jump in the physical quantity. Using the equations of the equivalent model, it’s possible to write \(v_A\) as a function of \(i_A\), \(v_B\) and the constant tensions \(v_{0,A}\), \(v_{0,B}\)
\begin{equation*}
\begin{split}\begin{aligned}
  v_A(t) 
  & = v_{0,A} + R_{AA} i_A + \frac{R_{AB}}{R_{BB}} \left[ v_B(t) - v_{0,B} - R_{BA} i_A  \right] = \\
  & = v_{0,A} - \frac{R_{AB}}{R_{BB}} v_{0,B} + \frac{R_{AB}}{R_{BB}} v_{B}(t) + \left[ R_{AA} - \frac{R_{BA} R_{AB}}{R_{BB}} \right] i_A(t)  = \\
\end{aligned}\end{split}
\end{equation*}\begin{equation*}
\begin{split}i_A(t) = \frac{R_{BB}}{\text{det} \mathbf{R}} v_A(t) - \frac{R_{BB} v_{0,A} - R_{AB} ( v_{0,B}- v_{B}(t) ) }{\text{det} \mathbf{R}} \end{split}
\end{equation*}
\sphinxAtStartPar
so that the dynamical equation for \(v_A(t)\) reads
\begin{equation*}
\begin{split}\begin{aligned}
  0 
  & = C \frac{d v_A}{dt} + i_A = \\
  & = C \frac{d v_A}{dt} + \frac{R_{BB}}{\det \mathbf{R}} v_A(t) - \frac{R_{BB} v_{0,A} - R_{AB} ( v_{0,B}- v_{B}(t) ) }{\det \mathbf{R}} \ ,
\end{aligned}\end{split}
\end{equation*}
\sphinxAtStartPar
and thus
\begin{equation*}
\begin{split}R_{eq} C \frac{d v_A}{dt} + v_A(t) =  v_{0,A} - \frac{R_{AB}}{R_{BB}} ( v_{0,B}- v_{B}(t) )  \end{split}
\end{equation*}
\sphinxAtStartPar
with \(R_{eq} := \dfrac{\det \mathbf{R}}{R_{BB}}\). If \(i_A(0^-) = i_B(0^-) = 0\), then
\begin{equation*}
\begin{split}\begin{aligned}
  v_{B}(t) & = v_{0,B} h(-t) \\
  v_{0,B} - v_{B}(t) & = v_{0,B} h(t)  \\
\end{aligned}\end{split}
\end{equation*}
\sphinxAtStartPar
With the initial conditions \(v_{A}(0^-) = v_{0,A}\), and defining the difference with the initial steady conditions, \(\delta v_A(t) := v_A(t) - v_{0,A}\), \(\delta v_B(t) = v_B(t) - v_{0,B}\), the initial condition for the difference reads \(\delta v_{A}(0^-) = 0\), so that the Cauchy problem to be solved reads
\begin{equation*}
\begin{split}\begin{cases}
  RC \delta \dot{v}_A + \delta v_{A} = \dfrac{R_{AB}}{R_{BB}} \delta v_{B}(t) = - \dfrac{R_{AB}}{R_{BB}} v_{0,B} h(t) \\
  \delta v_A(0^-) = 0 \ .
\end{cases}\end{split}
\end{equation*}\begin{equation*}
\begin{split}\delta v_A(t) = -\frac{R_{AB}}{R_{BB}} v_{0,B} \, \left[ 1 - \exp\left( - \dfrac{t}{RC} \right) \right] \, h(t) \ .\end{split}
\end{equation*}
\sphinxAtStartPar
Once the solution \(\delta v_A(t)\) is found,
\begin{itemize}
\item {} 
\sphinxAtStartPar
voltage across the capacitor reads
\begin{equation*}
\begin{split}\begin{aligned}
     v_A(t)
     & = \delta v_A(t) + v_{A,0} = \\
     & = v_{A,0} - \frac{R_{AB}}{R_{BB}} v_{0,B} \, \left[ 1 - \exp\left( - \dfrac{t}{RC} \right) \right] \, h(t) \ ,
   \end{aligned}\end{split}
\end{equation*}
\item {} 
\sphinxAtStartPar
current through the capacitor reads
\begin{equation*}
\begin{split}\begin{aligned}
     i_A(t) 
     & = C \dfrac{d v_A}{dt}(t) = \\
     & = \frac{R_{AB}}{\det \mathbf{R}} \exp\left( - \dfrac{t}{RC}  \right) \, h(t) \ ,
   \end{aligned}\end{split}
\end{equation*}
\sphinxAtStartPar
as \(R = \frac{\det \mathbf{R}}{R_{BB}}\);

\item {} 
\sphinxAtStartPar
current through the switch reads
\begin{equation*}
\begin{split}\begin{aligned}
     i_B(t)
     & = \frac{1}{R_{BB}} \left[ v_B(t) - v_{0,B} - R_{BA} i_A(t) \right] = \\
     & = - \frac{v_{0,B}}{R_{BB}} \, h(t) - \frac{R_{BA} R_{AB}}{R_{BB} \det \mathbf{R} } \exp \left( -\dfrac{t}{RC} \right) \, h(t) \ .
   \end{aligned}\end{split}
\end{equation*}
\item {} 
\sphinxAtStartPar
the desired current reads
\begin{equation*}
\begin{split}i(t) = i_0(\mathbf{e}, \mathbf{a}) + i_{/i_A} i_A(t) + i_{/i_B} i_B(t) \ .\end{split}
\end{equation*}
\end{itemize}
\end{sphinxadmonition}


\phantomsection \label{exercise:exam-25-02-11-exe-01}

\begin{sphinxadmonition}{note}{Exercise 10.7.1 (Exam 2025\sphinxhyphen{}02\sphinxhyphen{}11, Exercise 1.)}



\begin{figure}[htbp]
\centering

\noindent\sphinxincludegraphics{{exam-2025-02-11-ese-01}.png}
\end{figure}
\subsubsection*{Solution}

\sphinxAtStartPar
Following the \sphinxstylestrong{guidelines for the solution}, a {\hyperref[\detokenize{ch/electrical-engineering-networks:classical-electromagnetism-electrical-engineering-newtork-analysis-thevenin-n-port}]{\sphinxcrossref{\DUrole{std,std-ref}{many\sphinxhyphen{}port Thevenin equivalent circuit}}}} of the resistive part of the circuit is found, with two ports for interfacing with the capacitor (A) and with the switch. The dynamical equation of the system is written in state\sphinxhyphen{}space representation, writing the voltage at the ports and the unknown variable \(i(t)\) as outputs; the capacitor contitutive equation is used to find the time evolution of the system once the switch is closed

\begin{figure}[htbp]
\centering

\noindent\sphinxincludegraphics{{exam-2025-02-11-ese-01-b}.png}
\end{figure}
\subsubsection*{Internal generators on, open circuit}

\sphinxAtStartPar
Solution using two loop currents, \(i_1\) in the upper part of the circuit and \(i_2\) in the lower triangle. Using KVL
\begin{equation*}
\begin{split}\begin{aligned}
  0 & = e_1 - R_2 i_{1,0} - R_1 (a + i_{1,0}) \\
  \rightarrow \quad i_{1,0} & = \dfrac{1}{R_1+R_2} e_1 - \frac{R_1}{R_1 + R_2} a \\
\end{aligned}\end{split}
\end{equation*}
\sphinxAtStartPar
so that the desired variables read
\begin{equation*}
\begin{split}\begin{cases}
  v_{A,0} & = R_3 a - R_2 i_{1,0} = \left[ R_3 + \dfrac{R_1 R_2}{R_1 + R_2} \right] a - \dfrac{R_2}{R_1+R_2} e_1 \\
  v_{B,0} & = e_2 - (R_3 + R_4) a \\
  i_{0} & = a
\end{cases}\end{split}
\end{equation*}\begin{equation*}
\begin{split}\begin{cases}
 v_{A,0} & = \quad\ 7.67 \, V \\
 v_{B,0} & =      -13.00 \, V \\
   i_{0} & = \quad\ 3.00 \, A \\
\end{cases}\end{split}
\end{equation*}
\begin{figure}[htbp]
\centering

\noindent\sphinxincludegraphics{{exam-2025-02-11-ese-01-c}.png}
\end{figure}
\subsubsection*{Internal generators off, current generators at the ports}

\sphinxAtStartPar
Callling \(i_A\) and \(i_B\) the current passing through the current generators connected at the ports. The solution is found powering one generation at a time and then exploiting PSCE

\sphinxAtStartPar
\sphinxstyleemphasis{Powering A} …

\sphinxAtStartPar
\sphinxstyleemphasis{Powering B.} …



\sphinxAtStartPar
Currents in the two parallel branches in the upper part of the circuit (current dividers) read
\begin{equation*}
\begin{split}\begin{cases}
  i   & = i_A \\
  v_A & = \left[ R_3 + \dfrac{R_1 R_2}{R_1 + R_2} \right] \, i_A - R_3 \, i_B \\
  v_B & = -R_3 \, i_A + (R_3 + R_4) \, i_B
\end{cases}\end{split}
\end{equation*}
\begin{figure}[htbp]
\centering

\noindent\sphinxincludegraphics{{exam-2025-02-11-ese-01-a}.png}
\end{figure}

\sphinxAtStartPar
The equations of the equivalent algebraic system are
\begin{equation*}
\begin{split}\begin{cases}
 v_A & = v_{A,0} + R_{AA}   \, i_A + R_{AB}   \, i_B \\
 v_B & = v_{B,0} + R_{BA}   \, i_A + R_{BB}   \, i_B \\
 i   & = i_{ ,0} + i_{/i_A} \, i_A + i_{/i_B} \, i_B \\
\end{cases}\end{split}
\end{equation*}\begin{equation*}
\begin{split}
\begin{bmatrix} v_A(t) \\ v_B(t) \end{bmatrix} & =
\begin{bmatrix} v_{A0} \\ v_{B0} \end{bmatrix} + 
\begin{bmatrix}
  R_3 + \frac{R_1 R_2}{R_1 + R_2} & -R_3 \\ -R_3 & R_3 + R_4
\end{bmatrix}
\begin{bmatrix} i_A(t) \\ i_B(t) \end{bmatrix} \\
i(t) & = i_0 + i_A(t)
\end{split}
\end{equation*}\begin{equation*}
\begin{split}\begin{aligned}
  \det \mathbf{R}
  & = \left( R_3 + \frac{R_1 R_2}{R_1 + R_2}  \right) \left( R_3 + R_4 \right) - R_3^2 = \\
  & = ( R_3 + R_4 ) \left( R_3 + \frac{R_1 R_2}{R_1 + R_2} - \dfrac{R_3^2}{R_3 + R_4} \right) = \\
  & = ( R_3 + R_4 ) \left( \frac{R_1 R_2}{R_1 + R_2} + \dfrac{R_3 R_4}{R_3 + R_4} \right) \ .
\end{aligned}\end{split}
\end{equation*}
\sphinxAtStartPar
\sphinxstylestrong{Steady solution for \(t \le 0^-\).} With switch open \(i_B = 0\) and steady conditions \(i_A = C \dot{v}_A = 0\),
\begin{equation*}
\begin{split}\begin{cases}
  v_A(0^-) & = v_{A,0} = \quad\ 7.67 \, V \\
  v_B(0^-) & = v_{B,0} =      -13.00 \, V \\
    i(0^-) & = i_{ ,0} = \quad\ 3.00 \, A \\
\end{cases}\end{split}
\end{equation*}
\sphinxAtStartPar
\sphinxstylestrong{Transient dynamics}, when the switch closes \(v_B(t \ge 0^+) = 0\),
\begin{equation*}
\begin{split}i_A(t) = \dfrac{R_3 + R_4}{\det \mathbf{R}} \Delta v_A(t) + \dfrac{R_3}{\det \mathbf{R}} \Delta v_B(t)\end{split}
\end{equation*}\begin{itemize}
\item {} 
\sphinxAtStartPar
\sphinxstylestrong{Tension across the switch}
\begin{equation*}
\begin{split}\begin{aligned}
     v_{B}(t) & = v_{B,0} \, h(-t) \\
     \Delta v_{B}(t) & = v_B(t) - v_{B,0} = - v_{B,0} \, h(t) \ . 
   \end{aligned}\end{split}
\end{equation*}
\item {} 
\sphinxAtStartPar
\sphinxstylestrong{Tension across the capacitor.} The dynamical equation for the difference of the state variable reads
\begin{equation*}
\begin{split}\begin{aligned}
     0 & = i_A + C \dot{v}_A = \\
       & = \dfrac{R_3 + R_4}{\det \mathbf{R}} \Delta v_A(t) + \dfrac{R_3}{\det \mathbf{R}} \Delta v_B(t) + C \dot{v}_A \ .
   \end{aligned}\end{split}
\end{equation*}
\sphinxAtStartPar
As \(v_{A}(t=0) = v_{A,0}\) (no jump in state variables without impulsive forcing), \(\Delta v_A = v_A - v_{A,0}\), and \(\frac{d}{dt} \Delta v_A = \frac{d}{dt} v_A\), the dynamical equation reads
\begin{equation*}
\begin{split}\begin{cases}
     \dfrac{\det \mathbf{R}}{R_3 + R_4} C  \dfrac{d}{dt}\Delta v_A + \Delta v_{A} = - \dfrac{R_3}{R_3 + R_4} \Delta v_{B}(t) = \dfrac{R_3}{R_3 + R_4} \, v_{B,0} \, h(t) \\ \\
     \Delta v_A(0^-) = 0 \ .
   \end{cases}\end{split}
\end{equation*}\begin{equation*}
\begin{split}\begin{aligned}
     \Delta v_A(t)
     & = \frac{R_3}{R_3 + R_4} v_{B,0} \left[ 1 - \exp\left( - \dfrac{t}{\tau} \right) \right] \, h(t) \ ,
   \end{aligned}\end{split}
\end{equation*}
\sphinxAtStartPar
having defined the time constant and the equivalent reistence seen by the capacitor
\begin{equation*}
\begin{split}\begin{aligned}
     R_{eq} & := \dfrac{\det \mathbf{R}}{R_3+R_4} = \dfrac{R_1 R_2}{R_1 + R_2} + \dfrac{R_3 R_4}{R_3 + R_4} = \frac{50}{21} \, V = 2.381 \, V  \\ \\
     \tau   & := R_{eq} C = 1.1905 \, s 
   \end{aligned}\end{split}
\end{equation*}
\sphinxAtStartPar
Tension through the capacitor reads
\begin{equation*}
\begin{split}\begin{aligned}
     v_A(t) 
     & = v_{A,0} + \delta v_A(t) = \\
     & = v_{A,0} + \Delta v_{A,+\infty} \, \left[ 1 - \exp\left( - \dfrac{t}{\tau} \right) \right] \, h(t) \ , 
   \end{aligned}\end{split}
\end{equation*}
\sphinxAtStartPar
so that the values
\begin{equation*}
\begin{split}\begin{aligned} 
     v_A(0^+)     & = v_{A,0} = 7.67 \, V \\
     v_A(+\infty) & = v_{A,0} + \Delta v_{A,+\infty} = ( 7.667 - 5.571 ) \, V = 2.095 \, V \ .
   \end{aligned}\end{split}
\end{equation*}
\item {} 
\sphinxAtStartPar
\sphinxstylestrong{Current through the capacitor.}
\begin{equation*}
\begin{split}\begin{aligned}
     i_A(t)
     & = \dfrac{R_3+R_4}{\det \mathbf{R}} \Delta v_A(t) + \dfrac{R_3}{\det \mathbf{R}} \Delta v_B(t) = \\
     & = \dfrac{R_3+R_4}{\det \mathbf{R}} \frac{R_3}{R_3 + R_4} v_{B,0} \left[ 1 - \exp\left( - \dfrac{t}{\tau} \right) \right] \, h(t) - \dfrac{R_3}{\det \mathbf{R}} v_{B,0} \, h(t) = \\
     & = - \dfrac{R_3}{\det \mathbf{R}} v_{B,0} \exp\left( - \dfrac{t}{\tau} \right) \, h(t) \\
     & = 2.34 \, A \, \exp\left( - \dfrac{t}{\tau} \right) \, h(t) \ .
   \end{aligned}\end{split}
\end{equation*}
\sphinxAtStartPar
so that the values
\begin{equation*}
\begin{split}\begin{aligned}
     i_A(0^+)     & = 2.34 \, A \\
     i_A(+\infty) & = 0.00 \, A
   \end{aligned}\end{split}
\end{equation*}
\item {} 
\sphinxAtStartPar
\sphinxstylestrong{Current \(i(t)\)}
\begin{equation*}
\begin{split}\begin{aligned}
     i(t) 
     & = i_{,0} + i_A(t) = \\
     & = a - \dfrac{R_3}{\det \mathbf{R}} v_{B,0} \exp\left( - \dfrac{t}{\tau} \right) \, h(t) \\
     & = 3.00 \, A + 2.34 \, A \, e^{-\frac{t}{\tau}} h(t) \ ,
   \end{aligned}\end{split}
\end{equation*}
\sphinxAtStartPar
so that the values
\begin{equation*}
\begin{split}\begin{aligned}
     i(0^+)     & = 5.35 \, A \\
     i(+\infty) & = 3.00 \, A
   \end{aligned}\end{split}
\end{equation*}
\end{itemize}

\sphinxAtStartPar
\sphinxstylestrong{Energy stored in the capacitor at \(t = 0\).} Energy in the capacitor reads
\begin{equation*}
\begin{split}E_C(t) = \dfrac{1}{2} C v_A^2(t) \ .\end{split}
\end{equation*}
\sphinxAtStartPar
At \(t = 0\), \(v_A(0) = 7.667 \, V\) and \(E_C(0) = 14.694 \, J \).
\end{sphinxadmonition}


\phantomsection \label{exercise:exam-25-01-22-exe-01}

\begin{sphinxadmonition}{note}{Exercise 10.7.2 (Exam 2025\sphinxhyphen{}01\sphinxhyphen{}22, Exercise 1.)}



\begin{figure}[htbp]
\centering

\noindent\sphinxincludegraphics{{exam-2025-01-22-ese-01}.png}
\end{figure}
\subsubsection*{Solution}

\sphinxAtStartPar
Following the \sphinxstylestrong{guidelines for the solution}, a {\hyperref[\detokenize{ch/electrical-engineering-networks:classical-electromagnetism-electrical-engineering-newtork-analysis-thevenin-n-port}]{\sphinxcrossref{\DUrole{std,std-ref}{many\sphinxhyphen{}port Thevenin equivalent circuit}}}} of the resistive part of the circuit is found, with two ports for interfacing with the capacitor (A) and with the switch. The dynamical equation of the system is written in state\sphinxhyphen{}space representation, writing the voltage at the ports and the unknown variable \(i(t)\) as outputs; the capacitor contitutive equation is used to find the time evolution of the system once the switch is closed



\begin{figure}[htbp]
\centering

\noindent\sphinxincludegraphics{{exam-2025-01-22-ese-01-b}.png}
\end{figure}
\subsubsection*{Internal generators on, open circuit}

\sphinxAtStartPar
Solution using two loop currents, \(i_1\) in the upper part of the circuit and \(i_2\) in the lower triangle. Using KVL
\begin{equation*}
\begin{split}\begin{aligned}
  0 & = e_2 - (R_7+R_8+R_1+R_4+R_6) i_{2,0} \\
  0 & = e_1 - (R_2+R_3) i_{1,0} \\
\end{aligned}
\end{split}
\end{equation*}\begin{equation*}
\begin{split}
\begin{aligned}
  i_{2,0} & = \dfrac{1}{R_{14678}} \, e_2 \\
  i_{1,0} & = \dfrac{1}{R_{23}} \, e_1 \\
\end{aligned}\end{split}
\end{equation*}
\sphinxAtStartPar
with \(R_{14678} = R_1+R_4+R_6+R_7+R_8\), and \(R_{23} = R_2 + R_3\). The desired physical quantities are
\begin{equation*}
\begin{split}\begin{cases}
  v_{A,0} & = - R_8 i_{2,0} = -\dfrac{R_8}{R_{14678}} e_2 \\ 
  v_{B,0} & = - R_4 i_{2,0} + R_3 i_{1,0} = - \dfrac{R_4}{R_{14678}} e_2 + \dfrac{R_3}{R_{23}} e_1 \\
  i_{0} & = - i_{2,0} = - \dfrac{1}{R_{14678}} e_2
\end{cases}\end{split}
\end{equation*}
\sphinxAtStartPar
and their values
\begin{equation*}
\begin{split}\begin{cases}
 v_{A,0} & =     - 20.6900 \, V \\
 v_{B,0} & = \ \ \ 12.4750 \, V \\
   i_{0} & = \    - 1.7241 \, A \\
\end{cases}\end{split}
\end{equation*}
\begin{figure}[htbp]
\centering

\noindent\sphinxincludegraphics{{exam-2025-01-22-ese-01-c}.png}
\end{figure}
\subsubsection*{Internal generators off, current generators at the ports}

\sphinxAtStartPar
Callling \(i_A\) and \(i_B\) the current passing through the current generators connected at the ports. The solution is found powering one generation at a time and then exploiting PSCE

\sphinxAtStartPar
\sphinxstyleemphasis{Powering A}
\begin{equation*}
\begin{split}\begin{aligned}
  0 & = (i_2 - i_A) R_8 + i_2 (R_{14678}) \\
\rightarrow \quad 
 i_2 & = \frac{R_8}{R_{14678}} i_A
\end{aligned}\end{split}
\end{equation*}\begin{equation*}
\begin{split}\begin{aligned}
  v_{A,A} & =  -R_8 ( i_2 - i_A ) & = \dfrac{R_8 R_{1467}}{R_{14678}} i_A \\
  v_{B,A} & = - R_4 i_2 & = - \dfrac{R_4 R_8}{R_{14678}} i_A \\
    i_{,A}& = - i_2 & = - \dfrac{R_8}{R_{14678}} i_A \\
\end{aligned}\end{split}
\end{equation*}\begin{equation*}
\begin{split}\begin{aligned}
 v_{A,A} & = R_{AA}   \, i_A && = \ \ \ 7.0345  \, \Omega \, i_A \\ 
 v_{B,A} & = R_{BA}   \, i_A && = - 1.2414  \, \Omega \, i_A \\
   i_{,A} & = i_{/i_A}\, i_A && = - 0.4138 \ \  i_A \\
\end{aligned}\end{split}
\end{equation*}
\sphinxAtStartPar
\sphinxstyleemphasis{Powering B.}



\sphinxAtStartPar
Currents in the two parallel branches in the upper part of the circuit (current dividers) read
\begin{equation*}
\begin{split}\begin{aligned}
  i_{2,B} & = \frac{R_4}{R_{14678}} i_B \\
  i_{3,B} & = \frac{R_2}{R_{23}} i_B \\
\end{aligned}\end{split}
\end{equation*}
\sphinxAtStartPar
and the desired variables
\begin{equation*}
\begin{split}\begin{aligned}
  i_{,B}  & = i_{4,B} & = \dfrac{R_{1678}}{R_{14678}} i_B \\
  v_{A,B} & = - R_8 i_{2,B} & = - \dfrac{R_4 R_8}{R_{14678}} i_B  \\
  v_{B,B} & = R_4 i_{4,B} + R_3 i_{3,B} & = \left[ \dfrac{R_4(R_{1678})}{R_{14678}} + \dfrac{R_2 R_3}{R_{23}} \right] i_B \\
\end{aligned}\end{split}
\end{equation*}\begin{equation*}
\begin{split}\begin{aligned}
 v_{A,B} & = R_{AB}   \, i_B & =     - 1.2414  \, \Omega \, i_B \\ 
 v_{B,B} & = R_{BB}   \, i_B & = \ \ \ 6.8073  \, \Omega \, i_B \\
   i_{,B}& = i_{/i_B} \, i_B & = \ \ \ 0.8966 \ \          i_B \\
\end{aligned}\end{split}
\end{equation*}
\begin{figure}[htbp]
\centering

\noindent\sphinxincludegraphics{{exam-2025-01-22-ese-01-a}.png}
\end{figure}

\sphinxAtStartPar
The equations of the equivalent algebraic system are
\begin{equation*}
\begin{split}\begin{cases}
 v_A & = v_{A,0} + R_{AA}   \, i_A + R_{AB}   \, i_B \\
 v_B & = v_{B,0} + R_{BA}   \, i_A + R_{BB}   \, i_B \\
 i   & = i_{ ,0} + i_{/i_A} \, i_A + i_{/i_B} \, i_B \\
\end{cases}\end{split}
\end{equation*}


\sphinxAtStartPar
and they can be used to write the currents as a function of the tensions
\begin{equation*}
\begin{split}\begin{aligned}
  i_A & = \dfrac{1}{\det \mathbf{R}} \left( R_{BB} \, \Delta v_A(t) - R_{AB} \, \Delta v_B(t) \right) \\
  i_B & = \dfrac{1}{\det \mathbf{R}} \left(-R_{BA} \, \Delta v_A(t) + R_{AA} \, \Delta v_B(t) \right) \\
\end{aligned}\end{split}
\end{equation*}


\sphinxAtStartPar
The switch command is off for \(t \le 0^-\), on for \(t > 0\),
\begin{equation*}
\begin{split}i_B(t \le 0^{-}) = 0 \qquad , \qquad v_B (t \ge 0^+) = 0 \ .\end{split}
\end{equation*}
\sphinxAtStartPar
\sphinxstylestrong{Steady solution for \(t \le 0^-\).} With switch open \(i_B = 0\) and steady conditions \(i_A = C \dot{v}_A = 0\),
\begin{equation*}
\begin{split}\begin{cases}
  v_A(0^-) & = v_{A,0} =     - 20.6900 \, V \\
  v_B(0^-) & = v_{B,0} = \ \ \ 12.4750 \, V \\
    i(0^-) & = i_{ ,0} = \    - 1.7241 \, A \\
\end{cases}\end{split}
\end{equation*}
\sphinxAtStartPar
\sphinxstylestrong{Transient dynamics}. For \(t \ge 0\), the switch is closed and thus \(v_B(t\ge 0^+) = 0\).
\begin{itemize}
\item {} 
\sphinxAtStartPar
\sphinxstylestrong{Tension across the switch} as a function of time
\begin{equation*}
\begin{split}\begin{aligned}
     v_{B}(t) & = v_{B,0} \, h(-t) = v_{B,0} ( 1 - h(t) ) \\
     \Delta v_B(t) & = v_{B}(t) - v_{B,0} = - v_{B,0} \,  h(t) \ . 
   \end{aligned}\end{split}
\end{equation*}
\item {} 
\sphinxAtStartPar
\sphinxstylestrong{Tension across the capacitor.} Writing \(i_A\) across the capacitor as a function of the tensions, the constitutive equation of the capacitor becomes
\begin{equation*}
\begin{split}\begin{aligned}
  0 & = C \dfrac{d \Delta v_A}{d t} + i_A = \\
    & = C \dfrac{d \Delta v_A}{d t} + \dfrac{1}{\det \mathbf{R}} \left( R_{BB} \, \Delta v_A - R_{AB} \, \Delta v_B \right)  \\ \\
   \end{aligned}\end{split}
\end{equation*}\begin{equation*}
\begin{split}\begin{cases}
      R_{eq} C \dfrac{d \Delta v_A}{d t} + \Delta v_A = \dfrac{ R_{AB} }{ R_{BB} } \, \Delta v_B(t) = - \dfrac{ R_{AB} }{ R_{BB} } v_{B,0} \, h(t)  \\ \\
      \Delta v_A(0) = 0 \ ,
   \end{cases}\end{split}
\end{equation*}
\sphinxAtStartPar
with
\begin{equation*}
\begin{split}\begin{aligned}
     R_{eq} & = \frac{\det \mathbf{R}}{R_{BB}} = 6.8081 \, \Omega \\
     \tau & = R_{eq} C = 3.4041 \cdot 10^{-3} \, s \\
     \det \mathbf{R} & = 46.345 \, \Omega^2 \\
   \end{aligned}\end{split}
\end{equation*}
\sphinxAtStartPar
The solution of the differential equation provides the difference of the tension through the capacitor w.r.t. the initial steady condition
\begin{equation*}
\begin{split}\Delta v_A(t) = \Delta v_{A,+\infty} \left( 1 - e^{-\frac{t}{\tau}} \right) \, h(t) \ ,\end{split}
\end{equation*}
\sphinxAtStartPar
with \(\Delta v_{A,+\infty} = -\frac{R_{AB}}{R_{BB}} v_{B,0} = 2.2742 \, V\). The voltage across the capacitor as a function of time \(t\) thus reads
\begin{equation*}
\begin{split}\begin{aligned}
     v_A(t) 
     & = v_{A,0} + \Delta v_A(t) = \\
     & = v_{A,0} + \Delta v_{A,+\infty} \left( 1 - e^{-\frac{t}{\tau}} \right) \, h(t) \ ,
   \end{aligned}\end{split}
\end{equation*}
\sphinxAtStartPar
so that the values
\begin{equation*}
\begin{split}\begin{aligned} 
     v_A(0^+) & = v_{A,0}  && = -20.69 \, V \\
     v_A(+\infty) & = v_{A,0} + \Delta V = -20.69 \, V + 2.2742 \, V && = -18.4158 \, V
   \end{aligned}\end{split}
\end{equation*}
\item {} 
\sphinxAtStartPar
\sphinxstylestrong{Current through the capacitor.}
\begin{equation*}
\begin{split}\begin{aligned}
      i_A(t)
      & = \dfrac{1}{\det \mathbf{R}} \left( R_{BB} \, \Delta v_A(t) - R_{AB} \, \Delta v_B(t) \right) = \\
      & = \dfrac{1}{\det \mathbf{R}} \left[ R_{BB} \, \left( -\frac{R_{AB}}{R_{BB}} v_{B,0} \right) \left( 1 - e^{-\frac{t}{\tau}} \right) \, h(t)  + R_{AB} \, v_{B,0} \, h(t) \right] = \\
      & = \frac{R_{AB}}{\det \mathbf{R}} v_{B,0} e^{-\frac{t}{\tau}} \, h(t) \ .
   \end{aligned}\end{split}
\end{equation*}
\sphinxAtStartPar
so that the values
\begin{equation*}
\begin{split}\begin{aligned} 
     i_A(0^+) & = \frac{R_{AB}}{\det \mathbf{R}} v_{B,0} = \frac{-1.2414 \, \Omega}{46.908 \, \Omega^2} \, 12.475 \, V = -0.334 \, A \\
     i_A(+\infty) & = v_{A,0} + \Delta V = -20.69 \, V + 2.2742 \, V && = 0.0 \, A
   \end{aligned}\end{split}
\end{equation*}
\sphinxAtStartPar
or with \(i_A = - C \frac{d \Delta v_A}{dt}\)…

\item {} 
\sphinxAtStartPar
\sphinxstylestrong{Current across the switch}
\begin{equation*}
\begin{split}\begin{aligned}
     i_B(t) 
     & = \frac{1}{R_{BB}} \bigg[ v_B(t) - v_{B,0} - R_{BA} i_A(t) \bigg] = \\
     & = \frac{1}{R_{BB}} \bigg[ - v_{B,0} - R_{BA} \frac{R_{AB}}{\det \mathbf{R}} v_{B,0} e^{-\frac{t}{\tau}} \bigg] \, h(t) = \\
     & = - \frac{v_{B,0}}{R_{BB}} \bigg[ 1 + \frac{R_{BA} R_{AB}}{\det \mathbf{R}} \, e^{-\frac{t}{\tau}} \bigg] \, h(t) \ .
   \end{aligned}\end{split}
\end{equation*}
\sphinxAtStartPar
so that the values
\begin{equation*}
\begin{split}\begin{aligned}
     i_B(0^+)     & = -\frac{v_{B,0}}{R_{BB}} \left[ 1 + \frac{R_{BA} R_{AB}}{\det \mathbf{R}} \right] = -\frac{v_{B,0} R_{AA}}{\det \mathbf{R}} =  - \frac{7.0345 \, \Omega}{46.345 \, \Omega^2} \, 12.475 \, V = - 1.8929 \, A \\
     i_B(+\infty) & = -\frac{v_{B,0}}{R_{BB}} = - \dfrac{12.475 \, V}{6.8073 \, \Omega} = -1.8320 \, A \ .
   \end{aligned}\end{split}
\end{equation*}
\item {} 
\sphinxAtStartPar
\sphinxstylestrong{Current \(i(t)\)}
\begin{equation*}
\begin{split}\begin{aligned}
     i(t)
     & = i_{0} - 0.4138 \, i_A(t) + 0.8966 \, i_B(t) = \\
     & = i_{0} + \left[ - 0.4138 \, i_{A,0^+} e^{-\frac{t}{\tau}}  + 0.8966 \, \left( i_{B,+\infty} + (i_{B,0^+} - i_{B,+\infty}) e^{-\frac{t}{\tau}} \right) \right] \, h(t) \ ,
   \end{aligned}\end{split}
\end{equation*}
\sphinxAtStartPar
so that
\begin{equation*}
\begin{split}\begin{aligned}
     i(0^+)     & = i_{0} - 0.4138 \, i_{A,0^+} + 0.8966 \, i_{B,0^+} = \\
                & = -1.7214 \, A - 0.4138 \, (-0.334 \, A) + 0.8966 \, (-1.8929 \, A) = -3.2831 \, A \\
     i(+\infty) & = i_{0} + 0.8966 \, i_{B,+\infty} = \\ 
                & = -1.7214 \, A + 0.8966 \, (-1.8320 \, A) = - 3.3671 \, A 
   \end{aligned}\end{split}
\end{equation*}
\end{itemize}



\sphinxAtStartPar
\sphinxstylestrong{Energy stored in the capacitor.}
\begin{equation*}
\begin{split}E_C(t) = \frac{1}{2} C v_A^2(t) \ ,\end{split}
\end{equation*}
\sphinxAtStartPar
and for \(t = \tau\),
\begin{equation*}
\begin{split}\begin{aligned}
  v_A(t)
  & = v_{A,0} + \Delta v_{A,+\infty} \left( 1 - e^{-\frac{t}{\tau}} \right) \, h(t) = \\
  & = -20.69 \, V + 2.2742 \, V \left( 1 - e^{-\frac{t}{\tau}} \right) \, h(t) \ ,
\end{aligned}\end{split}
\end{equation*}
\sphinxAtStartPar
and thus \(v_A(\tau) = -19.25 \, V\)
\begin{equation*}
\begin{split}E_C(\tau) = 0.5 \cdot 5 \cdot 10^{-4} \, F \cdot ( 19.25 \, V )^2 = 9.26 \cdot 10^{-2} \, J \ .\end{split}
\end{equation*}\end{sphinxadmonition}
\phantomsection \label{exercise:exam-24-09-06-exe-01}

\begin{sphinxadmonition}{note}{Exercise 10.7.3 (Exam 2024\sphinxhyphen{}09\sphinxhyphen{}06, Exercise 1.)}



\begin{figure}[htbp]
\centering

\noindent\sphinxincludegraphics{{exam-2024-09-06-ese-01}.png}
\end{figure}
\subsubsection*{Solution}
\end{sphinxadmonition}
\phantomsection \label{exercise:exam-24-07-22-exe-01}

\begin{sphinxadmonition}{note}{Exercise 10.7.4 (Exam 2024\sphinxhyphen{}07\sphinxhyphen{}22, Exercise 1.)}



\begin{figure}[htbp]
\centering

\noindent\sphinxincludegraphics{{exam-2024-07-22-ese-01}.png}
\end{figure}
\subsubsection*{Solution}
\end{sphinxadmonition}
\phantomsection \label{exercise:exam-24-02-13-exe-01-b}

\begin{sphinxadmonition}{note}{Exercise 10.7.5 (Exam 2024\sphinxhyphen{}02\sphinxhyphen{}13, Exercise 1.)}



\begin{figure}[htbp]
\centering

\noindent\sphinxincludegraphics{{exam-2024-02-13-ese-01-b}.png}
\end{figure}
\subsubsection*{Solution}
\end{sphinxadmonition}

\sphinxstepscope


\subsection{Harmonic regime of linear electrical grids}
\label{\detokenize{ch/electrical-engineering-exercises-harmonic:harmonic-regime-of-linear-electrical-grids}}\label{\detokenize{ch/electrical-engineering-exercises-harmonic:classical-electromagnetism-electrical-engineering-exercises-harmonic}}\label{\detokenize{ch/electrical-engineering-exercises-harmonic::doc}}\phantomsection \label{exercise:exam-25-02-11-exe-02}

\begin{sphinxadmonition}{note}{Exercise 10.7.6 (Exam 2025\sphinxhyphen{}02\sphinxhyphen{}11, Exercise 2.)}



\begin{figure}[htbp]
\centering

\noindent\sphinxincludegraphics{{exam-2025-02-11-ese-02}.png}
\end{figure}
\subsubsection*{Solution}

\sphinxAtStartPar
First {\hyperref[\detokenize{ch/electrical-engineering-networks:classical-electromagnetism-electrical-engineering-newtork-analysis-thevenin-1-port}]{\sphinxcrossref{\DUrole{std,std-ref}{one\sphinxhyphen{}port equivalent Thevenin circuit}}}} of the circuit with port \(A-B\) is evaluated, then {\hyperref[\detokenize{ch/electrical-engineering-networks-harmonic:classical-electromagnetism-electrical-engineering-newtork-analysis-harmonic-power}]{\sphinxcrossref{\DUrole{std,std-ref}{power flow in harmonic regime}}}} is discussed.

\sphinxAtStartPar
\sphinxstylestrong{Thevenin equivalent: voltage.} With open circuit in \(A-B\), current \(a\) flows in the lower branch and in impedence \(Z_1\). Clockwise loop currents \(i_1\) and \(i_2\) flows in the left and right loop respectively. Kirchhoff voltage laws in the left and right loops give
\begin{equation*}
\begin{split}\begin{aligned}
  0 & = e_1 - Z_L (i_1 + a) - (R_1 + Z_C) i_1 \\
  0 & = -e_2 - Z_2 (i_2 + a) - Z_3 i_2 \\
\end{aligned}
\quad \rightarrow \quad
\begin{aligned}
  i_1 & = \frac{e_1 - Z_L a}{Z_L + Z_C + R_1} \\
  i_2 & = -\frac{e_2 + Z_2 a}{Z_2 + Z_3} \\
\end{aligned}\end{split}
\end{equation*}
\sphinxAtStartPar
and thus using Kirchhoff voltage law on the loop with nodes \(A-B\) and closing through \(Z_1\) and \(R_1\),
\begin{equation*}
\begin{split}V_{Th} = R_1 i_1 + Z_1 a = \dots\end{split}
\end{equation*}
\sphinxAtStartPar
\sphinxstylestrong{Thevenin equivalent: impedence.} Opening circuit at the current generator, and replace tension generators with short circuits, the equivalent impedence is
\begin{equation*}
\begin{split}Z_{Th} = ( (Z_C + Z_L) \parallel R_1) + Z_1 \ .\end{split}
\end{equation*}
\sphinxAtStartPar
\sphinxstylestrong{Equivalent circuit.} Kirchhoff voltage law on the equivalent circuit reads
\begin{equation*}
\begin{split}0 = V_{Th} - Z_{Th} i - Z_{x} i = 0 \ ,\end{split}
\end{equation*}
\sphinxAtStartPar
and thus
\begin{equation*}
\begin{split}I = \frac{V_{Th}}{Z_{Th} + Z_{x}} = \dots\end{split}
\end{equation*}
\sphinxAtStartPar
\sphinxstylestrong{Power.} Complex power reads
\begin{equation*}
\begin{split}S = V I^* = Z_x |I|^2 = \frac{Z_x}{|Z_{Th} + Z_x|^2} |V_{th}|^2 \ ,\end{split}
\end{equation*}
\sphinxAtStartPar
Writing the impedence as \(Z_x = R_x + i X_x\), the active power reads
\begin{equation*}
\begin{split}P = \frac{ R_x }{ (R_{Th} + R_x)^2 + (X_{Th} + X_x)^2} |V_{Th}|^2 \ .\end{split}
\end{equation*}
\sphinxAtStartPar
With the physical constraints \(R \ge 0\), the problem is a constrained optimization problem of finding the maximum value of the function \(P(R_x, X_x)\) subject to the constraint \(R_x \ge 0\),
\begin{equation*}
\begin{split}\text{find } \ \max_{R_x, X_x} P(R_x, X_x) \qquad \text{s.t.} \qquad R_x \ge 0 \ .\end{split}
\end{equation*}
\sphinxAtStartPar
The denominator is the sum of two non negative terms, one function of \(R_x\) and one function of \(X_x\). The independent variable \(X_x\) only appears in this term at the denominator, so that this term must vanish at the solution of the optimization problem, and thus
\begin{equation*}
\begin{split}\widetilde{X}_x = - X_{Th} \ .\end{split}
\end{equation*}
\sphinxAtStartPar
The remaining term is a function of \(R_x\) only and proportional to
\begin{equation*}
\begin{split}f(R_x) = \frac{R_x}{(R_{Th} + R_x)^2} \ .\end{split}
\end{equation*}
\sphinxAtStartPar
Local extremes of this function is attained where
\begin{equation*}
\begin{split}\begin{aligned}
  0 = f'(R_x) 
  & = \frac{(R_{Th} + R_x)^2 - 2 R_x (R_{Th} + R_x))}{(R_{Th} + R_x)^4} = \\
  & = \frac{R_{Th}^2 - R_x^2 }{(R_{Th} + R_x)^4} \\
\end{aligned}\end{split}
\end{equation*}
\sphinxAtStartPar
and thus, within the physical limit of the problem, the local and global maximum of the function (check that \(f''(\widetilde{R}_x) < 0\)), is attained for
\begin{equation*}
\begin{split}\begin{aligned}
  \widetilde{R}_{x} & = R_{Th} \\
  \widetilde{Z}_{x} & = R_{Th} - i X_{Th}
\end{aligned}\end{split}
\end{equation*}
\sphinxAtStartPar
and the maximum active power is
\begin{equation*}
\begin{split}P_{max} = P(\widetilde{Z}_x) = \frac{|V_{Th}|^2}{4 R_{Th} } \ .\end{split}
\end{equation*}
\sphinxAtStartPar
while the reactive power in this condition reads
\begin{equation*}
\begin{split}Q = - \frac{ X_{Th} }{4 R^2_{Th}} |V_{Th}|^2 \ .\end{split}
\end{equation*}\end{sphinxadmonition}
\phantomsection \label{exercise:exam-25-02-11-exe-03}

\begin{sphinxadmonition}{note}{Exercise 10.7.7 (Exam 2025\sphinxhyphen{}02\sphinxhyphen{}11, Exercise 3.)}



\begin{figure}[htbp]
\centering

\noindent\sphinxincludegraphics{{exam-2025-02-11-ese-03}.png}
\end{figure}
\subsubsection*{Solution}

\sphinxAtStartPar
First {\hyperref[\detokenize{ch/electrical-engineering-networks-harmonic:classical-electromagnetism-electrical-engineering-newtork-analysis-harmonic-power}]{\sphinxcrossref{\DUrole{std,std-ref}{power flow in harmonic regime}}}} is used to calculate load impedence, then the electrical circuit is solved, and the power on the tension generator is computed.

\sphinxAtStartPar
\sphinxstylestrong{Load impedence \(Z_L\)}. Load impedence appears in the load constitutive equation \(V_L = Z_L I_L\), and can be evalauted from data about complex power,
\begin{equation*}
\begin{split}
   S_L  & = |S_L| e^{i \phi_L} = V_L I_L^* = Z_L |I|^2 = \frac{1}{Z_L^*} |V_L|^2 \\ 
\end{split}
\end{equation*}\begin{equation*}
\begin{split}Z_L = \frac{|V_L|^2}{|S_L|} e^{i \varphi_L}\end{split}
\end{equation*}
\sphinxAtStartPar
\sphinxstylestrong{Current \(I_s\).} From data of load power, it’s possible to evaluate the current \(I_s\). The current \(I_L\) through the load reads
\begin{equation*}
\begin{split}S_L = V_L I_L^* \qquad \rightarrow \qquad I_L = \frac{S_L^*}{V_L^*} = \frac{|S_L|}{|V_L|} e^{i(-\phi_L + \phi_V)}\end{split}
\end{equation*}
\sphinxAtStartPar
The three parallel sides act as current divider so that
\begin{equation*}
\begin{split}I_L = \frac{(R_3+Z_L)^{-1}}{(R_3+Z_L)^{-1} + ( (i X_1 ) \parallel (R_2 + i X_2) )^{-1}} I_s\end{split}
\end{equation*}
\sphinxAtStartPar
and thus
\begin{equation*}
\begin{split}I_s = |I_s| e^{i \varphi_{I_s}} = \dots\end{split}
\end{equation*}
\sphinxAtStartPar
\sphinxstylestrong{Equivalent circuit.} The impedence of the circuit powered by the tension generatore is
\begin{equation*}
\begin{split}Z_{eq} = R_1 + ( i X_1 \parallel (R_2 + i X_2) \parallel (R_3 + Z_L) ) \ .\end{split}
\end{equation*}
\sphinxAtStartPar
Given the equivalent impedance, and the current \(I_s\) the voltage across the tension generator is
\begin{equation*}
\begin{split}E_s = Z_{eq} I_s = |E_s| e^{i \varphi_{E_s}} \dots \ .\end{split}
\end{equation*}
\sphinxAtStartPar
and the power factor is \(\cos \varphi_s = \dots\), where
\begin{equation*}
\begin{split}\varphi_s = \varphi_{E_s} - \varphi_{I_s} = \dots \ . \end{split}
\end{equation*}\end{sphinxadmonition}
\phantomsection \label{exercise:exam-25-01-22-exe-02}

\begin{sphinxadmonition}{note}{Exercise 10.7.8 (Exam 2025\sphinxhyphen{}01\sphinxhyphen{}22, Exercise 2.)}



\begin{figure}[htbp]
\centering

\noindent\sphinxincludegraphics{{exam-2025-01-22-ese-02}.png}
\end{figure}
\subsubsection*{Solution}

\sphinxAtStartPar
First {\hyperref[\detokenize{ch/electrical-engineering-networks:classical-electromagnetism-electrical-engineering-newtork-analysis-thevenin-1-port}]{\sphinxcrossref{\DUrole{std,std-ref}{one\sphinxhyphen{}port equivalent Thevenin circuit}}}} of the circuit with port \(A-B\) is evaluated, then the equivalent circuit is solved to find the tension \(v(t)\) across the current generator, and {\hyperref[\detokenize{ch/electrical-engineering-networks-harmonic:classical-electromagnetism-electrical-engineering-newtork-analysis-harmonic-power}]{\sphinxcrossref{\DUrole{std,std-ref}{power flow in harmonic regime}}}} is discussed.

\sphinxAtStartPar
\sphinxstylestrong{Thevenin equivalent: voltage.} With an open circuit, the network can be split into two parts: the triangle in the upper\sphinxhyphen{}left side and the section in the right part.

\sphinxAtStartPar
In the triangular part, a current \(I_{a}\) flows in counter\sphinxhyphen{}clockwise direction, while current \(I_b\) flows in the right part in clockwise direction,
\begin{equation*}
\begin{split}\begin{aligned}
  I_a & = \frac{E_1}{Z_1 + Z_2} \\
  I_b & = \frac{E_2 + i \Omega L_5 A_2}{Z_4 + Z_5 + Z_3} \\
\end{aligned}\end{split}
\end{equation*}
\sphinxAtStartPar
as
\begin{equation*}
\begin{split}E_2 + \bigg( Z_4 + Z_3 \underbrace{- i \frac{1}{\Omega C_5} + i \Omega L_5}_{=Z_5} \bigg) I_b + i \Omega L_5 A_2 = 0 \ . \end{split}
\end{equation*}
\sphinxAtStartPar
with \(Z_k\) being the impedence of the \(k\)\sphinxhyphen{}th side. Thevenin voltage thus reads
\begin{equation*}
\begin{split}\begin{aligned}
  V_{Th} & = E_2 - Z_3 I_b + Z_2 I_a \\ 
\end{aligned}\end{split}
\end{equation*}
\sphinxAtStartPar
\sphinxstylestrong{Thevenin equivalent: impedence.} Equivalent impedence reads
\begin{equation*}
\begin{split}Z_{Th} = (Z_1 \parallel Z_2 + ( Z_3 \parallel (Z_4 + Z_5)))\end{split}
\end{equation*}
\sphinxAtStartPar
\sphinxstylestrong{Equivalent circuit.} Prescribed current \(A_1\) flows in the equivalent circuit, and the voltage across the current generator is evaluated with Krichhoff voltage law
\begin{equation*}
\begin{split}V_{A_1} - V_{Th} - Z_{Th} A_1 = 0 \ ,\end{split}
\end{equation*}\begin{equation*}
\begin{split}V_{A_1} = V_{Th} + Z_{Th} A_1 = |V_A| e^{i \varphi_{V_{A_1}}} \ .\end{split}
\end{equation*}
\sphinxAtStartPar
Signal in time is reconstructed using using the relation between effective and maximum amplitude of the oscillation and evaluating the real part of the signal \(|V_{A_1}| e^{i(\Omega t + \varphi_{V_{A_1}})}\)
\begin{equation*}
\begin{split}v_{A_1}(t) = \sqrt{2} |V_{A_1}| \cos(\Omega t + \varphi_{V_{A_1}}) \ .\end{split}
\end{equation*}
\sphinxAtStartPar
\sphinxstylestrong{Poer.} Using definitions of {\hyperref[\detokenize{ch/electrical-engineering-networks-harmonic:classical-electromagnetism-electrical-engineering-newtork-analysis-harmonic-power}]{\sphinxcrossref{\DUrole{std,std-ref}{power in circuits in harmonic regime}}}},
\begin{equation*}
\begin{split}\begin{aligned}
   S_{A_1}  & = V_{A_1} I_{A_1}^* \\
  |S_{A_1}| & = |V_{A_1}| |I_{A_1}| \\
   P_{A_1}  & = \text{re} \{ S_{A_1} \} \\
   Q_{A_1}  & = \text{im} \{ S_{A_1} \} \\
\end{aligned}\end{split}
\end{equation*}\end{sphinxadmonition}
\phantomsection \label{exercise:exam-24-09-06-exe-02}

\begin{sphinxadmonition}{note}{Exercise 10.7.9 (Exam 2024\sphinxhyphen{}09\sphinxhyphen{}06, Exercise 2.)}



\begin{figure}[htbp]
\centering

\noindent\sphinxincludegraphics{{exam-2024-09-06-ese-02}.png}
\end{figure}
\subsubsection*{Solution}
\end{sphinxadmonition}
\phantomsection \label{exercise:exam-24-07-22-exe-02}

\begin{sphinxadmonition}{note}{Exercise 10.7.10 (Exam 2024\sphinxhyphen{}07\sphinxhyphen{}22, Exercise 2.)}



\begin{figure}[htbp]
\centering

\noindent\sphinxincludegraphics{{exam-2024-07-22-ese-02}.png}
\end{figure}
\subsubsection*{Solution}
\end{sphinxadmonition}

\sphinxstepscope


\subsection{Three\sphinxhyphen{}phase electrical circuits in harmonic regime}
\label{\detokenize{ch/electrical-engineering-exercises-three-phase:three-phase-electrical-circuits-in-harmonic-regime}}\label{\detokenize{ch/electrical-engineering-exercises-three-phase:classical-electromagnetism-electrical-engineering-exercises-three-phase}}\label{\detokenize{ch/electrical-engineering-exercises-three-phase::doc}}
\begin{sphinxadmonition}{note}{Guidelines for solution}

\sphinxAtStartPar
Analyze the network as a standard configuration of a three\sphinxhyphen{}phase network ({\hyperref[\detokenize{ch/electrical-engineering-three-phase:classical-electromagnetism-electrical-engineering-three-phase-star-star}]{\sphinxcrossref{\DUrole{std,std-ref}{star\sphinxhyphen{}star}}}},…) and rely on results derived for {\hyperref[\detokenize{ch/electrical-engineering-three-phase:classical-electromagnetism-electrical-engineering-three-phase}]{\sphinxcrossref{\DUrole{std,std-ref}{three\sphinxhyphen{}phase circuits}}}}.

\sphinxAtStartPar
As an example, for a \sphinxstylestrong{star\sphinxhyphen{}star configuration}:
\begin{enumerate}
\sphinxsetlistlabels{\arabic}{enumi}{enumii}{}{.}%
\item {} 
\sphinxAtStartPar
evaluate load impedances, impedances in parallel with the generators, interconnections between phases

\item {} 
\sphinxAtStartPar
evaluate voltage difference across the centers of the stars, \(v_{AB}\)

\item {} 
\sphinxAtStartPar
once \(v_{AB}\) is known, it should be easier to evaluate currents and voltages in the grid with KCL and KVL

\item {} 
\sphinxAtStartPar
use relations of {\hyperref[\detokenize{ch/electrical-engineering-networks-harmonic:classical-electromagnetism-electrical-engineering-newtork-analysis-harmonic-power}]{\sphinxcrossref{\DUrole{std,std-ref}{power in harmonic regime}}}}, to answer the questions about power: just remember the difference between maximum and effective values, and that a wattmeter measures the active power

\end{enumerate}
\end{sphinxadmonition}
\phantomsection \label{exercise:exam-24-09-06-exe-03}

\begin{sphinxadmonition}{note}{Exercise 10.7.11 (Exam 2024\sphinxhyphen{}09\sphinxhyphen{}06, Exercise 3.)}



\begin{figure}[htbp]
\centering

\noindent\sphinxincludegraphics{{exam-2024-09-06-ese-03}.png}
\end{figure}
\subsubsection*{Solution}

\sphinxAtStartPar
This network is a star\sphinxhyphen{}star connection with impedances
\begin{equation*}
\begin{split}\begin{aligned}
  Z_g & = ( R_1 + s L_1 ) \parallel \frac{1}{s C_1} \qquad g = 1:3 \\
  Z_4 & = R_2 + \frac{1}{s C_2}
\end{aligned}\end{split}
\end{equation*}
\sphinxAtStartPar
and inter\sphinxhyphen{}connection between phases \(2\) and \(3\) with impedance \(Z_4\).



\sphinxAtStartPar
\sphinxstylestrong{Voltage \(v_{AB}\).}
\begin{equation*}
\begin{split}v_{AB} = \dfrac{ \sum_{g=1}^{3} Y_g \, e_g }{\sum_{k=1}^{4} Y_4}\end{split}
\end{equation*}
\sphinxAtStartPar
Generation and loads are equilibrated, and thus \(\sum_{g=1}^{3} Y_g \, e_g = 0\), and \(v_{AB} = 0\).

\sphinxAtStartPar
\sphinxstylestrong{Current \(i_{Z_2}\).} As \(v_{AB}=0\), then \(i_{Z_2} = 0\), as in general it whould be \(i_{Z_2} = \frac{v_{AB}}{R_2 + \frac{1}{sC_2}}\).

\sphinxAtStartPar
\sphinxstylestrong{Current \(i_{Z_4}\).} With KVL on the loop with the two tension generators \(e_2\), \(e_3\) closed with \(Z_4\)
\begin{equation*}
\begin{split}\begin{aligned}
  0 & = e_3 + Z_4 i_{Z_4} - e_2 \\
  \rightarrow \quad i_{Z_4} & = \frac{e_2 - e_3}{Z_4}
\end{aligned}\end{split}
\end{equation*}
\sphinxAtStartPar
\sphinxstylestrong{Currents \(i_{e_2}\).} Current \(i_{e_2}\) through the generator are evaluated through KVL between the centers of the stars,
\begin{equation*}
\begin{split}\begin{aligned}
  0
  & = e_2 - \dfrac{1}{\frac{1}{R_1 + s L_1} + s C_1 } i_{e_2} - v_{AB} \\
  \rightarrow \quad i_{e_2} & = \left[ \frac{1}{R_1 + s L_1} + s C_1 \right] e_2 \\
\end{aligned}\end{split}
\end{equation*}
\sphinxAtStartPar
\sphinxstylestrong{Powers of generator \(2\).}
\begin{equation*}
\begin{split}\begin{aligned}
  S_2 & = V_2 I_2^* \\
  A_2 & = |S_2| \\
  P_2 & = \text{re} \{ S_2 \} \\
  Q_2 & = \text{im} \{ S_2 \} \ ,
\end{aligned}\end{split}
\end{equation*}
\sphinxAtStartPar
using the effective values of tension and current \(V_2\), \(I_2\).
\end{sphinxadmonition}
\phantomsection \label{exercise:exam-24-07-22-exe-03}

\begin{sphinxadmonition}{note}{Exercise 10.7.12 (Exam 2024\sphinxhyphen{}07\sphinxhyphen{}22, Exercise 3.)}



\begin{figure}[htbp]
\centering

\noindent\sphinxincludegraphics{{exam-2024-07-22-ese-03}.png}
\end{figure}
\subsubsection*{Solution}

\sphinxAtStartPar
This network is a star\sphinxhyphen{}star connection with impedances
\begin{equation*}
\begin{split}\begin{aligned}
  Z_1 & = ( R_1 + j X_{C_1} ) \parallel R_2 \\
  Z_2 & = 0 \\
  Z_3 & = ( R_3 + j X_{L_2} ) \parallel j X_{L_1} \\
  Z_4 & = j X_{C_2}
\end{aligned}\end{split}
\end{equation*}
\sphinxAtStartPar
and inter\sphinxhyphen{}connection between phase \(3\) and the netural with \sphinxstylestrong{resistance \(R_4\)}, before \(Z_4\), and thus \sphinxstylestrong{in parallel with the generator \(3\)}.

\sphinxAtStartPar
\sphinxstylestrong{Voltage \(v_{AB}\).} As \(Z_2 = 0\), it’s not possible to directly use
\begin{equation*}
\begin{split}v_{AB} = \dfrac{ \sum_{g=1}^{3} Y_g \, e_g }{\sum_{k=1}^{4} Y_4} \ ,\end{split}
\end{equation*}
\sphinxAtStartPar
or this must be used with the limit \(Y_2 \rightarrow + \infty\), and thus
\begin{equation*}
\begin{split}v_{AB} = e_2 \ .\end{split}
\end{equation*}
\sphinxAtStartPar
\sphinxstylestrong{Wattmeter tension \(v_W\).} KVL with the generators \(2\) and \(3\),
\begin{equation*}
\begin{split}v_W = e_2 - e_3 \ .\end{split}
\end{equation*}
\sphinxAtStartPar
\sphinxstylestrong{Wattmeter current \(i_w = i_{e_2}\).} KCL on the center of generation star, \(0 = i_{e_1} + i_{e_2} + i_{3} + i_{4}\), with
\begin{equation*}
\begin{split}\begin{aligned}
  i_{e_1} & =  \frac{1}{Z_1} ( e_1 - v_{AB} ) \\
  i_{3}   & =  \frac{1}{Z_3} ( e_3 - v_{AB} ) \\
  i_{4}   & = -\frac{1}{Z_4}   v_{AB}   \ ,
\end{aligned}\end{split}
\end{equation*}
\sphinxAtStartPar
being \(i_3 = i_{e_3} + i_{R_4}\) the sum of the current in the parallel connection on the branch \(3\) of the generation. Thus, current \(i_{e_2}\) reads
\begin{equation*}
\begin{split}\begin{aligned}
  i_{e_2} 
  & = - i_{e_1} - i_{3} - i_{4} = \\
  & = - \frac{e_1}{Z_1} - \frac{e_3}{Z_3} + \left(  \frac{1}{Z_1} + \frac{1}{Z_3} + \frac{1}{Z_4}  \right) v_{AB}
\end{aligned}\end{split}
\end{equation*}
\sphinxAtStartPar
\sphinxstylestrong{Wattmeter.} Wattmeter reading provides the active power
\begin{equation*}
\begin{split}P_w = \text{re} \{ S_w \} = \text{re} \{ v_w i_w^* \} \ .\end{split}
\end{equation*}
\sphinxAtStartPar
\sphinxstylestrong{Power on \(C_2\).} Current and voltage across \(C_2\) are
\begin{equation*}
\begin{split}\begin{aligned}
  i_{C_2} & = i_4 \\
  v_{C_2} & = Z_{C_2} i_{C_2} = \frac{1}{s C_2} i_{C_2} \ ,
\end{aligned}\end{split}
\end{equation*}
\sphinxAtStartPar
and the complex power is
\begin{equation*}
\begin{split}s = V_{C_2} I_{C_2}^* \ .\end{split}
\end{equation*}\end{sphinxadmonition}
\phantomsection \label{exercise:exam-24-06-19-exe-01}

\begin{sphinxadmonition}{note}{Exercise 10.7.13 (Exam 2024\sphinxhyphen{}06\sphinxhyphen{}19, Exercise 1.)}



\begin{figure}[htbp]
\centering

\noindent\sphinxincludegraphics{{exam-2024-06-19-ese-01}.png}
\end{figure}
\subsubsection*{Solution}

\sphinxAtStartPar
This network is a star\sphinxhyphen{}star connection with impedances
\begin{equation*}
\begin{split}\begin{aligned}
  Z_1 & = ( R_2 + j X_{L_2} ) \parallel ( j X_{C_1} ) \\
  Z_2 & = ( R_1 \parallel 0 ) \\
  Z_3 & = ( R_3 + j X_{C_2} ) \parallel j X_{L_1} \\
\end{aligned}\end{split}
\end{equation*}
\sphinxAtStartPar
with \(L_2\) and \(R_4\) in parallel with generator \(e_2\). As \(R_1\) is in parallel with a short\sphinxhyphen{}circuit in \(Z_2\), this impedance is zero and as it is the current through \(R_1\). There’s no neutral.

\sphinxAtStartPar
\sphinxstylestrong{Voltage \(v_{AB}\).} As \(Z_2 = 0\) (see previous exercise), the voltage between the centers of the stars is
\begin{equation*}
\begin{split}v_{AB} = e_2 \ .\end{split}
\end{equation*}
\sphinxAtStartPar
\sphinxstylestrong{Wattmeter tension \(v_W\).} KVL with the generators \(2\) and \(3\),
\begin{equation*}
\begin{split}v_W = e_1 - e_3 \ .\end{split}
\end{equation*}
\sphinxAtStartPar
\sphinxstylestrong{Wattmeter current \(i_w = i_{2}\).} KCL on the center of generation star, \(0 = i_{e_1} + i_{2} + i_{e_3}\), with
\begin{equation*}
\begin{split}\begin{aligned}
  i_{e_1} & =  \frac{1}{Z_1} ( e_1 - e_2 ) \\
  i_{e_3} & =  \frac{1}{Z_3} ( e_3 - e_2 ) \\
\end{aligned}\end{split}
\end{equation*}
\sphinxAtStartPar
being \(i_2 = i_{e_2} + i_{L_1} + i_{R_4}\) the sum of the current in the parallel connection on the branch \(2\) of the generation. Thus, current \(i_{w}\) reads
\begin{equation*}
\begin{split}\begin{aligned}
  i_w = i_{2} 
  & = - i_{e_1} - i_{e_3} = \\
  & = \frac{1}{Z_1} ( e_2 - e_1 ) + \frac{1}{Z_3} ( e_2 - e_3 ) \\
\end{aligned}\end{split}
\end{equation*}
\sphinxAtStartPar
\sphinxstylestrong{Wattmeter.} Wattmeter reading provides the active power
\begin{equation*}
\begin{split}P_w = \text{re} \{ S_w \} = \text{re} \{ v_w i_w^* \} \ .\end{split}
\end{equation*}
\sphinxAtStartPar
\sphinxstylestrong{Power of tension generator \(e_1\).}
\begin{equation*}
\begin{split}s_{e_1} = e_{2} i_{e_2}^* \ .\end{split}
\end{equation*}
\sphinxAtStartPar
…
\end{sphinxadmonition}
\phantomsection \label{exercise:exam-24-02-13-exe-02}

\begin{sphinxadmonition}{note}{Exercise 10.7.14 (Exam 2024\sphinxhyphen{}02\sphinxhyphen{}13, Exercise 2.)}



\begin{figure}[htbp]
\centering

\noindent\sphinxincludegraphics{{exam-2024-02-13-ese-02}.png}
\end{figure}
\subsubsection*{Solution}
\end{sphinxadmonition}

\sphinxstepscope

\sphinxAtStartPar
1classical\sphinxhyphen{}electromagnetism:electrical\sphinxhyphen{}engineering\sphinxhyphen{}exercises:electromagnetic)=


\subsection{Electromagnetic circuits}
\label{\detokenize{ch/electrical-engineering-exercises-electromagnetic:electromagnetic-circuits}}\label{\detokenize{ch/electrical-engineering-exercises-electromagnetic::doc}}\phantomsection \label{exercise:exam-25-01-22-exe-03}

\begin{sphinxadmonition}{note}{Exercise 10.7.15 (Exam 2025\sphinxhyphen{}01\sphinxhyphen{}22, Exercise 3.)}



\begin{figure}[htbp]
\centering

\noindent\sphinxincludegraphics{{exam-2025-01-22-ese-03}.png}
\end{figure}
\end{sphinxadmonition}
\phantomsection \label{exercise:exam-24-06-19-exe-02}

\begin{sphinxadmonition}{note}{Exercise 10.7.16 (Exam 2024\sphinxhyphen{}06\sphinxhyphen{}19, Exercise 2.)}



\begin{figure}[htbp]
\centering

\noindent\sphinxincludegraphics{{exam-2025-01-22-ese-03}.png}
\end{figure}
\end{sphinxadmonition}
\phantomsection \label{exercise:exam-24-02-13-exe-01-a}

\begin{sphinxadmonition}{note}{Exercise 10.7.17 (Exam 2024\sphinxhyphen{}02\sphinxhyphen{}13, Exercise 1a.)}



\begin{figure}[htbp]
\centering

\noindent\sphinxincludegraphics{{exam-2024-02-13-ese-01-a}.png}
\end{figure}
\end{sphinxadmonition}

\sphinxstepscope


\part{Metodi numerici}

\sphinxstepscope

\begin{sphinxuseclass}{sd-container-fluid}
\begin{sphinxuseclass}{sd-sphinx-override}
\begin{sphinxuseclass}{sd-p-0}
\begin{sphinxuseclass}{sd-mt-2}
\begin{sphinxuseclass}{sd-mb-4}
\begin{sphinxuseclass}{sd-row}
\begin{sphinxuseclass}{sd-row-cols-2}
\begin{sphinxuseclass}{sd-gx-2}
\begin{sphinxuseclass}{sd-gy-1}
\begin{sphinxuseclass}{sd-col}
\begin{sphinxuseclass}{sd-d-flex-row}
\begin{sphinxuseclass}{sd-align-minor-center}
\begin{sphinxuseclass}{sd-container-fluid}
\begin{sphinxuseclass}{sd-sphinx-override}
\begin{sphinxuseclass}{sd-row}
\begin{sphinxuseclass}{sd-row-cols-2}
\begin{sphinxuseclass}{sd-row-cols-xs-2}
\begin{sphinxuseclass}{sd-row-cols-sm-3}
\begin{sphinxuseclass}{sd-row-cols-md-3}
\begin{sphinxuseclass}{sd-row-cols-lg-3}
\begin{sphinxuseclass}{sd-gx-3}
\begin{sphinxuseclass}{sd-gy-1}
\begin{sphinxuseclass}{sd-col}
\begin{sphinxuseclass}{sd-col-auto}
\begin{sphinxuseclass}{sd-d-flex-row}
\begin{sphinxuseclass}{sd-align-minor-center}
\sphinxAtStartPar
basics

\end{sphinxuseclass}
\end{sphinxuseclass}
\end{sphinxuseclass}
\end{sphinxuseclass}
\begin{sphinxuseclass}{sd-col}
\begin{sphinxuseclass}{sd-col-auto}
\begin{sphinxuseclass}{sd-d-flex-row}
\begin{sphinxuseclass}{sd-align-minor-center}
\sphinxAtStartPar
26 apr 2025

\end{sphinxuseclass}
\end{sphinxuseclass}
\end{sphinxuseclass}
\end{sphinxuseclass}
\begin{sphinxuseclass}{sd-col}
\begin{sphinxuseclass}{sd-col-auto}
\begin{sphinxuseclass}{sd-d-flex-row}
\begin{sphinxuseclass}{sd-align-minor-center}
\sphinxAtStartPar
1 min read

\end{sphinxuseclass}
\end{sphinxuseclass}
\end{sphinxuseclass}
\end{sphinxuseclass}
\end{sphinxuseclass}
\end{sphinxuseclass}
\end{sphinxuseclass}
\end{sphinxuseclass}
\end{sphinxuseclass}
\end{sphinxuseclass}
\end{sphinxuseclass}
\end{sphinxuseclass}
\end{sphinxuseclass}
\end{sphinxuseclass}
\end{sphinxuseclass}
\end{sphinxuseclass}
\end{sphinxuseclass}
\end{sphinxuseclass}
\end{sphinxuseclass}
\end{sphinxuseclass}
\end{sphinxuseclass}
\end{sphinxuseclass}
\end{sphinxuseclass}
\end{sphinxuseclass}
\end{sphinxuseclass}
\end{sphinxuseclass}

\chapter{Green’s function method}
\label{\detokenize{ch/green-function:green-s-function-method}}\label{\detokenize{ch/green-function:classical-electromagnetism-green-function}}\label{\detokenize{ch/green-function::doc}}

\section{Poisson equation}
\label{\detokenize{ch/green-function:poisson-equation}}
\sphinxAtStartPar
General Poisson’s problem
\begin{equation*}
\begin{split}\begin{cases}
  - \nabla^2 \mathbf{u}(\mathbf{r}, t) = \mathbf{f}(\mathbf{r},t) \\
  \text{+ b.c.}
\end{cases}\end{split}
\end{equation*}
\sphinxAtStartPar
with common boundary conditions
\begin{equation*}
\begin{split}\begin{cases}
\mathbf{u} = \mathbf{g} & \quad \text{on $S_D$} \\
\hat{\mathbf{n}} \cdot \nabla \mathbf{u} = \mathbf{h} & \quad \text{on $S_N$}
\end{cases}\end{split}
\end{equation*}
\sphinxAtStartPar
over Dirichlet and Neumann regions of the boundary.

\sphinxAtStartPar
Poisson’s problem for Green’s function, in infinite domain
\begin{equation*}
\begin{split}
  - \nabla_{\mathbf{r}}^2 G(\mathbf{r}; \mathbf{r}_0) = \delta(\mathbf{r} - \mathbf{r}_0) \\
\end{split}
\end{equation*}
\sphinxAtStartPar
Green’s function method
\begin{equation*}
\begin{split}\begin{aligned}
  E(\mathbf{r}_0, t) u_i(\mathbf{r}_0, t) 
  & =   \int_{\mathbf{r} \in \Omega} u_i(\mathbf{r},t) \delta(\mathbf{r}-\mathbf{r}_0) = \\
  & = - \int_{\mathbf{r} \in \Omega} u_i(\mathbf{r},t) \nabla_{\mathbf{r}}^2 G(\mathbf{r}-\mathbf{r}_0) = \\
  & = - \int_{\mathbf{r} \in \Omega} \nabla_{\mathbf{r}} \cdot ( u_i \nabla_{\mathbf{r}} G - G \nabla_{\mathbf{r}} u_i) - \int_{\mathbf{r} \in \Omega} G \nabla^2 u_i= \\
  & = - \oint_{\mathbf{r} \in \partial \Omega} \hat{\mathbf{n}} \cdot ( u_i \nabla_{\mathbf{r}} G - G \nabla_{\mathbf{r}} u_i) + \int_{\mathbf{r} \in \Omega} G(\mathbf{r}-\mathbf{r}_0) f_i(\mathbf{r}, t) . \\
\end{aligned}\end{split}
\end{equation*}
\sphinxAtStartPar
An integro\sphinxhyphen{}differential boundary problem can be written using boundary conditions. As an example, using Dirichlet and Neumann boundary conditions, the integro\sphinxhyphen{}differential problem reads
\begin{equation*}
\begin{split}\begin{aligned}
&  E(\mathbf{r}_0, t) \mathbf{u}(\mathbf{r}_0, t) 
+ \int_{\mathbf{r} \in S_N} \mathbf{u}(\mathbf{r},t) \, \hat{\mathbf{n}} \cdot \nabla_{\mathbf{r}} G(\mathbf{r}-\mathbf{r}_0)
- \int_{\mathbf{r} \in S_D} G(\mathbf{r}-\mathbf{r}_0) \, \hat{\mathbf{n}} \cdot \nabla_{\mathbf{r}} \mathbf{u}(\mathbf{r},t) = \\ 
& =
- \int_{\mathbf{r} \in S_D} \mathbf{g}(\mathbf{r},t) \, \hat{\mathbf{n}} \cdot \nabla_{\mathbf{r}} G(\mathbf{r}-\mathbf{r}_0)
+ \int_{\mathbf{r} \in S_N} G(\mathbf{r}-\mathbf{r}_0) \, \mathbf{h}(\mathbf{r},t)  
+ \int_{\mathbf{r} \in \Omega} G(\mathbf{r}-\mathbf{r}_0) \, \mathbf{f}(\mathbf{r}, t) . \\
\end{aligned}\end{split}
\end{equation*}
\sphinxAtStartPar
Green’s function of the Poisson\sphinxhyphen{}Laplace equation reads
\begin{equation*}
\begin{split}G(\mathbf{r};\mathbf{r}_0) = \frac{1}{4 \pi} \frac{1}{\left| \mathbf{r}-\mathbf{r}_0 \right|} \ .\end{split}
\end{equation*}\subsubsection*{Green’s function of the Laplace equation}
\begin{equation*}
\begin{split}-\nabla^2 G = 0 \qquad \text{for $\mathbf{r} \ne \mathbf{r}_0$}\end{split}
\end{equation*}
\sphinxAtStartPar
Solutions with spherical symmetry,
\begin{equation*}
\begin{split}0 = \nabla^2 G = \frac{1}{r^2} \left( r^2 G' \right)'
\quad \rightarrow \quad
G'(r) = \frac{A}{r^2} \quad \rightarrow \quad G(r) = - \frac{A}{r} + B
\end{split}
\end{equation*}
\sphinxAtStartPar
Choosing \(B = 0\) s.t. \(G(r) \rightarrow 0\) as \(r \rightarrow \infty\), and integrating over a sphere centered in \(r=0\) to get \(A = -\frac{1}{4 \pi}\),
\begin{equation*}
\begin{split}1 = \int_{V} \delta(r) = - \int_{V} \nabla^2 G = - \oint_{\partial V} \hat{\mathbf{n}} \cdot \nabla G
= -\oint_{\partial V} \hat{\mathbf{r}} \cdot \hat{\mathbf{r}} \frac{A}{r^2} = - 4 \pi \, A \end{split}
\end{equation*}

\section{Helmholtz equation}
\label{\detokenize{ch/green-function:helmholtz-equation}}
\sphinxAtStartPar
\sphinxstylestrong{todo} from Fourier to Laplace trasnform in the first lines of this section

\sphinxAtStartPar
A Helmholtz’s equation can be thouoght as the time Fourier transform of a wave equation,
\begin{equation*}
\begin{split}\begin{cases}
  \dfrac{1}{c^2} \partial_{tt} \mathbf{u}(\mathbf{r},t) - \nabla^2 \mathbf{u}(\mathbf{r},t) = \mathbf{f}(\mathbf{r},t) \\
  \text{+ b.c.} \\
  \text{+ i.c.} \ ,
\end{cases}\end{split}
\end{equation*}
\sphinxAtStartPar
Fourier transform in time of field \(\mathbf{u}(\mathbf{r},t)\) reads
\begin{equation*}
\begin{split}\tilde{\mathbf{u}}(\mathbf{r}, \omega) = \mathscr{F}\{ \mathbf{u}(\mathbf{r},t) \} = \int_{t=-\infty}^{+\infty} \mathbf{u}(\mathbf{r},t) e^{-i \omega t} \, d \omega\end{split}
\end{equation*}
\sphinxAtStartPar
and, if \(\mathbf{u}(\mathbf{r},t)\) is compact in time, Fourier transform of its time partial derivatives read
\begin{equation*}
\begin{split}\begin{aligned}
  \mathscr{F}\{ \dot{\mathbf{u}}(\mathbf{r},t) \} 
  & = \int_{t=-\infty}^{+\infty} \dot{\mathbf{u}}(\mathbf{r},t) e^{-i \omega t} \, d \omega = \\
  & = \mathbf{u}(\mathbf{r},t) e^{-i \omega t} \big|_{t=-\infty}^{+\infty} + i \omega \int_{t=-\infty}^{+\infty} \mathbf{u}(\mathbf{r},t) e^{-i \omega t} \, d \omega = \\
  & = i \omega \mathscr{F}\{  \mathbf{u}(\mathbf{r},t) \}
\end{aligned}\end{split}
\end{equation*}\begin{equation*}
\begin{split}\mathscr{F}\{ \partial_t^n \mathbf{u}(\mathbf{r},t) \} = (i \omega)^n \tilde{\mathbf{u}} \ .\end{split}
\end{equation*}
\sphinxAtStartPar
The differential problem in the transformed domain thus reads
\begin{equation*}
\begin{split}- \frac{\omega^2}{c^2} \tilde{\mathbf{u}} - \nabla^2 \tilde{\mathbf{u}} = \tilde{\mathbf{f}}\end{split}
\end{equation*}
\sphinxAtStartPar
Green’s function of Helmholtz’e equation reads
\begin{equation*}
\begin{split}G(\mathbf{r}, s) =
  \alpha^+ \frac{e^{ \frac{s|\mathbf{r} - \mathbf{r}_0|}{c}}}{|\mathbf{r} - \mathbf{r}_0|} +
  \alpha^- \frac{e^{-\frac{s|\mathbf{r} - \mathbf{r}_0|}{c}}}{|\mathbf{r} - \mathbf{r}_0|}
\end{split}
\end{equation*}
\sphinxAtStartPar
with \(\alpha^+ + \alpha^- = \frac{1}{4 \pi}\).

\sphinxAtStartPar
Being the Laplace transform,
\begin{equation*}
\begin{split}\mathscr{L}\{ f(t) \} = \int_{t=0^-}^{+\infty} f(t) e^{-st} dt \ ,\end{split}
\end{equation*}
\sphinxAtStartPar
the Laplace transform of a causal function with time delay \(\tau \ge 0\) reads
\begin{equation*}
\begin{split}\mathscr{L}\{ f(t-\tau) \} = \int_{t=0^-}^{+\infty} f(t-\tau) e^{-st} dt = \int_{z = - \tau}^{+\infty} f(z) e^{-s(z+\tau)} \, dz = e^{-s\tau} \, \int_{z = 0}^{+\infty} f(z) e^{-s z} \, dz = e^{-s \tau} \, \mathscr{L}\{ f(t) \}\end{split}
\end{equation*}
\sphinxAtStartPar
having used causality \(f(t) = 0\) for \(t < 0\). Laplace transform of Dirac’s delta \(\delta(t)\) reads
\begin{equation*}
\begin{split}\mathscr{L}\{ \delta(t) \} = \int_{t=0^-}^{+\infty} \delta(t) \, dt = 1 \ ,\end{split}
\end{equation*}
\sphinxAtStartPar
so that \(e^{-s \tau} = e^{- s \tau} \, 1 = \mathscr{L}\{ \delta(t-\tau) \}\).

\sphinxAtStartPar
Thus, Green’s function for the wave equation reads
\begin{equation*}
\begin{split}G(\mathbf{r},t; \mathbf{r}_0, t_0) = 
  \alpha^+ \frac{ \delta \left( t - t_0 + \frac{|\mathbf{r}-\mathbf{r}_0|}{c} \right)}{|\mathbf{r} - \mathbf{r}_0|} +
  \alpha^- \frac{ \delta \left( t - t_0 - \frac{|\mathbf{r}-\mathbf{r}_0|}{c} \right)}{|\mathbf{r} - \mathbf{r}_0|}
\end{split}
\end{equation*}
\sphinxAtStartPar
If \(t \ge t_0\), and \(G(\mathbf{r}, t; \mathbf{r}_0, t_0)\) connects the past \(t_0\) with the future \(t\), the first term is not causal, and thus \(\alpha^+ = 0\) and
\begin{equation*}
\begin{split}G(\mathbf{r},t; \mathbf{r}_0, t_0) = \frac{1}{4 \pi} \frac{ \delta \left( t - t_0 - \frac{|\mathbf{r}-\mathbf{r}_0|}{c} \right)}{|\mathbf{r} - \mathbf{r}_0|} \ .\end{split}
\end{equation*}\subsubsection*{Green’s function of Helmholtz’s equation}
\begin{equation*}
\begin{split}\frac{s^2}{c^2} G - \nabla^2 G = \delta(r)\end{split}
\end{equation*}\begin{equation*}
\begin{split}G(r) = \frac{\alpha e^{k r} + \beta e^{-kr}}{r}\end{split}
\end{equation*}
\sphinxAtStartPar
Proof:
\begin{itemize}
\item {} 
\sphinxAtStartPar
Gradient
\begin{equation*}
\begin{split}\nabla G(r) = \hat{\mathbf{r}} \partial_r G = \hat{\mathbf{r}} \frac{\alpha (k r - 1) e^{kr} + \beta(-k r - 1)e^{-kr}}{r^2}\end{split}
\end{equation*}
\item {} 
\sphinxAtStartPar
Laplacian
\begin{equation*}
\begin{split}\begin{aligned}
    \nabla^2 G(r) & = \frac{1}{r^2} \left( r^2 G'(r) \right)' = \\
    & = \frac{1}{r^2} \left(  \alpha (k r - 1) e^{kr} + \beta(-k r - 1)e^{-kr}\right)' = \\
    & = \frac{1}{r^2} \left( \alpha k e^{kr} + \alpha k^2 r  e^{kr} - \alpha k e^{kr} - \beta k e^{-kr} + \beta k^2 r e^{-kr} + \beta k e^{-kr}  \right) = \\
    & = \frac{1}{r} \left( \alpha e^{kr} + \beta e^{-kr} \right) k^2 = k^2 G(r) \ .
  \end{aligned}\end{split}
\end{equation*}
\sphinxAtStartPar
and thus \(k^2 G(r) - \nabla^2 G = 0\), for \(r \ne 0\);

\item {} 
\sphinxAtStartPar
Unity
\begin{equation*}
\begin{split}1 = \int_{V} \delta(r) = \int_V \left( k^2 G - \nabla^2 G \right) = \int_V k^2 G - \oint_{\partial V} \hat{\mathbf{n}} \cdot \nabla G \end{split}
\end{equation*}
\sphinxAtStartPar
the second term is the sum of two contributions of the form
\begin{equation*}
\begin{split}\oint_{\partial V} \hat{\mathbf{n}} \cdot \nabla G^{\pm} = \oint_{\partial V} \frac{\alpha^{\pm}(\pm k r - 1) e^{\pm k r}}{r^2} = 4 \pi \alpha^{\pm} (\pm k r - 1) e^{\pm k r}\end{split}
\end{equation*}
\sphinxAtStartPar
the first term is the sum of two contributions of the form
\begin{equation*}
\begin{split}\begin{aligned}
    k^2 \int_{V} G(r)
      & = k^2 \int_{V} \frac{\alpha^{\pm} e^{\pm k r}}{r} = \\
      & = k^2 \alpha^{\pm} \int_{R = 0}^{r} \int_{\phi=0}^{\pi} \int_{\theta=0}^{2 \pi} \frac{e^{\pm k R}}{R} R^2 \sin \phi \, dR \, d \phi \, d \theta = \\
      & = k^2 \alpha^{\pm} \, 4 \pi \int_{R = 0}^{r} R \, e^{\pm k R} \, dR \ .
  \end{aligned}\end{split}
\end{equation*}
\sphinxAtStartPar
the last integral can be evaluated with integration by parts
\begin{equation*}
\begin{split}\begin{aligned}
    \int_{R = 0}^{r} R \, e^{\pm k R} \, dR
    & = \left.\left[ \frac{1}{\pm k} e^{\pm k R } R \right]\right|_{R=0}^{r} \mp \frac{1}{k} \int_{R=0}^{r} e^{\pm k R} \, dR = \\
    & = \frac{1}{\pm k} e^{\pm k r } r  - \frac{1}{k^2} e^{\pm k R} + \frac{1}{k^2} = \\
  \end{aligned}\end{split}
\end{equation*}
\sphinxAtStartPar
Thus summing everything together,
\begin{equation*}
\begin{split}\begin{aligned}
    1 & = \alpha^+ \left[ 4 \pi k^2 \left( \frac{r}{k} e^{k r} - \frac{1}{k^2} e^{kr} + \frac{1}{k^2} \right) - 4 \pi \left( k r - 1 \right) e^{kr} \right] + \alpha^- \left[ \dots \right] = \\
      & = 4 \pi \left( \alpha^+ + \alpha^- \right) \ .
  \end{aligned}\end{split}
\end{equation*}
\end{itemize}


\section{Wave equation}
\label{\detokenize{ch/green-function:wave-equation}}
\sphinxAtStartPar
Wave equation general problem
\begin{equation*}
\begin{split}\begin{cases}
  \dfrac{1}{c^2} \partial_{tt} \mathbf{u}(\mathbf{r},t) - \nabla^2 \mathbf{u}(\mathbf{r},t) = \mathbf{f}(\mathbf{r},t) \\
  \text{+ b.c.} \\
  \text{+ i.c.} \\
\end{cases}\end{split}
\end{equation*}
\sphinxAtStartPar
Green’s problem of the wave equation
\begin{equation*}
\begin{split}\frac{1}{c^2} \partial_{tt} G(\mathbf{r},t;\mathbf{r}_0,t_0) - \nabla_{\mathbf{r}}^2 G(\mathbf{r},t;\mathbf{r}_0,t_0) = \delta(\mathbf{r}-\mathbf{r}_0) \delta(t-t_0)\end{split}
\end{equation*}
\sphinxAtStartPar
Integration by parts
\begin{equation*}
\begin{split}\begin{aligned}
  E(\mathbf{r}_{\alpha}, t_{\alpha}) \mathbf{u}(\mathbf{r}_{\alpha},t_{\alpha}) & = \int_{t \in T} \int_{\mathbf{r} \in V} \delta(t-t_{\alpha}) \delta(\mathbf{r}-\mathbf{r}_{\alpha}) \mathbf{u}(\mathbf{r},t) = \\
  & = \int_{t \in T} \int_{\mathbf{r} \in V} \left\{ \frac{1}{c^2} \partial_{tt} G - \nabla^2_{\mathbf{r}} G \right\} \mathbf{u} = \\
  & = \int_{t \in T} \int_{\mathbf{r} \in V} \left\{ \frac{1}{c^2} \left[ \partial_t \left( \mathbf{u} \partial_t G - G \partial_t \mathbf{u} \right) + G \partial_{tt} \mathbf{u} \right] - \nabla_{\mathbf{r}} \cdot \left( \nabla_{\mathbf{r}} G \, \mathbf{u} - G \nabla_{\mathbf{r}} \mathbf{u} \right) - G \, \nabla^2_{\mathbf{r}} \mathbf{u} \right\} = \\
  & = \int_{\mathbf{r} \in V} \frac{1}{c^2} \left[ \mathbf{u}(\mathbf{r},t) \partial_t G(\mathbf{r},t; \mathbf{r}_{\alpha},t_{\alpha}) - G(\mathbf{r},t; \mathbf{r}_{\alpha},t_{\alpha}) \partial_t \mathbf{u}(\mathbf{r},t) \right] \bigg|_{t_0}^{t_1} + \\
  & \quad + \int_{t \in T} \oint_{\mathbf{r} \in \partial V} \left\{ - \hat{\mathbf{n}}(\mathbf{r},t) \cdot \nabla_{\mathbf{r}} G(\mathbf{r},t; \mathbf{r}_{\alpha},t_{\alpha}) \, \mathbf{u}(\mathbf{r},t) + G(\mathbf{r},t; \mathbf{r}_{\alpha},t_{\alpha}) \, \hat{\mathbf{n}}(\mathbf{r},t) \cdot \nabla_{\mathbf{r}} \mathbf{u}(\mathbf{r},t) \right\} + \\
  & \quad + \int_{t \in T}  \int_{\mathbf{r} \in V} G(\mathbf{r},t; \mathbf{r}_{\alpha},t_{\alpha}) \underbrace{ \left\{ \frac{1}{c^2} \partial_{tt} \mathbf{u}(\mathbf{r},t) - \nabla^2_{\mathbf{r}} \mathbf{u}(\mathbf{r},t) \right\}}_{= \mathbf{f}(\mathbf{r},t)} \\
\end{aligned}\end{split}
\end{equation*}\begin{equation*}
\begin{split}\begin{aligned}
  \int_{t \in T} \int_{\mathbf{r} \in V} \frac{1}{4 \pi}\frac{\delta\left( t-t_{\alpha} + \frac{|\mathbf{r} - \mathbf{r}_{\alpha}|}{c} \right)}{|\mathbf{r} - \mathbf{r}_{\alpha}|} \, \mathbf{f}(\mathbf{r},t) 
  & = \int_{\mathbf{r} \in V \cap B_{|\mathbf{r} - \mathbf{r}_{\alpha}| \le c (t_\alpha - t)}} \frac{1}{4 \pi |\mathbf{r} - \mathbf{r}_{\alpha}|} \mathbf{f}\left(\mathbf{r}, t_{\alpha} - \frac{|\mathbf{r}-\mathbf{r}_{\alpha}|}{c}\right)
\end{aligned}\end{split}
\end{equation*}
\sphinxstepscope

\begin{sphinxuseclass}{sd-container-fluid}
\begin{sphinxuseclass}{sd-sphinx-override}
\begin{sphinxuseclass}{sd-p-0}
\begin{sphinxuseclass}{sd-mt-2}
\begin{sphinxuseclass}{sd-mb-4}
\begin{sphinxuseclass}{sd-row}
\begin{sphinxuseclass}{sd-row-cols-2}
\begin{sphinxuseclass}{sd-gx-2}
\begin{sphinxuseclass}{sd-gy-1}
\begin{sphinxuseclass}{sd-col}
\begin{sphinxuseclass}{sd-d-flex-row}
\begin{sphinxuseclass}{sd-align-minor-center}
\begin{sphinxuseclass}{sd-container-fluid}
\begin{sphinxuseclass}{sd-sphinx-override}
\begin{sphinxuseclass}{sd-row}
\begin{sphinxuseclass}{sd-row-cols-2}
\begin{sphinxuseclass}{sd-row-cols-xs-2}
\begin{sphinxuseclass}{sd-row-cols-sm-3}
\begin{sphinxuseclass}{sd-row-cols-md-3}
\begin{sphinxuseclass}{sd-row-cols-lg-3}
\begin{sphinxuseclass}{sd-gx-3}
\begin{sphinxuseclass}{sd-gy-1}
\begin{sphinxuseclass}{sd-col}
\begin{sphinxuseclass}{sd-col-auto}
\begin{sphinxuseclass}{sd-d-flex-row}
\begin{sphinxuseclass}{sd-align-minor-center}
\sphinxAtStartPar
basics

\end{sphinxuseclass}
\end{sphinxuseclass}
\end{sphinxuseclass}
\end{sphinxuseclass}
\begin{sphinxuseclass}{sd-col}
\begin{sphinxuseclass}{sd-col-auto}
\begin{sphinxuseclass}{sd-d-flex-row}
\begin{sphinxuseclass}{sd-align-minor-center}
\sphinxAtStartPar
26 apr 2025

\end{sphinxuseclass}
\end{sphinxuseclass}
\end{sphinxuseclass}
\end{sphinxuseclass}
\begin{sphinxuseclass}{sd-col}
\begin{sphinxuseclass}{sd-col-auto}
\begin{sphinxuseclass}{sd-d-flex-row}
\begin{sphinxuseclass}{sd-align-minor-center}
\sphinxAtStartPar
1 min read

\end{sphinxuseclass}
\end{sphinxuseclass}
\end{sphinxuseclass}
\end{sphinxuseclass}
\end{sphinxuseclass}
\end{sphinxuseclass}
\end{sphinxuseclass}
\end{sphinxuseclass}
\end{sphinxuseclass}
\end{sphinxuseclass}
\end{sphinxuseclass}
\end{sphinxuseclass}
\end{sphinxuseclass}
\end{sphinxuseclass}
\end{sphinxuseclass}
\end{sphinxuseclass}
\end{sphinxuseclass}
\end{sphinxuseclass}
\end{sphinxuseclass}
\end{sphinxuseclass}
\end{sphinxuseclass}
\end{sphinxuseclass}
\end{sphinxuseclass}
\end{sphinxuseclass}
\end{sphinxuseclass}
\end{sphinxuseclass}

\chapter{Metodi numerici}
\label{\detokenize{ch/numerical-methods:metodi-numerici}}\label{\detokenize{ch/numerical-methods:classical-electromagnetism-numerics}}\label{\detokenize{ch/numerical-methods::doc}}

\section{Elettrostatica}
\label{\detokenize{ch/numerical-methods:elettrostatica}}
\sphinxAtStartPar
I problemi dell’elettrostatica sono governate dalle due equazioni di Maxwell per i campi \(\mathbf{e}\), \(\mathbf{d}\),
\begin{equation*}
\begin{split}\begin{cases}
  \nabla \cdot \mathbf{d} = \rho \\ \\
  \nabla \times \mathbf{e} = \mathbf{0 \ ,}
\end{cases}\end{split}
\end{equation*}
\sphinxAtStartPar
dotate delle opportune condizioni al contorno ed equazioni costitutive. Per un materiale lineare isotropo, ad esempio, \(\mathbf{d} = \varepsilon \mathbf{e}\). La condizione di irrotazionalità del campo elettrico, permette di scriverlo come gradiente di un potenziale scalare, \(\mathbf{e} = - \nabla v\), e di ottenere l’equazione di Poisson,
\begin{equation*}
\begin{split}-\nabla \cdot (\varepsilon \nabla v ) = \rho \ .\end{split}
\end{equation*}

\subsection{Sorgente}
\label{\detokenize{ch/numerical-methods:sorgente}}\begin{equation*}
\begin{split}\mathbf{e}(r) = \frac{q_i}{4 \pi \varepsilon}\frac{\mathbf{r} - \mathbf{r}_i}{|\mathbf{r} - \mathbf{r}_i|^3}\end{split}
\end{equation*}\begin{equation*}
\begin{split}\mathbf{e}(\mathbf{r}) = - \nabla_{\mathbf{r}} v(\mathbf{r})\end{split}
\end{equation*}\begin{equation*}
\begin{split}\varepsilon \, v(\mathbf{r}) = \frac{q_i}{4 \pi}\frac{1}{|\mathbf{r} - \mathbf{r}_i|}\end{split}
\end{equation*}

\subsection{Dipolo}
\label{\detokenize{ch/numerical-methods:dipolo}}
\sphinxAtStartPar
Un dipolo è definito come due cariche di intensità uguale e contraria \(-q_2 = q_1 = q > 0\), nei punti dello spazio \(P_1\), \(P_2 = P_1 + \mathbf{l}\), nelle condizioni limite \(|\mathbf{l}| \rightarrow 0\), \(q \rightarrow \infty\), in modo tale da avere \(q |\mathbf{l}|\) finito, \(\mathbf{p} = q \mathbf{l}\).

\sphinxAtStartPar
Il potenziale del dipolo è dato dal principio di sovrapposizione delle cause e degli effetti,
\begin{equation*}
\begin{split}\begin{aligned}
  \varepsilon \, v(\mathbf{r})
  & = - \frac{q}{4 \pi }\frac{1}{\left|\mathbf{r} - \mathbf{r}_0 + \frac{\mathbf{l}}{2} \right|} 
      + \frac{q}{4 \pi }\frac{1}{\left|\mathbf{r} - \mathbf{r}_0 - \frac{\mathbf{l}}{2} \right|} = \\
  & = \ ... \\
  & = \frac{q}{4 \pi} \left( 
  - \frac{1}{\left|\mathbf{r} - \mathbf{r}_0 \right|} + \frac{\mathbf{r} - \mathbf{r}_0}{\left|\mathbf{r} - \mathbf{r}_0 \right|^3} \cdot \frac{\mathbf{l}}{2}
  + \frac{1}{\left|\mathbf{r} - \mathbf{r}_0 \right|} + \frac{\mathbf{r} - \mathbf{r}_0}{\left|\mathbf{r} - \mathbf{r}_0 \right|^3} \cdot \frac{\mathbf{l}}{2} + o(|\mathbf{l}|) \right) = \\
  & = \ ... \\
  & = \frac{1}{4 \pi}
 \frac{\mathbf{r} - \mathbf{r}_0}{\left|\mathbf{r} - \mathbf{r}_0 \right|^3} \cdot \mathbf{P} \ ,
\end{aligned}\end{split}
\end{equation*}
\sphinxAtStartPar
avendo definito il vettore momento dipolo \(\mathbf{P} = q \mathbf{l}\).

\sphinxAtStartPar
\sphinxstylestrong{Polariazazione \sphinxhyphen{} Potenziale generato da una distribuzione di dipoli.}
\begin{equation*}
\begin{split}d \mathbf{P} = \mathbf{p} \, \Delta V\end{split}
\end{equation*}\begin{equation*}
\begin{split}\varepsilon v_P(\mathbf{r}) = \int_{\mathbf{r}_0 \in V_0} \frac{1}{4 \pi}
 \frac{\mathbf{r} - \mathbf{r}_0}{\left|\mathbf{r} - \mathbf{r}_0 \right|^3} \cdot \mathbf{p}(\mathbf{r}_0) \, dV_0 \end{split}
\end{equation*}\begin{equation*}
\begin{split}\begin{aligned}
\partial_i |\mathbf{r}|^2 & = 2 x_i \\
                          & = 2 |\mathbf{r}| \partial_i |\mathbf{r}|
\end{aligned}
\qquad \rightarrow \qquad \partial_i |\mathbf{r}| = \frac{x_i}{|\mathbf{r}|}\end{split}
\end{equation*}\begin{equation*}
\begin{split}\partial_i |\mathbf{r}|^n = n |\mathbf{r}|^{n-1} \, \partial_i |\mathbf{r}| = n x_i |\mathbf{r}|^{n-2}\end{split}
\end{equation*}\begin{equation*}
\begin{split}\frac{\mathbf{r}-\mathbf{r}_0}{|\mathbf{r}-\mathbf{r}_0|^3} = \nabla_{\mathbf{r}_0} \frac{1}{|\mathbf{r}-\mathbf{r}_0|}\end{split}
\end{equation*}\begin{equation*}
\begin{split}\begin{aligned}
\frac{\mathbf{r}- \mathbf{r}_0}{|\mathbf{r}- \mathbf{r}_0|^3} \cdot \mathbf{p}(\mathbf{r}_0) 
 & = \nabla_{\mathbf{r}_0} \frac{1}{|\mathbf{r}-\mathbf{r}_0|} \cdot \mathbf{p}(\mathbf{r}_0) = \\
 & = \nabla_{\mathbf{r}_0} \cdot \left( \frac{1}{|\mathbf{r}-\mathbf{r}_0|} \mathbf{p}(\mathbf{r}_0) \right) - \frac{1}{|\mathbf{r}- \mathbf{r}_0|} \nabla_{\mathbf{r}_0} \cdot \mathbf{p}(\mathbf{r}_0) = \\
\end{aligned}\end{split}
\end{equation*}
\sphinxAtStartPar
e quindi
\begin{equation*}
\begin{split}4 \, \pi \, \varepsilon v_P(\mathbf{r}) = \oint_{\mathbf{r}_0 \in \partial V_0} \frac{\hat{\mathbf{n}}(\mathbf{r}_0) \cdot \mathbf{p}(\mathbf{r}_0)}{|\mathbf{r}-\mathbf{r}_0|} - \oint_{\mathbf{r}_0 \in V_0} \frac{\nabla_{\mathbf{r}_0} \cdot \mathbf{p}(\mathbf{r}_0)}{|\mathbf{r} - \mathbf{r}_0|}\end{split}
\end{equation*}
\sphinxAtStartPar
I due contributi hanno la forma di sorgenti, essendo termini proporzionali a \(\frac{1}{|\mathbf{r}-\mathbf{r}_0|}\).
Il potenziale dovuto alla densità di volume di dipoli equivale alla somma dei due contributi delle cariche di:
\begin{itemize}
\item {} 
\sphinxAtStartPar
polarizzazione di superficie \(\sigma_p =   \hat{\mathbf{n}} \cdot \mathbf{p}\)

\item {} 
\sphinxAtStartPar
polarizzazione di volume     \(\rho_p   = - \nabla \cdot \mathbf{p}\)

\end{itemize}

\sphinxAtStartPar
\sphinxstylestrong{Oss.} Se la polarizzazione è uniforme nel volume, il contributo della polarizzazione nel volume si annulla e rimane solo il contributo della polarizzazione sul contorno del volume.

\sphinxAtStartPar
\sphinxstylestrong{Oss.} Legge di Gauss per il campo elettrico,
\begin{equation*}
\begin{split}\begin{aligned}
  \nabla \cdot \mathbf{e} & = \frac{1}{\varepsilon_0} \rho = \\
                          & = \frac{1}{\varepsilon_0} \left( \rho_l + \rho_p \right) = \\
                          & = \frac{1}{\varepsilon_0} \left( \rho_l - \nabla \cdot \mathbf{p} \right) \\
  \nabla \cdot \left( \varepsilon_0 \mathbf{e} + \mathbf{p} \right) & = \rho_l \\
  \nabla \cdot  \mathbf{d} & = \rho_l
\end{aligned}\end{split}
\end{equation*}
\sphinxstepscope


\part{Appendici}

\sphinxstepscope

\begin{sphinxuseclass}{sd-container-fluid}
\begin{sphinxuseclass}{sd-sphinx-override}
\begin{sphinxuseclass}{sd-p-0}
\begin{sphinxuseclass}{sd-mt-2}
\begin{sphinxuseclass}{sd-mb-4}
\begin{sphinxuseclass}{sd-row}
\begin{sphinxuseclass}{sd-row-cols-2}
\begin{sphinxuseclass}{sd-gx-2}
\begin{sphinxuseclass}{sd-gy-1}
\begin{sphinxuseclass}{sd-col}
\begin{sphinxuseclass}{sd-d-flex-row}
\begin{sphinxuseclass}{sd-align-minor-center}
\begin{sphinxuseclass}{sd-container-fluid}
\begin{sphinxuseclass}{sd-sphinx-override}
\begin{sphinxuseclass}{sd-row}
\begin{sphinxuseclass}{sd-row-cols-2}
\begin{sphinxuseclass}{sd-row-cols-xs-2}
\begin{sphinxuseclass}{sd-row-cols-sm-3}
\begin{sphinxuseclass}{sd-row-cols-md-3}
\begin{sphinxuseclass}{sd-row-cols-lg-3}
\begin{sphinxuseclass}{sd-gx-3}
\begin{sphinxuseclass}{sd-gy-1}
\begin{sphinxuseclass}{sd-col}
\begin{sphinxuseclass}{sd-col-auto}
\begin{sphinxuseclass}{sd-d-flex-row}
\begin{sphinxuseclass}{sd-align-minor-center}
\sphinxAtStartPar
basics

\end{sphinxuseclass}
\end{sphinxuseclass}
\end{sphinxuseclass}
\end{sphinxuseclass}
\begin{sphinxuseclass}{sd-col}
\begin{sphinxuseclass}{sd-col-auto}
\begin{sphinxuseclass}{sd-d-flex-row}
\begin{sphinxuseclass}{sd-align-minor-center}
\sphinxAtStartPar
26 apr 2025

\end{sphinxuseclass}
\end{sphinxuseclass}
\end{sphinxuseclass}
\end{sphinxuseclass}
\begin{sphinxuseclass}{sd-col}
\begin{sphinxuseclass}{sd-col-auto}
\begin{sphinxuseclass}{sd-d-flex-row}
\begin{sphinxuseclass}{sd-align-minor-center}
\sphinxAtStartPar
0 min read

\end{sphinxuseclass}
\end{sphinxuseclass}
\end{sphinxuseclass}
\end{sphinxuseclass}
\end{sphinxuseclass}
\end{sphinxuseclass}
\end{sphinxuseclass}
\end{sphinxuseclass}
\end{sphinxuseclass}
\end{sphinxuseclass}
\end{sphinxuseclass}
\end{sphinxuseclass}
\end{sphinxuseclass}
\end{sphinxuseclass}
\end{sphinxuseclass}
\end{sphinxuseclass}
\end{sphinxuseclass}
\end{sphinxuseclass}
\end{sphinxuseclass}
\end{sphinxuseclass}
\end{sphinxuseclass}
\end{sphinxuseclass}
\end{sphinxuseclass}
\end{sphinxuseclass}
\end{sphinxuseclass}
\end{sphinxuseclass}

\chapter{Ottica}
\label{\detokenize{ch/optics:ottica}}\label{\detokenize{ch/optics:classical-electromagnetism-optics}}\label{\detokenize{ch/optics::doc}}





\renewcommand{\indexname}{Proof Index}
\begin{sphinxtheindex}
\let\bigletter\sphinxstyleindexlettergroup
\bigletter{example\sphinxhyphen{}0}
\item\relax\sphinxstyleindexentry{example\sphinxhyphen{}0}\sphinxstyleindexextra{ch/circuits\sphinxhyphen{}electromechanic}\sphinxstyleindexpageref{ch/circuits-electromechanic:\detokenize{example-0}}
\indexspace
\bigletter{harmonic:effective\sphinxhyphen{}values}
\item\relax\sphinxstyleindexentry{harmonic:effective\sphinxhyphen{}values}\sphinxstyleindexextra{ch/electrical\sphinxhyphen{}engineering\sphinxhyphen{}networks\sphinxhyphen{}harmonic}\sphinxstyleindexpageref{ch/electrical-engineering-networks-harmonic:\detokenize{harmonic:effective-values}}
\end{sphinxtheindex}

\renewcommand{\indexname}{Indice}
\printindex
\end{document}